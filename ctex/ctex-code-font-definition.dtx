% \iffalse meta-comment
%
% Copyright (C) 2003--2020
% CTEX.ORG and any individual authors listed elsewhere in this file.
% --------------------------------------------------------------------------
%
% This work may be distributed and/or modified under the
% conditions of the LaTeX Project Public License, either
% version 1.3c of this license or (at your option) any later
% version. This version of this license is in
%    http://www.latex-project.org/lppl/lppl-1-3c.txt
% and the latest version of this license is in
%    http://www.latex-project.org/lppl.txt
% and version 1.3 or later is part of all distributions of
% LaTeX version 2005/12/01 or later.
%
% This work has the LPPL maintenance status `maintained'.
%
% The Current Maintainers of this work are Leo Liu, Qing Lee and Liam Huang.
%
% --------------------------------------------------------------------------
%
% \fi
%
% \section{字体定义文件}
%
% \subsection{传统定义方式}
%
%    \begin{macrocode}
%<*c19|c70>
%%
%% Chinese characters
%%
%<c19>%% character set: GBK (extension of GB 2312)
%<c70>%% character set: Unicode
%% font encoding: Unicode
%%
%</c19|c70>
%    \end{macrocode}
%
% \pkg{CJK} 宏包使用的字体族。
%    \begin{macrocode}
%<rm&c19>\DeclareFontFamily{C19}{rm}{\hyphenchar\font\m@ne}
%<rm&c70>\DeclareFontFamily{C70}{rm}{\hyphenchar\font\m@ne}
%<sf&c19>\DeclareFontFamily{C19}{sf}{\hyphenchar\font\m@ne}
%<sf&c70>\DeclareFontFamily{C70}{sf}{\hyphenchar\font\m@ne}
%<tt&c19>\DeclareFontFamily{C19}{tt}{\hyphenchar\font\m@ne}
%<tt&c70>\DeclareFontFamily{C70}{tt}{\hyphenchar\font\m@ne}
%    \end{macrocode}
%
% \upLaTeX{} 使用的字体族。\upLaTeX 在 NFSS 下使用字体编码 |JY2| 和 |JT2| 来分别
% 表示横排与直排的日文。
%    \begin{macrocode}
%<rm&jy2>\DeclareKanjiFamily{JY2}{zhrm}{}
%<rm&jt2>\DeclareKanjiFamily{JT2}{zhrm}{}
%<sf&jy2>\DeclareKanjiFamily{JY2}{zhsf}{}
%<sf&jt2>\DeclareKanjiFamily{JT2}{zhsf}{}
%<tt&jy2>\DeclareKanjiFamily{JY2}{zhtt}{}
%<tt&jt2>\DeclareKanjiFamily{JT2}{zhtt}{}
%    \end{macrocode}
%
%
%    \begin{macrocode}
%<*rm>
%<*c19>
\DeclareFontShape{C19}{rm}{m}{n}{<-> CJK * gbksong}{\CJKnormal}
\DeclareFontShape{C19}{rm}{b}{n}{<-> CJK * gbkhei}{\CJKnormal}
\DeclareFontShape{C19}{rm}{bx}{n}{<-> CJK * gbkhei}{\CJKnormal}
\DeclareFontShape{C19}{rm}{m}{sl}{<-> CJK * gbksongsl}{\CJKnormal}
\DeclareFontShape{C19}{rm}{b}{sl}{<-> CJK * gbkheisl}{\CJKnormal}
\DeclareFontShape{C19}{rm}{bx}{sl}{<-> CJK * gbkheisl}{\CJKnormal}
\DeclareFontShape{C19}{rm}{m}{it}{<-> CJK * gbkkai}{\CJKnormal}
\DeclareFontShape{C19}{rm}{b}{it}{<-> CJKb * gbkkai}{\CJKbold}
\DeclareFontShape{C19}{rm}{bx}{it}{<-> CJKb * gbkkai}{\CJKbold}
%</c19>
%<*c70>
\DeclareFontShape{C70}{rm}{m}{n}{<-> CJK * unisong}{\CJKnormal}
\DeclareFontShape{C70}{rm}{b}{n}{<-> CJK * unihei}{\CJKnormal}
\DeclareFontShape{C70}{rm}{bx}{n}{<-> CJK * unihei}{\CJKnormal}
\DeclareFontShape{C70}{rm}{m}{sl}{<-> CJK * unisongsl}{\CJKnormal}
\DeclareFontShape{C70}{rm}{b}{sl}{<-> CJK * uniheisl}{\CJKnormal}
\DeclareFontShape{C70}{rm}{bx}{sl}{<-> CJK * uniheisl}{\CJKnormal}
\DeclareFontShape{C70}{rm}{m}{it}{<-> CJK * unikai}{\CJKnormal}
\DeclareFontShape{C70}{rm}{b}{it}{<-> CJKb * unikai}{\CJKbold}
\DeclareFontShape{C70}{rm}{bx}{it}{<-> CJKb * unikai}{\CJKbold}
%</c70>
%<*jy2>
\DeclareFontShape{JY2}{zhrm}{m}{n}{<-> upzhserif-h}{}
\DeclareFontShape{JY2}{zhrm}{m}{it}{<-> upzhserifit-h}{}
\DeclareFontShape{JY2}{zhrm}{bx}{n}{<-> upzhserifb-h}{}
%</jy2>
%<*jt2>
\DeclareFontShape{JT2}{zhrm}{m}{n}{<-> upzhserif-v}{}
\DeclareFontShape{JT2}{zhrm}{m}{it}{<-> upzhserifit-v}{}
\DeclareFontShape{JT2}{zhrm}{bx}{n}{<-> upzhserifb-v}{}
%</jt2>
%</rm>
%    \end{macrocode}
%
%    \begin{macrocode}
%<*sf>
%<*c19>
\DeclareFontShape{C19}{sf}{m}{n}{<-> CJK * gbkyou}{\CJKnormal}
\DeclareFontShape{C19}{sf}{b}{n}{<-> CJKb * gbkyou}{\CJKbold}
\DeclareFontShape{C19}{sf}{bx}{n}{<-> CJKb * gbkyou}{\CJKbold}
\DeclareFontShape{C19}{sf}{m}{sl}{<-> CJK * gbkyousl}{\CJKnormal}
\DeclareFontShape{C19}{sf}{b}{sl}{<-> CJKb * gbkyousl}{\CJKbold}
\DeclareFontShape{C19}{sf}{bx}{sl}{<-> CJKb * gbkyousl}{\CJKbold}
\DeclareFontShape{C19}{sf}{m}{it}{<-> CJK * gbkyou}{\CJKnormal}
\DeclareFontShape{C19}{sf}{b}{it}{<-> CJKb * gbkyou}{\CJKbold}
\DeclareFontShape{C19}{sf}{bx}{it}{<-> CJKb * gbkyou}{\CJKbold}
%</c19>
%<*c70>
\DeclareFontShape{C70}{sf}{m}{n}{<-> CJK * uniyou}{\CJKnormal}
\DeclareFontShape{C70}{sf}{b}{n}{<-> CJKb * uniyou}{\CJKbold}
\DeclareFontShape{C70}{sf}{bx}{n}{<-> CJKb * uniyou}{\CJKbold}
\DeclareFontShape{C70}{sf}{m}{sl}{<-> CJK * uniyousl}{\CJKnormal}
\DeclareFontShape{C70}{sf}{b}{sl}{<-> CJKb * uniyousl}{\CJKbold}
\DeclareFontShape{C70}{sf}{bx}{sl}{<-> CJKb * uniyousl}{\CJKbold}
\DeclareFontShape{C70}{sf}{m}{it}{<-> CJK * uniyou}{\CJKnormal}
\DeclareFontShape{C70}{sf}{b}{it}{<-> CJKb * uniyou}{\CJKbold}
\DeclareFontShape{C70}{sf}{bx}{it}{<-> CJKb * uniyou}{\CJKbold}
%</c70>
%<*jy2>
\DeclareFontShape{JY2}{zhsf}{m}{n}{<-> upzhsans-h}{}
\DeclareFontShape{JY2}{zhsf}{bx}{n}{<-> upzhsansb-h}{}
%</jy2>
%<*jt2>
\DeclareFontShape{JT2}{zhsf}{m}{n}{<-> upzhsans-v}{}
\DeclareFontShape{JT2}{zhsf}{bx}{n}{<-> upzhsansb-v}{}
%</jt2>
%</sf>
%    \end{macrocode}
%
%    \begin{macrocode}
%<*tt>
%<*c19>
\DeclareFontShape{C19}{tt}{m}{n}{<-> CJK * gbkfs}{\CJKnormal}
\DeclareFontShape{C19}{tt}{b}{n}{<-> CJKb * gbkfs}{\CJKbold}
\DeclareFontShape{C19}{tt}{bx}{n}{<-> CJKb * gbkfs}{\CJKbold}
\DeclareFontShape{C19}{tt}{m}{sl}{<-> CJK * gbkfssl}{\CJKnormal}
\DeclareFontShape{C19}{tt}{b}{sl}{<-> CJKb * gbkfssl}{\CJKbold}
\DeclareFontShape{C19}{tt}{bx}{sl}{<-> CJKb * gbkfssl}{\CJKbold}
\DeclareFontShape{C19}{tt}{m}{it}{<-> CJK * gbkfs}{\CJKnormal}
\DeclareFontShape{C19}{tt}{b}{it}{<-> CJKb * gbkfs}{\CJKbold}
\DeclareFontShape{C19}{tt}{bx}{it}{<-> CJKb * gbkfs}{\CJKbold}
%</c19>
%<*c70>
\DeclareFontShape{C70}{tt}{m}{n}{<-> CJK * unifs}{\CJKnormal}
\DeclareFontShape{C70}{tt}{b}{n}{<-> CJKb * unifs}{\CJKbold}
\DeclareFontShape{C70}{tt}{bx}{n}{<-> CJKb * unifs}{\CJKbold}
\DeclareFontShape{C70}{tt}{m}{sl}{<-> CJK * unifssl}{\CJKnormal}
\DeclareFontShape{C70}{tt}{b}{sl}{<-> CJKb * unifssl}{\CJKbold}
\DeclareFontShape{C70}{tt}{bx}{sl}{<-> CJKb * unifssl}{\CJKbold}
\DeclareFontShape{C70}{tt}{m}{it}{<-> CJK * unifs}{\CJKnormal}
\DeclareFontShape{C70}{tt}{b}{it}{<-> CJKb * unifs}{\CJKbold}
\DeclareFontShape{C70}{tt}{bx}{it}{<-> CJKb * unifs}{\CJKbold}
%</c70>
%<*jy2>
\DeclareFontShape{JY2}{zhtt}{m}{n}{<-> upzhmono-h}{}
%</jy2>
%<*jt2>
\DeclareFontShape{JT2}{zhtt}{m}{n}{<-> upzhmono-v}{}
%</jt2>
%</tt>
%    \end{macrocode}
%
% \subsection{预定义字库}
%
% \begin{macro}{\@@_define_fontset:nnn}
% 3 个参数依次为 \pdfTeX{}、\upTeX{} 和 \XeTeX{}/\LuaTeX{}。
%    \begin{macrocode}
%<*fontset>
%<*!mac>
\cs_new_protected_nopar:Npx \@@_define_fontset:nnn #1#2#3
  {
    \sys_if_engine_pdftex:TF
      {#1} { \sys_if_engine_uptex:TF {#2} {#3} }
  }
%    \end{macrocode}
% \end{macro}
%
% \begin{macro}{\@@_define_fontset:nnnn}
% 4 个参数依次为 \pdfTeX{}(生成 PDF)、\pdfTeX{}(生成 DVI)、\upTeX{} 和
% \XeTeX{}/\LuaTeX{}。
%    \begin{macrocode}
\cs_new_protected_nopar:Npx \@@_define_fontset:nnnn #1#2#3#4
  {
    \sys_if_engine_pdftex:TF
      { \sys_if_output_pdf:TF   {#1} {#2} }
      { \sys_if_engine_uptex:TF {#3} {#4} }
  }
%</!mac>
%    \end{macrocode}
% \end{macro}
%
% \subsubsection{\opt{adobe}}
%
% \tn{pdfmapline} 不支持 OpenType 字体,因而 \opt{adobe} 字体集在 pdf 模式下
% 就没有定义。\opt{fandol} 的情况类似。
%    \begin{macrocode}
%<*adobe>
\@@_define_fontset:nnnn
  { \ctex_fontset_error:n { adobe } }
  {
    \ctex_zhmap_case:nnn
      {
        \setCJKmainfont  { AdobeSongStd-Light.otf }
          [
            cmap       = UniGB-UTF16-H,
            BoldFont   = AdobeHeitiStd-Regular.otf,
            ItalicFont = AdobeKaitiStd-Regular.otf
          ]
        \setCJKsansfont { AdobeHeitiStd-Regular.otf }
          [ cmap = UniGB-UTF16-H ]
        \setCJKmonofont { AdobeFangsongStd-Regular.otf }
          [ cmap = UniGB-UTF16-H ]
        \setCJKfamilyfont { zhsong } { AdobeSongStd-Light.otf       }
          [ cmap = UniGB-UTF16-H ]
        \setCJKfamilyfont { zhhei  } { AdobeHeitiStd-Regular.otf    }
          [ cmap = UniGB-UTF16-H ]
        \setCJKfamilyfont { zhkai  } { AdobeKaitiStd-Regular.otf    }
          [ cmap = UniGB-UTF16-H ]
        \setCJKfamilyfont { zhfs   } { AdobeFangsongStd-Regular.otf }
          [ cmap = UniGB-UTF16-H ]
        \ctex_punct_set:n { adobe }
        \ctex_punct_map_family:nn   { \CJKrmdefault } { zhsong }
        \ctex_punct_map_family:nn   { \CJKsfdefault } { zhhei  }
        \ctex_punct_map_family:nn   { \CJKttdefault } { zhfs   }
        \ctex_punct_map_bfseries:nn { \CJKrmdefault } { zhhei  }
        \ctex_punct_map_itshape:nn  { \CJKrmdefault } { zhkai  }
      }
      {
        \ctex_load_zhmap:nnnn { rm } { zhhei } { zhfs } { zhadobefonts }
        \ctex_punct_set:n { adobe }
        \ctex_punct_map_family:nn   { \CJKrmdefault } { zhsong }
        \ctex_punct_map_bfseries:nn { \CJKrmdefault } { zhhei  }
        \ctex_punct_map_itshape:nn  { \CJKrmdefault } { zhkai  }
      }
      { \ctex_fontset_error:n { adobe } }
  }
  {
    \ctex_set_upfonts:nnnnnn
      { AdobeSongStd-Light.otf       }
      { AdobeHeitiStd-Regular.otf    }
      { AdobeKaitiStd-Regular.otf    }
      { AdobeHeitiStd-Regular.otf    }
      { AdobeHeitiStd-Regular.otf    }
      { AdobeFangsongStd-Regular.otf }
    \ctex_set_upfamily:nnn { zhsong } { upzhserif   } {}
    \ctex_set_upfamily:nnn { zhhei  } { upzhsans    } {}
    \ctex_set_upfamily:nnn { zhfs   } { upzhmono    } {}
    \ctex_set_upfamily:nnn { zhkai  } { upzhserifit } {}
  }
  {
    \setCJKmainfont { AdobeSongStd-Light       }
      [ BoldFont = AdobeHeitiStd-Regular, ItalicFont = AdobeKaitiStd-Regular ]
    \setCJKsansfont { AdobeHeitiStd-Regular    }
    \setCJKmonofont { AdobeFangsongStd-Regular }
    \setCJKfamilyfont { zhsong } { AdobeSongStd-Light       }
    \setCJKfamilyfont { zhhei  } { AdobeHeitiStd-Regular    }
    \setCJKfamilyfont { zhfs   } { AdobeFangsongStd-Regular }
    \setCJKfamilyfont { zhkai  } { AdobeKaitiStd-Regular    }
  }
%</adobe>
%    \end{macrocode}
%
% \subsubsection{\opt{fandol}}
%
%    \begin{macrocode}
%<*fandol>
\@@_define_fontset:nnnn
  { \ctex_fontset_error:n { fandol } }
  {
    \ctex_zhmap_case:nnn
      {
        \setCJKmainfont { FandolSong-Regular.otf }
          [
            cmap       = UniGB-UTF16-H,
            BoldFont   = FandolSong-Bold.otf,
            ItalicFont = FandolKai-Regular.otf
          ]
        \setCJKsansfont { FandolHei-Regular.otf }
          [ cmap = UniGB-UTF16-H, BoldFont = FandolHei-Bold.otf ]
        \setCJKmonofont { FandolFang-Regular.otf }
          [ cmap = UniGB-UTF16-H ]
        \setCJKfamilyfont { zhsong } { FandolSong-Regular.otf }
          [ cmap = UniGB-UTF16-H, BoldFont = FandolSong-Bold.otf ]
        \setCJKfamilyfont { zhhei  } { FandolHei-Regular.otf  }
          [ cmap = UniGB-UTF16-H, BoldFont = FandolHei-Bold.otf  ]
        \setCJKfamilyfont { zhfs   } { FandolFang-Regular.otf }
          [ cmap = UniGB-UTF16-H ]
        \setCJKfamilyfont { zhkai  } { FandolKai-Regular.otf  }
          [ cmap = UniGB-UTF16-H ]
        \ctex_punct_set:n { fandol }
        \ctex_punct_map_family:nn   { \CJKrmdefault         } { zhsong  }
        \ctex_punct_map_family:nn   { \CJKsfdefault         } { zhhei   }
        \ctex_punct_map_family:nn   { \CJKttdefault         } { zhfs    }
        \ctex_punct_map_bfseries:nn { \CJKrmdefault, zhsong } { zhsongb }
        \ctex_punct_map_bfseries:nn { \CJKsfdefault, zhhei  } { zhheib  }
        \ctex_punct_map_itshape:nn  { \CJKrmdefault         } { zhkai   }
      }
      {
        \ctex_load_zhmap:nnnn { rm } { zhhei } { zhfs } { zhfandolfonts }
        \ctex_punct_set:n { fandol }
        \ctex_punct_map_family:nn   { \CJKrmdefault } { zhsong }
        \ctex_punct_map_bfseries:nn { \CJKrmdefault } { zhhei  }
        \ctex_punct_map_itshape:nn  { \CJKrmdefault } { zhkai  }
      }
      { \ctex_fontset_error:n { fandol } }
  }
  {
    \ctex_set_upfonts:nnnnnn
      { FandolSong-Regular.otf }
      { FandolSong-Bold.otf    }
      { FandolKai-Regular.otf  }
      { FandolHei-Regular.otf  }
      { FandolHei-Bold.otf     }
      { FandolFang-Regular.otf }
    \ctex_set_upfamily:nnn { zhsong } { upzhserif   } { upzhserifb }
    \ctex_set_upfamily:nnn { zhhei  } { upzhsans    } { upzhsansb  }
    \ctex_set_upfamily:nnn { zhfs   } { upzhmono    } {}
    \ctex_set_upfamily:nnn { zhkai  } { upzhserifit } {}
  }
  {
    \setCJKmainfont { FandolSong-Regular }
      [
        Extension  = .otf,
        BoldFont   = FandolSong-Bold,
        ItalicFont = FandolKai-Regular
      ]
    \setCJKsansfont { FandolHei-Regular  }
      [ Extension = .otf, BoldFont = FandolHei-Bold ]
    \setCJKmonofont { FandolFang-Regular }
      [ Extension = .otf ]
    \setCJKfamilyfont { zhsong } { FandolSong-Regular }
      [ Extension = .otf, BoldFont = FandolSong-Bold ]
    \setCJKfamilyfont { zhhei  } { FandolHei-Regular  }
      [ Extension = .otf, BoldFont = FandolHei-Bold  ]
    \setCJKfamilyfont { zhfs   } { FandolFang-Regular }
      [ Extension = .otf ]
    \setCJKfamilyfont { zhkai  } { FandolKai-Regular  }
      [ Extension = .otf ]
  }
%</fandol>
%    \end{macrocode}
%
% \subsubsection{\opt{founder}}
%
%    \begin{macrocode}
%<*founder>
\@@_define_fontset:nnn
  {
    \ctex_zhmap_case:nnn
      {
        \setCJKmainfont { FZSSK.TTF }
          [ BoldFont = FZXBSK.TTF, ItalicFont = FZKTK.TTF ]
        \setCJKsansfont { FZXH1K.TTF } [ BoldFont = FZHTK.TTF ]
        \setCJKmonofont { FZFSK.TTF }
        \setCJKfamilyfont { zhsong } { FZSSK.TTF } [ BoldFont = FZXBSK.TTF ]
        \setCJKfamilyfont { zhhei  } { FZHTK.TTF }
        \setCJKfamilyfont { zhkai  } { FZKTK.TTF }
        \setCJKfamilyfont { zhfs   } { FZFSK.TTF }
        \setCJKfamilyfont { zhli   } { FZLSK.TTF }
        \setCJKfamilyfont { zhyou  } { FZY1K.TTF } [ BoldFont = FZY3K.TTF ]
        \ctex_punct_set:n { founder }
        \ctex_punct_map_family:nn   { \CJKrmdefault         } { zhsong  }
        \ctex_punct_map_family:nn   { \CJKsfdefault         } { zhheil  }
        \ctex_punct_map_family:nn   { \CJKttdefault         } { zhfs    }
        \ctex_punct_map_itshape:nn  { \CJKrmdefault         } { zhkai   }
        \ctex_punct_map_bfseries:nn { \CJKrmdefault, zhsong } { zhsongb }
        \ctex_punct_map_bfseries:nn { \CJKsfdefault         } { zhhei   }
        \ctex_punct_map_bfseries:nn { zhyou                 } { zhyoub  }
      }
      {
        \ctex_load_zhmap:nnnn { rm } { zhhei } { zhfs } { zhfounderfonts }
        \ctex_punct_set:n { founder }
        \ctex_punct_map_family:nn   { \CJKrmdefault } { zhsong }
        \ctex_punct_map_bfseries:nn { \CJKrmdefault } { zhhei  }
        \ctex_punct_map_itshape:nn  { \CJKrmdefault } { zhkai  }
      }
      { \ctex_fontset_error:n { founder } }
  }
  {
    \ctex_set_upfonts:nnnnnn
      { FZSSK.TTF  }
      { FZXBSK.TTF }
      { FZKTK.TTF  }
      { FZXH1K.TTF }
      { FZHTK.TTF  }
      { FZFSK.TTF  }
    \ctex_set_upfamily:nnn { zhsong } { upzhserif   } { upzhserifb }
    \ctex_set_upfamily:nnn { zhhei  } { upzhsans    } { upzhsansb  }
    \ctex_set_upfamily:nnn { zhfs   } { upzhmono    } {}
    \ctex_set_upfamily:nnn { zhkai  } { upzhserifit } {}
    \ctex_set_upfamily:nnn { zhli   } { upschrm     } {}
    \ctex_set_upfamily:nnn { zhyou  } { upschgt     } {}
    \ctex_set_upmap:nnn    { upstsl } { FZLSK.TTF } {}
    \ctex_set_upmap:nnn    { upstht } { FZY1K.TTF } {}
  }
  {
    \setCJKmainfont { FZShuSong-Z01 }
      [ BoldFont = FZXiaoBiaoSong-B05, ItalicFont = FZKai-Z03 ]
    \setCJKsansfont { FZXiHeiI-Z08 } [ BoldFont = FZHei-B01 ]
    \setCJKmonofont { FZFangSong-Z02 }
    \setCJKfamilyfont { zhsong } { FZShuSong-Z01  }
      [ BoldFont = FZXiaoBiaoSong-B05 ]
    \setCJKfamilyfont { zhhei  } { FZHei-B01      }
    \setCJKfamilyfont { zhkai  } { FZKai-Z03      }
    \setCJKfamilyfont { zhfs   } { FZFangSong-Z02 }
    \setCJKfamilyfont { zhli   } { FZLiShu-S01    }
    \setCJKfamilyfont { zhyou  } { FZXiYuan-M01   }
      [ BoldFont = FZZhunYuan-M02 ]
  }
%</founder>
%    \end{macrocode}
%
% \subsubsection{\opt{mac} 相关}
%
% 按 \href{https://github.com/CTeX-org/ctex-kit/issues/351}{Issue 351}
% 的讨论,以 El Capitan 为分界,分别设置 |macold|(El Capitan 之前)
% 和 |macnew|(El Capitan 及之后)。检测方式则以 El Capitan 及之后
% 的苹方字体为准。
%
%    \begin{macrocode}
%<*mac>
\file_if_exist:nTF { /System/Library/Fonts/PingFang.ttc }
  { \ctex_file_input:n { ctex-fontset-macnew.def } }
  { \ctex_file_input:n { ctex-fontset-macold.def } }
%</mac>
%    \end{macrocode}
%
% |macold| 的设置参考了
% \href{https://github.com/CTeX-org/ctex-kit/wiki/OS-X-Mavericks-(10.9)-预装的主要简体中文字体}^^A
% {OS X Mavericks (10.9) 预装的主要简体中文字体列表}。
%
% |macnew| 在默认字体设置方面,引入了多字重的宋体作为罗马字族,
% 以及引入了苹方黑体作为无衬线字族。
% 由于 Songti SC Light 的字重与 STSong 及 Windows 上的 SimSun 更接近,故默认字重
% 使用 Songti SC Light,而不带后缀的正常字重事实上没有使用。黑体、圆体等设置
% 也有类似的情况。
%
%    \begin{macrocode}
%<*macold|macnew>
\@@_define_fontset:nnnn
  { \ctex_fontset_error:n { mac } }
  {
%<macold>    \ctex_fontset_error:n { macold }
%<*macnew>
    \ctex_zhmap_case:nnn
      {
        \setCJKmainfont { :3:Songti.ttc }
          [
            BoldFont       = :1:Songti.ttc,
            ItalicFont     = :0:Kaiti.ttc,
            BoldItalicFont = :3:Kaiti.ttc,
          ]
        \setCJKsansfont { :2:PingFang.ttc } [ BoldFont = :8:PingFang.ttc ]
        \setCJKmonofont { STFANGSO.ttf    }
        \setCJKfamilyfont { zhsong } { :3:Songti.ttc   } [ BoldFont = :1:Songti.ttc   ]
        \setCJKfamilyfont { zhhei  } { :2:PingFang.ttc } [ BoldFont = :8:PingFang.ttc ]
        \setCJKfamilyfont { zhkai  } { :0:Kaiti.ttc    } [ BoldFont = :3:Kaiti.ttc    ]
        \setCJKfamilyfont { zhfs   } { STFANGSO.ttf    }
        \setCJKfamilyfont { zhli   } { :0:Baoli.ttc    }
        \setCJKfamilyfont { zhyou  } { :4:Yuanti.ttc   } [ BoldFont = :0:Yuanti.ttc   ]
        \ctex_punct_set:n { mac }
        \ctex_punct_map_family:nn   { \CJKrmdefault         } { zhsong  }
        \ctex_punct_map_family:nn   { \CJKsfdefault         } { zhheil  }
        \ctex_punct_map_family:nn   { \CJKttdefault         } { zhfs    }
        \ctex_punct_map_itshape:nn  { \CJKrmdefault         } { zhkai   }
        \ctex_punct_map_bfseries:nn { \CJKrmdefault, zhsong } { zhsongb }
        \ctex_punct_map_bfseries:nn { \CJKsfdefault         } { zhhei   }
        \ctex_punct_map_bfseries:nn { zhyou                 } { zhyoub  }
      }
      {
        \ctex_load_zhmap:nnnn { rm } { zhhei } { zhfs } { zhmacfonts }
        \ctex_punct_set:n { mac }
        \ctex_punct_map_family:nn   { \CJKrmdefault } { zhsong }
        \ctex_punct_map_bfseries:nn { \CJKrmdefault } { zhhei  }
        \ctex_punct_map_itshape:nn  { \CJKrmdefault } { zhkai  }
      }
      { \ctex_fontset_error:n { macnew } }
%</macnew>
  }
  {
%<macold>    \ctex_fontset_error:n { macold }
%<*macnew>
    \ctex_set_upmap:nnn { upserif   } { :3:Songti.ttc } { :1:Songti.ttc }
    \ctex_set_upmap:nnn { upserifit } { :0:Kaiti.ttc  } { }
    \ctex_set_upmap:nnn { upstsl    } { :0:Baoli.ttc  } { }
    \ctex_set_upmap:nnn { upstht    } { :4:Yuanti.ttc } { }
    \ctex_set_upmap_unicode:nnn { upsans } { :2:PingFang.ttc } { :8:PingFang.ttc }
    \ctex_set_upmap_unicode:nnn { upmono } { STFANGSO.ttf    } { }
    \ctex_set_upfamily:nnn { zhsong } { upzhserif   } { upzhserifb }
    \ctex_set_upfamily:nnn { zhhei  } { upzhsans    } { upzhsansb  }
    \ctex_set_upfamily:nnn { zhfs   } { upzhmono    } { }
    \ctex_set_upfamily:nnn { zhkai  } { upzhserifit } { }
    \ctex_set_upfamily:nnn { zhli   } { upschrm     } { }
    \ctex_set_upfamily:nnn { zhyou  } { upschgt     } { }
%</macnew>
  }
  {
%<*macold>
    \setCJKmainfont { STSong     }
      [ BoldFont = STHeiti, ItalicFont = STKaiti ]
    \setCJKsansfont { STXihei    } [ BoldFont = STHeiti ]
    \setCJKmonofont { STFangsong }
    \setCJKfamilyfont { zhsong } { STSong     }
    \setCJKfamilyfont { zhhei  } { STHeiti    }
    \setCJKfamilyfont { zhfs   } { STFangsong }
    \setCJKfamilyfont { zhkai  } { STKaiti    }
%</macold>
%<*macnew>
    \setCJKmainfont { Songti~SC~Light }
      [
        BoldFont       = Songti~SC~Bold,
        ItalicFont     = Kaiti~SC,
        BoldItalicFont = Kaiti~SC~Bold
      ]
    \setCJKsansfont { PingFang~SC }
    \setCJKmonofont { STFangsong  }
    \setCJKfamilyfont { zhsong } { Songti~SC~Light } [ BoldFont = Songti~SC~Bold ]
    \setCJKfamilyfont { zhhei  } { Heiti~SC~Light  } [ BoldFont = Heiti~SC~Medium ]
    \setCJKfamilyfont { zhpf   } { PingFang~SC     }
    \setCJKfamilyfont { zhfs   } { STFangsong      }
    \setCJKfamilyfont { zhkai  } { Kaiti~SC        } [ BoldFont = Kaiti~SC~Bold ]
    \setCJKfamilyfont { zhli   } { Baoli~SC        }
    \setCJKfamilyfont { zhyou  } { Yuanti~SC~Light } [ BoldFont = Yuanti~SC~Regular ]
%</macnew>
  }
%</macold|macnew>
%    \end{macrocode}
%
% \subsubsection{\opt{ubuntu}}
%
%    \begin{macrocode}
%<*ubuntu>
\@@_define_fontset:nnnn
  { \ctex_fontset_error:n { ubuntu } }
  {
    \ctex_zhmap_case:nnn
      {
        \setCJKmainfont { :2:NotoSerifCJK-Regular.ttc }
          [ BoldFont = :2:NotoSerifCJK-Bold.ttc, ItalicFont = gkai00mp.ttf ]
        \setCJKsansfont { :2:NotoSansCJK-Regular.ttc  }
          [ BoldFont = :2:NotoSansCJK-Bold.ttc  ]
        \setCJKmonofont { :2:NotoSerifCJK-Regular.ttc }
          [ BoldFont = :2:NotoSerifCJK-Bold.ttc ]
        \setCJKfamilyfont { zhsong } { :2:NotoSerifCJK-Regular.ttc }
          [ BoldFont = :2:NotoSerifCJK-Bold.ttc ]
        \setCJKfamilyfont { zhhei  } { :2:NotoSansCJK-Regular.ttc  }
          [ BoldFont = :2:NotoSansCJK-Bold.ttc  ]
        \setCJKfamilyfont { zhkai  } { gkai00mp.ttf  }
        \ctex_punct_set:n { fandol }
        \ctex_punct_map_family:nn   { \CJKrmdefault         } { zhsong  }
        \ctex_punct_map_family:nn   { \CJKsfdefault         } { zhhei   }
        \ctex_punct_map_bfseries:nn { \CJKrmdefault, zhsong } { zhsongb }
        \ctex_punct_map_bfseries:nn { \CJKsfdefault, zhhei  } { zhheib  }
        \ctex_punct_map_itshape:nn  { \CJKrmdefault         } { zhkai   }
      }
      {
        \ctex_load_zhmap:nnnn { rm } { zhhei } { zhsong } { zhubuntufonts }
        \ctex_punct_set:n { ubuntu }
        \ctex_punct_map_family:nn   { \CJKrmdefault } { zhsong }
        \ctex_punct_map_bfseries:nn { \CJKrmdefault } { zhhei  }
        \ctex_punct_map_itshape:nn  { \CJKrmdefault } { zhkai  }
      }
      { \ctex_fontset_error:n { ubuntu } }
  }
  {
    \ctex_set_upmap_unicode:nnn { upserif }
      { :2:NotoSerifCJK-Regular.ttc } { :2:NotoSerifCJK-Bold.ttc }
    \ctex_set_upmap_unicode:nnn { upsans  }
      { :2:NotoSansCJK-Regular.ttc  } { :2:NotoSansCJK-Bold.ttc  }
    \ctex_set_upmap_unicode:nnn { upmono  }
      { :2:NotoSerifCJK-Regular.ttc } { }
    \ctex_set_upmap:nnn { upserifit } { gkai00mp.ttf  } { }
    \ctex_set_upfamily:nnn { zhsong } { upzhserif   } { upzhserifb }
    \ctex_set_upfamily:nnn { zhhei  } { upzhsans    } { upzhsansb  }
    \ctex_set_upfamily:nnn { zhkai  } { upzhserifit } { }
  }
  {
    \setCJKmainfont { Noto~Serif~CJK~SC } [ ItalicFont = AR~PL~KaitiM~GB ]
    \setCJKsansfont { Noto~Sans~CJK~SC  }
    \setCJKmonofont { Noto~Serif~CJK~SC }
    \setCJKfamilyfont { zhsong } { Noto~Serif~CJK~SC }
    \setCJKfamilyfont { zhhei  } { Noto~Sans~CJK~SC  }
    \setCJKfamilyfont { zhkai  } { AR~PL~KaitiM~GB   }
  }
%</ubuntu>
%    \end{macrocode}
%
% \subsubsection{\opt{windows}}
%
% \begin{variable}{ \c_@@_msyh_suffix_str}
% Windows 8 以后,微软雅黑由原来的 \file{.ttf} 后缀改为 \file{.ttc} 后缀,需要
% 加以区分。
%    \begin{macrocode}
%<*windows>
\bool_lazy_or:nnTF
  { \file_if_exist_p:n { \c_dollar_str WINDIR/Fonts/msyh.ttc } }
  { \file_if_exist_p:n {                            msyh.ttc } }
  { \str_const:Nn \c_@@_msyh_suffix_str { ttc } }
  { \str_const:Nn \c_@@_msyh_suffix_str { ttf } }
%    \end{macrocode}
% \end{variable}
%
%    \begin{macrocode}
\@@_define_fontset:nnn
  {
    \ctex_zhmap_case:nnn
      {
        \ctex_punct_set:n { windows }
        \setCJKmainfont { simsun.ttc }
          [ BoldFont = simhei.ttf, ItalicFont = simkai.ttf ]
        \setCJKsansfont { msyh.\c_@@_msyh_suffix_str }
          [ BoldFont = msyhbd.\c_@@_msyh_suffix_str ]
        \setCJKmonofont { simfang.ttf }
        \setCJKfamilyfont { zhsong  } { simsun.ttc  }
        \setCJKfamilyfont { zhhei   } { simhei.ttf  }
        \setCJKfamilyfont { zhfs    } { simfang.ttf }
        \setCJKfamilyfont { zhkai   } { simkai.ttf  }
        \setCJKfamilyfont { zhyahei } { msyh.\c_@@_msyh_suffix_str }
          [ BoldFont = msyhbd.\c_@@_msyh_suffix_str ]
        \setCJKfamilyfont { zhli    } { simli.ttf   }
        \setCJKfamilyfont { zhyou   } { simyou.ttf  }
        \ctex_punct_map_family:nn   { \CJKrmdefault          } { zhsong   }
        \ctex_punct_map_bfseries:nn { \CJKrmdefault          } { zhhei    }
        \ctex_punct_map_itshape:nn  { \CJKrmdefault          } { zhkai    }
        \ctex_punct_map_family:nn   { \CJKsfdefault          } { zhyahei  }
        \ctex_punct_map_bfseries:nn { \CJKsfdefault, zhyahei } { zhyaheib }
        \ctex_punct_map_family:nn   { \CJKttdefault          } { zhfs     }
      }
      {
        \ctex_load_zhmap:nnnn { rm } { zhhei } { zhfs } { zhwindowsfonts }
        \ctex_punct_set:n { windows }
        \ctex_punct_map_family:nn   { \CJKrmdefault } { zhsong }
        \ctex_punct_map_bfseries:nn { \CJKrmdefault } { zhhei  }
        \ctex_punct_map_itshape:nn  { \CJKrmdefault } { zhkai  }
      }
      {
        \tl_set:Nn \CJKrmdefault { rm }
        \tl_set:Nn \CJKsfdefault { sf }
        \tl_set:Nn \CJKttdefault { tt }
      }
  }
  {
    \ctex_set_upfonts:nnnnnn
      { simsun.ttc                   }
      { simhei.ttf                   }
      { simkai.ttf                   }
      { msyh.\c_@@_msyh_suffix_str   }
      { msyhbd.\c_@@_msyh_suffix_str }
      { simfang.ttf                  }
    \ctex_set_upfamily:nnn { zhsong  } { upzhserif   } {}
    \ctex_set_upfamily:nnn { zhhei   } { upzhserifb  } {}
    \ctex_set_upfamily:nnn { zhfs    } { upzhmono    } {}
    \ctex_set_upfamily:nnn { zhkai   } { upzhserifit } {}
    \ctex_set_upfamily:nnn { zhyahei } { upzhsans    } { upzhsansb }
    \ctex_set_upfamily:nnn { zhli    } { upschrm     } {}
    \ctex_set_upfamily:nnn { zhyou   } { upschgt     } {}
    \ctex_set_upmap:nnn { upstsl } { simli.ttf  } {}
    \ctex_set_upmap:nnn { upstht } { simyou.ttf } {}
  }
  {
    \setCJKmainfont   { SimSun } [ BoldFont = SimHei , ItalicFont = KaiTi ]
    \setCJKsansfont   { Microsoft~YaHei } [ BoldFont = *~Bold ]
    \setCJKmonofont   { FangSong }
    \setCJKfamilyfont { zhsong  } { SimSun          }
    \setCJKfamilyfont { zhhei   } { SimHei          }
    \setCJKfamilyfont { zhfs    } { FangSong        }
    \setCJKfamilyfont { zhkai   } { KaiTi           }
    \setCJKfamilyfont { zhyahei } { Microsoft~YaHei } [ BoldFont = *~Bold ]
    \setCJKfamilyfont { zhli    } { LiSu            }
    \setCJKfamilyfont { zhyou   } { YouYuan         }
  }
%</windows>
%    \end{macrocode}
%
% \subsection{中文字体命令}
% \begin{macro}{\songti,\heiti,\fangsong,\kaishu,\lishu,\youyuan,\yahei,\pingfang}
% 使用 \upLaTeX{} 编译时,\opt{macnew} 字库中由于传统黑体(黑体-简)无法使用,
% 我们用苹方来代替。同时 \tn{yahei}、\tn{pingfang} 命令被设置为与 \tn{heiti} 相同。
%    \begin{macrocode}
%<*!mac>
\NewDocumentCommand \songti   { } { \CJKfamily { zhsong  } }
\NewDocumentCommand \heiti    { } { \CJKfamily { zhhei   } }
%<!ubuntu>\NewDocumentCommand \fangsong { } { \CJKfamily { zhfs    } }
\NewDocumentCommand \kaishu   { } { \CJKfamily { zhkai   } }
%<*windows|founder|macnew>
\NewDocumentCommand \lishu    { } { \CJKfamily { zhli    } }
\NewDocumentCommand \youyuan  { } { \CJKfamily { zhyou   } }
%</windows|founder|macnew>
%<windows>\NewDocumentCommand \yahei    { } { \CJKfamily { zhyahei } }
%<*macnew>
\bool_lazy_or:nnTF
  { \sys_if_engine_pdftex_p: }
  { \sys_if_engine_uptex_p:  }
  {
    \cs_new_eq:NN \yahei    \heiti
    \cs_new_eq:NN \pingfang \heiti
  }
  {
    \NewDocumentCommand \yahei    { } { \CJKfamily { zhpf } }
    \NewDocumentCommand \pingfang { } { \CJKfamily { zhpf } }
  }
%</macnew>
%</!mac>
%</fontset>
%    \end{macrocode}
% \end{macro}
%
% \subsection{\pkg{zhmetrics} 的字体映射}
%
% 确认 \tn{catcode},没有重复载入检查。
%    \begin{macrocode}
%<*zhmap>
\begingroup\catcode61\catcode48\catcode32=10\relax%
  \catcode 35=6 % #
  \catcode123=1 % {
  \catcode125=2 % }
  \toks0{\endlinechar=\the\endlinechar\relax}%
  \toks2{\endlinechar=13 }%
  \def\x#1 #2 {%
    \toks0\expandafter{\the\toks0 \catcode#1=\the\catcode#1\relax}%
    \toks2\expandafter{\the\toks2 \catcode#1=#2 }}%
  \x  13  5 % carriage return
  \x  32 10 % space
  \x  35  6 % #
  \x  40 12 % (
  \x  41 12 % )
  \x  45 12 % -
  \x  46 12 % .
  \x  47 12 % /
  \x  58 12 % :
  \x  60 12 % <
  \x  61 12 % =
  \x  64 11 % @
  \x  91 12 % [
  \x  93 12 % ]
  \x 123  1 % {
  \x 125  2 % }
  \edef\x#1{\endgroup%
    \edef\noexpand#1{%
      \the\toks0 %
      \let\noexpand\noexpand\noexpand#1%
          \noexpand\noexpand\noexpand\undefined%
      \noexpand\noexpand\noexpand\endinput}%
    \the\toks2}%
\expandafter\x\csname ctex@zhmap@endinput\endcsname
%    \end{macrocode}
%
% \begin{macro}[int]{\ifzhmappdf}
%    \begin{macrocode}
\begingroup\expandafter\endgroup
\expandafter\let\csname ifzhmappdf\expandafter\endcsname\csname
  \expandafter\ifx\csname ifctexpdf\endcsname\relax
    \expandafter\ifx\csname pdfoutput\endcsname\relax
      iffalse\else\ifnum\pdfoutput < 1 iffalse\else iftrue\fi\fi
  \else ifctexpdf\fi
\endcsname
%    \end{macrocode}
% \end{macro}
%
% \begin{macro}[int]{\ProvidesFile}
% 提供非 \LaTeX{} 格式下的 \tn{ProvidesFile}。
%    \begin{macrocode}
\begingroup
\expandafter\ifx\csname ProvidesFile\endcsname\relax
  \long\def\x#1\ProvidesFile#2[#3]{%
    #1%
    \immediate\write-1{File: #2 #3}%
    \expandafter\xdef\csname ver@#2\endcsname{#3}}
  \expandafter\x%
\fi
\endgroup
%    \end{macrocode}
% \end{macro}
%
% 文件标识信息。
%    \begin{macrocode}
%<adobe>\ProvidesFile{zhadobefonts.tex}%
%<adobe>  [2019/05/29 v2.4.16 Adobe font map loader for DVIPDFMx (CTEX)]
%<fandol>\ProvidesFile{zhfandolfonts.tex}%
%<fandol>  [2019/05/29 v2.4.16 Fandol font map loader for DVIPDFMx (CTEX)]
%<founder>\ProvidesFile{zhfounderfonts.tex}%
%<founder>  [2019/05/29 v2.4.16 Founder font map loader for pdfTeX and DVIPDFMx (CTEX)]
%<mac>\ProvidesFile{zhmacfonts.tex}%
%<mac>  [2019/05/29 v2.4.16 Mac font map loader for pdfTeX and DVIPDFMx (CTEX)]
%<ubuntu>\ProvidesFile{zhubuntufonts.tex}%
%<ubuntu>  [2019/05/29 v2.4.16 Ubuntu font map loader for pdfTeX and DVIPDFMx (CTEX)]
%<windows>\ProvidesFile{zhwindowsfonts.tex}%
%<windows>  [2019/05/29 v2.4.16 Windows font map loader for pdfTeX and DVIPDFMx (CTEX)]
%    \end{macrocode}
%
% \subsubsection{\pkg{zhadobefonts.tex}}
%
%    \begin{macrocode}
%<*adobe>
\ifzhmappdf
%% pdfTeX does not support OTF fonts
\else
  \special{pdf:mapline gbk@UGBK@          UniGB-UTF16-H AdobeSongStd-Light.otf}
  \special{pdf:mapline gbksong@UGBK@      UniGB-UTF16-H AdobeSongStd-Light.otf}
  \special{pdf:mapline gbkkai@UGBK@       UniGB-UTF16-H AdobeKaitiStd-Regular.otf}
  \special{pdf:mapline gbkhei@UGBK@       UniGB-UTF16-H AdobeHeitiStd-Regular.otf}
  \special{pdf:mapline gbkfs@UGBK@        UniGB-UTF16-H AdobeFangsongStd-Regular.otf}
  \special{pdf:mapline cyberb@Unicode@    UniGB-UTF16-H AdobeSongStd-Light.otf}
  \special{pdf:mapline unisong@Unicode@   UniGB-UTF16-H AdobeSongStd-Light.otf}
  \special{pdf:mapline unikai@Unicode@    UniGB-UTF16-H AdobeKaitiStd-Regular.otf}
  \special{pdf:mapline unihei@Unicode@    UniGB-UTF16-H AdobeHeitiStd-Regular.otf}
  \special{pdf:mapline unifs@Unicode@     UniGB-UTF16-H AdobeFangsongStd-Regular.otf}
  \special{pdf:mapline gbksongsl@UGBK@    UniGB-UTF16-H AdobeSongStd-Light.otf       -s .167}
  \special{pdf:mapline gbkkaisl@UGBK@     UniGB-UTF16-H AdobeKaitiStd-Regular.otf    -s .167}
  \special{pdf:mapline gbkheisl@UGBK@     UniGB-UTF16-H AdobeHeitiStd-Regular.otf    -s .167}
  \special{pdf:mapline gbkfssl@UGBK@      UniGB-UTF16-H AdobeFangsongStd-Regular.otf -s .167}
  \special{pdf:mapline unisongsl@Unicode@ UniGB-UTF16-H AdobeSongStd-Light.otf       -s .167}
  \special{pdf:mapline unikaisl@Unicode@  UniGB-UTF16-H AdobeKaitiStd-Regular.otf    -s .167}
  \special{pdf:mapline uniheisl@Unicode@  UniGB-UTF16-H AdobeHeitiStd-Regular.otf    -s .167}
  \special{pdf:mapline unifssl@Unicode@   UniGB-UTF16-H AdobeFangsongStd-Regular.otf -s .167}
\fi
%</adobe>
%    \end{macrocode}
%
% \subsubsection{\pkg{zhfandolfonts.tex}}
%
%    \begin{macrocode}
%<*fandol>
\ifzhmappdf
%% pdfTeX does not support OTF fonts
\else
  \special{pdf:mapline gbk@UGBK@          UniGB-UTF16-H FandolSong-Regular.otf}
  \special{pdf:mapline gbksong@UGBK@      UniGB-UTF16-H FandolSong-Regular.otf}
  \special{pdf:mapline gbkkai@UGBK@       UniGB-UTF16-H FandolKai-Regular.otf}
  \special{pdf:mapline gbkhei@UGBK@       UniGB-UTF16-H FandolHei-Regular.otf}
  \special{pdf:mapline gbkfs@UGBK@        UniGB-UTF16-H FandolFang-Regular.otf}
  \special{pdf:mapline cyberb@Unicode@    UniGB-UTF16-H FandolSong-Regular.otf}
  \special{pdf:mapline unisong@Unicode@   UniGB-UTF16-H FandolSong-Regular.otf}
  \special{pdf:mapline unikai@Unicode@    UniGB-UTF16-H FandolKai-Regular.otf}
  \special{pdf:mapline unihei@Unicode@    UniGB-UTF16-H FandolHei-Regular.otf}
  \special{pdf:mapline unifs@Unicode@     UniGB-UTF16-H FandolFang-Regular.otf}
  \special{pdf:mapline gbksongsl@UGBK@    UniGB-UTF16-H FandolSong-Regular.otf -s .167}
  \special{pdf:mapline gbkkaisl@UGBK@     UniGB-UTF16-H FandolKai-Regular.otf  -s .167}
  \special{pdf:mapline gbkheisl@UGBK@     UniGB-UTF16-H FandolHei-Regular.otf  -s .167}
  \special{pdf:mapline gbkfssl@UGBK@      UniGB-UTF16-H FandolFang-Regular.otf -s .167}
  \special{pdf:mapline unisongsl@Unicode@ UniGB-UTF16-H FandolSong-Regular.otf -s .167}
  \special{pdf:mapline unikaisl@Unicode@  UniGB-UTF16-H FandolKai-Regular.otf  -s .167}
  \special{pdf:mapline uniheisl@Unicode@  UniGB-UTF16-H FandolHei-Regular.otf  -s .167}
  \special{pdf:mapline unifssl@Unicode@   UniGB-UTF16-H FandolFang-Regular.otf -s .167}
\fi
%</fandol>
%    \end{macrocode}
%
% \subsubsection{\pkg{zhfounderfonts.tex}}
%
%    \begin{macrocode}
%<*founder>
\ifzhmappdf
  \pdfmapline{=gbk@UGBK@          <FZSSK.TTF}
  \pdfmapline{=gbksong@UGBK@      <FZSSK.TTF}
  \pdfmapline{=gbkkai@UGBK@       <FZKTK.TTF}
  \pdfmapline{=gbkhei@UGBK@       <FZHTK.TTF}
  \pdfmapline{=gbkfs@UGBK@        <FZFSK.TTF}
  \pdfmapline{=gbkli@UGBK@        <FZLSK.TTF}
  \pdfmapline{=gbkyou@UGBK@       <FZY1K.TTF}
  \pdfmapline{=cyberb@Unicode@    <FZSSK.TTF}
  \pdfmapline{=unisong@Unicode@   <FZSSK.TTF}
  \pdfmapline{=unikai@Unicode@    <FZKTK.TTF}
  \pdfmapline{=unihei@Unicode@    <FZHTK.TTF}
  \pdfmapline{=unifs@Unicode@     <FZFSK.TTF}
  \pdfmapline{=unili@Unicode@     <FZLSK.TTF}
  \pdfmapline{=uniyou@Unicode@    <FZY1K.TTF}
  \pdfmapline{=gbksongsl@UGBK@    <FZSSK.TTF}
  \pdfmapline{=gbkkaisl@UGBK@     <FZKTK.TTF}
  \pdfmapline{=gbkheisl@UGBK@     <FZHTK.TTF}
  \pdfmapline{=gbkfssl@UGBK@      <FZFSK.TTF}
  \pdfmapline{=gbklisl@UGBK@      <FZLSK.TTF}
  \pdfmapline{=gbkyousl@UGBK@     <FZY1K.TTF}
  \pdfmapline{=unisongsl@Unicode@ <FZSSK.TTF}
  \pdfmapline{=unikaisl@Unicode@  <FZKTK.TTF}
  \pdfmapline{=uniheisl@Unicode@  <FZHTK.TTF}
  \pdfmapline{=unifssl@Unicode@   <FZFSK.TTF}
  \pdfmapline{=unilisl@Unicode@   <FZLSK.TTF}
  \pdfmapline{=uniyousl@Unicode@  <FZY1K.TTF}
\else
  \special{pdf:mapline gbk@UGBK@          unicode FZSSK.TTF}
  \special{pdf:mapline gbksong@UGBK@      unicode FZSSK.TTF}
  \special{pdf:mapline gbkkai@UGBK@       unicode FZKTK.TTF}
  \special{pdf:mapline gbkhei@UGBK@       unicode FZHTK.TTF}
  \special{pdf:mapline gbkfs@UGBK@        unicode FZFSK.TTF}
  \special{pdf:mapline gbkli@UGBK@        unicode FZLSK.TTF}
  \special{pdf:mapline gbkyou@UGBK@       unicode FZY1K.TTF}
  \special{pdf:mapline cyberb@Unicode@    unicode FZSSK.TTF}
  \special{pdf:mapline unisong@Unicode@   unicode FZSSK.TTF}
  \special{pdf:mapline unikai@Unicode@    unicode FZKTK.TTF}
  \special{pdf:mapline unihei@Unicode@    unicode FZHTK.TTF}
  \special{pdf:mapline unifs@Unicode@     unicode FZFSK.TTF}
  \special{pdf:mapline unili@Unicode@     unicode FZLSK.TTF}
  \special{pdf:mapline uniyou@Unicode@    unicode FZY1K.TTF}
  \special{pdf:mapline gbksongsl@UGBK@    unicode FZSSK.TTF -s .167}
  \special{pdf:mapline gbkkaisl@UGBK@     unicode FZKTK.TTF -s .167}
  \special{pdf:mapline gbkheisl@UGBK@     unicode FZHTK.TTF -s .167}
  \special{pdf:mapline gbkfssl@UGBK@      unicode FZFSK.TTF -s .167}
  \special{pdf:mapline gbklisl@UGBK@      unicode FZLSK.TTF -s .167}
  \special{pdf:mapline gbkyousl@UGBK@     unicode FZY1K.TTF -s .167}
  \special{pdf:mapline unisongsl@Unicode@ unicode FZSSK.TTF -s .167}
  \special{pdf:mapline unikaisl@Unicode@  unicode FZKTK.TTF -s .167}
  \special{pdf:mapline uniheisl@Unicode@  unicode FZHTK.TTF -s .167}
  \special{pdf:mapline unifssl@Unicode@   unicode FZFSK.TTF -s .167}
  \special{pdf:mapline unilisl@Unicode@   unicode FZLSK.TTF -s .167}
  \special{pdf:mapline uniyousl@Unicode@  unicode FZY1K.TTF -s .167}
\fi
%</founder>
%    \end{macrocode}
%
% \subsubsection{\pkg{zhmacfonts.tex}}
%
%    \begin{macrocode}
%<*mac>
\ifzhmappdf
%% pdfTeX does not support OTF fonts
\else
  \special{pdf:mapline gbk@UGBK@          UniGB-UTF16-H :3:Songti.ttc}
  \special{pdf:mapline gbksong@UGBK@      UniGB-UTF16-H :3:Songti.ttc}
  \special{pdf:mapline gbkkai@UGBK@       UniGB-UTF16-H :0:Kaiti.ttc}
  \special{pdf:mapline gbkhei@UGBK@       unicode       :2:PingFang.ttc}
  \special{pdf:mapline gbkfs@UGBK@        unicode       STFANGSO.ttf}
  \special{pdf:mapline gbkli@UGBK@        UniGB-UTF16-H :0:Baoli.ttc}
  \special{pdf:mapline gbkyou@UGBK@       UniGB-UTF16-H :4:Yuanti.ttc}
  \special{pdf:mapline cyberb@Unicode@    UniGB-UTF16-H :3:Songti.ttc}
  \special{pdf:mapline unisong@Unicode@   UniGB-UTF16-H :3:Songti.ttc}
  \special{pdf:mapline unikai@Unicode@    UniGB-UTF16-H :0:Kaiti.ttc}
  \special{pdf:mapline unihei@Unicode@    unicode       :2:PingFang.ttc}
  \special{pdf:mapline unifs@Unicode@     unicode       STFANGSO.ttf}
  \special{pdf:mapline unili@Unicode@     UniGB-UTF16-H :0:Baoli.ttc}
  \special{pdf:mapline uniyou@Unicode@    UniGB-UTF16-H :4:Yuanti.ttc}
  \special{pdf:mapline gbksongsl@UGBK@    UniGB-UTF16-H :3:Songti.ttc   -s .167}
  \special{pdf:mapline gbkkaisl@UGBK@     UniGB-UTF16-H :0:Kaiti.ttc    -s .167}
  \special{pdf:mapline gbkheisl@UGBK@     unicode       :2:PingFang.ttc -s .167}
  \special{pdf:mapline gbkfssl@UGBK@      unicode       STFANGSO.ttf    -s .167}
  \special{pdf:mapline gbklisl@UGBK@      UniGB-UTF16-H :0:Baoli.ttc    -s .167}
  \special{pdf:mapline gbkyousl@UGBK@     UniGB-UTF16-H :4:Yuanti.ttc   -s .167}
  \special{pdf:mapline unisongsl@Unicode@ UniGB-UTF16-H :3:Songti.ttc   -s .167}
  \special{pdf:mapline unikaisl@Unicode@  UniGB-UTF16-H :0:Kaiti.ttc    -s .167}
  \special{pdf:mapline uniheisl@Unicode@  unicode       :2:PingFang.ttc -s .167}
  \special{pdf:mapline unifssl@Unicode@   unicode       STFANGSO.ttf    -s .167}
  \special{pdf:mapline unilisl@Unicode@   UniGB-UTF16-H :0:Baoli.ttc    -s .167}
  \special{pdf:mapline uniyousl@Unicode@  UniGB-UTF16-H :4:Yuanti.ttc   -s .167}
\fi
%</mac>
%    \end{macrocode}
%
% \subsubsection{\pkg{zhubuntufonts.tex}}
%
%    \begin{macrocode}
%<*ubuntu>
\ifzhmappdf
%% pdfTeX does not support OTF fonts
\else
  \special{pdf:mapline gbk@UGBK@          unicode :2:NotoSerifCJK-Regular.ttc}
  \special{pdf:mapline gbksong@UGBK@      unicode :2:NotoSerifCJK-Regular.ttc}
  \special{pdf:mapline gbkkai@UGBK@       unicode gkai00mp.ttf}
  \special{pdf:mapline gbkhei@UGBK@       unicode :2:NotoSansCJK-Regular.ttc}
  \special{pdf:mapline gbkfs@UGBK@        unicode :2:NotoSerifCJK-Regular.ttc}
  \special{pdf:mapline cyberb@Unicode@    unicode :2:NotoSerifCJK-Regular.ttc}
  \special{pdf:mapline unisong@Unicode@   unicode :2:NotoSerifCJK-Regular.ttc}
  \special{pdf:mapline unikai@Unicode@    unicode gkai00mp.ttf}
  \special{pdf:mapline unihei@Unicode@    unicode :2:NotoSansCJK-Regular.ttc}
  \special{pdf:mapline unifs@Unicode@     unicode :2:NotoSerifCJK-Regular.ttc}
  \special{pdf:mapline gbksongsl@UGBK@    unicode :2:NotoSerifCJK-Regular.ttc -s .167}
  \special{pdf:mapline gbkkaisl@UGBK@     unicode gkai00mp.ttf                -s .167}
  \special{pdf:mapline gbkheisl@UGBK@     unicode :2:NotoSansCJK-Regular.ttc  -s .167}
  \special{pdf:mapline gbkfssl@UGBK@      unicode :2:NotoSerifCJK-Regular.ttc -s .167}
  \special{pdf:mapline unisongsl@Unicode@ unicode :2:NotoSerifCJK-Regular.ttc -s .167}
  \special{pdf:mapline unikaisl@Unicode@  unicode gkai00mp.ttf                -s .167}
  \special{pdf:mapline uniheisl@Unicode@  unicode :2:NotoSansCJK-Regular.ttc  -s .167}
  \special{pdf:mapline unifssl@Unicode@   unicode :2:NotoSerifCJK-Regular.ttc -s .167}
\fi
%</ubuntu>
%    \end{macrocode}
%
% \subsubsection{\pkg{zhwindowsfonts.tex}}
%
%    \begin{macrocode}
%<*windows>
\ifzhmappdf
  \pdfmapline{=gbk@UGBK@          <simsun.ttc}
  \pdfmapline{=gbksong@UGBK@      <simsun.ttc}
  \pdfmapline{=gbkkai@UGBK@       <simkai.ttf}
  \pdfmapline{=gbkhei@UGBK@       <simhei.ttf}
  \pdfmapline{=gbkfs@UGBK@        <simfang.ttf}
  \pdfmapline{=gbkli@UGBK@        <simli.ttf}
  \pdfmapline{=gbkyou@UGBK@       <simyou.ttf}
  \pdfmapline{=cyberb@Unicode@    <simsun.ttc}
  \pdfmapline{=unisong@Unicode@   <simsun.ttc}
  \pdfmapline{=unikai@Unicode@    <simkai.ttf}
  \pdfmapline{=unihei@Unicode@    <simhei.ttf}
  \pdfmapline{=unifs@Unicode@     <simfang.ttf}
  \pdfmapline{=unili@Unicode@     <simli.ttf}
  \pdfmapline{=uniyou@Unicode@    <simyou.ttf}
  \pdfmapline{=gbksongsl@UGBK@    <simsun.ttc}
  \pdfmapline{=gbkkaisl@UGBK@     <simkai.ttf}
  \pdfmapline{=gbkheisl@UGBK@     <simhei.ttf}
  \pdfmapline{=gbkfssl@UGBK@      <simfang.ttf}
  \pdfmapline{=gbklisl@UGBK@      <simli.ttf}
  \pdfmapline{=gbkyousl@UGBK@     <simyou.ttf}
  \pdfmapline{=unisongsl@Unicode@ <simsun.ttc}
  \pdfmapline{=unikaisl@Unicode@  <simkai.ttf}
  \pdfmapline{=uniheisl@Unicode@  <simhei.ttf}
  \pdfmapline{=unifssl@Unicode@   <simfang.ttf}
  \pdfmapline{=unilisl@Unicode@   <simli.ttf}
  \pdfmapline{=uniyousl@Unicode@  <simyou.ttf}
\else
  \special{pdf:mapline gbk@UGBK@          unicode :0:simsun.ttc -v 50}
  \special{pdf:mapline gbksong@UGBK@      unicode :0:simsun.ttc -v 50}
  \special{pdf:mapline gbkkai@UGBK@       unicode simkai.ttf    -v 70}
  \special{pdf:mapline gbkhei@UGBK@       unicode simhei.ttf    -v 150}
  \special{pdf:mapline gbkfs@UGBK@        unicode simfang.ttf   -v 50}
  \special{pdf:mapline gbkli@UGBK@        unicode simli.ttf     -v 150}
  \special{pdf:mapline gbkyou@UGBK@       unicode simyou.ttf    -v 60}
  \special{pdf:mapline cyberb@Unicode@    unicode :0:simsun.ttc -v 50}
  \special{pdf:mapline unisong@Unicode@   unicode :0:simsun.ttc -v 50}
  \special{pdf:mapline unikai@Unicode@    unicode simkai.ttf    -v 70}
  \special{pdf:mapline unihei@Unicode@    unicode simhei.ttf    -v 150}
  \special{pdf:mapline unifs@Unicode@     unicode simfang.ttf   -v 50}
  \special{pdf:mapline unili@Unicode@     unicode simli.ttf     -v 150}
  \special{pdf:mapline uniyou@Unicode@    unicode simyou.ttf    -v 60}
  \special{pdf:mapline gbksongsl@UGBK@    unicode :0:simsun.ttc -v 50  -s .167}
  \special{pdf:mapline gbkkaisl@UGBK@     unicode simkai.ttf    -v 70  -s .167}
  \special{pdf:mapline gbkheisl@UGBK@     unicode simhei.ttf    -v 150 -s .167}
  \special{pdf:mapline gbkfssl@UGBK@      unicode simfang.ttf   -v 50  -s .167}
  \special{pdf:mapline gbklisl@UGBK@      unicode simli.ttf     -v 150 -s .167}
  \special{pdf:mapline gbkyousl@UGBK@     unicode simyou.ttf    -v 60  -s .167}
  \special{pdf:mapline unisongsl@Unicode@ unicode :0:simsun.ttc -v 50  -s .167}
  \special{pdf:mapline unikaisl@Unicode@  unicode simkai.ttf    -v 70  -s .167}
  \special{pdf:mapline uniheisl@Unicode@  unicode simhei.ttf    -v 150 -s .167}
  \special{pdf:mapline unifssl@Unicode@   unicode simfang.ttf   -v 50  -s .167}
  \special{pdf:mapline unilisl@Unicode@   unicode simli.ttf     -v 150 -s .167}
  \special{pdf:mapline uniyousl@Unicode@  unicode simyou.ttf    -v 60  -s .167}
\fi
%</windows>
%    \end{macrocode}
%
%    \begin{macrocode}
\ctex@zhmap@endinput
%</zhmap>
%    \end{macrocode}
%
% \subsection{制作 \texttt{spa} 文件}
%
% 我们通过 \XeTeX{} 的 \tn{XeTeXglyphbounds} 取得字体中标点符号的边界信息,为
% \pkg{CJKpunct} 宏包制作 \file{spa}。
%
%    \begin{macrocode}
%<*spa>
%<*macro>
\input expl3-generic %
\ExplSyntaxOn
\sys_if_engine_xetex:F
  {
    \msg_new:nnn { ctex } { xetex }
      { XeTeX~is~required~to~compile~this~document! }
    \msg_fatal:nn { ctex } { xetex }
  }
%    \end{macrocode}
%
% \pkg{CJKpunct} 定义的标点符号是:
% \begin{verbatim}
%   ‘“「『〔([{〈《〖【
%   —…、。,.:;!?%〕)]}〉》〗】’”」』
% \end{verbatim}
% 注意顺序不能改变。
%    \begin{macrocode}
\seq_const_from_clist:Nn \c_@@_punct_seq
  {
    "2018 , "201C , "300C , "300E , "3014 , "FF08 , "FF3B , "FF5B ,
    "3008 , "300A , "3016 , "3010 ,
    "2014 , "2026 , "3001 , "3002 , "FF0C , "FF0E , "FF1A , "FF1B ,
    "FF01 , "FF1F , "FF05 , "3015 , "FF09 , "FF3D , "FF5D , "3009 ,
    "300B , "3017 , "3011 , "2019 , "201D , "300D , "300F
  }
%    \end{macrocode}
%
% \begin{macro}[int]{\ctex_make_spa:nn}
% |#1| 是 \file{spa} 文件名,|#2| 是由 CJK 族名与字体构成的逗号列表。
%    \begin{macrocode}
\cs_new_protected_nopar:Npn \ctex_make_spa:nn #1#2
  {
    \iow_open:Nn \g_@@_spa_iow {#1}
    \clist_map_inline:nn {#2}
      { \@@_write_family:nn ##1 }
    \iow_close:N \g_@@_spa_iow
  }
\iow_new:N \g_@@_spa_iow
\cs_new_eq:NN \MAKESPA \ctex_make_spa:nn
%    \end{macrocode}
% \end{macro}
%
%    \begin{macrocode}
\cs_new_protected:Npn \@@_write_family:nn #1#2
  {
    \group_begin:
      \tex_font:D \l_@@_punct_font = "#2" ~ at ~ 100 pt \scan_stop:
      \l_@@_punct_font
      \clist_clear:N \l_@@_punct_bounds_clist
      \seq_map_inline:Nn \c_@@_punct_seq
        { \exp_args:No \@@_save_bounds:n { \int_use:N \tex_XeTeXcharglyph:D ##1 } }
      \iow_now:Nx \g_@@_spa_iow
        {
          \token_to_str:N \ctexspadef {#1}
%    \end{macrocode}
% 最后这三个逗号对 \pkg{CJKpunct} 来说是必要的。
%    \begin{macrocode}
            { \l_@@_punct_bounds_clist , , , }
        }
    \group_end:
  }
\cs_new_protected_nopar:Npn \@@_save_bounds:n #1
  {
    \clist_put_right:Nx \l_@@_punct_bounds_clist
      {
        \@@_calc_bounds:nn { 1 } {#1} ,
        \@@_calc_bounds:nn { 3 } {#1}
      }
  }
\clist_new:N \l_@@_punct_bounds_clist
%    \end{macrocode}
%
% \pkg{CJKpunct} 要求的格式是边界空白宽度与 1\,em 的比值的一百倍。
%    \begin{macrocode}
\cs_new_nopar:Npn \@@_calc_bounds:nn #1#2
  {
    \fp_eval:n
      {
        round
          (
            \dim_to_decimal_in_unit:nn
              { 100 \tex_XeTeXglyphbounds:D #1 ~ #2 }
              { 1 em }
          )
      }
  }
\ExplSyntaxOff
%</macro>
%    \end{macrocode}
%
% 下面是 \CTeX{} 定义的一些字体。
%    \begin{macrocode}
%<*make>
\input ctexspamacro %
\MAKESPA {ctexpunct.spa}
  {
    {adobezhsong}     {AdobeSongStd-Light} ,
    {adobezhhei}      {AdobeHeitiStd-Regular} ,
    {adobezhkai}      {AdobeKaitiStd-Regular} ,
    {adobezhfs}       {AdobeFangsongStd-Regular} ,
%
    {fandolzhsong}    {FandolSong} ,
    {fandolzhsongb}   {FandolSong-Bold} ,
    {fandolzhhei}     {FandolHei} ,
    {fandolzhheib}    {FandolHei-Bold} ,
    {fandolzhkai}     {FandolKai} ,
    {fandolzhfs}      {FandolFang} ,
%
    {founderzhsong}   {FZShuSong-Z01} ,
    {founderzhsongb}  {FZXiaoBiaoSong-B05} ,
    {founderzhhei}    {FZHei-B01} ,
    {founderzhheil}   {FZXiHeiI-Z08} ,
    {founderzhkai}    {FZKai-Z03} ,
    {founderzhfs}     {FZFangSong-Z02} ,
    {founderzhli}     {FZLiShu-S01} ,
    {founderzhyou}    {FZXiYuan-M01} ,
    {founderzhyoub}   {FZZhunYuan-M02} ,
%
    {maczhsong}       {Songti SC Light} ,
    {maczhsongb}      {Songti SC Bold} ,
    {maczhhei}        {Heiti SC Medium} ,
    {maczhheil}       {Heiti SC Light} ,
    {maczhkai}        {Kaiti SC} ,
    {maczhkaib}       {Kaiti SC Bold} ,
    {maczhfs}         {STFangsong} ,
    {maczhli}         {Baoli SC} ,
    {maczhyou}        {Yuanti SC Light} ,
    {maczhyoub}       {Yuanti SC Regular} ,
    {maczhpf}         {PingFang SC} ,
    {maczhpfb}        {PingFang SC Semibold} ,
%
    {ubuntuzhsong}    {Noto Serif CJK SC} ,
    {ubuntuzhsongb}   {Noto Serif CJK SC Bold} ,
    {ubuntuzhhei}     {Noto Sans CJK SC} ,
    {ubuntuzhheib}    {Noto Sans CJK SC Bold} ,
    {ubuntuzhkai}     {AR PL KaitiM GB} ,
%
    {windowszhsong}   {SimSun} ,
    {windowszhhei}    {SimHei} ,
    {windowszhkai}    {KaiTi} ,
    {windowszhfs}     {FangSong} ,
    {windowszhli}     {LiSu} ,
    {windowszhyou}    {YouYuan} ,
    {windowszhyahei}  {Microsoft YaHei} ,
    {windowszhyaheib} {Microsoft YaHei Bold}
  }
\primitive\end
%</make>
%</spa>
%    \end{macrocode}
