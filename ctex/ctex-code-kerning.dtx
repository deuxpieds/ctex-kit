% \iffalse meta-comment
%
% Copyright (C) 2003--2020
% CTEX.ORG and any individual authors listed elsewhere in this file.
% --------------------------------------------------------------------------
%
% This work may be distributed and/or modified under the
% conditions of the LaTeX Project Public License, either
% version 1.3c of this license or (at your option) any later
% version. This version of this license is in
%    http://www.latex-project.org/lppl/lppl-1-3c.txt
% and the latest version of this license is in
%    http://www.latex-project.org/lppl.txt
% and version 1.3 or later is part of all distributions of
% LaTeX version 2005/12/01 or later.
%
% This work has the LPPL maintenance status `maintained'.
%
% The Current Maintainers of this work are Leo Liu, Qing Lee and Liam Huang.
%
% --------------------------------------------------------------------------
%
% \fi
%
% \section{字距与缩进}
%
% \begin{macro}{autoindent}
% \opt{autoindent} 也是可以用在正文中的选项,意义与宏包选项 |option/autoindent| 相同。
%    \begin{macrocode}
%<*class|ctex>
\keys_define:nn { ctex }
  {
    autoindent .choice: ,
    autoindent .default:n = { true } ,
    autoindent / true    .code:n =
      {
        \tl_set:Nn \l_@@_autoindent_tl { 2 \ccwd }
        \ctex_select_size:
      } ,
    autoindent / false   .code:n =
      { \tl_clear:N \l_@@_autoindent_tl } ,
    autoindent / unknown .code:n =
      {
        \ctex_set_default_ccwd:Nn \l_@@_autoindent_tl {#1}
        \ctex_select_size:
      }
  }
%    \end{macrocode}
% \end{macro}
%
% \begin{macro}{\CTEXsetfont}
% 无论字体大小是否变化都更新相关信息。
%    \begin{macrocode}
\NewDocumentCommand \CTEXsetfont { } { \ctex_select_size: }
\cs_new_protected_nopar:Npn \ctex_select_size:
  { \cs_if_free:NTF \size@update { \ctex_update_size: } { \selectfont } }
%    \end{macrocode}
% \end{macro}
%
% \begin{macro}[int]{\ctex_update_size:}
% 在字号变化时更新 \tn{ccwd}、\tn{parindent} 和汉字间距。字距为零则恢复正常设置。
%    \begin{macrocode}
\cs_new_protected_nopar:Npn \ctex_update_size:
  {
    \tl_if_eq:NNTF \l_@@_ziju_tl \c_@@_zero_tl
      {
        \ctex_update_stretch:
        \ctex_update_parindent:
      }
      { \ctex_update_ziju: }
  }
\tl_const:Nx \c_@@_zero_tl { \fp_use:N \c_zero_fp }
\tl_new:N \l_@@_ziju_tl
\tl_set_eq:NN \l_@@_ziju_tl \c_@@_zero_tl
%    \end{macrocode}
% 在 \tn{selectfont} 中,若 \tn{size@update} 为 \tn{relax},说明字体大小没有变化,
% 我们也就不用更新相关参数。
%    \begin{macrocode}
\ctex_add_to_selectfont:n
  { \cs_if_free:NF \size@update { \ctex_update_size: } }
%    \end{macrocode}
% \end{macro}
%
% \begin{macro}{linestretch}
% 若行宽不是汉字宽度的整数倍,自然要求伸展它们之间的差。这里设置的是在此基础上的
% 额外伸展量。初始化为一个汉字的宽度。若设置为 \tn{maxdimen},则禁用此功能。
% 参数的默认单位是汉字的宽度 \tn{ccwd}。
%    \begin{macrocode}
\keys_define:nn { ctex }
  {
    linestretch .code:n =
      {
        \ctex_set_default_ccwd:Nn \l_@@_line_stretch_tl {#1}
        \ctex_select_size:
      } ,
    linestretch .value_required:n = true
  }
\tl_new:N \l_@@_line_stretch_tl
\tl_set:Nn \l_@@_line_stretch_tl { \ccwd }
%    \end{macrocode}
% \end{macro}
%
% \begin{macro}[int]{\ctex_update_stretch:}
% 首先计算一行上汉字的字数,\tn{CJKglue} 相当于将 \tn{linewidth} 与汉字总宽度之差
% 均匀地填充到汉字之间。\hologo{eTeX} 的除法是四舍五入,而我们这里应该用截断。由于
% 没有可展性的要求,直接用原语 \cs{tex_divide:D} 要比 \cs{int_div_truncate:nn}
% 快一些。下面的算法还兼顾到了 \tn{linewidth} 不为汉字字宽的整数倍的情况。
% 若用户禁用 \opt{linestretch} 并且修改过 \tn{CJKglue},则只更新
% \tn{ccwd},否则设置伸展量为 $0.08$ 倍 \tn{baselineskip}。注意 \pkg{everysel} 的
% 钩子位于 \tn{size@update} 之前,\tn{baselineskip} 还未更新,不能直接使用它。
%    \begin{macrocode}
\cs_new_protected_nopar:Npn \ctex_update_stretch:
  {
    \ctex_update_em_unit:
    \dim_set:Nn \l_@@_tmp_dim { \l_@@_line_stretch_tl }
    \dim_compare:nNnTF \l_@@_tmp_dim = \c_max_dim
      {
        \ctex_if_ccglue_touched:TF
          { \ctex_update_ccwd: }
          {
            \dim_set:Nn \l_@@_tmp_dim
              { \baselinestretch \tex_glueexpr:D \f@baselineskip \scan_stop: }
            \skip_set:Nn \l_@@_ccglue_skip
              { \c_zero_dim plus .08 \l_@@_tmp_dim }
            \ctex_update_ccglue:
          }
      }
      {
        \int_set:Nn \l_@@_tmp_int
          { \tex_dimexpr:D \linewidth - \ccwd - \l_@@_tmp_dim \scan_stop: }
        \tex_divide:D \l_@@_tmp_int \ccwd
        \int_compare:nNnTF \l_@@_tmp_int > \c_zero_int
          {
            \skip_set:Nn \l_@@_ccglue_skip
              {
                \c_zero_dim plus \dim_eval:n
                  {
                    ( \linewidth - \ccwd - \l_@@_tmp_int \ccwd ) /
                    \l_@@_tmp_int
                  }
              }
          }
          { \skip_zero:N \l_@@_ccglue_skip }
        \ctex_update_ccglue:
      }
  }
%    \end{macrocode}
% \end{macro}
%
% \begin{macro}[int]{\ctex_update_parindent:}
% 更新段落首行缩进。此函数在字号变化时调用。
%    \begin{macrocode}
\cs_new_protected_nopar:Npn \ctex_update_parindent:
  {
    \tl_if_empty:NF \l_@@_autoindent_tl
      {
        \dim_compare:nNnF \parindent = \c_zero_dim
          { \dim_set:Nn \parindent { \l_@@_autoindent_tl } }
      }
  }
%    \end{macrocode}
% \end{macro}
%
% \begin{macro}{\ziju}
% 若参数为 $0$,则恢复正常间距。
%    \begin{macrocode}
\NewDocumentCommand \ziju { m }
  { \exp_args:Nx \ctex_ziju:n {#1} \tex_ignorespaces:D }
\cs_new_protected_nopar:Npn \ctex_ziju:n #1
  {
    \tl_set:Nx \l_@@_ziju_tl { \fp_eval:n {#1} }
    \ctex_select_size:
  }
%    \end{macrocode}
% \end{macro}
%
% \begin{macro}[int]{\ctex_update_ziju:}
% 更新字距。若字距不大于 $-1$,即 \tn{ccwd} 为非正值,则不计算伸缩值。
% 否则,首先假定汉字的宽度为正常宽度加上字距,看一行上能正常放下多少个汉字。
%    \begin{macrocode}
\cs_new_protected_nopar:Npn \ctex_update_ziju:
  {
    \ctex_update_em_unit:
    \dim_set:Nn \l_@@_ziju_dim { \l_@@_ziju_tl \ccwd }
    \dim_add:Nn \ccwd { \l_@@_ziju_dim }
    \dim_compare:nNnTF \ccwd > \c_zero_dim
%    \end{macrocode}
% 伸展量保证行内的剩余空白能够被均匀地填充到汉字之间,收缩的最大限度是让当前行
% 还能够再挤下一个汉字并且不会出现负间距。由 \TeX{} 决定伸展还是收缩。
%    \begin{macrocode}
      {
        \dim_set:Nn \l_@@_tmp_dim
          { \linewidth - \ccwd + \l_@@_ziju_dim }
        \int_set:Nn \l_@@_tmp_int { \l_@@_tmp_dim }
        \tex_divide:D \l_@@_tmp_int \ccwd
        \dim_sub:Nn \l_@@_tmp_dim { \l_@@_tmp_int \ccwd }
%    \end{macrocode}
% 由于 \tn{parindent} 是一个固定值,并不参与伸缩,容易导致第一行出现坏盒子。
% 我们在这里将字数减去 $2$,以此放大伸缩值。
%    \begin{macrocode}
        \dim_compare:nNnF \parindent = \c_zero_dim
          {
            \int_compare:nNnF \l_@@_tmp_int < 3
              { \int_sub:Nn \l_@@_tmp_int { 2 } }
          }
        \skip_set:Nn \l_@@_ccglue_skip
          {
            \l_@@_ziju_dim
            plus  \dim_eval:n { \l_@@_tmp_dim / \l_@@_tmp_int }
            minus \dim_min:nn { \dim_abs:n { \l_@@_ziju_dim } }
              { ( \ccwd - \l_@@_tmp_dim ) / ( \l_@@_tmp_int + 1 ) }
          }
      }
      { \skip_set:Nn \l_@@_ccglue_skip { \l_@@_ziju_dim } }
    \ctex_update_ccglue:
%    \end{macrocode}
% 字距设置得比较大时,为了尽量保证段首缩进能够与下一行对齐,应该需要相应地加上
% 或者减去伸缩值。但是这里并不清楚 \TeX{} 是伸展还是收缩,之前以“当前行是否还
% 放得下一个汉字”为标准加上或减去伸缩值的做法也未必与实际结果一致,所以只好还
% 是设置为 |2\ccwd|。
%    \begin{macrocode}
    \ctex_update_parindent:
  }
\dim_new:N \l_@@_ziju_dim
%    \end{macrocode}
% \end{macro}
%
% \begin{macro}{\CTEXindent,\CTEXnoindent}
% 过时命令。
%    \begin{macrocode}
\NewDocumentCommand \CTEXindent { }
  {
    \msg_warning:nnnn { ctex } { deprecated-command } { \CTEXindent }
      { \parindent is~ set~ to~ 2\ccwd. }
    \ctex_update_ccwd: \dim_set:Nn \parindent { 2 \ccwd }
  }
\NewDocumentCommand \CTEXnoindent { }
  {
    \msg_warning:nnnn { ctex } { deprecated-command } { \CTEXnoindent }
      { \parindent is~ set~ to~ 0pt. }
    \dim_zero:N \parindent
  }
%</class|ctex>
%    \end{macrocode}
% \end{macro}
