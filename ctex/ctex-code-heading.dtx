% \iffalse meta-comment
%
% Copyright (C) 2003--2020
% CTEX.ORG and any individual authors listed elsewhere in this file.
% --------------------------------------------------------------------------
%
% This work may be distributed and/or modified under the
% conditions of the LaTeX Project Public License, either
% version 1.3c of this license or (at your option) any later
% version. This version of this license is in
%    http://www.latex-project.org/lppl/lppl-1-3c.txt
% and the latest version of this license is in
%    http://www.latex-project.org/lppl.txt
% and version 1.3 or later is part of all distributions of
% LaTeX version 2005/12/01 or later.
%
% This work has the LPPL maintenance status `maintained'.
%
% The Current Maintainers of this work are Leo Liu, Qing Lee and Liam Huang.
%
% --------------------------------------------------------------------------
%
% \fi
%
% \section{中文化的标题结构}
%
% 本节内容在 \CTeX{} 文档类或打开 \opt{heading} 选项下生效。
%    \begin{macrocode}
%<*class|heading>
%    \end{macrocode}
%
% \subsection{定义标题格式选项}
%
% \begin{variable}{\c_@@_section_headings_seq}
% 保存 \tn{section} 级以下标题名字。
%    \begin{macrocode}
%<*article|book|report>
\seq_const_from_clist:Nn \c_@@_section_headings_seq
  { section , subsection , subsubsection , paragraph , subparagraph }
%</article|book|report>
%    \end{macrocode}
% \end{variable}
%
% \begin{variable}{\c_@@_headings_seq}
%    \begin{macrocode}
%<*article|book|report>
\seq_new:N \c_@@_headings_seq
\seq_gset_eq:NN \c_@@_headings_seq \c_@@_section_headings_seq
%<book|report>\seq_gput_left:Nn \c_@@_headings_seq { chapter }
\seq_gput_left:Nn \c_@@_headings_seq { part }
%</article|book|report>
%<*beamer>
\seq_const_from_clist:Nn \c_@@_headings_seq
  { part , section , subsection }
%</beamer>
%    \end{macrocode}
% \end{variable}
%
% \begin{macro}{\@@_initial_heading:n}
%    \begin{macrocode}
\cs_new_protected_nopar:Npn \@@_initial_heading:n #1
  {
    \tl_new:c { CTEX@pre#1 }
    \tl_new:c { CTEX@post#1 }
    \tl_const:cx { CTEXthe#1 }
      {
        \exp_not:c { CTEX@pre#1 }
        \exp_not:c { CTEX@the#1 }
        \exp_not:c { CTEX@post#1 }
      }
    \tl_const:cx { CTEX@#1name }
      {
        \group_begin:
          \exp_not:c { CTEX@#1@nameformat }
            {
              \exp_not:c { CTEX@pre#1 }
              \exp_not:N \tl_if_empty:NTF
              \exp_not:c { CTEX@#1@numberformat }
                { \exp_not:c { CTEX@the#1 } }
                {
                  \group_begin:
                    \exp_not:c { CTEX@#1@numberformat }
                    \exp_not:c { CTEX@the#1 }
                  \group_end:
                }
              \exp_not:c { CTEX@post#1 }
            }
        \group_end:
      }
  }
%    \end{macrocode}
% \end{macro}
%
% \begin{macro}{\@@_def_heading_keys:n}
%    \begin{macrocode}
\cs_new_protected_nopar:Npn \@@_def_heading_keys:n #1
  {
    \tl_put_right:Nx \l_@@_tmp_tl
      {
        #1                  .meta:nn = { ctex / #1 } { ####1 } ,
        #1 / name            .code:n =
          { \ctex_assign_heading_name:nn {#1} { ####1 } } ,
        #1 / number        .tl_set:N = \exp_not:c { CTEX@the#1 } ,
        #1 / beforeskip    .tl_set:N = \exp_not:c { CTEX@#1@beforeskip } ,
        #1 / afterskip     .tl_set:N = \exp_not:c { CTEX@#1@afterskip} ,
        #1 / indent        .tl_set:N = \exp_not:c { CTEX@#1@indent } ,
        #1 / numbering   .bool_set:N = \exp_not:c { CTEX@#1@numbering } ,
        #1 / numbering    .initial:n = true ,
        #1 / beforeskip   .initial:n = \c_zero_skip ,
        #1 / afterskip    .initial:n = \c_zero_skip ,
        #1 / indent       .initial:n = \c_zero_dim ,
        #1 / beforeskip   .value_required:n = true ,
        #1 / afterskip    .value_required:n = true ,
        #1 / indent       .value_required:n = true ,
%<*article|book|report>
        #1 / afterindent .bool_set:N = \exp_not:c { CTEX@#1@afterindent } ,
        #1 / fixskip     .bool_set:N = \exp_not:c { CTEX@#1@fixskip } ,
        #1 / hang        .bool_set:N = \exp_not:c { CTEX@#1@hang } ,
        #1 / hang         .initial:n = true ,
        #1 / runin       .bool_set:N = \exp_not:c { CTEX@#1@runin } ,
        #1 / tocline         .code:n =
          {
            \cs_set:Npn \exp_not:c { CTEX@#1@tocline}
                        \exp_not:n { ####1####2 } { ####1 }
          } ,
        \@@_plus_key_aux:nn {#1} { break } ,
%</article|book|report>
        \@@_plus_key_aux:nn {#1} { format } ,
        \@@_plus_key_aux:nn {#1} { nameformat } ,
        \@@_plus_key_aux:nn {#1} { numberformat } ,
        \@@_plus_key_aux:nn {#1} { titleformat } ,
        \@@_plus_key_aux:nn {#1} { aftername } ,
        \@@_plus_key_aux:nn {#1} { aftertitle } ,
      }
  }
\cs_new_nopar:Npn \@@_plus_key_aux:nn #1#2
  {
    #1 / #2   .tl_set:N = \exp_not:c { CTEX@#1@#2 } ,
    #1 / #2 +   .code:n =
      { \tl_put_right:Nn \exp_not:c { CTEX@#1@#2 } { ####1 } } ,
    #1 / #2 ~ + .code:n =
      { \tl_put_right:Nn \exp_not:c { CTEX@#1@#2 } { ####1 } }
  }
%    \end{macrocode}
% \end{macro}
%
% \begin{macro}[int]{\ctex_assign_heading_name:nn}
% \begin{macro}{\@@_assign_heading_name:nnn}
% \opt{name} 的值是一个至多两个元素的逗号分隔列表。由于 \LaTeXiii{} 的
% \texttt{clist} 总是会自动忽略空元素,所以设置 |name={,章}| 后,第一个元素将会
% 是“章”,必须用空的分组保护空元素:|name={{},章}|,这在使用中有些许不便。我们
% 可以改用 \texttt{seq} 或者手写函数解析参数来加以改进。为实现的简单起见,这里用
% 了 \pkg{xparse} 的 \tn{SplitArgument},它带有参数的长度检查。
%    \begin{macrocode}
\NewDocumentCommand \ctex_assign_heading_name:nn
  { m > { \SplitArgument { 1 } { , } } +m }
  { \@@_assign_heading_name:nnn {#1} #2 }
\cs_new_protected:Npn \@@_assign_heading_name:nnn #1#2#3
  {
    \tl_set:cn { CTEX@pre#1 } {#2}
    \IfNoValueTF {#3}
      { \tl_clear:c { CTEX@post#1 } }
      { \tl_set:cn { CTEX@post#1 } {#3} }
  }
%    \end{macrocode}
% \end{macro}
% \end{macro}
%
% \begin{macro}{part/pagestyle,chapter/pagestyle,chapter/lofskip,chapter/lotskip}
% 只在 \cls{ctexbook} 和 \cls{ctexrep} 下有定义。
%    \begin{macrocode}
\group_begin:
%<*book|report>
\tl_set:Nn \l_@@_tmp_tl
  {
    part    / pagestyle .tl_set:N = \CTEX@part@pagestyle ,
    chapter / pagestyle .tl_set:N = \CTEX@chapter@pagestyle ,
    chapter / lofskip   .tl_set:N = \CTEX@chapter@lofskip ,
    chapter / lotskip   .tl_set:N = \CTEX@chapter@lotskip ,
    chapter / lofskip  .initial:n = \c_zero_skip ,
    chapter / lotskip  .initial:n = \c_zero_skip ,
    chapter / lofskip  .value_required:n = true ,
    chapter / lotskip  .value_required:n = true ,
  }
%</book|report>
%<*article|beamer>
\tl_clear:N \l_@@_tmp_tl
%</article|beamer>
%    \end{macrocode}
% \end{macro}
%
% 定义标题键值选项。
%    \begin{macrocode}
\seq_map_inline:Nn \c_@@_headings_seq
  {
    \@@_initial_heading:n {#1}
    \@@_def_heading_keys:n {#1}
  }
\use:x
  {
    \group_end:
    \keys_define:nn { ctex } { \exp_not:o { \l_@@_tmp_tl } }
  }
%    \end{macrocode}
%
%    \begin{macrocode}
%<*article|book|report>
%    \end{macrocode}
%
% \subsection{标准标题命令的修改}
%
% \begin{macro}[int]{\CTEX@fixtopskip}
% 修正 \cls{book} 和 \cls{report} 类的 \tn{part} 和 \tn{chapter} 标题之前的多余空行。
%    \begin{macrocode}
%<*book|report>
\cs_new_protected_nopar:Npn \CTEX@fixtopskip
  {
    \CTEX@fixheadingskip
    \dim_compare:nNnF \tex_pagegoal:D < \c_max_dim
      { \skip_sub:Nn \l_@@_heading_skip { \tex_topskip:D } }
  }
%</book|report>
%    \end{macrocode}
% \end{macro}
%
% \begin{macro}[int]{\CTEX@fixheadingskip}
% 抑制行间粘连,修正标题前后的多余间距。事实上,减掉 \tn{parskip},有一定的风险。
% 如果接下来的内容不会进入水平模式(例如在 \opt{format} 选项中使用 \tn{hrule} 或者 \tn{hbox}),
% \TeX{} 就不会加上 \tn{parskip}。这时候就需要用户把 \tn{parskip} 加到 \opt{beforeskip}
% 或者 \opt{afterskip} 作为修正。
%    \begin{macrocode}
\cs_new_protected_nopar:Npn \CTEX@fixheadingskip
  {
    \par
    \dim_set:Nn \tex_prevdepth:D { -1000pt }
    \skip_sub:Nn \l_@@_heading_skip { \tex_parskip:D }
  }
\skip_new:N \l_@@_heading_skip
\cs_new_protected_nopar:Npn \CTEX@setheadingskip
  { \skip_set:Nn \l_@@_heading_skip }
\cs_new_eq:NN \CTEX@headingskip \l_@@_heading_skip
%    \end{macrocode}
% \end{macro}
%
% \begin{macro}[int]{\partmark}
% 提供 \tn{partmark}。
%    \begin{macrocode}
\ProvideDocumentCommand \partmark { m }
  { \markboth { } { } }
%    \end{macrocode}
% \end{macro}
%
% \begin{macro}{\CTEXifname}
% \begin{macro}[int]{\CTEX@ifnametrue,\CTEX@ifnamefalse}
% 用于判断当前标题是否有编号。
%    \begin{macrocode}
\cs_new_eq:NN \CTEXifname \use_ii:nn
\cs_new_protected_nopar:Npn \CTEX@ifnametrue
  { \cs_set_eq:NN \CTEXifname \use_i:nn }
\cs_new_protected_nopar:Npn \CTEX@ifnamefalse
  { \cs_set_eq:NN \CTEXifname \use_ii:nn }
%    \end{macrocode}
% \end{macro}
% \end{macro}
%
% \begin{macro}[int]{\CTEX@addloflotskip}
% 往插图和表格目录中加入额外间距。如果间距为零,则不加入。
%    \begin{macrocode}
%<*book|report>
\cs_new_protected_nopar:Npn \CTEX@addloflotskip #1
  {
    \skip_set:Nn \l_@@_heading_skip { \use:c { CTEX@#1@lofskip } }
    \skip_if_eq:nnF { \l_@@_heading_skip } { \c_zero_skip }
      {
        \addtocontents { lof }
          { \protect \addvspace { \skip_use:N \l_@@_heading_skip } }
      }
    \skip_set:Nn \l_@@_heading_skip { \use:c { CTEX@#1@lotskip } }
    \skip_if_eq:nnF { \l_@@_heading_skip } { \c_zero_skip }
      {
        \addtocontents { lot }
          { \protect \addvspace { \skip_use:N \l_@@_heading_skip } }
      }
  }
%</book|report>
%    \end{macrocode}
% \end{macro}
%
% \begin{macro}[int]{\CTEX@addtocline}
%    \begin{macrocode}
\cs_new_protected:Npn \CTEX@addtocline #1#2
  { \addcontentsline { toc } {#1} { \use:c { CTEX@#1@tocline } {#1} {#2} } }
%    \end{macrocode}
% \end{macro}
%
% \begin{macro}[int]{\CTEX@disableautoindent}
% 禁用自动调整首行缩进。
%    \begin{macrocode}
\cs_new_protected_nopar:Npn \CTEX@disableautoindent
  { \tl_clear:N \l_@@_autoindent_tl }
%    \end{macrocode}
% \end{macro}
%
% \subsubsection{part 的标题}
%
%    \begin{macrocode}
%<@@=>
%    \end{macrocode}
%
% \begin{macro}[int]{\part}
%    \begin{macrocode}
%<*article>
\renewcommand\part{%
   \if@noskipsec \leavevmode \fi
   \par
   \CTEX@part@break
%  \addvspace{4ex}%
   \CTEX@setheadingskip \CTEX@part@beforeskip
   \ifodd \CTEX@part@fixskip \CTEX@fixheadingskip \fi
   \addvspace \CTEX@headingskip
   \ifodd \CTEX@part@afterindent
     \@afterindenttrue
   \else
     \@afterindentfalse
   \fi
   \secdef\@part\@spart}
%</article>
%<*book|report>
\renewcommand\part{%
% \if@openright
%   \cleardoublepage
% \else
%   \clearpage
% \fi
  \CTEX@part@break
% \thispagestyle{plain}%
  \thispagestyle{\CTEX@part@pagestyle}%
  \if@twocolumn
    \onecolumn
    \@tempswatrue
  \else
    \@tempswafalse
  \fi
% \null\vfil
  \CTEX@setheadingskip \CTEX@part@beforeskip
  \ifodd \CTEX@part@fixskip \CTEX@fixtopskip \fi
  \vspace*{\CTEX@headingskip}%
  \secdef\@part\@spart}
%</book|report>
%    \end{macrocode}
% \end{macro}
%
% \begin{macro}[int]{\@part}
%    \begin{macrocode}
%<*article>
\def\@part[#1]#2{%
  \ifnum \c@secnumdepth >\m@ne
    \ifodd \CTEX@part@numbering
      \CTEX@ifnametrue
      \refstepcounter{part}%
%     \addcontentsline{toc}{part}{\thepart\hspace{1em}#1}%
    \else
      \CTEX@ifnamefalse
      \CTEX@makeanchor{part*}%
%     \addcontentsline{toc}{part}{#1}%
    \fi
  \else
    \CTEX@ifnamefalse
    \CTEX@makeanchor{part*}%
%   \addcontentsline{toc}{part}{#1}%
  \fi
  \CTEX@gettitle{#1}%
  \CTEX@addtocline{part}{#1}%
  {\interlinepenalty \@M
%  \normalfont \parindent \z@ \raggedright
   \CTEX@disableautoindent
   \normalfont \CTEX@part@format
%  \ifnum \c@secnumdepth >\m@ne
%    \Large\bfseries\partname\nobreakspace\thepart\par\nobreak
%  \fi
   \CTEX@hangindent{part}%
     {\CTEXifname{\CTEX@partname\CTEX@part@aftername}{}}%
%  \huge\bfseries #2%
   \CTEX@part@titleformat{#2}%
%  \markboth{}{}%
   \partmark{#1}%
   \CTEX@part@aftertitle}%
  \nobreak
% \vskip 3ex
  \CTEX@setheadingskip \CTEX@part@afterskip
  \ifodd \CTEX@part@fixskip \CTEX@fixheadingskip \fi
  \vskip \CTEX@headingskip
  \@afterheading}
%</article>
%<*book|report>
\def\@part[#1]#2{%
  \ifnum \c@secnumdepth >-2\relax
    \ifodd \CTEX@part@numbering
      \CTEX@ifnametrue
      \refstepcounter{part}%
%     \addcontentsline{toc}{part}{\thepart\hspace{1em}#1}%
    \else
      \CTEX@ifnamefalse
      \CTEX@makeanchor{part*}%
%     \addcontentsline{toc}{part}{#1}%
    \fi
  \else
    \CTEX@ifnamefalse
    \CTEX@makeanchor{part*}%
%   \addcontentsline{toc}{part}{#1}%
  \fi
  \CTEX@gettitle{#1}%
  \CTEX@addtocline{part}{#1}%
%  \markboth{}{}%
   \partmark{#1}%
  {\interlinepenalty \@M
%  \normalfont \centering
   \CTEX@disableautoindent
   \normalfont \CTEX@part@format
%  \ifnum \c@secnumdepth >-2\relax
%    \huge\bfseries\partname\nobreakspace\thepart\par\vskip 20\p@
%  \fi
   \CTEX@hangindent{part}%
     {\CTEXifname{\CTEX@partname\CTEX@part@aftername}{}}%
%  \Huge\bfseries #2\par}%
   \CTEX@part@titleformat{#2}%
   \CTEX@part@aftertitle}%
  \@endpart}
%</book|report>
%    \end{macrocode}
% \end{macro}
%
% \begin{macro}[int]{\@spart}
%    \begin{macrocode}
%<*article>
\def\@spart#1{%
    \CTEX@ifnamefalse
    \CTEX@makeanchor@spart{part*}%
    \CTEX@gettitle{#1}%
    {\interlinepenalty \@M
%    \normalfont \parindent \z@ \raggedright
     \CTEX@disableautoindent
     \normalfont \CTEX@part@format
     \CTEX@hangindent{part}{}%
%    \huge \bfseries #1\par}%
     \CTEX@part@titleformat{#1}%
     \CTEX@part@aftertitle}%
     \nobreak
%    \vskip 3ex
     \CTEX@setheadingskip \CTEX@part@afterskip
     \ifodd \CTEX@part@fixskip \CTEX@fixheadingskip \fi
     \vskip \CTEX@headingskip
     \@afterheading}
%</article>
%<*book|report>
\def\@spart#1{%
    \CTEX@ifnamefalse
    \CTEX@makeanchor@spart{part*}%
    \CTEX@gettitle{#1}%
    {\interlinepenalty \@M
%    \normalfont \centering
     \CTEX@disableautoindent
     \normalfont \CTEX@part@format
     \CTEX@hangindent{part}{}%
%    \Huge \bfseries #1\par}%
     \CTEX@part@titleformat{#1}%
     \CTEX@part@aftertitle}%
    \@endpart}
%</book|report>
%    \end{macrocode}
% \end{macro}
%
% \begin{macro}[int]{\@endpart}
%    \begin{macrocode}
%<*book|report>
\def\@endpart{%
%             \vfil
              \CTEX@setheadingskip \CTEX@part@afterskip
              \ifodd \CTEX@part@fixskip \CTEX@fixheadingskip \fi
              \vskip \CTEX@headingskip
              \newpage
              \if@twoside
               \if@openright
                \null
                \thispagestyle{empty}%
                \newpage
               \fi
              \fi
              \if@tempswa
                \twocolumn
              \fi}
%</book|report>
%    \end{macrocode}
% \end{macro}
%
% \subsubsection{chapter 的标题}
%
%    \begin{macrocode}
%<*book|report>
%    \end{macrocode}
%
% \begin{macro}[int]{\chapter}
%    \begin{macrocode}
\renewcommand\chapter{%
%                   \if@openright\cleardoublepage\else\clearpage\fi
%                   \thispagestyle{plain}%
                    \CTEX@chapter@break
                    \thispagestyle{\CTEX@chapter@pagestyle}%
                    \global\@topnum\z@
%                   \@afterindentfalse
                    \ifodd \CTEX@chapter@afterindent
                      \@afterindenttrue
                    \else
                      \@afterindentfalse
                    \fi
                    \secdef\@chapter\@schapter}
%    \end{macrocode}
% \end{macro}
%
% \begin{macro}[int]{\@chapter}
%    \begin{macrocode}
\def\@chapter[#1]#2{%
  \ifnum \c@secnumdepth >\m@ne
%<*book>
    \if@mainmatter
%</book>
      \ifodd \CTEX@chapter@numbering
        \CTEX@ifnametrue
        \refstepcounter{chapter}%
%       \typeout{\@chapapp\space\thechapter.}%
        \typeout{\CTEXthechapter}%
%       \addcontentsline{toc}{chapter}
%         {\protect\numberline{\thechapter}#1}%
      \else
        \CTEX@ifnamefalse
        \CTEX@makeanchor{\Hy@chapapp*}%
%       \addcontentsline{toc}{chapter}{#1}%
      \fi
%<*book>
    \else
      \CTEX@ifnamefalse
      \CTEX@makeanchor@chapter{\Hy@chapapp*}%
%     \addcontentsline{toc}{chapter}{#1}%
    \fi
%</book>
  \else
    \CTEX@ifnamefalse
    \CTEX@makeanchor@chapter{\Hy@chapapp*}%
%   \addcontentsline{toc}{chapter}{#1}%
  \fi
  \CTEX@gettitle{#1}%
  \CTEX@addtocline{chapter}{#1}%
  \chaptermark{#1}%
% \addtocontents{lof}{\protect\addvspace{10\p@}}%
% \addtocontents{lot}{\protect\addvspace{10\p@}}%
  \CTEX@addloflotskip{chapter}%
  \if@twocolumn
    \@topnewpage[\@makechapterhead{#2}]%
  \else
    \@makechapterhead{#2}%
  \@afterheading
  \fi}
%    \end{macrocode}
% \end{macro}
%
% \begin{macro}[int]{\@schapter}
%    \begin{macrocode}
\def\@schapter#1{%
  \CTEX@ifnamefalse
  \CTEX@makeanchor@schapter{\Hy@chapapp*}%
  \CTEX@gettitle{#1}%
  \if@twocolumn
    \@topnewpage[\@makeschapterhead{#1}]%
  \else
    \@makeschapterhead{#1}%
    \@afterheading
  \fi}
%    \end{macrocode}
% \end{macro}
%
% \begin{macro}[int]{\@makechapterhead}
%    \begin{macrocode}
\def\@makechapterhead#1{%
% \vspace*{50\p@}%
  \CTEX@setheadingskip \CTEX@chapter@beforeskip
  \ifodd \CTEX@chapter@fixskip \CTEX@fixtopskip \fi
  \vspace*{\CTEX@headingskip}%
% {\normalfont \parindent \z@ \raggedright
  {\CTEX@disableautoindent
   \normalfont \CTEX@chapter@format
   \interlinepenalty\@M
%  \ifnum \c@secnumdepth >\m@ne
%    \if@mainmatter
%      \huge\bfseries\@chapapp\space\thechapter\par\nobreak\vskip 20\p@
%    \fi
%  \fi
   \CTEX@hangindent{chapter}%
     {\CTEXifname{\CTEX@chaptername\CTEX@chapter@aftername}{}}%
%  \Huge \bfseries #1\par\nobreak
   \CTEX@chapter@titleformat{#1}%
   \CTEX@chapter@aftertitle
   \nobreak
%  \vskip 40\p@
   \CTEX@setheadingskip \CTEX@chapter@afterskip
   \ifodd \CTEX@chapter@fixskip \CTEX@fixheadingskip \fi
   \vskip \CTEX@headingskip
  }}
%    \end{macrocode}
% \end{macro}
%
% \begin{macro}[int]{\@makeschapterhead}
%    \begin{macrocode}
\def\@makeschapterhead#1{%
% \vspace*{50\p@}%
  \CTEX@setheadingskip \CTEX@chapter@beforeskip
  \ifodd \CTEX@chapter@fixskip \CTEX@fixtopskip \fi
  \vspace*{\CTEX@headingskip}%
% {\normalfont \parindent \z@ \raggedright
  {\CTEX@disableautoindent
   \normalfont \CTEX@chapter@format
   \interlinepenalty\@M
   \CTEX@hangindent{chapter}{}%
%  \Huge \bfseries  #1\par\nobreak
   \CTEX@chapter@titleformat{#1}%
   \CTEX@chapter@aftertitle
   \nobreak
%  \vskip 40\p@
   \CTEX@setheadingskip \CTEX@chapter@afterskip
   \ifodd \CTEX@chapter@fixskip \CTEX@fixheadingskip \fi
   \vskip \CTEX@headingskip
  }}
%    \end{macrocode}
% \end{macro}
%
%    \begin{macrocode}
%</book|report>
%    \end{macrocode}
%
% \subsubsection{section 类的标题}
%
% \begin{macro}[int]{\@startsection}
%    \begin{macrocode}
\def\@startsection#1#2#3#4#5#6{%
  \if@noskipsec \leavevmode \fi
  \par
% \@tempskipa #4\relax
% \@afterindenttrue
% \ifdim \@tempskipa <\z@
%   \@tempskipa -\@tempskipa \@afterindentfalse
% \fi
  \CTEX@update@sectionformat@n{#1}%
  \ifodd \CTEX@afterindent
    \@afterindenttrue
  \else
    \@afterindentfalse
  \fi
  \if@nobreak
    \everypar{}%
  \else
%   \addpenalty\@secpenalty\addvspace\@tempskipa
    \csname CTEX@#1@break\endcsname
    \CTEX@setheadingskip{#4}%
    \ifodd \CTEX@fixskip \CTEX@fixheadingskip \fi
    \addvspace \CTEX@headingskip
  \fi
  \@ifstar
    {\CTEX@makeanchor@ssect{#1*}\@ssect{#3}{#4}{#5}{#6}}%
    {\@dblarg{\@sect{#1}{#2}{#3}{#4}{#5}{#6}}}}
%    \end{macrocode}
% \end{macro}
%
% \begin{macro}[int]{\@seccntformat}
%    \begin{macrocode}
\def\@seccntformat#1{%
% \csname the#1\endcsname\quad}%
  \csname CTEX@#1name\endcsname
  \csname CTEX@#1@aftername\endcsname}
%    \end{macrocode}
% \end{macro}
%
% \begin{macro}[int]{\@sect}
%    \begin{macrocode}
\def\@sect#1#2#3#4#5#6[#7]#8{%
  \ifnum #2>\c@secnumdepth
    \CTEX@ifnamefalse
    \CTEX@makeanchor@sect{#1*}%
    \let\@svsec\@empty
  \else
    \ifodd \csname CTEX@#1@numbering\endcsname
      \CTEX@ifnametrue
      \refstepcounter{#1}%
      \protected@edef\@svsec{\@seccntformat{#1}\relax}%
    \else
      \CTEX@ifnamefalse
      \CTEX@makeanchor{#1*}%
      \let\@svsec\@empty
    \fi
  \fi
  \CTEX@gettitle{#7}%
% \@tempskipa #5\relax
% \ifdim \@tempskipa>\z@
  \unless \ifodd \CTEX@runin
    \begingroup
      #6{%
        \CTEX@hangfrom{\hskip\glueexpr #3\relax\@svsec}%
%       \interlinepenalty \@M #8\@@par}%
        \interlinepenalty \@M
        \csname CTEX@#1@titleformat\endcsname{#8}%
        \csname CTEX@#1@aftertitle\endcsname}%
    \endgroup
    \csname #1mark\endcsname{#7}%
%   \addcontentsline{toc}{#1}{%
%     \ifnum #2>\c@secnumdepth \else
%       \protect\numberline{\csname the#1\endcsname}%
%     \fi
%     #7}%
    \CTEX@addtocline{#1}{#7}%
  \else
    \def\@svsechd{%
    #6{\hskip\glueexpr #3\relax
%     \@svsec #8}%
      \@svsec
      \csname CTEX@#1@titleformat\endcsname{#8}%
      \csname CTEX@#1@aftertitle\endcsname}%
    \csname #1mark\endcsname{#7}%
%   \addcontentsline{toc}{#1}{%
%     \ifnum #2>\c@secnumdepth \else
%       \protect\numberline{\csname the#1\endcsname}%
%     \fi
%     #7}%
    \CTEX@addtocline{#1}{#7}}%
  \fi
  \@xsect{#5}}
%    \end{macrocode}
% \end{macro}
%
% \begin{macro}[int]{\@ssect}
%    \begin{macrocode}
\def\@ssect#1#2#3#4#5{%
  \CTEX@ifnamefalse
  \CTEX@gettitle{#5}%
% \@tempskipa #3\relax
% \ifdim \@tempskipa>\z@
  \unless \ifodd \CTEX@runin
    \begingroup
      #4{%
        \CTEX@hangfrom{\hskip\glueexpr #1\relax}%
%         \interlinepenalty \@M #5\@@par}%
          \interlinepenalty \@M
          \CTEX@titleformat@n{#5}%
          \CTEX@aftertitle}%
    \endgroup
  \else
%   \def\@svsechd{#4{\hskip\glueexpr #1\relax #5}}%
    \def\@svsechd{#4{\hskip\glueexpr #1\relax
                     \CTEX@titleformat@n{#5}\CTEX@aftertitle}}%
  \fi
  \@xsect{#3}}
%    \end{macrocode}
% \end{macro}
%
% \begin{macro}[int]{\@xsect}
%    \begin{macrocode}
\def\@xsect#1{%
% \@tempskipa #1\relax
% \ifdim \@tempskipa>\z@
  \unless \ifodd \CTEX@runin
    \par \nobreak
%   \vskip \@tempskipa
    \CTEX@setheadingskip{#1}%
    \ifodd \CTEX@fixskip \CTEX@fixheadingskip \fi
    \vskip \CTEX@headingskip
    \@afterheading
  \else
    \@nobreakfalse
    \global\@noskipsectrue
    \everypar{%
      \if@noskipsec
        \global\@noskipsecfalse
       {\setbox\z@\lastbox}%
        \clubpenalty\@M
        \begingroup \@svsechd \endgroup
        \unskip
%       \@tempskipa #1\relax
%       \hskip -\@tempskipa
        \hskip\glueexpr #1\relax
      \else
        \clubpenalty \@clubpenalty
        \everypar{}%
      \fi}%
  \fi
  \ignorespaces}
%    \end{macrocode}
% \end{macro}
%
%    \begin{macrocode}
%<@@=ctex>
%    \end{macrocode}
%
% \begin{macro}[int]{\CTEX@hangindent}
% 用于实现 \tn{part} 和 \tn{chapter} 标题的 \opt{indent} 和 \opt{hang} 选项。
%    \begin{macrocode}
\cs_new_protected:Npn \CTEX@hangindent #1#2
  {
    \dim_set:Nn \parindent { \use:c { CTEX@#1@indent } }
    \bool_if:cTF { CTEX@#1@hang }
      { \@hangfrom { \@@_indent_aux: #2 } }
      {#2}
  }
\cs_new_protected_nopar:Npn \@@_indent_aux:
  {
    \dim_compare:nNnF \parindent = \c_zero_dim
      { \skip_horizontal:n { \parindent } }
  }
%    \end{macrocode}
% \end{macro}
%
% \begin{macro}[int]{\CTEX@hangfrom}
% \opt{hang} 选项控制是否采用悬挂缩进。
%    \begin{macrocode}
\cs_new_protected_nopar:Npn \CTEX@hangfrom
  {
    \bool_if:NTF \CTEX@hang
      { \@hangfrom }
      { \noindent \use:n }
  }
%    \end{macrocode}
% \end{macro}
%
% \begin{macro}[int]{\CTEX@update@sectionformat@n}
% 在 \tn{@startsection} 中设置 \tn{CTEX@titleformat@n} 等为相应函数。
%    \begin{macrocode}
\cs_new_protected_nopar:Npn \CTEX@update@sectionformat@n #1
  {
    \cs_set_eq:Nc \CTEX@titleformat@n { CTEX@#1@titleformat }
    \cs_set_eq:Nc \CTEX@aftertitle    { CTEX@#1@aftertitle }
    \cs_set_eq:Nc \CTEX@afterindent   { CTEX@#1@afterindent }
    \cs_set_eq:Nc \CTEX@fixskip       { CTEX@#1@fixskip }
    \cs_set_eq:Nc \CTEX@hang          { CTEX@#1@hang }
    \cs_set_eq:Nc \CTEX@runin         { CTEX@#1@runin }
  }
\cs_new_eq:NN \CTEX@titleformat@n \use:n
\cs_new_eq:NN \CTEX@aftertitle \prg_do_nothing:
\cs_new_eq:NN \CTEX@afterindent \c_true_bool
\cs_new_eq:NN \CTEX@fixskip \c_false_bool
\cs_new_eq:NN \CTEX@hang \c_true_bool
\cs_new_eq:NN \CTEX@runin \c_false_bool
%    \end{macrocode}
% \end{macro}
%
% \begin{macro}[int]{\CTEX@part@tocline, \CTEX@chapter@tocline}
%    \begin{macrocode}
\cs_new:Npn \CTEX@part@tocline #1#2
  {
    \CTEXifname
      { \CTEXthepart \hspace { 1em } }
      { }
    #2
  }
%<*book|report>
\cs_new:Npn \CTEX@chapter@tocline #1#2
  {
    \CTEXifname
      { \protect \numberline { \CTEXthechapter \hspace { .3em } } }
      { }
    #2
  }
%</book|report>
%    \end{macrocode}
% \end{macro}
%
% \begin{macro}[int]{\CTEXnumberline}
%    \begin{macrocode}
\cs_new_nopar:Npn \CTEXnumberline #1
  {
    \CTEXifname
      { \protect \numberline { \use:c { CTEXthe #1 } } }
      { }
  }
%    \end{macrocode}
% \end{macro}
%
%    \begin{macrocode}
\int_zero:N \l_@@_tmp_int
\seq_map_inline:Nn \c_@@_section_headings_seq
  {
    \int_incr:N \l_@@_tmp_int
    \cs_gset_protected_nopar:cpx  {#1}
      {
        \exp_not:N \@startsection {#1}
          { \int_use:N \l_@@_tmp_int }
          { \exp_not:c { CTEX@#1@indent } }
          { \exp_not:c { CTEX@#1@beforeskip } }
          { \exp_not:c { CTEX@#1@afterskip } }
          { \exp_not:N \normalfont \exp_not:c { CTEX@#1@format } }
      }
    \cs_new:cpn { CTEX@#1@tocline } ##1##2
      { \CTEXnumberline { ##1 } ##2 }
  }
%    \end{macrocode}
%
% \subsubsection{附录标题}
%
% \begin{macro}[int]{appendix/name,appendix/number,appendix/numbering}
%    \begin{macrocode}
\keys_define:nn { ctex }
  {
    appendix                .meta:nn = { ctex / appendix } {#1} ,
    appendix / name          .code:n =
      { \ctex_assign_heading_name:nn { appendix } {#1} } ,
    appendix / number      .tl_set:N = \CTEX@appendix@number ,
    appendix / numbering .bool_set:N = \CTEX@appendix@numbering ,
    appendix / numbering  .initial:n = true
  }
\tl_new:N \CTEX@preappendix
\tl_new:N \CTEX@postappendix
%    \end{macrocode}
% \end{macro}
%
% \begin{macro}[int]{\appendix}
%    \begin{macrocode}
\cs_new_eq:NN \CTEX@save@appendix \appendix
\cs_gset_protected_nopar:Npn \appendix
  {
    \CTEX@save@appendix
%<*article>
    \gdef \CTEX@presection { \CTEX@preappendix }
    \gdef \CTEX@thesection { \CTEX@appendix@number }
    \gdef \CTEX@postsection { \CTEX@postappendix }
    \gdef \CTEX@section@numbering { \CTEX@appendix@numbering }
%</article>
%<*book|report>
    \gdef \CTEX@prechapter { \CTEX@preappendix }
    \gdef \CTEX@thechapter { \CTEX@appendix@number }
    \gdef \CTEX@postchapter { \CTEX@postappendix }
    \gdef \CTEX@chapter@numbering { \CTEX@appendix@numbering }
%</book|report>
  }
%    \end{macrocode}
% \end{macro}
%
% \subsubsection{设置 \pkg{hyperref} 宏包的标题锚点}
%
% \begin{macro}[int]{\CTEX@makeanchor}
% 设置超链接跳转锚点,在 \pkg{hyperref} 载入后才有意义。
%    \begin{macrocode}
\cs_new_protected_nopar:Npn \CTEX@makeanchor #1
  { }
%    \end{macrocode}
% \end{macro}
%
% \begin{variable}{\c_@@_headings_cs_seq}
% 保存内部标题命令的 \CTeX{} 定义,用于随后比较。
%    \begin{macrocode}
\seq_const_from_clist:Nn \c_@@_headings_cs_seq
%<article>  { part , spart , sect , ssect }
%<book|report>  { part , spart , chapter , schapter , sect , ssect }
\seq_map_inline:Nn \c_@@_headings_cs_seq
  {
    \cs_new_eq:cc { CTEX@ #1 } { @ #1 }
    \cs_new_eq:cN { CTEX@makeanchor@ #1 } \CTEX@makeanchor
  }
%    \end{macrocode}
% \end{variable}
%
% \begin{macro}[int]{\CTEX@hyperheadinghook}
% \pkg{hyperref} 会重定义内部标题命令,目的在于为没有编号的标题设置锚点(这一功能受他的
% \opt{implicit} 选项的控制)。我们在上面对标题命令的修改已经包含这一功能,如果这些标题命令在
% \pkg{hyperref} 载入之前没有被修改过,则恢复 \CTeX{} 的定义。
%    \begin{macrocode}
\cs_new_protected_nopar:Npn \CTEX@hyperheadinghook
  {
    \group_begin:
      \ifHy@implicit
        \cs_set_eq:NN \H@old@chapter \Hy@org@chapter
        \seq_map_inline:Nn \c_@@_headings_cs_seq
          {
            \cs_if_eq:ccT { H@old@ ##1 } { CTEX@ ##1 }
              {
                \cs_gset_eq:cc { @ ##1 } { CTEX@ ##1 }
                \cs_gset_eq:cN { CTEX@makeanchor@ ##1 } \CTEX@makeanchor
              }
          }
      \else:
        \seq_map_inline:Nn \c_@@_headings_cs_seq
          { \cs_gset_eq:cN { CTEX@makeanchor@ ##1 } \CTEX@makeanchor }
      \fi:
    \group_end:
  }
%    \end{macrocode}
% \end{macro}
%
%    \begin{macrocode}
\ctex_at_end_package:nn { hyperref }
  {
    \cs_gset_protected_nopar:Npn \CTEX@makeanchor #1
      {
        \Hy@MakeCurrentHrefAuto {#1}
        \Hy@raisedlink
          {
            \hyper@anchorstart { \@currentHref }
            \hyper@anchorend
          }
      }
    \CTEX@hyperheadinghook
  }
%    \end{macrocode}
%
% \subsubsection{兼容 \pkg{nameref} 宏包}
%
% \begin{macro}[int]{\CTEX@gettitle}
% 在 \pkg{nameref} 载入后才有意义,与上述 \pkg{hyperref} 的处理类似。
%    \begin{macrocode}
\cs_new_protected:Npn \CTEX@gettitle #1
  { }
\ctex_at_end_package:nn { nameref }
  {
    \cs_gset_protected_nopar:Npn \CTEX@gettitle { \NR@gettitle }
    \seq_map_inline:Nn \c_@@_headings_cs_seq
      {
        \cs_if_eq:ccT { NR@ #1 } { CTEX@ #1 }
          { \cs_gset_eq:cc { @ #1 } { CTEX@ #1 } }
      }
  }
%    \end{macrocode}
% \end{macro}
%
% \subsubsection{兼容 \pkg{titlesec} 宏包}
%
% 我们修改了 \tn{@startsection} 的定义,它的第四个(\meta{beforeskip})和
% 第五个(\meta{afterskip})参数的符号不再有特殊意义,改由相应的选项
% \opt{afterindent} 和 \opt{runin} 来控制。
%
% 引入 \pkg{titlesec} 宏包,并且未设置它的 \opt{loadonly} 选项时,\pkg{titlesec}
% 会展开 section 类标题获取它们的参数,进行初始设置。我们需要进行一些调整。
%
% \begin{macro}[int]{\ctex_titlesec_hook:}
% \tn{titleformat} 的设置保存在名为 |\ttlf@|\meta{section} 的宏中备用,它的内容是
% \begin{quote}\small
%   |\ttlh@|\meta{shape}\Arg{format}\Arg{label}\Arg{sep}\Arg{before}\Arg{after}
% \end{quote}
% 我们这里的 \meta{shape} 为 |hang| 或者 |runin|。\tn{titlespacing} 的设置保存在
% |\ttls@|\meta{section} 之中,它的内容是
% \begin{quote}\small
%   \Arg{left}\Arg{right}\Arg{before}\Arg{after}\Arg{afterindent}
% \end{quote}
% 其中 \meta{afterindent} 为 |1| 或 |0|,分别对应是否保留段首缩进。
% 我们需要根据 \CTeX{} 的 \opt{runin} 和 \opt{afterindent} 选项调整
% |\ttlh@|\meta{shape} 和 \meta{afterindent}。注意,由 \tn{ttl@extract} 得的
% \meta{before} 和 \meta{after} 的值总是非负的,而 \CTeX{} 的 \opt{beforeskip}
% 和 \opt{afterskip} 是可以取负值的,但我们不打算调整它们了。
% 如果使用了 \pkg{titlesec} 的 \opt{indentafter} 等选项,也不需要调整
% |\ttls@|\meta{section}。
%    \begin{macrocode}
\cs_new_protected_nopar:Npn \ctex_titlesec_hook:
  {
    \@ifpackagewith { titlesec } { explicit }
      {
        \cs_set_eq:NN \@@_titlesec_format:Nn
                      \@@_titlesec_format_explicit:Nn
      }
      { }
    \clist_map_inline:nn
      { indentafter , noindentafter , indentfirst , nonindentfirst }
      {
        \@ifpackagewith { titlesec } { ##1 }
          {
            \clist_map_break:n
              { \cs_set_eq:NN \@@_titlesec_hook:n \@@_titlesec_format:n }
          }
          { }
      }
    \seq_map_function:NN \c_@@_section_headings_seq \@@_titlesec_hook:n
  }
\cs_new_protected_nopar:Npn \@@_titlesec_hook:n #1
  {
    \@@_titlesec_format:n {#1}
    \exp_args:Nc \@@_titlesec_spacing:Nn { ttls@#1 } {#1}
  }
\cs_new_protected_nopar:Npn \@@_titlesec_format:n #1
  {
    \cs_if_free:cF { ttlf@#1 }
      { \exp_args:Nc \@@_titlesec_format:Nn { ttlf@#1 } {#1} }
  }
\cs_new_protected_nopar:Npn \@@_titlesec_format:Nn #1#2
  {
    \tl_set:Nx #1
      {
        \bool_if:cTF { CTEX@#2@runin }
          { \exp_not:N \ttlh@runin }
          { \exp_not:N \ttlh@hang }
        \tl_tail:N #1
      }
  }
\cs_new_protected_nopar:Npn \@@_titlesec_format_explicit:Nn #1#2
  {
    \cs_set_nopar:Npx #1 ##1
      {
        \bool_if:cTF { CTEX@#2@runin }
          { \exp_not:N \ttlh@runin }
          { \exp_not:N \ttlh@hang }
        \exp_args:No \tl_tail:n { #1 { } }
      }
  }
\cs_new_protected_nopar:Npn \@@_titlesec_spacing:Nn #1#2
  { \tl_set:Nx #1 { \exp_after:wN \@@_titlesec_spacing:nnnnnn #1 {#2} } }
\cs_new:Npn \@@_titlesec_spacing:nnnnnn #1#2#3#4#5#6
  {
    \exp_not:n { {#1} {#2} {#3} {#4} }
    { \bool_if:cTF { CTEX@#6@afterindent } { \@ne } { \z@ } }
  }
%    \end{macrocode}
% \end{macro}
%
%    \begin{macrocode}
\@ifpackageloaded { titlesec }
  { }
  {
    \ctex_at_end_package:nn { titlesec }
      {
        \@ifpackagewith { titlesec } { loadonly }
          { }
          { \ctex_titlesec_hook: }
      }
  }
%    \end{macrocode}
%
% 让编译时终端显示  \tn{CTEXthechapter},目录使用 |\CTEXtheXXX| 编号。
%    \begin{macrocode}
\ctex_at_end_package:nn { titlesec }
  {
%<*book|report>
    \tl_set:Nn \ttl@chapterout { \typeout { \CTEXthechapter } }
%</book|report>
    \cs_if_free:NF \ttl@tocpart
      {
        \cs_set_protected_nopar:Npn \ttl@tocpart
          { \tl_set:Nn \ttl@a { \CTEXthepart \hspace { 1em } } }
      }
    \seq_map_inline:Nn \c_@@_headings_seq
      {
        \cs_if_exist:cF { ttl@toc #1 }
          {
            \cs_new_protected_nopar:cpx { ttl@toc #1 }
              {
                \tl_set:Nn \exp_not:N \ttl@a
                  {
                    \exp_not:N \protect
                    \exp_not:N \numberline { \exp_not:c { CTEXthe #1 } }
                  }
              }
          }
      }
  }
%    \end{macrocode}
%
% \subsubsection{兼容 \pkg{titleps} 宏包}
%
% 按照 \pkg{titleps} 宏包的实现机制,|\CTEXtheXXX| 等宏直到页眉排版时才会被展开,
% 这可能会造成问题\footnote{\url{https://github.com/CTeX-org/ctex-kit/issues/217}}。
%
% \begin{macro}[int]{\ctex_titleps_hook:}
% 我们修改 \pkg{titleps} 包的内部命令 \tn{ttl@settopmark} 和 \tn{ttl@setsubmark},
% 将 |\CTEXtheXXX| 等加入更新队列中。
%    \begin{macrocode}
\group_begin:
\char_set_catcode_other:N \#
\cs_new_protected_nopar:Npn \ctex_titleps_hook:
  {
    \ctex_patch_cmd:Nnn \ttl@settopmark
      { \protect \@namedef { the#1 } { \@nameuse { the#1 } } }
      {
        \protect \@namedef { the#1 } { \@nameuse { the#1 } }
        \CTEX@titlepslabel@set {#1}
      }
    \ctex_patch_cmd:Nnn \ttl@setsubmark
      { \protect \@namedef { the#1 } { } }
      {
        \protect \@namedef { the#1 } { }
        \CTEX@titlepslabel@clear {#1}
      }
    \ctex_patch_cmd:Nnn \ttl@setsubmark
      { \protect \@namedef { the#2 } { \@nameuse { the#2 } } }
      {
        \protect \@namedef { the#2 } { \@nameuse { the#2 } }
        \CTEX@titlepslabel@set {#2}
      }
  }
\group_end:
%    \end{macrocode}
% \end{macro}
%
% \begin{macro}[int]{\CTEX@titlepslabel@set,\CTEX@titlepslabel@clear}
% 这两个函数要在随后被 \tn{xdef} 展开来获得 |\CTEXtheXXX| 的内容,不应该用
% \tn{protected} 来定义。
%    \begin{macrocode}
\cs_new_nopar:Npn \CTEX@titlepslabel@set #1
  {
    \cs_if_free:cF { CTEXthe#1 }
      { \protect \@namedef { CTEXthe#1 } { \@nameuse { CTEXthe#1 } } }
  }
\cs_new_nopar:Npn \CTEX@titlepslabel@clear #1
  {
    \cs_if_free:cF { CTEXthe#1 }
      { \protect \@namedef { CTEXthe#1 } { } }
  }
%    \end{macrocode}
% \end{macro}
%
% \pkg{titleps} 宏包的功能可以由 \pkg{titlesec} 的选项 \opt{pagestyles} 引入。
%    \begin{macrocode}
\ctex_at_end_package:nn { titlesec }
  { \cs_if_free:NF \ttl@settopmark { \ctex_titleps_hook: } }
\ctex_at_end_package:nn { titleps } { \ctex_titleps_hook: }
%    \end{macrocode}
%
% 除此之外,也可以使用 \pkg{titleps} 提供的命令 \tn{newtitlemark} 来完成:
% \begin{verbatim}
%   \newtitlemark { \CTEXthechapter }
%   \newtitlemark { \CTEXthesection }
% \end{verbatim}
% 但 \tn{newtitlemark} 不包含章节间的层次信息,功能上不及修改内部命令完整。
%
% \begin{macro}[int]{\ttl@setifthe}
% 使 |\iftheXXX| 等命令在页眉设置中可用。
%    \begin{macrocode}
\ctex_at_end_package:nn { titleps }
  {
    \cs_set_protected_nopar:Npn \ttl@setifthe #1
      {
        \exp_args:Nco \cs_set_nopar:Npn { ifthe #1 }
          {
            \CTEXifname
              { \protect \@firstoftwo }
              { \protect \@secondoftwo }
          }
      }
    \seq_map_function:NN \c_@@_headings_seq \ttl@setifthe
  }
%    \end{macrocode}
% \end{macro}
%
%
% \subsection{目录标签的宽度}
%
% \begin{macro}{\CTEX@toc@width@n}
%   \begin{macrocode}
\cs_new_protected:Npn \CTEX@toc@width@n #1
  {
    \hbox_set:Nn \l_@@_tmp_box {#1}
    \dim_set:Nn \@tempdima
      {
        \dim_max:nn { \@tempdima }
          { \box_wd:N \l_@@_tmp_box + \f@size \p@ / 2 }
      }
  }
%    \end{macrocode}
% \end{macro}
%
% \begin{macro}{\numberline,\@@_patch_toc_width:n}
% 为 \tn{numberline} 命令打补丁,并兼容 \pkg{tocloft} 和 \pkg{titletoc} 宏包。
%
% 这里需要替换 |#| 本身,因此需要先切换为 other 类。表示参数的 |#| 用
% \cs{c_parameter_token} 代替。
%   \begin{macrocode}
\group_begin:
\char_set_catcode_other:N \#
\use:n
  {
    \group_end:
    \ctex_preto_cmd:NnnTF \numberline { \ExplSyntaxOff }
      { \CTEX@toc@width@n {#1} }
      { }
      { \ctex_patch_failure:N \numberline }
    \cs_new_protected:Npn \@@_patch_toc_width:n \c_parameter_token 1
      {
        \@ifpackageloaded { \c_parameter_token 1 }
          { }
          {
            \ctex_at_end_package:nn { \c_parameter_token 1 }
              {
                \ctex_preto_cmd:NnnTF \numberline
                  { \char_set_catcode_letter:n { 64 } }
                  { \CTEX@toc@width@n {#1} }
                  { }
                  { \ctex_patch_failure:N \numberline }
              }
          }
      }
  }
\@@_patch_toc_width:n { tocloft  }
\@@_patch_toc_width:n { titletoc }
%    \end{macrocode}
% \end{macro}
%
% \subsection{页眉信息的修改}
%
% \begin{macro}[int]{\ps@headings}
%    \begin{macrocode}
%<*article>
\if@twoside
  \ctex_patch_cmd:Nnn \ps@headings
    { \ifnum \c@secnumdepth > \z@ \thesection \quad \fi }
    { \CTEXifname { \CTEXthesection \quad } { } }
  \ctex_patch_cmd:Nnn \ps@headings
    { \ifnum \c@secnumdepth > \@ne \thesubsection \quad \fi }
    { \CTEXifname { \CTEXthesubsection \quad } { } }
\else:
%    \end{macrocode}
% 不知为何,标准文档类此处对 \texttt{secnumdepth} 的判断为 $0$,与 \tn{section} 的层次 $1$ 不符。
%    \begin{macrocode}
  \ctex_patch_cmd:Nnn \ps@headings
    { \ifnum \c@secnumdepth > \m@ne \thesection \quad \fi }
    { \CTEXifname { \CTEXthesection \quad } { } }
\fi:
%</article>
%<*book|report>
\ctex_patch_cmd:Nnn \ps@headings
  {
%<book>    \ifnum \c@secnumdepth > \m@ne \if@mainmatter
%<report>    \ifnum \c@secnumdepth > \m@ne
      \@chapapp \ \thechapter . ~ \ %
%<report>    \fi
%<book>    \fi \fi
  }
  { \CTEXifname { \CTEXthechapter \quad } { } }
\if@twoside
  \ctex_patch_cmd:Nnn \ps@headings
    { \ifnum \c@secnumdepth > \z@ \thesection . ~ \ \fi }
    { \CTEXifname { \CTEXthesection \quad } { } }
\fi:
%</book|report>
%    \end{macrocode}
% \end{macro}
%
%
% \begin{macro}[int]{\ps@fancy}
% 这里对 \pkg{fancyhdr} 宏包打补丁。原来 \pkg{fancyhdr} 宏包中使用
% \tn{thesection} 等宏表示页眉中的章节编号,这里改用 \pkg{ctex} 包所用的
% \tn{CTEXthesection} 系列宏。
%    \begin{macrocode}
\ctex_at_end_package:nn { fancyhdr }
  {
%<*article>
    \ctex_patch_cmd:Nnn \ps@fancy
      { \ifnum \c@secnumdepth > \z@ \thesection \hskip 1em \relax \fi }
      { \CTEXifname { \CTEXthesection \quad } { } }
    \ctex_patch_cmd:Nnn \ps@fancy
      { \ifnum \c@secnumdepth > \@ne \thesubsection \hskip 1em \relax \fi }
      { \CTEXifname { \CTEXthesubsection \quad } { } }
%</article>
%<*book|report>
    \ctex_patch_cmd:Nnn \ps@fancy
      { \ifnum \c@secnumdepth > \m@ne \@chapapp \ \thechapter . ~ \ \fi }
      { \CTEXifname { \CTEXthechapter \quad } { } }
    \ctex_patch_cmd:Nnn \ps@fancy
      { \ifnum \c@secnumdepth > \z@ \thesection . ~ \ \fi }
      { \CTEXifname { \CTEXthesection \quad } { } }
%</book|report>
  }
%    \end{macrocode}
% \end{macro}
%
%    \begin{macrocode}
%</article|book|report>
%    \end{macrocode}
%
% \subsection{\cls{beamer} 标题页模板的修改}
%
%    \begin{macrocode}
%<*beamer>
%    \end{macrocode}
%
%    \begin{macrocode}
\ExplSyntaxOff
%    \end{macrocode}
%
% 对应 \tn{partpage}。
%    \begin{macrocode}
\defbeamertemplate*{part page}{CTEX}[1][]{%
  \begingroup
    \CTEX@disableautoindent
%    \centering
%    {\usebeamerfont{part name}%
%     \usebeamercolor[fg]{part name}\partname~\insertromanpartnumber}
%    \vskip1em\par
    \par \addvspace{\glueexpr\CTEX@part@beforeskip\relax}%
    \CTEX@part@format
    \parindent \dimexpr \CTEX@part@indent \relax
    \ifodd \CTEX@part@numbering
      \CTEX@partname \CTEX@part@aftername
    \fi
    \begin{beamercolorbox}[sep=16pt,center,#1]{part title}
%      \usebeamerfont{part title}\insertpart\par
      \CTEX@part@titleformat \insertpart \CTEX@part@aftertitle
    \end{beamercolorbox}%
    \par \addvspace{\glueexpr\CTEX@part@afterskip\relax}%
  \endgroup
}
%    \end{macrocode}
%
% 对应 \tn{sectionpage}。
%    \begin{macrocode}
\defbeamertemplate*{section page}{CTEX}[1][]{%
  \begingroup
    \CTEX@disableautoindent
%    \centering
%    {\usebeamerfont{section name}%
%     \usebeamercolor[fg]{section name}\sectionname~\insertsectionnumber}
%    \vskip1em\par
    \par \addvspace{\glueexpr\CTEX@section@beforeskip\relax}%
    \CTEX@section@format
    \parindent \dimexpr \CTEX@section@indent \relax
    \ifodd \CTEX@section@numbering
      \CTEX@sectionname \CTEX@section@aftername
    \fi
    \begin{beamercolorbox}[sep=12pt,center,#1]{part title}
%      \usebeamerfont{section title}\insertsection\par
      \CTEX@section@titleformat \insertsection \CTEX@section@aftertitle
    \end{beamercolorbox}%
    \par \addvspace{\glueexpr\CTEX@section@afterskip\relax}%
  \endgroup
}
%    \end{macrocode}
%
% 对应 \tn{subsectionpage}。
%    \begin{macrocode}
\defbeamertemplate*{subsection page}{CTEX}[1][]{%
  \begingroup
    \CTEX@disableautoindent
%    \centering
%    {\usebeamerfont{subsection name}%
%     \usebeamercolor[fg]{subsection name}\subsectionname~\insertsubsectionnumber}
%    \vskip1em\par
    \par \addvspace{\glueexpr\CTEX@subsection@beforeskip\relax}%
    \CTEX@subsection@format
    \parindent \dimexpr \CTEX@subsection@indent \relax
    \ifodd \CTEX@subsection@numbering
      \CTEX@subsectionname \CTEX@subsection@aftername
    \fi
    \begin{beamercolorbox}[sep=8pt,center,#1]{part title}
%      \usebeamerfont{subsection title}\insertsubsection\par
      \CTEX@subsection@titleformat \insertsubsection \CTEX@subsection@aftertitle
    \end{beamercolorbox}%
    \par \addvspace{\glueexpr\CTEX@subsection@afterskip\relax}%
  \endgroup
}
%    \end{macrocode}
%
% 将 \cls{beamer} 的默认模板重定向为 \texttt{CTEX} 模板。
%    \begin{macrocode}
\defbeamertemplatealias{part page}{default}{CTEX}
\defbeamertemplatealias{section page}{default}{CTEX}
\defbeamertemplatealias{subsection page}{default}{CTEX}
%    \end{macrocode}
%
%    \begin{macrocode}
\ExplSyntaxOn
%    \end{macrocode}
%
%    \begin{macrocode}
%</beamer>
%    \end{macrocode}
%
% \subsection{标签引用数字的汉化}
%
% \begin{macro}[int]{\refstepcounter}
% 对标题进行引用时,设置标签为通过 \opt{number} 选项设置的形式。
%    \begin{macrocode}
\cs_new_protected_nopar:Npn \CTEX@setcurrentlabel@n #1
  {
    \protected@edef \@currentlabel
      {
        \cs_if_exist:cTF { CTEX@the#1 }
          { \exp_args:cc { p@#1 } { CTEX@the#1 } }
          { \exp_not:o { \@currentlabel } }
      }
  }
%    \end{macrocode}
% \end{macro}
%
% \begin{macro}[int]{\ctex_varioref_hook:}
% 关于标签引用的宏包可能会修改 \tn{refstepcounter}。其中 \pkg{cleveref} 和
% \pkg{hyperref} 宏包都会保存之前的定义,并且它们都要求尽可能晚的被载入,所以
% 对我们上述的修改影响不大。需要注意的是 \pkg{varioref} 宏包,如果它在
% \CTeX{} 之后被载入,我们之前的修改将会被覆盖。
%    \begin{macrocode}
\cs_new_protected_nopar:Npn \ctex_varioref_hook:
  {
    \seq_map_inline:Nn \c_@@_headings_seq
      { \ctex_fix_varioref_label:n { ##1 } }
  }
%    \end{macrocode}
% \end{macro}
%
% \begin{macro}{\@@_fix_varioref_label:n}
% \pkg{varioref} 宏包的 \tn{labelformat} 实际上是定义一个以 |\the<#1>| 为参数的宏
% |\p@<#1>|。\LaTeX{} 在定义计数器 |<#1>| 时,都会将 |\p@<#1>| 初始化为 \tn{@empty}。
% 如果这个宏非空,说明用户自定义了标签格式,我们就不再修改。这里不能使用
% \cs{exp_args:Nnc},因为 \texttt{c} 这种展开格式不会将参数放在花括号内。而
% \tn{labelformat} 的定义是
% \begin{verbatim}
%   \def\labelformat#1{\expandafter\def\csname p@#1\endcsname##1}
% \end{verbatim}
% 它的第二个参数必须放在花括号内,否则将会被作为宏的定界符号。
%    \begin{macrocode}
\cs_new_protected_nopar:Npn \ctex_fix_varioref_label:n #1
  {
    \tl_if_empty:cT { p@#1 }
      { \exp_args:Nno \labelformat {#1} { \cs:w CTEX@the#1 \cs_end: } }
  }
%    \end{macrocode}
% \end{macro}
%
% 如果 \pkg{varioref} 已经被载入,则使用它来设置。
%    \begin{macrocode}
\@ifpackageloaded { varioref }
  { \ctex_varioref_hook: }
  {
    \cs_new_eq:NN \CTEX@save@refstepcounter \refstepcounter
    \RenewDocumentCommand \refstepcounter { m }
      {
        \CTEX@save@refstepcounter {#1}
        \CTEX@setcurrentlabel@n {#1}
      }
    \ctex_at_end_package:nn { varioref } { \ctex_varioref_hook: }
  }
%    \end{macrocode}
%
% \subsection{载入 \meta{scheme} 文件}
%
%    \begin{macrocode}
\ctex_scheme_input:o { \l_@@_scheme_tl }
%    \end{macrocode}
%
%    \begin{macrocode}
%</class|heading>
%    \end{macrocode}
%
% \subsection{标题格式的 \opt{scheme} 定义}
%
% 下面使用 \CTeX 文档类的设置方式,\opt{plain} 模拟标准文档类直接定义或以
% \tn{@startsection} 设定的章节标题格式,\opt{chinese} 汉化的标题格式。
%
%    \begin{macrocode}
%<*scheme&(article|book|report|beamer)>
%    \end{macrocode}
%
%    \begin{macrocode}
\keys_set:nn { ctex / part }
  {
    aftertitle  = \par ,
%<*article|book|report>
    hang        = false ,
%</article|book|report>
%<*plain>
    name        = \partname \space ,
%<*article|book|report>
    number      = \thepart ,
%</article|book|report>
%<*beamer>
    number      = \insertromanpartnumber ,
%</beamer>
%</plain>
%<*chinese>
    number      = \chinese { part } ,
%</chinese>
%<*article>
    beforeskip  = 4ex ,
    afterskip   = 3ex ,
%<*plain>
    format      = \raggedright ,
    nameformat  = \Large \bfseries ,
    aftername   = \par \nobreak ,
    titleformat = \huge \bfseries ,
    afterindent = false
%</plain>
%<*chinese>
    format      = \Large \bfseries \centering ,
    aftername   = \quad ,
    afterindent = true
%</chinese>
%</article>
%<*book|report>
    aftername   = \par \vskip 20 \p@ ,
    beforeskip  = 0pt \@plus 1fil ,
    afterskip   = 0pt \@plus 1fil ,
    pagestyle   = plain ,
    break       = \if@openright \cleardoublepage \else \clearpage \fi ,
%<*plain>
    format      = \centering ,
    nameformat  = \huge \bfseries ,
    titleformat = \Huge \bfseries
%</plain>
%<*chinese>
    format      = \huge \bfseries \centering
%</chinese>
%</book|report>
%<*beamer>
    format      = \centering ,
    nameformat  = \usebeamerfont { part ~ name }
                  \usebeamercolor [fg] { part ~ name } ,
    aftername   = \vskip 1em \par ,
    titleformat = \usebeamerfont { part ~ title }
%</beamer>
  }
%    \end{macrocode}
%
%    \begin{macrocode}
%<*book|report>
\keys_set:nn { ctex / chapter }
  {
    pagestyle   = plain ,
    aftertitle  = \par ,
    hang        = false ,
    beforeskip  = 50 \p@ ,
    afterskip   = 40 \p@ ,
    lofskip     = 10 \p@ ,
    lotskip     = 10 \p@ ,
    break       = \if@openright \cleardoublepage \else \clearpage \fi ,
%<*plain>
    name        = \chaptername \space ,
    number      = \thechapter ,
    format      = \raggedright ,
    nameformat  = \huge \bfseries ,
    aftername   = \par \nobreak \vskip 20 \p@ ,
    titleformat = \Huge \bfseries ,
    afterindent = false ,
    tocline     = \CTEXnumberline {#1} #2
%</plain>
%<*chinese>
    number      = \chinese { chapter } ,
    format      = \huge \bfseries \centering ,
    aftername   = \quad ,
    afterindent = true
%</chinese>
  }
%</book|report>
%    \end{macrocode}
%
%    \begin{macrocode}
%<@@=>
%    \end{macrocode}
%
%    \begin{macrocode}
\keys_set:nn { ctex / section }
  {
%<*article|book|report>
    number      = \thesection ,
    aftername   = \quad ,
    aftertitle  = \@@par ,
    beforeskip  = 3.5ex \@plus 1ex \@minus .2ex ,
    afterskip   = 2.3ex \@plus .2ex ,
    runin       = false ,
    break       = \addpenalty \@secpenalty ,
%<*plain>
    format      = \Large \bfseries ,
    afterindent = false
%</plain>
%<*chinese>
    format      = \Large \bfseries \centering ,
    afterindent = true
%</chinese>
%</article|book|report>
%<*beamer>
%<*plain>
    name        = \sectionname \space ,
%</plain>
    format      = \centering ,
    number      = \insertsectionnumber ,
    nameformat  = \usebeamerfont { section ~ name }
                  \usebeamercolor [fg] { section ~ name } ,
    aftername   = \vskip 1em \par ,
    titleformat = \usebeamerfont { section ~ title } ,
    aftertitle  = \par
%</beamer>
  }
%    \end{macrocode}
%
%    \begin{macrocode}
\keys_set:nn { ctex / subsection }
  {
%<*article|book|report>
    number      = \thesubsection ,
    format      = \large \bfseries ,
    aftername   = \quad ,
    aftertitle  = \@@par ,
    beforeskip  = 3.25ex \@plus 1ex \@minus .2ex ,
    afterskip   = 1.5ex  \@plus .2ex ,
    runin       = false ,
    break       = \addpenalty \@secpenalty ,
%<*plain>
    afterindent = false
%</plain>
%<*chinese>
    afterindent = true
%</chinese>
%</article|book|report>
%<*beamer>
%<*plain>
    name        = \subsectionname \space ,
    number      = \insertsubsectionnumber ,
%</plain>
%<*chinese>
    number      = \arabic { section } . \arabic { subsection } ,
%</chinese>
    format      = \centering ,
    nameformat  = \usebeamerfont { subsection ~ name }
                  \usebeamercolor [fg] { subsection ~ name } ,
    aftername   = \vskip 1em \par ,
    titleformat = \usebeamerfont { subsection ~ title } ,
    aftertitle  = \par
%</beamer>
  }
%    \end{macrocode}
%
%    \begin{macrocode}
%<*article|book|report>
%    \end{macrocode}
%
%    \begin{macrocode}
\keys_set:nn { ctex / subsubsection }
  {
    number      = \thesubsubsection ,
    format      = \normalsize \bfseries ,
    aftername   = \quad ,
    aftertitle  = \@@par ,
    beforeskip  = 3.25ex \@plus 1ex \@minus .2ex ,
    afterskip   = 1.5ex \@plus .2ex ,
    runin       = false ,
    break       = \addpenalty \@secpenalty ,
%<*plain>
    afterindent = false
%</plain>
%<*chinese>
    afterindent = true
%</chinese>
  }
%    \end{macrocode}
%
%    \begin{macrocode}
\keys_set:nn { ctex / paragraph }
  {
    number      = \theparagraph ,
    format      = \normalsize \bfseries ,
    aftername   = \quad ,
    beforeskip  = 3.25ex \@plus 1ex \@minus .2ex ,
    break       = \addpenalty \@secpenalty ,
%<*plain>
    afterindent = false
%</plain>
%<*chinese>
    afterindent = true
%</chinese>
  }
%    \end{macrocode}
%
%    \begin{macrocode}
\keys_set:nn { ctex / subparagraph }
  {
    number      = \thesubparagraph ,
    format      = \normalsize \bfseries ,
    aftername   = \quad ,
    beforeskip  = 3.25ex \@plus 1ex \@minus .2ex ,
    break       = \addpenalty \@secpenalty ,
%<*plain>
    afterindent = false
%</plain>
%<*chinese>
    afterindent = true
%</chinese>
  }
%    \end{macrocode}
%
% 处理 \opt{sub3section} 与 \opt{sub4section} 的格式。
%    \begin{macrocode}
\int_compare:nNnTF \g__ctex_section_depth_int > 2
  {
    \keys_set:nn { ctex / paragraph }
      {
        aftertitle  = \@@par ,
        afterskip   = 1ex \@plus .2ex ,
        runin       = false
      }
  }
  {
    \keys_set:nn { ctex / paragraph }
      {
        afterskip   = 1em ,
        runin       = true
      }
  }
\int_compare:nNnTF \g__ctex_section_depth_int > 3
  {
    \keys_set:nn { ctex / subparagraph }
      {
        aftertitle  = \@@par ,
        afterskip   = 1ex \@plus .2ex ,
        runin       = false
      }
  }
  {
    \keys_set:nn { ctex / subparagraph }
      {
        afterskip   = 1em ,
        runin       = true
      }
  }
\int_compare:nNnTF \g__ctex_section_depth_int > 2
  { \keys_set:nn { ctex / subparagraph } { indent = \c_zero_dim } }
  { \keys_set:nn { ctex / subparagraph } { indent = \parindent } }
%    \end{macrocode}
%
%    \begin{macrocode}
%<@@=ctex>
%    \end{macrocode}
%
% 处理附录的格式。
%    \begin{macrocode}
\keys_set:nn { ctex / appendix }
%<*article>
  { number      = \@Alph \c@section }
%</article>
%<*book|report>
  {
    name        = \appendixname \space ,
    number      = \@Alph \c@chapter
  }
%</book|report>
%    \end{macrocode}
%
%    \begin{macrocode}
%</article|book|report>
%    \end{macrocode}
%
%    \begin{macrocode}
%</scheme&(article|book|report|beamer)>
%    \end{macrocode}
%
% \subsection{\pkg{ctex.sty} 的 \opt{heading} 选项}
%
%    \begin{macrocode}
%<*ctex|ctexheading>
%    \end{macrocode}
%
% \begin{variable}{\c_@@_std_class_tl}
% 用于记录被引入的标准文档类。
%    \begin{macrocode}
\clist_map_inline:nn { article , book , report , beamer }
  {
    \@ifclassloaded {#1}
      { \clist_map_break:n { \tl_const:Nn \c_@@_std_class_tl {#1} } }
      { }
  }
%    \end{macrocode}
% \end{variable}
%
% 若标准文档类被引入,则载入对应的标题定义文件。否则视 \tn{chapter} 是否有定义来
% 引入 \cls{book} 或者 \cls{article}。
%    \begin{macrocode}
\msg_new:nnn { ctex } { not-standard-class }
  {
    None~of~the~standard~document~classes~was~loaded.\\
    Heading~`#1'~is~selected.\\
    ctex~may~not~work~as~expected.
  }
%<ctex>\bool_if:NTF \l_@@_heading_bool
%<ctexheading>\use:n
  {
    \tl_if_exist:NTF \c_@@_std_class_tl
      { \cs_new_eq:NN \c_@@_class_tl \c_@@_std_class_tl }
      {
        \cs_if_exist:NTF \chapter
          {
            \cs_if_exist:NF \if@mainmatter
              { \cs_new_eq:NN \if@mainmatter \tex_iftrue:D }
            \tl_const:Nn \c_@@_class_tl { book }
          }
          { \tl_const:Nn \c_@@_class_tl { article } }
        \msg_warning:nnx { ctex } { not-standard-class } { \c_@@_class_tl }
      }
    \ctex_file_input:n { ctex- \c_@@_class_tl .def }
  }
%<ctex>  { \ctex_scheme_input:o { \l_@@_scheme_tl } }
%    \end{macrocode}
%
%    \begin{macrocode}
%</ctex|ctexheading>
%    \end{macrocode}
%
% \subsection{标题配置文件}
%
%    \begin{macrocode}
%<*name>
%    \end{macrocode}
%
%    \begin{macrocode}
\keys_set_known:nn { ctex }
  {
    contentsname   = 目录 ,
    listfigurename = 插图 ,
    listtablename  = 表格 ,
    figurename     = 图 ,
    tablename      = 表 ,
    abstractname   = 摘要 ,
    indexname      = 索引 ,
    bibname        = 参考文献 ,
    appendixname   = 附录 ,
    proofname      = 证明 ,
    algorithmname  = 算法 ,
    refname        = 参考文献 ,
    continuation   = (续) ,
    part    / name = { 第 , 部分 } ,
    chapter / name = { 第 , 章 }
  }
%    \end{macrocode}
%
%    \begin{macrocode}
%</name>
%    \end{macrocode}
