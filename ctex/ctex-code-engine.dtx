% \iffalse meta-comment
%
% Copyright (C) 2003--2020
% CTEX.ORG and any individual authors listed elsewhere in this file.
% --------------------------------------------------------------------------
%
% This work may be distributed and/or modified under the
% conditions of the LaTeX Project Public License, either
% version 1.3c of this license or (at your option) any later
% version. This version of this license is in
%    http://www.latex-project.org/lppl/lppl-1-3c.txt
% and the latest version of this license is in
%    http://www.latex-project.org/lppl.txt
% and version 1.3 or later is part of all distributions of
% LaTeX version 2005/12/01 or later.
%
% This work has the LPPL maintenance status `maintained'.
%
% The Current Maintainers of this work are Leo Liu, Qing Lee and Liam Huang.
%
% --------------------------------------------------------------------------
%
% \fi
%
% \section{特定引擎支持与设置}
%
% \subsection{\pkg{ctex-engine-pdftex.def}}
%
% \begin{macro}[int]{\ctex_set_zhmap:n}
% 设置 \upTeX{} 字体映射,同时作用于 \tn{AtBeginDvi} 与
% \tn{AtBeginShipoutFirst}。该宏对 \pdfTeX{} 和 \upTeX{} 均有用。
%     \begin{macrocode}
%<*pdftex|uptex|aptex>
\cs_new_protected_nopar:Npn \ctex_set_zhmap:n #1
  {
    \AtBeginDvi {#1}
    \ctex_at_end_package:nn { atbegshi }
      { \AtBeginShipoutFirst {#1} }
  }
\@onlypreamble \ctex_set_zhmap:n
%</pdftex|uptex|aptex>
%    \end{macrocode}
% \end{macro}
%
%    \begin{macrocode}
%<*pdftex>
%    \end{macrocode}
%
% \begin{variable}{\c_@@_cmap_encoding_seq}
% 需要加上 CMap 的 CJK 字体编码。
%    \begin{macrocode}
\seq_const_from_clist:Nn \c_@@_cmap_encoding_seq
  { C19 , C10 , C00 , C09 , C40 , C60 }
%    \end{macrocode}
% \end{variable}
%
% \begin{macro}[int]{\ctex_family_cmap:nn}
% 在 \tn{DeclareFontFamily} 的 \meta{loading-settings} 中给 CJK 字体族加上 CMap。
%    \begin{macrocode}
\cs_new_protected_nopar:Npn \ctex_family_cmap:nn #1#2
  {
    \cs_if_free:cF { #1 + #2 }
      {
        \seq_if_in:NnT \c_@@_cmap_encoding_seq {#1}
          { \tl_gput_right:cn { #1 + #2 } { \ctex_add_cmap:n {#1} } }
      }
  }
\cs_generate_variant:Nn \ctex_family_cmap:nn { x }
\cs_new_eq:NN \CTEX@Family@CMap \ctex_family_cmap:xn
%    \end{macrocode}
% \end{macro}
%
% \begin{macro}[int]{\ctex_add_cmap:n}
% 给 |#1| 编码的 CJK 字体加上 CMap。
%    \begin{macrocode}
\cs_new_protected_nopar:Npn \ctex_add_cmap:n #1
  {
    \cs_if_free:NF \CJK@plane
      { \ctex_add_cmap:cn { @@_add_cmap_ #1 \CJK@plane : } {#1} }
  }
\cs_new_protected_nopar:Npn \ctex_add_cmap:Nn #1#2
  {
    \cs_if_exist:NF #1 { \@@_save_cmap:Nn #1 {#2} }
    #1
  }
\cs_generate_variant:Nn \ctex_add_cmap:Nn { c }
\cs_new_protected_nopar:Npn \@@_save_cmap:Nn #1#2
  {
    \tl_set:Nx \l_@@_tmp_tl { \str_lower_case:n {#2} \CJK@plane }
    \tex_immediate:D \tex_pdfobj:D stream ~ file { \l_@@_tmp_tl .cmap }
    \cs_new_protected_nopar:Npx #1
      {
        \exp_not:N \tex_pdffontattr:D \exp_not:N \tex_font:D
          { /ToUnicode ~ \int_use:N \tex_pdflastobj:D \c_space_tl 0 ~ R }
      }
  }
%    \end{macrocode}
% \end{macro}
%
% \begin{macro}[int]{\DeclareFontFamily}
% 只在 \pdfLaTeX{} 下加 CMap。如 \pkg{cmap} 宏包被引入,则不重复设置。
%    \begin{macrocode}
\group_begin:
\char_set_catcode_other:N \#
\sys_if_output_pdf:TF
  {
    \group_end:
    \ctex_appto_cmd:NnnTF \DeclareFontFamily { \ExplSyntaxOff }
      { \CTEX@Family@CMap {#1} {#2} }
      {
        \ctex_at_end_package:nn { cmap }
          { \cs_gset_eq:NN \CTEX@Family@CMap \use_none:nn }
      }
      { \ctex_patch_failure:N \DeclareFontFamily }
  }
  { \group_end: }
%    \end{macrocode}
% \end{macro}
%
% 首先检查选项,决定是否载入 \pkg{zhmCJK} 宏包。
%    \begin{macrocode}
\if_bool:N \g_@@_zhmCJK_bool
  \PassOptionsToPackage { encoding = \g_@@_encoding_tl } { zhmCJK }
  \RequirePackage { zhmCJK }
%    \end{macrocode}
% 不载入 \pkg{zhmCJK} 宏包时直接调用 \pkg{CJK} 及相关宏包。
%    \begin{macrocode}
\else:
  \str_if_eq:onTF { \g_@@_encoding_tl } { GBK }
    { \RequirePackage { CJK } }
    { \RequirePackage { CJKutf8 } }
  \RequirePackage { CJKpunct , CJKspace }
%    \end{macrocode}
%
% \begin{macro}[int]{\ctex_load_zhmap:nnnn}
% 载入 \pkg{zhmetrics} 的字体映射文件,同时设置 \tn{CJKrmdefault} 等。
%    \begin{macrocode}
  \cs_new_protected_nopar:Npn \ctex_load_zhmap:nnnn #1#2#3#4
    {
      \tl_set:Nn \CJKrmdefault {#1}
      \tl_set:Nn \CJKsfdefault {#2}
      \tl_set:Nn \CJKttdefault {#3}
      \ctex_set_zhmap:n { \ctex_zhmap_input:n {#4} }
    }
  \@onlypreamble \ctex_load_zhmap:nnnn
%    \end{macrocode}
% \end{macro}
%
% \begin{macro}[int]{\ctex_zhmap_input:n}
% 载入字体映射文件时,确认 |%| 和 |\| 的 \tn{catcode}。
%    \begin{macrocode}
  \cs_new_protected_nopar:Npn \ctex_zhmap_input:n #1
    {
      \ctex_file_wrapper:nnn
        {
          \char_set_catcode_comment:n { 37 } % %
          \char_set_catcode_escape:n  { 92 } % \
        }
        { \file_input:n {#1} }
        {
          \char_set_catcode:nn { 37 } { \char_value_catcode:n { 37 } }
          \char_set_catcode:nn { 92 } { \char_value_catcode:n { 92 } }
        }
    }
%    \end{macrocode}
% \end{macro}
%
%    \begin{macrocode}
  \tl_if_exist:NF \CJKfamilydefault
    { \tl_const:Nn \CJKfamilydefault { \CJKrmdefault } }
  \tl_if_exist:NF \CJKrmdefault { \tl_new:N \CJKrmdefault }
  \tl_if_exist:NF \CJKsfdefault { \tl_new:N \CJKsfdefault }
  \tl_if_exist:NF \CJKttdefault { \tl_new:N \CJKttdefault }
  \ctex_preto_cmd:NnnTF \rmfamily { \ExplSyntaxOff }
    { \CJKfamily { \CJKrmdefault } }
    { }
    { \ctex_patch_failure:N \rmfamily }
  \ctex_preto_cmd:NnnTF \sffamily { \ExplSyntaxOff }
    { \CJKfamily { \CJKsfdefault } }
    { }
    { \ctex_patch_failure:N \sffamily }
  \ctex_preto_cmd:NnnTF \ttfamily { \ExplSyntaxOff }
    { \CJKfamily { \CJKttdefault } }
    { }
    { \ctex_patch_failure:N \ttfamily }
  \ctex_preto_cmd:NnnTF \normalfont { \ExplSyntaxOff }
    { \CJKfamily { \CJKfamilydefault } }
    { \cs_set_eq:NN \reset@font \normalfont }
    { \ctex_patch_failure:N \normalfont }
%    \end{macrocode}
%
% \pkg{zhmCJK} 判断结束。
%    \begin{macrocode}
\fi:
%    \end{macrocode}
%
% \begin{macro}[int]{\ctex_CJK_input:n,\CJK@input}
% \pkg{breqn} 包可能会在正文中将 |^| 的 \tn{catcode} 改为 $12$ 或 $13$,这将
% 破坏 \pkg{CJK} 对汉字的首字节的定义(\tn{CJK@loadBinding} 和
% \tn{CJK@loadEncoding})。因此需要确保载入 \file{.enc} 和 \file{.bdg} 文件时,
% |^| 的 \tn{catcode} 为 $7$。
%    \begin{macrocode}
\cs_new_protected_nopar:Npn \ctex_CJK_input:n #1
  {
    \ctex_file_wrapper:nnn
      {
        \char_set_catcode_other:n            { 60 } % <
        \char_set_catcode_math_superscript:n { 94 } % ^
        \int_set:Nn \tex_endlinechar:D { -1 }
      }
      { \file_input:n {#1} }
      {
        \char_set_catcode:nn { 60 } { \char_value_catcode:n { 60 } }
        \char_set_catcode:nn { 94 } { \char_value_catcode:n { 94 } }
        \int_set:Nn \tex_endlinechar:D { \int_use:N \tex_endlinechar:D }
      }
  }
\cs_set_eq:NN \CJK@input \ctex_CJK_input:n
%    \end{macrocode}
% \end{macro}
%
% \begin{macro}[int]{\ctex_plane_to_utfxvibe:Nn,\CJK@surr}
% \pkg{fancyhdr} 宏包的 \tn{nouppercase} 会将 \tn{uppercase} 定义为 \tn{relax},而
% \tn{CJK@surr} 需要用它将 \tn{CJK@plane} 转化成大写字母,这就造成了冲突^^A
% \footnote{\url{https://github.com/CTeX-org/ctex-kit/issues/146}}。
% 我们在这里给出 \tn{CJK@surr} 的一个不依赖 \tn{uppercase} 的实现。
%    \begin{macrocode}
\if_cs_exist:N \CJK@surr
  \cs_new_protected_nopar:Npn \ctex_plane_to_utfxvibe:Nn #1#2
    {
      \tl_set:Nx \l_@@_tmp_tl {#2}
      \int_set:Nn \l_@@_tmp_int
        { \exp_args:No \int_from_hex:n { \l_@@_tmp_tl } }
      \int_compare:nNnTF \l_@@_tmp_int < { 256 }
        { \tl_gset:Nx #1 { \int_to_Hex:n { \l_@@_tmp_int } } }
        {
          \int_sub:Nn \l_@@_tmp_int { 256 }
          \tl_gset:Nx #1
            {
              \int_to_Hex:n
                { \int_div_truncate:nn { \l_@@_tmp_int } { 4 } + "D800 }
              \int_to_Hex:n
                { \int_mod:nn { \l_@@_tmp_int } { 4 } + "DC }
            }
        }
    }
  \cs_set_eq:NN \CJK@surr \ctex_plane_to_utfxvibe:Nn
\fi:
%    \end{macrocode}
% \end{macro}
%
% \pkg{CJKpunct} 宏包会在 \tn{AtBeginDocument} 的里设置标点格式为 \opt{quanjiao}。
%    \begin{macrocode}
\AtBeginDocument
  {
    \str_if_eq:eeF { \l_@@_punct_tl } { quanjiao }
      { \punctstyle { \l_@@_punct_tl } }
  }
%    \end{macrocode}
%
% 在导言区末尾更新 \tn{CJKfamilydefault},注意要在 \tn{CJK@envStart} 之前使用。
%    \begin{macrocode}
\ctex_at_end_preamble:n { \ctex_update_default_family: }
%    \end{macrocode}
%
% 启用中文字符功能。\opt{GBK} 编码时,将汉字的首字节设置为活动字符,并对这些
% 字符初始化;\opt{UTF8} 编码时,上游宏包已经处理好。
% \tn{CJK@makeActive} 应该先于 \file{ctex-name-gbk.cfg} 等文件的载入。
% 注意 \tn{CJK@loadBinding} 需要调用补丁后的 \tn{CJK@input}。使用
% \pkg{zhmCJK} 时,此功能已经被启用。
%    \begin{macrocode}
\reverse_if:N \if_bool:N \g_@@_zhmCJK_bool
  \str_if_eq:onF { \g_@@_encoding_tl } { UTF8 }
    {
      \CJK@makeActive
      \CJK@loadBinding { standard }
    }
%    \end{macrocode}
%
% 在导言区结束时调用 \tn{CJK@envStart} 启用完整的中文功能。
%
% \tn{CJK@envStart} 的定义是
% \begin{verbatim}
%   \def\CJK@envStart#1#2#3{
%     \CJK@upperReset
%     \ifCJK@lowercase@
%       \CJK@lowerReset
%     \fi%
%     \CJK@makeActive%
%     \CJK@global\let\CJK@selectFamily \CJK@selFam
%     \CJK@global\let\CJK@selectEnc \CJK@selEnc%
%     \def\CJK@@@enc{#2}
%     \ifx\CJK@@@enc \@empty
%       \PackageInfo{CJK}{
%         no encoding parameter given,\MessageBreak
%         waiting for \protect\CJKenc\space commands}
%     \else
%       \CJKenc{#2}
%     \fi
%     \CJKfontenc{#2}{#1}
%     \CJKfamily{#3}
%     \def\CJK@series{\f@series}
%     \def\CJK@shape{\f@shape}%
%     \csname CJKhook\endcsname}
% \end{verbatim}
% \tn{CJK@upperReset} 可能会有一定风险,因此我们直到导言区末尾才使用
% \tn{CJK@envStart}。这样可以避免将 \env{CJK} 环境内置入 \env{document} 环境的
% 最里层,最后也就不需要 \tn{clearpage}。\pkg{zhmCJK} 已经提供类似功能。
% 注意先使用 \cs{ctex_update_default_family:} 更新 \tn{CJKfamilydefault}。
%    \begin{macrocode}
  \exp_args:Nx \ctex_at_end_preamble:n
    {
      \exp_not:N \CJK@envStart
        { } { \g_@@_encoding_tl } { \exp_not:N \CJKfamilydefault }
      \exp_not:N \CJKtilde
    }
%    \end{macrocode}
%
% \pkg{zhmCJK} 判断结束。
%    \begin{macrocode}
\fi:
%    \end{macrocode}
%
% \begin{macro}[int]{\ctex_auto_ignorespaces:}
% 保存 \tn{CJK@@ignorespaces} 的定义,方便使用。
%    \begin{macrocode}
\cs_new_eq:NN \ctex_auto_ignorespaces: \CJK@@@@ignorespaces
%    \end{macrocode}
% \end{macro}
%
% \begin{macro}[int]{\ctex_ignorespaces_case:N,\ctex_set_ignorespaces:}
% 设置忽略空格的的方式。根据 \opt{space} 选项的值重定义 \tn{CJK@ignorespaces},
% 并保存起来供 \tn{CJKhook} 备用。
%    \begin{macrocode}
\cs_new_protected_nopar:Npn \ctex_ignorespaces_case:N #1
  {
    \cs_set_protected_nopar:Npn \ctex_set_ignorespaces:
      { \cs_set_eq:NN \CJK@ignorespaces #1 }
    \ctex_set_ignorespaces:
  }
\cs_new_protected_nopar:Npn \ctex_set_ignorespaces:
  { \cs_set_eq:NN \CJK@ignorespaces \ctex_auto_ignorespaces: }
%    \end{macrocode}
% \end{macro}
%
% \begin{macro}[int]{\CJKhook}
% \env{CJK} 和 \env{CJK*} 环境都会重新定义 \tn{CJK@ignorespaces}。我们在 \pkg{CJK}
% 宏包提供的 \tn{CJKhook} 里重新设置它,让这两个环境忽略空格的方式都受 \opt{space}
% 选项的控制。这对 \pkg{zhmCJK} 是必要的。
%    \begin{macrocode}
\ctex_gadd_hook:Nn \CJKhook { \ctex_set_ignorespaces: }
%    \end{macrocode}
% \end{macro}
%
% \begin{macro}[int]{\ctex_punct_set:n}
% 设置 CJK 族对应到实际的字体。|#1| 是 \opt{fontset} 的名字。
%    \begin{macrocode}
\cs_new_protected_nopar:Npn \ctex_punct_set:n #1
  {
    \clist_map_inline:Nn \c_@@_punct_family_clist
      {
        \cs_if_free:cF { c_@@_ #1 ##1 _punct_spaces_tl }
          {
            \cs_set_eq:cc
              { CJKpunct@ ##1 @spaces }
              { c_@@_ #1 ##1 _punct_spaces_tl }
          }
      }
  }
\clist_const:Nn \c_@@_punct_family_clist
  {
    zhsong , zhhei , zhfs , zhkai , zhli , zhyou ,
    zhsongb , zhheil , zhheib , zhyoub , zhyahei , zhyaheib
  }
%    \end{macrocode}
% \end{macro}
%
% \begin{macro}[int]{\ctex_punct_map_family:nn}
% CJK 族 |#1| 使用族 |#2| 的边界信息。
%    \begin{macrocode}
\cs_new_protected_nopar:Npn \ctex_punct_map_family:nn #1#2
  {
    \cs_if_free:cF { CJKpunct@ #2 @spaces }
      { \cs_set_eq:cc { CJKpunct@ #1 @spaces } { CJKpunct@ #2 @spaces } }
  }
%    \end{macrocode}
% \end{macro}
%
% \begin{macro}[int]{\ctex_punct_map_bfseries:nn}
% CJK 族 |#1| 的 \tn{bfseries} 使用族 |#2| 的边界信息。
%    \begin{macrocode}
\cs_new_protected_nopar:Npn \ctex_punct_map_bfseries:nn #1#2
  {
    \clist_map_inline:nn {#1}
      {
        \ctex_punct_map_series:nnn { ##1 } { b } {#2}
        \ctex_punct_map_series:nnn { ##1 } { bx } {#2}
      }
  }
\cs_new_protected_nopar:Npn \ctex_punct_map_series:nnn #1#2#3
  {
    \CJKpunctmapfamily { C19 } {#1} {#2} { m }  {#3}
    \CJKpunctmapfamily { C19 } {#1} {#2} { it } {#3}
    \CJKpunctmapfamily { C19 } {#1} {#2} { sl } {#3}
    \CJKpunctmapfamily { C70 } {#1} {#2} { m }  {#3}
    \CJKpunctmapfamily { C70 } {#1} {#2} { it } {#3}
    \CJKpunctmapfamily { C70 } {#1} {#2} { sl } {#3}
  }
%    \end{macrocode}
% \end{macro}
%
% \begin{macro}[int]{\ctex_punct_map_itshape:nn}
% CJK 族 |#1| 的 \tn{itshape} 使用族 |#2| 的边界信息。
%    \begin{macrocode}
\cs_new_protected_nopar:Npn \ctex_punct_map_itshape:nn #1#2
  {
    \CJKpunctmapfamily { C19 } {#1} { m }  { it } {#2}
    \CJKpunctmapfamily { C19 } {#1} { b }  { it } {#2}
    \CJKpunctmapfamily { C19 } {#1} { bx } { it } {#2}
    \CJKpunctmapfamily { C70 } {#1} { m }  { it } {#2}
    \CJKpunctmapfamily { C70 } {#1} { b }  { it } {#2}
    \CJKpunctmapfamily { C70 } {#1} { bx } { it } {#2}
  }
%    \end{macrocode}
% \end{macro}
%
% \begin{macro}[int]{\ctex_punct_space:nn,\ctexspadef}
% 定义标点的边界信息。
%    \begin{macrocode}
\cs_new_protected_nopar:Npn \ctex_punct_space:nn #1#2
  { \tl_const:cn { c_@@_ #1 _punct_spaces_tl } {#2} }
\cs_new_eq:NN \ctexspadef \ctex_punct_space:nn
%    \end{macrocode}
% \end{macro}
%
% 载入边界信息文件。
%     \begin{macrocode}
\ctex_file_input:n { ctexspa.def }
%    \end{macrocode}
%
%    \begin{macrocode}
%</pdftex>
%    \end{macrocode}
%
% \subsection{\pkg{ctex-engine-xetex.def}}
%
%    \begin{macrocode}
%<*xetex>
%    \end{macrocode}
%
%    \begin{macrocode}
\RequirePackage { xeCJK }
\exp_args:Nx \xeCJKsetup
  {
    LoadFandol   = false ,
    PunctStyle   = \l_@@_punct_tl
  }
%    \end{macrocode}
%
% 最新版本的 \pkg{fontspec} 默认对 \tn{rmfamily} 和 \tn{sffamily} 设置
% |Ligatures=TeX|,对 \tn{ttfamily} 设置 |WordSpace={1,0,0}| 和
% |PunctuationSpace=WordSpace|。
%    \begin{macrocode}
\@ifpackagelater { fontspec } { 2014/05/25 } { }
  { \msg_error:nnn { ctex } { package-too-old } { fontspec } }
%    \end{macrocode}
%
%    \begin{macrocode}
%</xetex>
%    \end{macrocode}
%
% \subsection{\pkg{ctex-engine-luatex.def}}
%
%    \begin{macrocode}
%<*luatex>
%    \end{macrocode}
%
% \pkg{LuaTeX-ja} 为了兼容 p\LaTeX 的使用习惯,对 \LaTeXe 的 \pkg{NFSS} 作了不少
% 修改和扩充,这对于简体中文用户来说不是必要的。我们在这里禁用它。
%    \begin{macrocode}
\msg_new:nnn { ctex } { luatexja-loaded }
  {
    Package~`luatexja'~can~not~be~loaded~before~`ctex'.\\
    Loading~file~`#1'~will~abort!
  }
\@ifpackageloaded { luatexja }
  { \msg_critical:nnx { ctex } { luatexja-loaded } { \g_file_curr_name_str } }
  { \tl_const:cn { ver@ltj-latex.\@pkgextension } { 9999/99/99 } }
%    \end{macrocode}
%
%    \begin{macrocode}
\RequirePackage { luatexja }
\@ifpackagelater { luatexja } { 2015/09/21 } { }
  { \msg_error:nnn { ctex } { package-too-old } { luatexja } }
%    \end{macrocode}
%
%    \begin{macrocode}
\RequirePackage { fontspec }
\@ifpackagelater { fontspec } { 2014/05/25 } { }
  { \msg_error:nnn { ctex } { package-too-old } { fontspec } }
%    \end{macrocode}
%
% \subsubsection{\pkg{LuaTeX-ja} 的默认设置}
%
%    \begin{macrocode}
\ExplSyntaxOff
%    \end{macrocode}
%
% 以下设置抄录自 \file{lltjdefs.sty},略有改动。
%
% U+2460--U+24FF (Enclosed Alphanumerics) 原属于字符范围 6,是 JAchar,
% 我们把它们归入字符范围 3,改成  ALchar。
%    \begin{macrocode}
\ltjdefcharrange{1}{"80-"36F, "1E00-"1EFF}
\ltjdefcharrange{2}{"370-"4FF, "1F00-"1FFF}
\ltjdefcharrange{3}{%
  "2000-"206F, "2070-"243F, "2460-"24FF, "2500-"27BF, "2900-"29FF,
  "2B00-"2BFF}
\ltjdefcharrange{4}{%
   "500-"10FF, "1200-"1DFF, "2440-"245F, "27C0-"28FF, "2A00-"2AFF,
  "2C00-"2E7F, "4DC0-"4DFF, "A4D0-"A82F, "A840-"ABFF, "FB00-"FE0F,
  "FE20-"FE2F, "FE70-"FEFF, "10000-"1FFFF, "E000-"F8FF} % non-Japanese
\ltjdefcharrange{5}{"D800-"DFFF, "E0000-"E00FF, "E01F0-"10FFFF}
\ltjdefcharrange{6}{%
  "2E80-"2EFF, "3000-"30FF, "3190-"319F, "31F0-"4DBF,
  "4E00-"9FFF, "F900-"FAFF, "FE10-"FE6F, "20000-"2FFFF, "E0100-"E01EF}
\ltjdefcharrange{7}{%
  "1100-"11FF, "2F00-"2FFF, "3100-"31EF, "A000-"A4CF,
  "A830-"A83F, "AC00-"D7FF}
\ltjdefcharrange{8}{"A7, "A8, "B0, "B1, "B4, "B6, "D7, "F7}
%    \end{macrocode}
% 将引号、破折号等中西文公用的标点符号归入字符范围 9,将他们设置为 JAchar。
%    \begin{macrocode}
\ltjdefcharrange{9}{%
  "2018, "2019, "201C, "201D, "2013, "2014, "2025, "2026, "2027, "2E3A}
%    \end{macrocode}
% \pkg{LuaTeX-ja} 默认把字符范围 2 和 3 设置为 JAchar,我们这里把它们都改成 ALchar。
%    \begin{macrocode}
\ltjsetparameter{jacharrange={-1, -2, -3, -4, -5, +6, +7, -8, +9}}
\directlua{for x=128,255 do luatexja.math.is_math_letters[x] = true end}
%    \end{macrocode}
%
% 以下设置抄录自 \file{ltj-latex.sty}。
%    \begin{macrocode}
\directlua{
  local s = kpse.find_file('ltj-kinsoku.lua', 'tex')
  luatexja.stack.charprop_stack_table[0] = s and dofile(s) or {}
}
\ltjsetparameter{kanjiskip=\z@ plus .4pt minus .5pt,
  xkanjiskip=.25\zw plus 1pt minus 1pt,
  autospacing, autoxspacing, jacharrange={-1},
  yalbaselineshift=\z@, yjabaselineshift=\z@,
  jcharwidowpenalty=500, differentjfm=paverage
}
%    \end{macrocode}
%
%    \begin{macrocode}
\ExplSyntaxOn
%    \end{macrocode}
%
% \subsubsection{\pkg{LuaTeX-ja} 的补丁}
%
%    \begin{macrocode}
%<@@=ctex_ltj>
%    \end{macrocode}
%
% 在 \LaTeX{} 下,\pkg{LuaTeX-ja} 对 \pkg{fontspec}、\pkg{xunicode}、\pkg{unicode-math}
% 和 \pkg{listings} 打了补丁。其中前三个是把 \tn{char} 换成 \tn{ltjalchar},确保
% 字符是 ALchar 类。我们这里用 \pkg{xunicode-addon} 来处理 \pkg{xunicode}。
%    \begin{macrocode}
\RequirePackage { xunicode-addon }
\AtBeginUTFCommand
  {
    \group_begin:
    \lua_now:e { tex.globaldefs = 0 }
    \ltj@allalchar
  }
\AtEndUTFCommand { \group_end: }
%    \end{macrocode}
%
% \begin{macro}[int]{\fontspec_visible_space:}
% 我们不使用 \pkg{luatexja} 对 \pkg{fontspec} 的补丁,直接处理。
%    \begin{macrocode}
\cs_set_protected:Npn \fontspec_visible_space:
  {
    \tex_iffontchar:D \tex_font:D "2423 \exp_stop_f:
      \ltjalchar "2423 \exp_stop_f:
    \else:
      \fontspec_visible_space_fallback:
    \fi:
  }
%    \end{macrocode}
% \end{macro}
%
% 对 \pkg{listings} 的补丁是让代码环境支持 JAchar 类。\pkg{LuaTeX-ja} 的补丁会将
% 代码目录标题改为日文,我们不需要。
%    \begin{macrocode}
\ctex_at_end_package:nn { listings }
  {
    \use:x
      {
        \exp_not:N \RequirePackage { lltjp-listings }
        \tl_set:Nn \exp_not:N \lstlistingname
          { \exp_not:o { \lstlistingname } }
        \tl_set:Nn \exp_not:N \lstlistlistingname
          { \exp_not:o { \lstlistlistingname } }
      }
  }
%    \end{macrocode}
%
% \subsubsection{字体切换方式}
%
% \begin{macro}[int]{\ctex_ltj_select_font:,\CJK@family}
% \tn{CJK@family} 保存的是当前 CJK 实际的字体族名,如果为空表示没有设置过字体。
%    \begin{macrocode}
\cs_new_protected_nopar:Npn \ctex_ltj_select_font:
  {
    \cs_if_exist_use:cF { \l_@@_current_font_tl }
      { \tl_if_empty:NF \CJK@family { \@@_select_font_aux: } }
  }
\tl_new:N \CJK@family
\tl_new:N \l_@@_current_font_tl
\tl_set:Nn \l_@@_current_font_tl
  { \CJK@encoding / \CJK@family / \f@series / \f@shape / \f@size }
%    \end{macrocode}
% \end{macro}
%
% \begin{macro}{\@@_select_font_aux:}
% 使用 \tn{pickup@font} 取得字体名称前,总需要先设置 \tn{font@name}。在这里将
% \tn{f@family} 换成 CJK 字体族,并确保编码正确。
%    \begin{macrocode}
\cs_new_protected_nopar:Npn \@@_select_font_aux:
  {
    \group_begin:
      \tl_set_eq:NN \f@encoding \CJK@encoding
      \tl_set_eq:NN \f@family \CJK@family
      \@@_push_fontname:n { \use:c { \curr@fontshape / \f@size } }
      \ctex_ltj_pickup_font:
    \group_end:
    \font@name
    \@@_pop_fontname:
%    \end{macrocode}
% 当字形未定义的时候,\textsf{NFSS} 就会启动替换机制(\tn{wrong@fontshape})。
% 第一次启动后,\cs{l_@@_current_font_tl} 还是没有定义。为此,我们再次选择字体,
% 确保它有定义和指向正确的 \texttt{font.id}。这对 \opt{AlternateFont} 的设置
% 特别重要。
%    \begin{macrocode}
    \cs_if_exist:cF { \l_@@_current_font_tl }
      { \@@_select_font_aux: }
  }
\cs_new_protected_nopar:Npn \@@_push_fontname:n #1
  {
    \cs_gset_eq:NN \@@_save_fontname:w \font@name
    \cs_gset_nopar:Npx \font@name {#1}
  }
\cs_new_protected_nopar:Npn \@@_pop_fontname:
  { \cs_gset_eq:NN \font@name \@@_save_fontname:w }
%    \end{macrocode}
% \end{macro}
%
% \begin{macro}[int]{\ctex_ltj_pickup_font:}
% 替换 \tn{define@newfont} 内部调用的 \tn{extract@font} 和 \tn{do@subst@correction}。
%    \begin{macrocode}
\cs_new_protected_nopar:Npn \ctex_ltj_pickup_font:
  {
    \exp_after:wN \cs_if_exist:NF \font@name
      {
        \group_begin:
          \cs_set_eq:NN \extract@font \ctex_ltj_extract_font:
          \cs_set_eq:NN \do@subst@correction \ctex_ltj_subst_font:
          \define@newfont
        \group_end:
      }
  }
\cs_new_eq:NN \pickup@jfont \ctex_ltj_pickup_font:
%    \end{macrocode}
% \end{macro}
%
% \begin{macro}[int]{\ctex_ltj_extract_font:}
% \pkg{LuaTeX-ja} 的 \tn{globaljfont} 在 \pkg{luatexja-core} 中定义:
% \begin{verbatim}
%   %%%%%%%% \jfont\CS={...:...;jfm=metric;...}, \globaljfont
%   \protected\def\jfont#1{%
%     \afterassignment\ltj@@jfont
%     \directlua{luatexja.jfont.jfontdefX(false, 'yoko','\luatexluaescapestring{\noexpand#1}')}}
%   \protected\def\globaljfont#1{%
%     \afterassignment\ltj@@jfont
%     \directlua{luatexja.jfont.jfontdefX(true,  'yoko','\luatexluaescapestring{\noexpand#1}')}}
%   \def\ltj@@jfont{\directlua{luatexja.jfont.jfontdefY()}}
% \end{verbatim}
% \texttt{jfontdefX} 函数的作用是把 \tn{CS} 定义为其后的字体,\texttt{jfontdefY}
% 的作用是更新 \texttt{JFM} 和记录相关字体信息。最后的工作是:
% \begin{verbatim}
%   tex.sprint(cat_lp, global_flag, '\\protected\\expandafter\\def\\csname ',
%     (cstemp==' ') and '\\space' or cstemp, '\\endcsname{\\ltj@cur'..
%     (jfm_dir == 'yoko' and 'j' or 't') .. 'fnt', fn, '\\relax}')
% \end{verbatim}
% \tn{CS} 的作用就是把 \tn{ltj@curjfnt} 设置为刚才定义的字体的 \texttt{font.id}。
%    \begin{macrocode}
\cs_new_protected_nopar:Npn \ctex_ltj_extract_font:
  {
    \get@external@font
    \ctex_ltj_if_alternate_shape_exist:nT { \curr@fontshape }
      {
        \tl_set:Nx \external@font
          { \exp_after:wN \@@_patch_external_font:w \external@font }
      }
    \exp_after:wN \globaljfont \font@name \external@font \scan_stop:
%    \end{macrocode}
% 这里 \tn{font@name} 不会直接改变当前字体,而 \tn{DeclareFontFamily} 和
% \tn{DeclareFontShape} 的最后一个参数通常要使用 \tn{font} 来引用当前字体。
% 为此,我们在分组内启用之前定义的字体,以便能得到正确的 \tn{font}。对字体参数的
% 赋值总是全局的,不会受到分组的影响。
%    \begin{macrocode}
    \font@name
    \lua_now:e { font.current(tex.getattribute('ltj@curjfnt')) }
    \use:c { \f@encoding + \f@family }
    \use:c { \curr@fontshape }
  }
%    \end{macrocode}
% \end{macro}
%
% \begin{macro}[int]{\ctex_ltj_subst_font:}
% \tn{do@subst@correction} 在设置通过 \texttt{sub} 或者 \texttt{ssub} 函数定义的
% 字体时会用到。如果没有设置 \opt{SlantedFont},\pkg{fontspec} 会设置
% \tn{itdefault} 作为 \tn{sldefault} 的替代字形,因而会用到这个函数。它的本来定义是:
% \begin{verbatim}
%   \def\do@subst@correction{%
%       \xdef\subst@correction{%
%          \font@name
%          \global\expandafter\font
%            \csname \curr@fontshape/\f@size\endcsname
%            \noexpand\fontname\font
%           \relax}%
%       \aftergroup\subst@correction
%   }
% \end{verbatim}
% 我们在这里不需要定义新字体,而是设置对应字体的命令。
%    \begin{macrocode}
\cs_new_protected_nopar:Npn \ctex_ltj_subst_font:
  {
    \ctex_ltj_if_alternate_shape_exist:nF { \curr@fontshape }
      {
        \group_begin:
        \tl_set_eq:NN \CJK@family \f@family
        \cs_if_exist:cF { \l_@@_current_font_tl  }
          {
            \cs_gset_protected_nopar:Npx \subst@correction
              {
                \cs_new_eq:NN
                  \exp_not:c { \l_@@_current_font_tl }
                  \font@name
              }
            \group_insert_after:N \group_insert_after:N
            \group_insert_after:N \subst@correction
          }
        \group_end:
      }
  }
%    \end{macrocode}
% \end{macro}
%
% \begin{macro}[int,TF]{\ctex_ltj_if_alternate_shape_exist:n}
% 即 \pkg{LuaTeX-ja} 中的 \tn{ltj@@does@alt@set},判断是否存在替代字体。
%    \begin{macrocode}
\prg_new_conditional:Npnn \ctex_ltj_if_alternate_shape_exist:n #1 { T , F , TF }
  {
    \lua_now:e { luatexja.jfont.does_alt_set ('\lua_escape:e {#1}') }
      \prg_return_true: \else: \prg_return_false: \fi:
  }
%    \end{macrocode}
% \end{macro}
%
% \begin{macro}{\@@_patch_external_font:w}
% 若对字体的定义完全相同,则它们有相同的 \texttt{font.id}。因此如果字形是由
% \textsf{NFSS} 的替换机制定义的,它们就有相同的 \texttt{font.id}。
% |print_aftl_address| 函数的定义是
% \begin{verbatim}
%   function print_aftl_address()
%     tex.sprint(cat_lp, ';ltjaltfont' .. tostring(aftl_base):sub(8))
%   end
% \end{verbatim}
% 主要目的是,如果当前字形有替代字体,则往字形的定义中加入一些标志,确保
% \texttt{font.id} 唯一。
%    \begin{macrocode}
\cs_new_nopar:Npn \@@_patch_external_font:w #1 ~ at
  { #1 \lua_now:e { luatexja.jfont.print_aftl_address() } ~ at }
%    \end{macrocode}
% \end{macro}
%
% \begin{macro}[int]{\ctex_ltj_select_alternate_font:}
% 在 \tn{selectfont} 中更新替代字体。
%    \begin{macrocode}
\cs_new_protected_nopar:Npn \ctex_ltj_select_alternate_font:
  {
    \ctex_ltj_if_alternate_shape_exist:nT { \l_@@_current_shape_tl }
      {
        \lua_now:e
          {
            luatexja.jfont.output_alt_font_cmd
              ('y', '\lua_escape:e { \l_@@_current_shape_tl }')
          }
        \lua_now:e { luatexja.jfont.pickup_alt_font_a ('\f@size') }
      }
  }
\tl_new:N \l_@@_current_shape_tl
\tl_set:Nn \l_@@_current_shape_tl
  { \CJK@encoding / \CJK@family / \f@series / \f@shape }
%    \end{macrocode}
% \end{macro}
%
% \begin{macro}[int]{\ltj@pickup@altfont@auxy}
% 被用在函数 |output_alt_font_cmd| 中,作用是定义替代字体。
%    \begin{macrocode}
\cs_new_protected_nopar:Npn \ltj@pickup@altfont@auxy #1
  {
    \cs_if_exist:cF { #1/\f@size }
      {
        \group_begin:
          \use:x { \exp_not:N \split@name #1 / \f@size } \@nil
          \@@_push_fontname:n { \use:c { \curr@fontshape / \f@size } }
          \ctex_ltj_pickup_font:
        \group_end:
        \@@_pop_fontname:
      }
  }
%    \end{macrocode}
% \end{macro}
%
% \begin{macro}[int]{\ltj@pickup@altfont@copy}
% 被用在函数 |pickup_alt_font_a| 中。\tn{ltj@@getjfontnumber} 的作用是将字体命令
% |#1| 对应的 \texttt{font.id} 保存到 \tn{ltj@tempcntc} 中。
%    \begin{macrocode}
\cs_new_protected_nopar:Npn \ltj@pickup@altfont@copy #1#2
  {
    \ltj@@@@getjfontnumber #1
    \lua_now:e
      {
        luatexja.jfont.pickup_alt_font_b
          ( \int_use:N \ltj@tempcntc, '\lua_escape:e {#2}' )
      }
  }
%    \end{macrocode}
% \end{macro}
%
% \subsubsection{数学字体族}
%
% 以下内容来自 \file{lltjfont.sty},目的是让汉字可以在数学环境中直接使用。
%
% \begin{macro}[int]{\ctex_ltj_if_jfont:nTF}
% 参数 |#1| 是一个 \LaTeXe{} 编码名称或者字体命令。\LaTeXe{} 字体命令的一般形式是:
% \begin{quote}\ttfamily\small
%   \textbackslash\meta{encoding}/\meta{family}/\meta{series}/\meta{shape}
% \end{quote}
% 通过截取名字中的 \meta{encoding} 来判断是否是 jfont。
% 最后会设置 \tn{ifin@} 为对应的 \tn{iftrue} 或者 \tn{iffalse}。
%    \begin{macrocode}
\cs_new:Npn \ctex_ltj_if_jfont:nTF #1
  {
    \lua_now:e
      { luatexja.jfont.is_kenc( string.match('\lua_escape:e {#1}', '[^/]+') ) }
    \ifin@ \exp_after:wN \use_i:nn \else: \exp_after:wN \use_ii:nn \fi:
  }
%    \end{macrocode}
% \end{macro}
%
% \begin{macro}[int]{\ctex_ltj_if_jfont_math:NTF}
% |#1| 是一个形式为 |\M@|\meta{encoding} 的命令,它由 \tn{DeclareFontEncoding} 的
% 第三个参数来定义。
%    \begin{macrocode}
\cs_new:Npn \ctex_ltj_if_jfont_math:NTF #1
  { \exp_after:wN \@@_if_jfont_math:w \token_to_str:N #1 \q_stop }
\group_begin:
  \char_set_catcode_other:N M
  \cs_new:Npn \@@_if_jfont_math:w #1 M #2#3 \q_stop
    { \ctex_ltj_if_jfont:nTF {#3} }
\group_end:
%    \end{macrocode}
% \end{macro}
%
% \begin{macro}[int]{\getanddefine@fonts,\ctex_ltj_get_and_define_fonts:nN}
% 在使用的场合,\tn{escapechar} 已经被设置成 $-1$,使用 \cs{token_to_str:N} 就
% 可以得到名字,不必使用 \cs{cs_to_str:N}。
%    \begin{macrocode}
\cs_new_protected_nopar:Npn \ctex_ltj_get_and_define_fonts:nN #1#2
  {
    \ctex_ltj_if_jfont:nTF { \token_to_str:N #2 }
      { \ctex_ltj_get_and_define_fonts_ja:nN }
      { \ctex_ltj_get_and_define_fonts_al:nN }
      {#1} #2
  }
\cs_new_eq:NN \ctex_ltj_get_and_define_fonts_al:nN \getanddefine@fonts
\cs_set_eq:NN \getanddefine@fonts \ctex_ltj_get_and_define_fonts:nN
\cs_new_protected_nopar:Npn \ctex_ltj_get_and_define_fonts_ja:nN #1#2
  {
    \tl_gset:Nx \font@name { \use:c { \token_to_str:N #2 / \tf@size } }
    \ctex_ltj_pickup_font: \tl_set_eq:NN \textfont@name \font@name
    \tl_gset:Nx \font@name { \use:c { \token_to_str:N #2 / \sf@size } }
    \ctex_ltj_pickup_font: \tl_set_eq:NN \scriptfont@name \font@name
    \tl_gset:Nx \font@name { \use:c { \token_to_str:N #2 / \ssf@size } }
    \ctex_ltj_pickup_font:
    \tl_put_right:Nx \math@fonts
      {
        \ltj@setpar@global
        \ltj@@@@set@stackfont #1 , \textfont@name   \c_colon_str { MJT }
        \ltj@@@@set@stackfont #1 , \scriptfont@name \c_colon_str { MJS }
        \ltj@@@@set@stackfont #1 , \font@name       \c_colon_str { MJSS }
      }
  }
%    \end{macrocode}
% \end{macro}
%
% \begin{macro}[int]{\use@mathgroup,\ctex_ltj_use_math_group:Nn}
% 在使用 \pkg{unicode-math} 宏包时,\cs{ctex_ltj_math_group_hook:} 将被重定义。
%    \begin{macrocode}
\cs_new_protected_nopar:Npn \ctex_ltj_use_math_group:Nn #1#2
  {
    \mode_if_math:T
      {
        \math@bgroup
          \cs_if_eq:cNF { M@ \f@encoding } #1 {#1}
          \ctex_ltj_math_group_hook:
          \ctex_ltj_if_jfont_math:NTF #1
            { \jfam } { \mathgroup } #2 \scan_stop:
        \math@egroup
      }
  }
\cs_new_eq:NN \ctex_ltj_math_group_hook: \prg_do_nothing:
\cs_set_eq:NN \use@mathgroup \ctex_ltj_use_math_group:Nn
%    \end{macrocode}
% \end{macro}
%
% 对 \pkg{unicode-math} 的补丁主要是将 \file{unicode-math-table.tex} 中的数学符号设置为
% \pkg{luatexja} 中的数学字母。
% 本段代码应放在 \cs{ctex_ltj_math_group_hook:} 的定义之后,避免因宏包载入顺序而造成的编译错误。
%    \begin{macrocode}
\cs_new_protected:Npn \ctex_ltj_set_math_letter:NN #1#2
  {
    \group_begin:
      \cs_set_protected:Npn #1 ##1##2##3
        { \ltjsetmathletter { ##1 } }
      #2
    \group_end:
  }
\ctex_at_end_package:nn { unicode-math }
  {
    \cs_if_exist:NTF \um_input_math_symbol_table:
      {
        \ctex_ltj_set_math_letter:NN
          \um_sym:nnn
          \um_input_math_symbol_table:
      }
      {
        \cs_set_eq:NN \use@mathgroup \ctex_ltj_use_math_group:Nn
        \cs_set_protected_nopar:Npn \ctex_ltj_math_group_hook:
          { \__um_switchto_literal: }
        \ctex_ltj_set_math_letter:NN
          \__um_sym:nnn
          \__um_input_math_symbol_table:
      }
  }
%    \end{macrocode}
%
% \subsubsection{字体族的定义与使用}
%
% \begin{macro}[int]{\ctex_mono_jfm:n}
% \begin{variable}{\l_@@_jfm_tl}
% \pkg{LuaTeX-ja} 中与标点格式 \opt{plain} 对应的 \texttt{JFM} 是 \opt{mono}。
%    \begin{macrocode}
\cs_new_protected_nopar:Npn \ctex_mono_jfm:n #1
  {
    \str_if_eq:nnTF {#1} { plain }
      { \tl_set:Nn \l_@@_jfm_tl { mono } }
      { \tl_set:Nn \l_@@_jfm_tl {#1} }
  }
\tl_new:N \l_@@_jfm_tl
\cs_generate_variant:Nn \ctex_mono_jfm:n { o }
\ctex_mono_jfm:o { \l__ctex_punct_tl }
%    \end{macrocode}
% \end{variable}
% \end{macro}
%
% \begin{macro}[int]{\CJK@encoding}
% \begin{macro}{\@@_change_encoding:}
% 在 \LaTeX 下,\pkg{LuaTeX-ja} 依赖字体编码来实现特殊设置。例如上述的
% \cs{ctex_ltj_if_jfont:nTF} 就是通过判断编码来实现的,它在设置数学字体时会用到。所以
% 不应该与西文共用 \texttt{EU2}。定义字体族 song 为 \tn{CJK@encoding} 的默认替换
% 字体。下划线 |_| 不在 \tn{nfss@catcodes} 里,可以放心使用。
%    \begin{macrocode}
\tl_const:Nn \CJK@encoding { LTJY3 }
\DeclareFontEncoding { \CJK@encoding } { } { }
\use:x
  {
    \exp_not:N \DeclareFontSubstitution
      { \CJK@encoding } { song } { \mddefault } { \updefault }
  }
\lua_now:e { luatexja.jfont.add_kyenc_list('\CJK@encoding') }
\cs_new_protected_nopar:Npn \@@_change_encoding:
  { \tl_set_eq:NN \g_fontspec_encoding_tl \CJK@encoding }
\DeclareFontFamily { \CJK@encoding } { song } { }
\DeclareFontShape { \CJK@encoding } { song } { \mddefault } { \updefault }
  { <-> psft:SimSun:cid=Adobe-GB1-5;jfm=\l_@@_jfm_tl } { }
\DeclareFontShape { \CJK@encoding } { song } { \bfdefault } { \updefault }
  { <-> psft:SimHei:cid=Adobe-GB1-5;jfm=\l_@@_jfm_tl } { }
\tl_const:Nn \c_@@_math_tl { CJKmath }
\DeclareSymbolFont { \c_@@_math_tl }
  { \CJK@encoding } { song } { \mddefault } { \updefault }
\SetSymbolFont { \c_@@_math_tl } { bold }
  { \CJK@encoding } { song } { \bfdefault } { \updefault }
\int_const:Nn \c_@@_math_fam_int { \use:c { sym \c_@@_math_tl } }
\jfam \c_@@_math_fam_int
%    \end{macrocode}
% \end{macro}
% \end{macro}
%
% 这是 \pkg{luatexja-fontspec} 中新增的一些字体选项。
%    \begin{macrocode}
\newfontfeature { CID }     {    cid = #1 }
\newfontfeature { JFM }     {    jfm = #1 }
\newfontfeature { JFM-var } { jfmvar = #1 }
%    \end{macrocode}
%
% 在新版本的 \pkg{fontspec} 中,\cs{__fontspec_fontname_wrap:n} 变成了私有函数。
%    \begin{macrocode}
\keys_define:nn { fontspec-preparse-external }
  {
    NoEmbed .code:n =
      { \cs_set_eq:NN \__fontspec_fontname_wrap:n \@@_noembed_wrap:n }
  }
\cs_new:Npn \@@_noembed_wrap:n #1 { psft: #1 }
%    \end{macrocode}
%
% \begin{macro}[int]{\ctex_ltj_set_family:nnn}
% 将自定义的字体族名与 \pkg{fontspec} 实际设置的名字对应起来。
%    \begin{macrocode}
\cs_new_protected:Npn \ctex_ltj_set_family:nnn #1#2#3
  {
    \group_begin:
    \clist_clear:N \l_@@_char_range_clist
    \seq_clear:N \l_@@_alternate_seq
    \tl_set:Nn \l_@@_base_CJKfamily_tl {#1}
    \keys_set_known:nnN { ctex_ltj / fontspec } {#2} \l_@@_tmp_tl
    \clist_set:No \l_@@_font_options_clist { \l_@@_tmp_tl }
    \ctex_ltj_set_alternate_family:nnF {#1} {#3}
      {
        \prop_gput:Nnn \g_@@_family_font_name_prop {#1} {#3}
        \prop_gput:Nno \g_@@_family_font_options_prop
          {#1} { \l_@@_font_options_clist }
        \@@_update_family_uid:N \l_@@_font_options_clist
        \@@_use_global_options:N \l_@@_font_options_clist
        \@@_gset_family_cs:nn {#1} {#3}
      }
    \group_end:
  }
\tl_new:N \l_@@_base_CJKfamily_tl
\clist_new:N \l_@@_font_options_clist
\cs_new_protected_nopar:Npn \@@_use_global_options:N #1
  {
    \clist_concat:NNN #1 \g_@@_default_features_clist #1
    \clist_put_left:Nx #1 { JFM = \l_@@_jfm_tl }
  }
%    \end{macrocode}
% \end{macro}
%
% \begin{variable}
%  {\g_@@_family_name_prop,\g_@@_family_font_name_prop,\g_@@_family_font_options_prop}
% 分别保存 \pkg{fontspec} 设置的字体族名、字体名称和字体选项。
%    \begin{macrocode}
\prop_new:N \g_@@_family_name_prop
\prop_new:N \g_@@_family_font_name_prop
\prop_new:N \g_@@_family_font_options_prop
%    \end{macrocode}
% \end{variable}
%
% \begin{macro}{\@@_check_family:n}
% 删除重复的定义,清除替代字体的先前设置。
%   \begin{macrocode}
\cs_new_protected_nopar:Npn \@@_check_family:n #1
  {
    \prop_gpop:NnNT \g_@@_family_font_name_prop {#1} \l_@@_tmp_tl
      {
        \cs_undefine:c { \@@_family_csname:n {#1} }
        \cs_undefine:c { \@@_alternate_cs:n {#1} }
        \prop_gpop:NnNT \g_@@_family_name_prop {#1} \l_@@_base_family_tl
          {
            \use:c { \@@_alternate_cs:n { clear / #1 } }
            \cs_undefine:c { \@@_alternate_cs:n { clear / #1 } }
            \cs_undefine:c { \@@_alternate_cs:n { reset / #1 } }
            \prop_gremove:Nn \g_@@_reset_alternate_prop {#1}
          }
        \msg_warning:nnxx { ctex } { redefine-family } {#1} { \l_@@_tmp_tl }
      }
  }
\tl_new:N \l_@@_tmp_tl
\msg_new:nnn { ctex } { redefine-family }
  { Redefining~CJKfamily~`\@@_msg_family_map:n {#1}'~(#2). }
%    \end{macrocode}
% \end{macro}
%
% \begin{macro}{\@@_gset_family_cs:nn}
% 在设置字体时,实际上并不是马上就定义。而是只保存相关参数,在通过 \tn{CJKfamily}
% 第一次使用时才定义。需要注意将编码改为 \tn{CJK@encoding}。
%    \begin{macrocode}
\cs_new_protected_nopar:Npn \@@_gset_family_cs:nn #1#2
  {
    \cs_gset_protected_nopar:cpx { \@@_family_csname:n {#1} }
      {
        \group_begin:
        \@@_change_encoding:
        \exp_not:n { \cs_set_eq:NN \CJKfamily \use_none:n }
        \exp_not:n { \cs_set_eq:NN \CTEX@fontfamily \use_none:n }
        \exp_not:n { \fontspec_set_family:Nnn \g_@@_fontspec_family_tl }
          { \exp_not:o { \l_@@_font_options_clist } } {#2}
        \prop_gput:Nno \exp_not:N \g_@@_family_name_prop {#1}
          { \exp_not:N \g_@@_fontspec_family_tl }
        \tl_gset_eq:NN \exp_not:N \g_@@_fontspec_family_tl
          \exp_not:N \g_@@_fontspec_family_tl
        \@@_set_alternate_family:n {#1}
        \group_end:
      }
  }
\tl_new:N \l_@@_base_family_tl
\tl_new:N \g_@@_fontspec_family_tl
\cs_new_nopar:Npn \@@_family_csname:n #1 { ctex_ltj/family/#1 }
\cs_new_protected_nopar:Npn \@@_set_alternate_family:n #1
  {
    \tl_set:Nn \l_@@_base_CJKfamily_tl {#1}
    \tl_set_eq:NN \l_@@_base_family_tl \g_@@_fontspec_family_tl
    \cs_if_exist_use:c { \@@_alternate_cs:n { reset / #1 } }
    \cs_if_exist_use:c { \@@_alternate_cs:n {#1} }
  }
\cs_new:Npn \@@_alternate_cs:n #1 { ctex_ltj/alternate_family/#1 }
%    \end{macrocode}
% \end{macro}
%
% \begin{macro}[int]{\CJKfamily}
% 切换字体。
%    \begin{macrocode}
\NewDocumentCommand \CJKfamily { m }
  { \ctex_ltj_switch_family:x {#1} \tex_ignorespaces:D }
\cs_new_protected_nopar:Npn \ctex_ltj_switch_family:n #1
  {
    \ctex_ltj_family_if_exist:nNTF {#1} \CJK@family
      {
        \tl_set:Nn \l_ctex_ltj_family_tl {#1}
        \selectfont
      }
      { \@@_family_unknown_warning:n {#1} }
  }
\tl_new:N \l_ctex_ltj_family_tl
\cs_generate_variant:Nn \ctex_ltj_switch_family:n { x }
%    \end{macrocode}
% \end{macro}
%
% \begin{macro}[int,TF]{\ctex_ltj_family_if_exist:nN}
% 判断 CJK 字体族 |#1| 是否存在,若存在则把实际族名保存到 |#2| 中。
%    \begin{macrocode}
\prg_new_protected_conditional:Npnn \ctex_ltj_family_if_exist:nN #1#2 { T , F , TF }
  {
    \prop_get:NnNTF \g_@@_family_name_prop {#1} #2
      { \prg_return_true: }
      {
        \cs_if_exist_use:cTF { \@@_family_csname:n {#1} }
          {
            \tl_set_eq:NN #2 \g_@@_fontspec_family_tl
            \prg_return_true:
          }
          { \prg_return_false: }
      }
  }
\prg_generate_conditional_variant:Nnn \ctex_ltj_family_if_exist:nN { x } { T , F , TF }
%    \end{macrocode}
% \end{macro}
%
% \begin{macro}{\@@_family_unknown_warning:n}
%    \begin{macrocode}
\cs_new_protected_nopar:Npn \@@_family_unknown_warning:n #1
  {
    \prop_if_empty:NF \g_@@_family_font_name_prop
      {
        \seq_if_in:NnF \g_@@_unknown_family_seq {#1}
          {
            \seq_gput_right:Nn \g_@@_unknown_family_seq {#1}
            \msg_warning:nnn { ctex } { family-unknown } {#1}
          }
      }
  }
\seq_new:N \g_@@_unknown_family_seq
\msg_new:nnn { ctex } { family-unknown }
  {
    Unknown~CJK~family~`\@@_msg_family_map:n {#1}'~is~being~ignored.\\
    Try~to~use~`\@@_msg_def_family_map:n {#1}'~to~define~it.
  }
\cs_new_nopar:Npn \@@_msg_def_family_map:n #1
  {
    \str_case_e:nnF {#1}
      {
        \CJKrmdefault { \token_to_str:N \setCJKmainfont }
        \CJKsfdefault { \token_to_str:N \setCJKsansfont }
        \CJKttdefault { \token_to_str:N \setCJKmonofont }
      }
      { \token_to_str:N \setCJKfamilyfont \{ #1 \} }
    [...]\{...\}
  }
\cs_new_nopar:Npn \@@_msg_family_map:n #1
  {
    \str_case_e:nnF {#1}
      {
        \CJKrmdefault { \token_to_str:N \CJKrmdefault }
        \CJKsfdefault { \token_to_str:N \CJKsfdefault }
        \CJKttdefault { \token_to_str:N \CJKttdefault }
      }
      {#1}
  }
%    \end{macrocode}
% \end{macro}
%
% \begin{macro}[int]{\ctex_ltj_fontspec:nn}
%    \begin{macrocode}
\cs_new_protected_nopar:Npn \ctex_ltj_fontspec:nn #1#2
  {
    \prop_get:NnNTF \g_@@_fontspec_prop
      { CJKfontspec/#1/#2/id } \l_ctex_ltj_family_tl
      { \ctex_ltj_switch_family:x { \l_ctex_ltj_family_tl } }
      {
        \int_gincr:N \g_@@_family_int
        \@@_fontspec:xnn
          { CJKfontspec ( \int_use:N \g_@@_family_int ) }
          {#1} {#2}
      }
  }
\cs_new_protected_nopar:Npn \ctex_ltj_fontspec:xx #1#2
  { \use:x { \ctex_ltj_fontspec:nn {#1} {#2} } }
\cs_new_protected_nopar:Npn \@@_fontspec:nnn #1#2#3
  {
    \bool_if:NT \l_@@_add_alternate_bool
      {
        \cs_if_free:cF
          { \@@_alternate_cs:n { reset / \l_ctex_ltj_family_tl } }
          {
            \cs_gset_eq:cc
              { \@@_alternate_cs:n { reset / #1 } }
              { \@@_alternate_cs:n { reset / \l_ctex_ltj_family_tl } }
            \cs_gset_eq:cc
              { \@@_alternate_cs:n { clear / #1 } }
              { \@@_alternate_cs:n { clear / \l_ctex_ltj_family_tl } }
          }
        \bool_set_false:N \l_@@_add_alternate_bool
      }
    \prop_gput:Nnn \g_@@_fontspec_prop { CJKfontspec/#2/#3/id } {#1}
    \ctex_ltj_set_family:nnn {#1} {#2} {#3}
    \ctex_ltj_switch_family:n {#1}
  }
\cs_generate_variant:Nn \@@_fontspec:nnn { x }
\prop_new:N \g_@@_fontspec_prop
%    \end{macrocode}
% \end{macro}
%
% \begin{macro}[int]
% {\ctex_ltj_add_font_features:n,\ctex_ltj_add_font_features:nn}
%    \begin{macrocode}
\cs_new_protected_nopar:Npn \ctex_ltj_add_font_features:n #1
  { \ctex_ltj_add_font_features:xn { \l_ctex_ltj_family_tl } {#1} }
\cs_new_protected_nopar:Npn \ctex_ltj_add_font_features:nn #1#2
  {
    \prop_get:NnNTF \g_@@_family_font_name_prop
      {#1} \l_@@_tmp_tl
      {
        \prop_get:NnN \g_@@_family_font_options_prop
          {#1} \l_@@_font_options_clist
        \clist_put_right:Nn \l_@@_font_options_clist {#2}
        \bool_set_true:N \l_@@_add_alternate_bool
        \ctex_ltj_fontspec:xx
          { \exp_not:o { \l_@@_font_options_clist } }
          { \exp_not:o { \l_@@_tmp_tl } }
      }
      { \msg_warning:nn { ctex } { addCJKfontfeature-ignored } }
  }
\bool_new:N \l_@@_add_alternate_bool
\cs_generate_variant:Nn \ctex_ltj_add_font_features:n  { x }
\cs_generate_variant:Nn \ctex_ltj_add_font_features:nn { x }
\msg_new:nnn { ctex } { addCJKfontfeature-ignored }
  {
    \token_to_str:N \addCJKfontfeature (s)~ignored.\\
    It~cannot~be~used~with~a~font~that~wasn't~selected~by~ctex.
  }
%    \end{macrocode}
% \end{macro}
%
% \begin{macro}{\@@_pass_args:nnnn}
% 为了支持字体属性可选项在前在后两种语法,给出两个辅助工具,自带展开功能。
%    \begin{macrocode}
\cs_new_protected:Npn \@@_pass_args:nnnn #1#2#3#4
  {
    \IfNoValueTF {#2}
      { \@@_post_arg:w {#1} {#3} {#4} }
      {
        \use:x { #1 {#2} {#3} }
        #4
      }
  }
\NewDocumentCommand \@@_post_arg:w { m m m O { } }
  {
    \use:x { #1 {#4} {#2} }
    #3
  }
%    \end{macrocode}
% \end{macro}
%
% \begin{macro}[int]
% {\setCJKfamilyfont,\newCJKfontfamily,\CJKfontspec,\addCJKfontfeatures}
%    \begin{macrocode}
\NewDocumentCommand \setCJKfamilyfont { m o m }
  {
    \@@_pass_args:nnnn
      { \ctex_ltj_set_family:nnn {#1} } {#2} {#3}
      { }
  }
\NewDocumentCommand \newCJKfontfamily { o m o m }
  {
    \tl_set:Nx \l_@@_tmp_tl
      { \IfNoValueTF {#1} { \cs_to_str:N #2 } {#1} }
    \cs_new_protected_nopar:Npx #2
      { \ctex_ltj_switch_family:n { \l_@@_tmp_tl } }
    \@@_pass_args:nnnn
      { \ctex_ltj_set_family:nnn { \l_@@_tmp_tl } } {#3} {#4}
      { }
  }
\NewDocumentCommand \CJKfontspec { o m }
  {
    \@@_pass_args:nnnn
      { \ctex_ltj_fontspec:nn } {#1} {#2}
      { \tex_ignorespaces:D }
  }
\NewDocumentCommand \addCJKfontfeatures { m }
  {
    \ctex_ltj_add_font_features:x {#1}
    \tex_ignorespaces:D
  }
\cs_new_eq:NN \addCJKfontfeature \addCJKfontfeatures
%    \end{macrocode}
% \end{macro}
%
% \begin{macro}[int]
% {\setCJKmainfont,\setCJKsansfont,\setCJKmonofont,
%  \setCJKmathfont,\defaultCJKfontfeatures}
%    \begin{macrocode}
\NewDocumentCommand \setCJKmainfont { o m }
  {
    \@@_pass_args:nnnn
      { \ctex_ltj_set_family:nnn { \CJKrmdefault } } {#1} {#2}
      { \normalfont }
  }
\cs_new_eq:NN \setCJKromanfont \setCJKmainfont
\NewDocumentCommand \setCJKsansfont { o m }
  {
    \@@_pass_args:nnnn
      { \ctex_ltj_set_family:nnn { \CJKsfdefault } } {#1} {#2}
      { \normalfont }
  }
\NewDocumentCommand \setCJKmonofont { o m }
  {
    \@@_pass_args:nnnn
      { \ctex_ltj_set_family:nnn { \CJKttdefault } } {#1} {#2}
      { \normalfont }
  }
\NewDocumentCommand \setCJKmathfont { o m }
  {
    \@@_pass_args:nnnn
      { \ctex_ltj_set_family:nnn { \c_@@_math_tl } } {#1} {#2}
      { }
  }
\NewDocumentCommand \defaultCJKfontfeatures { m }
  { \clist_gset:Nn \g_@@_default_features_clist {#1} }
\clist_new:N \g_@@_default_features_clist
\@onlypreamble \setCJKmainfont
\@onlypreamble \setCJKsansfont
\@onlypreamble \setCJKmonofont
\@onlypreamble \setCJKmathfont
\@onlypreamble \setCJKromanfont
\@onlypreamble \defaultCJKfontfeatures
%    \end{macrocode}
% \end{macro}
%
%    \begin{macrocode}
\tl_if_exist:NF \CJKfamilydefault
  { \tl_const:Nn \CJKfamilydefault { \CJKrmdefault } }
\tl_if_exist:NF \CJKrmdefault { \tl_const:Nn \CJKrmdefault { rm } }
\tl_if_exist:NF \CJKsfdefault { \tl_const:Nn \CJKsfdefault { sf } }
\tl_if_exist:NF \CJKttdefault { \tl_const:Nn \CJKttdefault { tt } }
%    \end{macrocode}
%
% \begin{macro}[int]{\fontfamily}
% \begin{macro}[int]{\CTEX@fontfamily}
% 修改 \tn{fontfamily},使主要 |CJK| 字体族能随西文主要字体更新。
%    \begin{macrocode}
\RenewDocumentCommand \fontfamily { m }
  {
    \tl_set:Nx \f@family {#1}
    \CTEX@fontfamily {#1}
  }
\cs_new_protected_nopar:Npn \CTEX@fontfamily #1
  {
    \str_if_eq:nnTF {#1} { \familydefault }
      { \CJKfamily { \CJKfamilydefault } }
      { \@@_update_family_aux: }
  }
\cs_new_protected_nopar:Npn \@@_update_family_aux:
  {
    \str_case_e:nn { \f@family }
      {
        { \rmdefault }     { \CJKfamily { \CJKrmdefault } }
        { \sfdefault }     { \CJKfamily { \CJKsfdefault } }
        { \ttdefault }     { \CJKfamily { \CJKttdefault } }
        { \familydefault } { \CJKfamily { \CJKfamilydefault } }
      }
  }
%    \end{macrocode}
% \end{macro}
% \end{macro}
%
% 在导言区末尾更新 \tn{CJKfamilydefault}。
%    \begin{macrocode}
\ctex_at_end_preamble:n { \ctex_update_default_family: }
%    \end{macrocode}
%
% \begin{macro}[int]{\ctex_ltj_ensure_default_family:}
% 在导言区结束确认 \tn{CJKfamilydefault} 确实存在。
%    \begin{macrocode}
\cs_new_protected_nopar:Npn \ctex_ltj_ensure_default_family:
  {
    \prop_if_empty:NF \g_@@_family_font_name_prop
      {
        \ctex_ltj_family_if_exist:xNF { \CJKfamilydefault } \l_@@_tmp_tl
          {
            \str_if_eq:eeTF { \CJKfamilydefault } { \CJKrmdefault }
              { \use:n }
              {
                \ctex_ltj_family_if_exist:xNTF { \CJKrmdefault } \l_@@_tmp_tl
                  { \tl_gset:Nn \CJKfamilydefault { \CJKrmdefault } \use_none:n }
                  { \use:n }
              }
              {
                \prop_map_inline:Nn \g_@@_family_font_name_prop
                  {
                    \prop_map_break:n
                      { \tl_gset_rescan:Nnn \CJKfamilydefault { } { ##1 } }
                  }
              }
          }
        \normalfont
        \ctex_ltj_update_mathfont:
      }
  }
%    \end{macrocode}
% \end{macro}
%
% \begin{macro}[int]{\ctex_ltj_update_mathfont:}
% 更新数学字体为实际的字体。
%    \begin{macrocode}
\cs_new_protected_nopar:Npn \ctex_ltj_update_mathfont:
  {
    \ctex_ltj_family_if_exist:xNTF { \c_@@_math_tl } \l_@@_tmp_tl
      { \ctex_ltj_update_mathfont:n { \l_@@_tmp_tl } }
      {
        \ctex_ltj_family_if_exist:xNT { \CJKfamilydefault } \l_@@_tmp_tl
          { \ctex_ltj_update_mathfont:n { \l_@@_tmp_tl } }
      }
  }
\cs_new_protected_nopar:Npn \ctex_ltj_update_mathfont:n #1
  {
    \tl_const:Nx \c_@@_math_family_tl {#1}
    \DeclareSymbolFont { \c_@@_math_tl } { \CJK@encoding }
      { \c_@@_math_family_tl } { \mddefault } { \updefault }
    \cs_if_free:cTF
      { \CJK@encoding/\c_@@_math_family_tl/\bfdefault/\updefault }
      {
        \SetSymbolFont { \c_@@_math_tl } { bold } { \CJK@encoding }
          { \c_@@_math_family_tl } { \mddefault } { \updefault }
      }
      {
        \SetSymbolFont { \c_@@_math_tl } { bold } { \CJK@encoding }
          { \c_@@_math_family_tl } { \bfdefault } { \updefault }
      }
  }
%    \end{macrocode}
% \end{macro}
%
% \subsubsection{替代字体的设置}
%
% \begin{macro}{AlternateFont,CharRange}
% 设置替代字体的选项。
%    \begin{macrocode}
\keys_define:nn { ctex_ltj / fontspec }
  {
    AlternateFont  .code:n = \ctex_ltj_set_alternate_seq:n {#1} ,
    AlternateFont  .value_required:n = true ,
    CharRange .clist_set:N = \l_@@_char_range_clist ,
    CharRange .value_required:n = true
  }
%    \end{macrocode}
% \end{macro}
%
% \begin{macro}[int]{\ctex_ltj_set_alternate_seq:n}
% 我们使用 \verb=||= 作为替代字体序列的分隔标志。它可能被设置为活动字符,为此
% 需要先“消毒”,同时过滤掉空元素。
%    \begin{macrocode}
\group_begin:
  \char_set_catcode_other:N \|
  \cs_set:Npn \@@_tmp:w #1
    {
      \cs_new_protected:Npn \ctex_ltj_set_alternate_seq:n ##1
        {
          \clist_if_empty:NT \l_@@_char_range_clist
            {
              \tl_set:Nn \l_@@_tmp_tl { ##1 }
              \tl_replace_all:Nnn \l_@@_tmp_tl {#1} { || }
              \seq_set_split:NnV \l_@@_tmp_seq { || } \l_@@_tmp_tl
              \seq_set_filter:NNn \l_@@_tmp_seq \l_@@_tmp_seq
                { ! \tl_if_blank_p:n { ####1 } }
              \seq_concat:NNN \l_@@_alternate_seq
                \l_@@_alternate_seq \l_@@_tmp_seq
            }
        }
    }
  \char_set_catcode_active:N \|
  \@@_tmp:w { || }
\group_end:
\seq_new:N \l_@@_tmp_seq
\seq_new:N \l_@@_alternate_seq
%    \end{macrocode}
% \end{macro}
%
% \begin{macro}[int]{\ctex_ltj_set_alternate_family:nnF}
% 如果在字体的选项中设置了 \opt{CharRange},则只设置替代字体。
%    \begin{macrocode}
\cs_new_protected_nopar:Npn \ctex_ltj_set_alternate_family:nnF #1#2#3
  {
    \clist_if_empty:NTF \l_@@_char_range_clist
      {
        \@@_check_family:n {#1}
        \seq_if_empty:NF \l_@@_alternate_seq
          { \ctex_ltj_save_alternate_seq:cn { \@@_alternate_cs:n {#1} } {#2} }
        #3
      }
      { \ctex_ltj_set_alternate_family:nn {#1} {#2} }
  }
%    \end{macrocode}
% \end{macro}
%
% \begin{macro}[int]
% {\ctex_ltj_save_alternate_seq:Nn,\ctex_ltj_save_alternate_seq:Nnnwn}
% 保存由 \opt{AlternateFont} 设置的替代字体序列。
%    \begin{macrocode}
\cs_new_protected_nopar:Npn \ctex_ltj_save_alternate_seq:Nn #1#2
  {
    \seq_map_inline:Nn \l_@@_alternate_seq
      { \ctex_ltj_save_alternate_seq:Nnnwnw #1 {#2} ##1 { } \q_stop }
  }
\cs_generate_variant:Nn \ctex_ltj_save_alternate_seq:Nn { c }
\NewDocumentCommand \ctex_ltj_save_alternate_seq:Nnnwnw
  { m m m +O{ } m u{ \q_stop } }
  {
    \clist_set:Nn \l_@@_char_range_clist {#3}
    \clist_set:Nn \l_@@_alternate_options_clist {#4}
    \@@_use_global_options:N \l_@@_alternate_options_clist
    \tl_if_blank:nTF {#5}
      { \tl_set:Nn \l_@@_tmp_tl {#2} }
      {
        \tl_set:Nn \l_@@_tmp_tl {#5}
        \tl_replace_all:Nnn \l_@@_tmp_tl { * } {#2}
      }
    \use:x
      {
        \ctex_ltj_save_alternate_family:Nnnn \exp_not:N #1
          { \exp_not:o { \l_@@_char_range_clist } }
          { \exp_not:o { \l_@@_alternate_options_clist } }
          { \exp_not:o { \l_@@_tmp_tl } }
      }
  }
\clist_new:N \l_@@_alternate_options_clist
%    \end{macrocode}
% \end{macro}
%
% \begin{macro}[int]{\ctex_ltj_set_alternate_family:nn}
% 设置选项 \opt{CharRange} 范围内的替代字体。如果已经定义了主字体,我们也马上
% 定义替代字体,否则只保存起来备用。
%    \begin{macrocode}
\cs_new_protected_nopar:Npn \ctex_ltj_set_alternate_family:nn #1#2
  {
    \@@_update_family_uid:N \l_@@_font_options_clist
    \@@_use_global_options:N \l_@@_font_options_clist
    \ctex_ltj_set_alternate_family:coonn
      { \@@_alternate_cs:n {#1} }
      { \l_@@_char_range_clist }
      { \l_@@_font_options_clist } {#2} {#1}
  }
\cs_new_protected_nopar:Npn \ctex_ltj_set_alternate_family:Nnnnn #1#2#3#4#5
  {
    \prop_get:NnNT \g_@@_family_name_prop {#5} \l_@@_base_family_tl
      { \ctex_ltj_set_alternate_family:nnn {#2} {#3} {#4} }
    \ctex_ltj_save_alternate_family:Nnnn #1 {#2} {#3} {#4}
  }
\cs_generate_variant:Nn \ctex_ltj_set_alternate_family:Nnnnn { coo }
%    \end{macrocode}
% \end{macro}
%
% \begin{macro}[int]{\ctex_ltj_save_alternate_family:Nnnn}
% 保存替代字体序列的定义,以备定义主字体时使用。
%    \begin{macrocode}
\cs_new_protected_nopar:Npn \ctex_ltj_save_alternate_family:Nnnn #1#2#3#4
  {
    \cs_if_exist:NF #1 { \cs_set_eq:NN #1 \prg_do_nothing: }
    \cs_gset_protected_nopar:Npx #1
      { \exp_not:o { #1 \ctex_ltj_set_alternate_family:nnn {#2} {#3} {#4} } }
  }
%    \end{macrocode}
% \end{macro}
%
% \begin{macro}[int]{\ctex_ltj_set_alternate_family:nnn}
% 实际定义替代字体族。
%    \begin{macrocode}
\cs_new_protected_nopar:Npn \ctex_ltj_set_alternate_family:nnn #1#2#3
  {
    \group_begin:
    \@@_change_encoding:
    \cs_set_eq:NN \CJKfamily \use_none:n
    \ctex_ltj_swap_cs:NN
      \DeclareFontShape@ \ctex_ltj_declare_alternate_shape:nnnnnn
    \tl_set:Nn \l_@@_char_range_clist {#1}
    \fontspec_set_family:Nnn \l_@@_alternate_family_tl {#2} {#3}
    \group_end:
  }
\tl_new:N \l_@@_alternate_family_tl
%    \end{macrocode}
% \end{macro}
%
% \begin{macro}[int]{\ctex_ltj_swap_cs:NN}
% 交换两个控制序列的意义。
%    \begin{macrocode}
\cs_new_protected:Npn \ctex_ltj_swap_cs:NN #1#2
  {
    \cs_set_eq:NN \@@_tmp:w #1
    \cs_set_eq:NN #1 #2
    \cs_set_eq:NN #2 \@@_tmp:w
    \cs_undefine:N \@@_tmp:w
  }
%    \end{macrocode}
% \end{macro}
%
% \begin{macro}[int]{LTJFONTUID}
% \begin{macro}{\@@_update_family_uid:N}
% \pkg{fontspec} 在一个字体族的选项和字体名称相同的时候,就不定义新字体。为了
% 避免混淆替代字体的设置,我们新定义一个虚拟的选项 \opt{LTJFONTUID},确保
% \pkg{fontspec} 对 CJK 字体族总是定义新字体。
%    \begin{macrocode}
\keys_define:nn { fontspec } { LTJFONTUID .code:n = }
\cs_new_protected_nopar:Npn \@@_update_family_uid:N #1
  {
    \int_gincr:N \g_@@_family_int
    \clist_put_right:Nx #1 { LTJFONTUID = \int_use:N \g_@@_family_int }
  }
\int_new:N \g_@@_family_int
%    \end{macrocode}
% \end{macro}
% \end{macro}
%
% \begin{macro}[int]{\ctex_ltj_declare_alternate_shape:nnnnnn}
% 在定义替代字体的字形时,通过字符范围与主字体的对应字形关联起来。
% \tn{DeclareFontShape@} 一个有六个参数,我们只需要使用它的第三个参数 \meta{series}
% 和第四个参数 \meta{shape}。
%    \begin{macrocode}
\cs_new_protected:Npn \ctex_ltj_declare_alternate_shape:nnnnnn #1#2#3#4#5#6
  {
    \ctex_ltj_declare_alternate_shape:nnnnnn {#1} {#2} {#3} {#4} {#5} {#6}
    \ctex_ltj_set_alternate_shape:Nnnnnnn \l_@@_char_range_clist
      { \l_@@_base_family_tl } {#3} {#4}
      { \l_fontspec_family_tl } {#3} {#4}
  }
%    \end{macrocode}
% \end{macro}
%
% \begin{macro}[int]{\ctex_ltj_set_alternate_shape:Nnnnnnn}
% 与 \pkg{LuaTeX-ja} 的 \tn{DeclareAlternateKanjiFont} 的功能类似,区别是固定编码
% 为 \tn{CJK@encoding}。这个设置总是全局的。
%    \begin{macrocode}
\cs_new_protected_nopar:Npn \ctex_ltj_set_alternate_shape:Nnnnnnn #1#2#3#4#5#6#7
  {
    \clist_map_inline:Nn #1
      {
        \prop_get:NnNTF \g_@@_char_range_prop { ##1 } \l_@@_char_range_tl
          {
            \ctex_ltj_set_alternate_shape:nnN { #2/#3/#4 } { #5/#6/#7 }
              \l_@@_char_range_tl
          }
          { \ctex_ltj_set_alternate_shape:nnn { #2/#3/#4 } { #5/#6/#7 } { ##1 } }
      }
    \@@_save_alternate_shape:cnn
      { \@@_alternate_cs:n { clear / \l_@@_base_CJKfamily_tl } }
      { luatexja.jfont.clear_alt_font_latex }
      { '\lua_escape:e { \CJK@encoding/#2/#3/#4 }' }
  }
%    \end{macrocode}
% \end{macro}
%
% \begin{macro}[int]{\ctex_ltj_set_alternate_shape:nnn}
% 我们使用 \texttt{->} 而不是像 \pkg{LuaTeX-ja} 一样使用 \texttt{-} 作为区间的
% 分隔符。\pkg{LuaTeX-ja} 支持使用负数来引用由 \texttt{JFM} 设置的字符类。如果
% 使用 \texttt{-} 作为分隔符,那么负数单独使用时,就需要把它放在两层花括号之内
% (例如 |{{-1}}|),或者使用类似 |{-1}-{-1}| 的形式才不会解释错误。
%    \begin{macrocode}
\NewDocumentCommand \ctex_ltj_set_alternate_shape:nnn
  { m m > { \SplitArgument { 1 } { -> } } m }
  { \ctex_ltj_set_alternate_shape:nnnn {#1} {#2} #3 }
\cs_new_protected_nopar:Npn \ctex_ltj_set_alternate_shape:nnnn #1#2#3#4
  {
    \ctex_ltj_set_alternate_shape:n
      {
        \IfNoValueTF {#4}
          { \int_eval:n {#3} , \int_eval:n {#3} , }
          {
            \int_eval:n { \tl_if_blank:nTF {#3} { "80 } {#3} } ,
            \int_eval:n { \tl_if_blank:nTF {#4} { "10FFFF } {#4} } ,
          }
        '\lua_escape:e { \CJK@encoding/#2 }' ,
        '\lua_escape:e { \CJK@encoding/#1 }'
      }
  }
\cs_new_protected_nopar:Npn \ctex_ltj_set_alternate_shape:n #1
  {
    \lua_now:e { luatexja.jfont.set_alt_font_latex ( #1 ) }
    \@@_save_alternate_shape:cnn
      { \@@_alternate_cs:n { reset / \l_@@_base_CJKfamily_tl } }
      { luatexja.jfont.set_alt_font_latex } {#1}
  }
%    \end{macrocode}
% \end{macro}
%
% \begin{macro}[int]{\ctex_ltj_set_alternate_shape:nnN}
% 若字符范围预先由 \texttt{declarecharrange} 声明,则可以直接使用。
%    \begin{macrocode}
\cs_new_protected_nopar:Npn \ctex_ltj_set_alternate_shape:nnN #1#2#3
  {
    \tl_map_inline:Nn #3
      {
        \ctex_ltj_set_alternate_shape:n
          {
            ##1 ,
            '\lua_escape:e { \CJK@encoding/#2 }' ,
            '\lua_escape:e { \CJK@encoding/#1 }'
          }
      }
  }
%    \end{macrocode}
% \end{macro}
%
% \begin{macro}{\@@_save_alternate_shape:Nnn}
% 将实际设置的替换字形保存起来用于清除或恢复。
% 暂时令 \cs{l_@@_base_family_tl} 为 \cs{scan_stop:} 是让它不被展开,使得替换
% 字体的设置可以在 \tn{addCJKfontfeature} 中直接使用。
%    \begin{macrocode}
\cs_new_protected_nopar:Npn \@@_save_alternate_shape:Nnn #1#2#3
  {
    \group_begin:
    \cs_if_exist:NF #1 { \cs_set_eq:NN #1 \prg_do_nothing: }
    \cs_set_eq:NN \l_@@_base_family_tl \scan_stop:
    \cs_set_eq:NN \lua_escape:e \scan_stop:
    \cs_gset_protected_nopar:Npx #1
      { \exp_not:o {#1} \exp_not:N \lua_now:e { #2 ( #3 ) } }
    \group_end:
  }
\cs_generate_variant:Nn \@@_save_alternate_shape:Nnn { c }
%    \end{macrocode}
% \end{macro}
%
% \begin{macro}{clearalternatefont,resetalternatefont}
% 清除和重置操作总是全局的。
%    \begin{macrocode}
\keys_define:nn { ctex }
  {
    clearalternatefont    .code:n =
      { \clist_map_function:xN {#1} \ctex_ltj_clear_alternate_font:n } ,
    resetalternatefont    .code:n =
      { \clist_map_function:xN {#1} \ctex_ltj_reset_alternate_font:n } ,
    clearalternatefont .default:n = \l_ctex_ltj_family_tl ,
    resetalternatefont .default:n = \l_ctex_ltj_family_tl
  }
\cs_new_protected_nopar:Npn \ctex_ltj_clear_alternate_font:n #1
  {
    \group_begin:
      \ctex_ltj_family_if_exist:xNTF {#1} \l_@@_base_family_tl
        {
          \cs_if_exist_use:cT { \@@_alternate_cs:n { clear / #1 } }
            {
              \prop_gput:Nno \g_@@_reset_alternate_prop
                {#1} { \l_@@_base_family_tl }
              \tl_set_eq:NN \CJK@family \l_@@_base_family_tl
              \selectfont
            }
        }
        { \@@_family_unknown_warning:n {#1} }
    \group_end:
  }
\cs_new_protected_nopar:Npn \ctex_ltj_reset_alternate_font:n #1
  {
    \group_begin:
      \prop_gpop:NnNT \g_@@_reset_alternate_prop {#1} \CJK@family
        {
          \tl_set_eq:NN \l_@@_base_family_tl \CJK@family
          \use:c { \@@_alternate_cs:n { reset / #1 } }
          \selectfont
        }
    \group_end:
  }
\prop_new:N \g_@@_reset_alternate_prop
\cs_generate_variant:Nn \clist_map_function:nN { x }
%    \end{macrocode}
% \end{macro}
%
% \begin{macro}{declarecharrange}
% 预先声明字符范围。
%    \begin{macrocode}
\keys_define:nn { ctex }
  {
    declarecharrange .code:n = \ctex_ltj_declare_char_range:x {#1} ,
    declarecharrange .value_required:n = true
  }
\cs_new_protected_nopar:Npn \ctex_ltj_declare_char_range:n #1
  { \clist_map_inline:nn {#1} { \@@_declare_char_range:nn ##1 } }
\cs_generate_variant:Nn \ctex_ltj_declare_char_range:n { x }
\cs_new_protected_nopar:Npn \@@_declare_char_range:nn #1#2
  { \use:x { \ctex_ltj_declare_char_range:nn { \tl_trim_spaces:n {#1} } } {#2} }
%    \end{macrocode}
% \end{macro}
%
% \begin{macro}[int]{\ctex_ltj_declare_char_range:nn}
% \begin{variable}{\g_@@_char_range_prop}
% |#1| 是名字,|#2| 是范围。
%    \begin{macrocode}
\cs_new_protected_nopar:Npn \ctex_ltj_declare_char_range:nn #1#2
  {
    \tl_clear:N \l_@@_char_range_tl
    \clist_map_function:nN {#2} \ctex_ltj_save_char_range:n
    \prop_gput:Nno \g_@@_char_range_prop {#1} { \l_@@_char_range_tl }
    \ctex_ltj_def_char_range_key:n {#1}
    \tl_clear:N \l_@@_char_range_tl
  }
\tl_new:N \l_@@_char_range_tl
\prop_new:N \g_@@_char_range_prop
%    \end{macrocode}
% \end{variable}
% \end{macro}
%
% \begin{macro}[int]{\ctex_ltj_save_char_range:n}
% 预先解释字符区间的意义。
%    \begin{macrocode}
\NewDocumentCommand \ctex_ltj_save_char_range:n
  { > { \SplitArgument { 1 } { -> } } m }
  { \ctex_ltj_save_char_range:nn #1 }
\cs_new_protected_nopar:Npn \ctex_ltj_save_char_range:nn #1#2
  {
    \tl_put_right:Nx \l_@@_char_range_tl
      { {
          \IfNoValueTF {#2}
            { \int_eval:n {#1} , \int_eval:n {#1} }
            {
              \int_eval:n { \tl_if_blank:nTF {#1} { "80 } {#1} } ,
              \int_eval:n { \tl_if_blank:nTF {#2} { "10FFFF } {#2} }
            }
      } }
  }
%    \end{macrocode}
% \end{macro}
%
% \begin{macro}[int]{\ctex_ltj_def_char_range_key:n}
% 在字体设置选项中定义字符范围键。
%    \begin{macrocode}
\cs_new_protected_nopar:Npn \ctex_ltj_def_char_range_key:n #1
  {
    \keys_if_exist:nnF { ctex_ltj / fontspec } {#1}
      {
        \keys_define:nn { ctex_ltj / fontspec }
          { #1 .code:n = \ctex_ltj_char_range_key:nn {#1} { ##1 } }
      }
  }
%    \end{macrocode}
% \end{macro}
%
% \begin{macro}[int]{\ctex_ltj_char_range_key:nn}
% 如果字符范围键没有值,则只设置的这个字符范围内的替代字体。
%    \begin{macrocode}
\cs_new_protected:Npn \ctex_ltj_char_range_key:nn #1#2
  {
    \tl_if_blank:nTF {#2}
      { \tl_set:Nn \l_@@_char_range_clist {#1} }
      {
        \clist_if_empty:NT \l_@@_char_range_clist
          {
            \tl_set:Nn \l_@@_tmp_tl { {#1} }
            \@@_char_range_parse_feature:w #2 \q_stop
          }
      }
  }
%    \end{macrocode}
% \end{macro}
%
% \begin{macro}{\@@_char_range_parse_feature:w}
% 可以使用加方括号的方式,通过文件名来调用字体。这容易与字体选项混淆。例如,需要
% 将 |[simsun.ttc]| 设置为 \opt{range} 的主字体,就需要使用
% |range={{[simsun.ttc]}}| 或者 |[]{[simsun.ttc]}|。下面的目的是,支持直接使用
% |[simsun.ttc]| 和 |[...][simsun.ttc]|。
%    \begin{macrocode}
\NewDocumentCommand \@@_char_range_parse_feature:w
  { +o o u { \q_stop } }
  {
    \exp_args:NNf \tl_put_right:Nn \l_@@_tmp_tl
      {
        \IfNoValueTF {#1} { {#3} }
          {
            \IfNoValueTF {#2}
              { \tl_if_blank:nTF {#3} { { [#1] } } { [ {#1} ] {#3} } }
              { [ {#1} ] { [#2] } }
          }
      }
    \seq_put_right:No \l_@@_alternate_seq { \l_@@_tmp_tl }
  }
%    \end{macrocode}
% \end{macro}
%
% \subsubsection{其它设置}
%
% 在抄录环境中禁用 \opt{autospacing} 和 \opt{autoxspacing}。然而,\pkg{LuaTeX-ja}
% 还是会使 JAchar 自动折行。没有看到有简单的禁用折行的办法,可能需要设置所有的
% JAchar 的 \opt{prebreakpenalty} 或 \opt{postbreakpenalty} 为 \texttt{10000}:
% \begin{verbatim}
%   \directlua
%     {
%       luatexja.isglobal = tex.globaldefs > 0 and "global" or ""
%       for i = 0x80, 0x10FFFF do
%         if luatexja.charrange.jcr_table_main[i] > 0 and
%            luatexja.charrange.jcr_table_main[i] < 218 and
%            luatexja.charrange.is_japanese_char_curlist(i) then
%           luatexja.stack.set_stack_table(luatexja.stack_table_index.PRE + i, 10000)
%         end
%       end
%     }
% \end{verbatim}
%    \begin{macrocode}
\AtBeginDocument
  {
    \ctex_appto_cmd:NnnTF \verbatim@font
      { \char_set_catcode_letter:n { 64 } }
      { \CTEX@verbatim@font@hook }
      { }
      { \ctex_patch_failure:N \verbatim@font }
  }
\cs_new_protected_nopar:Npn \CTEX@verbatim@font@hook
  { \ltjsetparameter { autospacing = false , autoxspacing = false } }
%    \end{macrocode}
%
% \begin{macro}{\@@@@italiccorr}
% \LaTeX{} 的倾斜校正也要重新定义。
%    \begin{macrocode}
\cs_set_eq:NN \@@@@italiccorr \/
%    \end{macrocode}
% \end{macro}
%
% \begin{macro}[int]{\ctex_ltj_set_kanjiskip:N,\ctex_ltj_set_xkanjiskip:N}
% \tn{ltjsetkanjiskip} 和 \tn{ltjsetxkanjiskip} 是相应的 \tn{ltjsetparameter}
% 的快捷方式,在使用他们时,要注意先使用 \tn{ltj@setpar@global}。
%    \begin{macrocode}
\cs_new_protected_nopar:Npn \ctex_ltj_set_kanjiskip:N
  { \ltj@setpar@global \ltjsetkanjiskip }
\cs_new_protected_nopar:Npn \ctex_ltj_set_xkanjiskip:N
  { \ltj@setpar@global \ltjsetxkanjiskip }
%    \end{macrocode}
% \end{macro}
%
%    \begin{macrocode}
%<@@=ctex>
%</luatex>
%    \end{macrocode}
%
% \subsection{\pkg{ctex-engine-uptex.def}}
%
%    \begin{macrocode}
%<*uptex|aptex>
%    \end{macrocode}
%
% 按 \pkg{CJK} 的命名习惯模拟部分命令,并设置默认字体。
%    \begin{macrocode}
\tl_set:Nn \CJKrmdefault { zhrm }
\tl_set:Nn \CJKsfdefault { zhsf }
\tl_set:Nn \CJKttdefault { zhtt }
\tl_set:Nn \CJKfamilydefault { \CJKrmdefault }
\tl_set:Nn \kanjifamilydefault { \CJKfamilydefault }
\RenewDocumentCommand \rmfamily { }
  {
    \not@math@alphabet \rmfamily \mathrm
    \romanfamily \rmdefault
    \kanjifamily \CJKrmdefault
    \selectfont
  }
\RenewDocumentCommand \sffamily { }
  {
    \not@math@alphabet \sffamily \mathsf
    \romanfamily \sfdefault
    \kanjifamily \CJKsfdefault
    \selectfont
  }
\RenewDocumentCommand \ttfamily { }
  {
    \not@math@alphabet \ttfamily \mathtt
    \romanfamily \ttdefault
    \kanjifamily \CJKttdefault
    \selectfont
  }
\NewDocumentCommand \CJKfamily { m }
  {
    \kanjifamily {#1}
    \selectfont
  }
%    \end{macrocode}
%
% 将 \upLaTeX 的默认字体由 |mc| 改为 |zhrm|,并启用 \tn{jfam}。
%    \begin{macrocode}
\DeclareErrorKanjiFont{JY2}{zhrm}{m}{n}{10}
\DeclareKanjiSubstitution{JY2}{zhrm}{m}{n}
\DeclareKanjiSubstitution{JT2}{zhrm}{m}{n}
\DeclareSymbolFont{mincho}{JY2}{zhrm}{m}{n}
\SetSymbolFont{mincho}{bold}{JY2}{zhrm}{bx}{n}
\jfam \symmincho
%    \end{macrocode}
%
% 在导言区末尾更新 \tn{CJKfamilydefault}。
%    \begin{macrocode}
\ctex_at_end_preamble:n { \ctex_update_default_family: }
%    \end{macrocode}
%
% 使修改立刻生效,保证导言区字体族正确。
%    \begin{macrocode}
\normalfont
%    \end{macrocode}
%
% \begin{macro}[int]{\em}
% 取消 \upLaTeX{} 对 \tn{em} 使用 |\mcfamily|、|\gtfamily| 命令的重定义,恢复
% \LaTeXe{} 对 \tn{em} 的原始定义。如果用户已经重定义了 \tn{em},则新定义保持
% 不变。\upLaTeX{} 2016/05/07u00 的定义有所变化,这一行为可以由用户通过 \pkg{platexrelease}
% 包改变,需要分支处理。
%    \begin{macrocode}
\ctex_patch_cmd_once:NnnnTF \em
  { \ExplSyntaxOff }
  { \eminnershape \else \gtfamily \itshape }
  { \eminnershape \else \itshape }
  { }
  {
    \ctex_patch_cmd:Nnn \em
      { \mcfamily \upshape \else \gtfamily \itshape }
      { \eminnershape \else \itshape }
  }
\cs_set_nopar:Npn \eminnershape { \upshape }
%    \end{macrocode}
% \end{macro}
%
% \begin{macro}[int]{\ctex_set_upfamily:nnn}
% 将 NFSS 字体族 |#1| 设置为 JFM 字体名 |#2|,粗体形式字体名 |#3|。其中字体名
% 形如 |upzhserif|,不包括表示方向的后缀 |-h| 与 |-v|。粗体字体名为空时不设置该
% 字形。本命令不设置字体映射,需要复用已有的字体映射或另行设置。
%    \begin{macrocode}
\cs_new_protected_nopar:Npn \ctex_set_upfamily:nnn #1#2#3
  {
    \DeclareKanjiFamily{JY2}{#1}{}
    \DeclareKanjiFamily{JT2}{#1}{}
    \DeclareFontShape{JY2}{#1}{m}{n}{<->~ #2-h}{}
    \DeclareFontShape{JT2}{#1}{m}{n}{<->~ #2-v}{}
    \tl_if_empty:nF {#3}
      {
        \DeclareFontShape{JY2}{#1}{bx}{n}{<->~ #3-h}{}
        \DeclareFontShape{JT2}{#1}{bx}{n}{<->~ #3-v}{}
      }
  }
%    \end{macrocode}
% \end{macro}
%
% \begin{macro}[int]{\ctex_set_upmap:nnn}
% 设置 \upTeX{} 字体映射。|#1| 是形如 |upserif| 的 PS TFM 字体名,不带表示粗体
% 的后缀 |b| 与表示排版方向的后缀 |-h| 与 |-v|。|#2| 与 |#3| 是普通与粗体的实际
% 字体名。
%    \begin{macrocode}
\cs_new_protected_nopar:Npn \ctex_set_upmap:nnn #1#2#3
  {
    \ctex_set_zhmap:n
      {
        \special{ pdf:mapline~ #1-h~ UniGB-UTF16-H~ #2 }
        \special{ pdf:mapline~ #1-v~ UniGB-UTF16-V~ #2 }
        \tl_if_empty:nF {#3}
          {
            \special{ pdf:mapline~ #1b-h~ UniGB-UTF16-H~ #3 }
            \special{ pdf:mapline~ #1b-v~ UniGB-UTF16-V~ #3 }
          }
      }
  }
%    \end{macrocode}
% \end{macro}
%
% \begin{macro}[int]{\ctex_set_upmap_unicode:nnn}
% 设置 \upTeX{} 字体映射,使用 |unicode| CMap。参数同上。
%    \begin{macrocode}
\cs_new_protected_nopar:Npn \ctex_set_upmap_unicode:nnn #1#2#3
  {
    \ctex_set_zhmap:n
      {
        \special{ pdf:mapline~ #1-h~ unicode~ #2 }
        \special{ pdf:mapline~ #1-v~ unicode~ #2 }
        \tl_if_empty:nF {#3}
          {
            \special{ pdf:mapline~ #1b-h~ unicode~ #3 }
            \special{ pdf:mapline~ #1b-v~ unicode~ #3 }
          }
      }
  }
%    \end{macrocode}
% \end{macro}
%
% \begin{macro}[int]{\ctex_set_upfonts:nnnnnn}
% 设置 \upTeX{} 基本字体映射,按 \pkg{zhmetrics-uptex} 的定义,依次设置衬线体
% 正、粗、意大利,无衬线体正、粗,等宽体正——共 6 种字体,并分横排及直排。
%    \begin{macrocode}
\cs_new_protected_nopar:Npn \ctex_set_upfonts:nnnnnn #1#2#3#4#5#6
  {
    \ctex_set_upmap:nnn { upserif   } {#1} {#2}
    \ctex_set_upmap:nnn { upserifit } {#3} {}
    \ctex_set_upmap:nnn { upsans    } {#4} {#5}
    \ctex_set_upmap:nnn { upmono    } {#6} {}
  }
%    \end{macrocode}
% \end{macro}
%
% 以下命令只能在导言区使用。
%    \begin{macrocode}
\@onlypreamble \ctex_set_upfamily:nnn
\@onlypreamble \ctex_set_upmap:nnn
\@onlypreamble \ctex_set_upmap_unicode:nnn
\@onlypreamble \ctex_set_upfonts:nnnnnn
%    \end{macrocode}
%
% \pkg{everysel} 宏包(2011/10/28)未考虑 \upLaTeX{} 对 \tn{selectfont} 的修
% 改,需要引入 \pkg{pxeverysel} 宏包。
%    \begin{macrocode}
\RequirePackage { pxeverysel }
%    \end{macrocode}
%
%    \begin{macrocode}
%</uptex|aptex>
%    \end{macrocode}
%
% \subsection{调整 \tn{CJKfamilydefault}}
%
% \begin{macro}[int]{\ctex_update_default_family:}
% 在导言区结束,如果 \tn{CJKfamilydefault} 没有被更改,则在此时根据西文字体的情况
% 更新 \tn{CJKfamilydefault}。\pkg{xeCJK} 已经有这个功能,不需要再调整。
%    \begin{macrocode}
%<*pdftex|luatex|uptex|aptex>
\cs_new_protected_nopar:Npn \ctex_update_default_family:
  {
    \tl_if_eq:NNT \CJKfamilydefault \l_@@_family_default_init_tl
      {
        \group_begin:
          \cs_set_eq:NN \@@_family_default_wrap:n \exp_not:n
          \tl_gset:Nx \CJKfamilydefault
            {
              \str_case:onF { \familydefault }
                {
                  { \rmdefault } { \exp_not:N \CJKrmdefault }
                  { \sfdefault } { \exp_not:N \CJKsfdefault }
                  { \ttdefault } { \exp_not:N \CJKttdefault }
                }
                { \CJKfamilydefault }
            }
        \group_end:
      }
%    \end{macrocode}
% 使用 \LuaLaTeX{} 时,自动调整得到的 \tn{CJKfamilydefault} 可能没有定义,需要
% 确认它的存在性。使用 \pkg{CJK} 宏包或 \upLaTeX{}
% 时,\texttt{C19rm}、\texttt{JY2rm} 等总是有定义的,不需要确认。
%    \begin{macrocode}
%<*luatex>
    \ctex_ltj_ensure_default_family:
%</luatex>
  }
%    \end{macrocode}
% \end{macro}
%
% \begin{variable}{\l_@@_family_default_init_tl}
% 往 \tn{CJKfamilydefault} 中加入标志,用于判断它是否被更改。
%    \begin{macrocode}
\tl_new:N \l_@@_family_default_init_tl
\cs_new_eq:NN \@@_family_default_wrap:n \use:n
\tl_set:Nx \l_@@_family_default_init_tl
  {
    \exp_not:N \@@_family_default_wrap:n
      { \exp_not:o { \CJKfamilydefault } }
  }
\tl_gset_eq:NN \CJKfamilydefault \l_@@_family_default_init_tl
%</pdftex|luatex|uptex|aptex>
%    \end{macrocode}
% \end{variable}
%
% \subsection{\pkg{hyperref} 兼容性处理}
%
% 现在处理各个引擎下的 PDF 中文书签问题。根据编译引擎与文件编码的不
% 同,\pkg{ctex} 向 \pkg{hyperref} 传递适当的参数,完成中文书签的正确设置。用
% 户仍需要自己载入 \pkg{hyperref} 宏包。
%
% \begin{macro}[int]{\ctex_hypersetup:n}
% 如果已经载入 \pkg{hyperref} 宏包,则直接使用其定义设置选项;否则
% \cs{ctex_hypersetup:n} 的效果与 \tn{PassOptionsToPackage} 一致,只传递宏包参
% 数。如果用户不载入 \pkg{hyperref} 宏包,相关参数即被丢弃。
%    \begin{macrocode}
%<*class|ctex>
\@ifpackageloaded { hyperref }
  {
    \cs_new_protected_nopar:Npn \ctex_hypersetup:n #1
      { \hypersetup {#1} }
  }
  {
    \cs_new_protected_nopar:Npn \ctex_hypersetup:n #1
      { \PassOptionsToPackage {#1} { hyperref } }
  }
%</class|ctex>
%    \end{macrocode}
% \end{macro}
%
% 在 \pdfTeX{} 下使用 \texttt{GBK} 编码,\dvipdfmx{} 驱动可以直接用它的
% \tn{special} 命令,其它模式用 \pkg{xCJK2uni} 宏包处理。使用 \texttt{UTF-8} 编
% 码时,\pkg{CJKutf8} 已经处理了书签问题,但仍需要设置 \opt{pdfencoding} 为
% \opt{unicode},目的是在书签的开头写入 BOM (|\376\377|),提示这是
% \texttt{UTF-16BE} 字节流。
%    \begin{macrocode}
%<*pdftex>
\ctex_hypersetup:n { driverfallback = dvipdfmx }
\str_if_eq:onTF { \g_@@_encoding_tl } { GBK }
  {
    \ctex_hypersetup:n { CJKbookmarks = true }
    \sys_if_output_pdf:TF
      { \ctex_at_end_package:nn { hyperref } { \RequirePackage { xCJK2uni } } }
      {
        \ctex_at_end_package:nn { hyperref }
          {
            \str_if_eq:onTF { \Hy@driver } { hdvipdfm }
              {
                \AtBeginShipoutFirst
                  { \special { pdf:tounicode~GBK-EUC-UCS2 } }
              }
              { \RequirePackage { xCJK2uni } }
          }
      }
  }
  { \ctex_hypersetup:n { pdfencoding = unicode } }
%</pdftex>
%    \end{macrocode}
% 在 \XeTeX{} 下,\pkg{hyperref} 在处理带有非 ASCII 字符和 |\%| 的书签时有问题^^A
% \footnote{\url{https://github.com/CTeX-org/ctex-kit/issues/39}}。
% 事实上,\pkg{hyperref} 在驱动文件 \file{hxetex.def} 中设置了
% \tn{Hy@unicodetrue},从而书签总是会被 \tn{HyPsd@ConvertToUnicode} 转化成
% \texttt{UTF-16BE} 编码的形式(抄录自 \tn{pdfstringdef}的定义):
% \begin{verbatim}
%       \ifHy@unicode
%         \HyPsd@ConvertToUnicode#1%
%         \ifx\HyPsd@pdfencoding\HyPsd@pdfencoding@auto
%           \ltx@IfUndefined{StringEncodingConvertTest}{%
%           }{%
%             \EdefUnescapeString\HyPsd@temp#1%
%             \ifxetex
%               \let\HyPsd@UnescapedString\HyPsd@temp
%               \StringEncodingConvertTest\HyPsd@temp\HyPsd@temp
%                                         {utf16be}{ascii-print}{%
%                 \EdefEscapeString\HyPsd@temp\HyPsd@temp
%                 \global\let#1\HyPsd@temp
%                 \HyPsd@EscapeTeX#1%
%                 \Hy@unicodefalse
%               }{%
%                \HyPsd@ToBigChars#1%
%               }%
% \end{verbatim}
% 通过宏包选项 \opt{pdfencoding=unicode} 设置 \tn{HyPsd@pdfencoding} 为
% \opt{unicode},可以避免随后再将书签从 \texttt{UTF-16BE} 字节流转化回正常
% 字符(其中使用的 \tn{HyPsd@ToBigChars} 没有考虑书签中含有 |\%| 的情况)。
% Heiko Oberdiek 在 \file{README} 中说明了将书签转化回正常字符的意图:避免
% XDVIPDFMX 的警告^^A
% \footnote{\url{http://project.ktug.org/dvipdfmx/mailman/dvipdfmx/2009-December/000153.html}}:
% \begin{verbatim}
%   ** WARNING ** Failed to convert input string to UTF16...
% \end{verbatim}
% \XeTeX{} 的维护者 Khaled Hosny 已经注意到了这个问题^^A
% \footnote{\url{http://tug.org/pipermail/tex-live/2013-December/034613.html}}。
% 需要注意的是,\file{hxetex.def} 重载了宏包选项 \opt{unicode},目的是不能设置
% 它为 \opt{false},但也导致它不会改变 \tn{HyPsd@pdfencoding}。如果
% \pkg{hyperref} 先于 \CTeX{} 被载入,那么 \opt{unicode} 选项是没有意义的。
% 因此要通过意义相同但在 \XeTeX{} 下更保险的 \opt{pdfencoding} 选项来设置。
% 为了与 \XeTeX 下的行为一致(使用 \tn{HyPsd@LoadUnicode} 载入 \file{puenc.def}),
% 在 \LuaTeX{} 下也启用这个选项。
%    \begin{macrocode}
%<*xetex|luatex>
\ctex_hypersetup:n { pdfencoding = unicode }
%</xetex|luatex>
%    \end{macrocode}
%
% 我们假定 \upTeX{} 使用 \dvipdfmx{} 驱动输出,于是使用与 \pdfTeX{} 类似的设
% 置。注意 \upTeX{} 需要使用 UTF8-UTF16 的编码转换。
%    \begin{macrocode}
%<*uptex|aptex>
\ctex_hypersetup:n { driverfallback = dvipdfmx }
\ctex_at_end_package:nn { hyperref }
  { \AtBeginShipoutFirst { \special { pdf:tounicode~ UTF8-UTF16 } } }
%</uptex|aptex>
%    \end{macrocode}
%
%    \begin{macrocode}
%<*pdftex|xetex|luatex|uptex|aptex>
%    \end{macrocode}
%
% \subsection{\pkg{CJKfntef}、\pkg{xeCJKfntef} 相关设置}
%
% \CTeX{} 宏集对 \pdfTeX{} 与 \XeTeX{} 引擎,分别载入 \pkg{CJKfntef} 或
% \pkg{xeCJKfntef} 宏包,并关闭宏包默认的彩色等多余格式。
%
% 载入 \pkg{CJKfntef} 或 \pkg{xeCJKfntef} 并做适当格式设置。有关 |\CTEX| 开头
% 的宏定义是过时命令,仅做兼容性保留。
%
%    \begin{macrocode}
%<*pdftex>
\ctex_at_end_package:nn { CJKfntef }
  {
    \normalem
    \cs_new_protected_nopar:Npn \@@_clear_fntef_color:n #1
      { \tl_clear:c { CJK#1color } }
%</pdftex>
%<*xetex>
\ctex_at_end_package:nn { xeCJKfntef }
  {
    \@ifpackagelater { xeCJKfntef } { 2014/11/04 }
      {
        \cs_new_protected_nopar:Npn \@@_clear_fntef_color:n #1
          { \xeCJKsetup { #1 / format = { } } }
      }
      {
        \cs_new_protected_nopar:Npn \@@_clear_fntef_color:n #1
          { \tl_clear:c { CJK#1color } }
      }
%</xetex>
%<*pdftex|xetex>
    \clist_map_inline:nn
      { underdot , underline , underdblline , underwave , sout , xout }
      { \@@_clear_fntef_color:n {#1} }
  }
%</pdftex|xetex>
%    \end{macrocode}
%
% \subsection{\tn{ccwd} 的更新}
%
% \begin{macro}[int]{\ctex_update_ccwd:,\ccwd}
%    \begin{macrocode}
\cs_new_protected_nopar:Npn \ctex_update_ccwd:
%<*pdftex|xetex>
  {
    \hbox_set:Nn \l_@@_tmp_box { \CJKglue }
    \dim_set:Nn \ccwd { \box_wd:N \l_@@_tmp_box + \f@size \p@ }
  }
%</pdftex|xetex>
%<*luatex>
  { \skip_set:Nn \ccwd { \ltjgetparameter { kanjiskip } + \zw } }
%</luatex>
%<*uptex|aptex>
  { \skip_set:Nn \ccwd { 1zw + \tex_kanjiskip:D } }
%</uptex|aptex>
\dim_new:N \ccwd
%    \end{macrocode}
% \end{macro}
%
% \begin{macro}[int]{\ctex_update_ccglue:}
% 更新字间距。
%    \begin{macrocode}
\cs_new_protected_nopar:Npn \ctex_update_ccglue:
%<*pdftex|xetex>
  {
    \cs_set_protected_nopar:Npn \CJKglue
      { \skip_horizontal:N \l_@@_ccglue_skip }
  }
%</pdftex|xetex>
%<*luatex>
  { \ctex_ltj_set_kanjiskip:N \l_@@_ccglue_skip }
%</luatex>
%<*uptex|aptex>
  { \skip_set_eq:NN \tex_kanjiskip:D \l_@@_ccglue_skip }
%</uptex|aptex>
\skip_new:N \l_@@_ccglue_skip
%    \end{macrocode}
% \end{macro}
%
% \begin{macro}[int,pTF]{\ctex_if_ccglue_touched:}
% 检查用户是否修改过汉字间距。
%    \begin{macrocode}
\prg_new_conditional:Npnn \ctex_if_ccglue_touched: { TF }
  {
%<*pdftex|xetex>
    \if_meaning:w \CJKglue \@@_ccglue:
      \prg_return_false: \else: \prg_return_true: \fi:
%</pdftex|xetex>
%<*luatex>
    \skip_if_eq:nnTF { \l_@@_ccglue_skip } { \ltjgetparameter { kanjiskip } }
      { \prg_return_false: } { \prg_return_true: }
%</luatex>
%<*uptex|aptex>
    \skip_if_eq:nnTF { \l_@@_ccglue_skip } { \tex_kanjiskip:D }
      { \prg_return_false: } { \prg_return_true: }
%</uptex|aptex>
  }
%<pdftex|xetex>\ctex_at_end:n { \cs_new_eq:NN \@@_ccglue: \CJKglue }
%    \end{macrocode}
% \end{macro}
%
% \begin{macro}[int]{\ctex_update_em_unit:}
% 将当前汉字的宽度保存到 \tn{ccwd} 中备用。不采用 \texttt{1em},因为这时的
% \texttt{1em} 实际上来自西文字体的信息,未必等于汉字的宽度,这似乎在传统的
% \file{.tfm} 字体上表现更明显。在 \pdfTeX{} 和 \XeTeX{} 下,直接使用 |\f@size\p@|
% 作为汉字的宽度,这应该对大多数汉字字体都成立,但不适用于诸如“方正兰亭黑长”之类
% 的特殊字体。在 \XeTeX{} 可以用 \tn{fontcharwd} 来改进。而在 \pdfTeX{} 下,若使用
% \pkg{zhmetrics} 技术,所有的汉字共享同一个 \file{.tfm},\tn{fontcharwd} 也就没有
% 意义。在 \LuaTeX{} 下,\pkg{LuaTeX-ja} 总是按照 JFM 中的设置输出汉字的宽度,可以
% 直接用 \tn{zw} 作为汉字宽度。\upTeX{} 可以直接使用原生的长度单位 |zw|。
%    \begin{macrocode}
\cs_new_protected_nopar:Npn \ctex_update_em_unit:
%<pdftex|xetex>  { \dim_set:Nn \ccwd { \f@size \p@ } }
%<luatex>  { \dim_set:Nn \ccwd { \zw } }
%<uptex|aptex>  { \dim_set:Nn \ccwd { 1zw } }
%    \end{macrocode}
% \end{macro}
%
% \subsection{其它}
%
% \begin{macro}[int]{\ctex_add_to_selectfont:n,\CTEX@selectfont@hook}
% \tn{EverySelectfont} 直到文档开始时才有效。为了 \tn{ccwd} 和 \pkg{LuaTeX-ja} 的
% 字体设置在导言区也可用,我们还需要在这里手工修改 \tn{selectfont}。\pkg{everysel}
% 宏包会用 \tn{CheckCommand} 来检查 \tn{selectfont} 是否为标准定义。我们修改了
% \tn{selectfont},所以会给出一个警告。为了消除这个警告,在它检查之前,还原本来
% 定义。\pkg{pxeverysel} 宏包取消了检查,但也需要恢复定义,避免重复使用钩子。
%    \begin{macrocode}
\cs_new_protected:Npn \ctex_add_to_selectfont:n #1
  {
    \cs_set_protected_nopar:Npx \CTEX@selectfont@hook
      { \exp_not:o { \CTEX@selectfont@hook #1 } }
  }
\cs_new_eq:NN \CTEX@selectfont@hook \prg_do_nothing:
\if_cs_exist:N \@EverySelectfont@Init
  \group_begin:
    \cs_set:Npn \@@_tmp:N #1
      {
        \tl_set:Nn \l_@@_tmp_tl {#1}
        \cs_new_eq:NN \CTEX@selectfont@save #1
        \cs_new_protected_nopar:Npn \@@_restore_selectfont:
          {
            \tl_put_left:Nn \@EverySelectfont@Init
              { \let #1 \CTEX@selectfont@save }
            \cs_undefine:N \@@_restore_selectfont:
          }
      }
    \ctex_parse_name:NN \@@_tmp:N \selectfont
  \exp_last_unbraced:NNo \group_end:
  \ctex_patch_cmd_once:NnnnTF { \l_@@_tmp_tl }
    { \ExplSyntaxOff }
    { \size@update }
    { \CTEX@selectfont@hook \size@update }
    { \@@_restore_selectfont: }
    { \ctex_patch_failure:N \selectfont }
\fi:
%    \end{macrocode}
% \end{macro}
%
% \tn{CJK@plane} 有定义,说明处于 \pkg{CJK} 宏包的 \tn{CJKsymbol} 之内,不必使用钩子。
%    \begin{macrocode}
%<*pdftex>
\EverySelectfont { \cs_if_exist:NF \CJK@plane { \CTEX@selectfont@hook } }
%</pdftex>
%<*xetex|luatex|uptex|aptex>
\EverySelectfont { \CTEX@selectfont@hook }
%</xetex|luatex|uptex|aptex>
%    \end{macrocode}
%
% Attribute 寄存器 \tn{ltj@curjfnt} 的初始值是 $-1$,必须把它设置为一个有效的
% \texttt{font.id},否则编译时会直接退出。
%    \begin{macrocode}
%<*luatex>
\ctex_add_to_selectfont:n
  {
    \ctex_ltj_select_font:
    \ctex_ltj_select_alternate_font:
  }
\tl_set:Nn \CJK@family { song } \selectfont
\tl_clear:N \CJK@family
%</luatex>
%    \end{macrocode}
%
% \begin{macro}[int]{\ctex_update_xkanjiskip:}
% \begin{variable}{\l_@@_xkanjiskip_skip}
% \upTeX{} 和 \pkg{LuaTeX-ja} 对 \tn{xkanjiskip} 都是即时赋值。单位 \opt{zw} 与字体相关,因此
% 需要每次 \tn{selectfont} 的时候更新一次 \tn{xkanjiskip}。如果用户设置过
% \tn{xkanjiskip},就不更新。注意,同 \TeX{} 的 \tn{baselineskip} 一样,如果在
% 一个段落内多次设置了 \tn{kanjiskip} 或 \tn{xkanjiskip},只有最后的设置会影响
% 全段。
%    \begin{macrocode}
%<*luatex|uptex|aptex>
\cs_new_protected_nopar:Npn \ctex_update_xkanjiskip:
  {
    \skip_if_eq:nnT
%<luatex>      { \ltjgetparameter { xkanjiskip } } { \l_@@_xkanjiskip_skip }
%<uptex|aptex>      { \tex_xkanjiskip:D } { \l_@@_xkanjiskip_skip }
      {
        \skip_set:Nn \l_@@_xkanjiskip_skip { \l_@@_xkanjiskip_tl }
%<luatex>        \ctex_ltj_set_xkanjiskip:N \l_@@_xkanjiskip_skip
%<uptex|aptex>        \skip_set_eq:NN \tex_xkanjiskip:D \l_@@_xkanjiskip_skip
      }
  }
\tl_new:N \l_@@_xkanjiskip_tl
\tl_set:Nn \l_@@_xkanjiskip_tl
%<luatex>  { .25\zw plus 1pt minus 1pt }
%<uptex|aptex>  { .25zw plus 1pt minus 1pt }
\skip_new:N \l_@@_xkanjiskip_skip
\skip_set:Nn \l_@@_xkanjiskip_skip
%<luatex>  { \ltjgetparameter { xkanjiskip } }
%<uptex|aptex>  { \tex_xkanjiskip:D }
%    \end{macrocode}
% \end{variable}
% \end{macro}
%
%    \begin{macrocode}
\ctex_add_to_selectfont:n { \ctex_update_xkanjiskip: }
%</luatex|uptex|aptex>
%    \end{macrocode}
%
% \begin{macro}[int]{\cht,\cdp,\cwd}
% 分别从 \file{.jfm} 中读取字符高度、深度和宽度,目前仅考虑横排的情况。
%    \begin{macrocode}
%<*luatex>
\dim_new:N \cht
\dim_new:N \cdp
\dim_new:N \cwd
\newluafunction \g_@@_kanjisize_func
\group_begin:
\char_set_catcode_space:n { 32 }
\lua_now:e
  {
    local nulltable = { }
    local t = lua.get_functions_table()
    local fmt = luatexja.jfont.font_metric_table
    local getattribute = tex.getattribute
    local setdimen = tex.setdimen
    t[\int_use:N \g_@@_kanjisize_func] = function ()
      local ft = fmt[getattribute('ltj@curjfnt')] or nulltable
      local ft = ft and ft.char_type or nulltable
      local fk = ft and ft[0] or nulltable
      setdimen('cht', fk.height or 0)
      setdimen('cdp', fk.depth or 0)
      setdimen('cwd', fk.width or ft.zw or 0)
    end
  }
\group_end:
\cs_new_protected_nopar:Npn \ctex_update_kanjisize:
 { \tex_luafunction:D \g_@@_kanjisize_func }
\ctex_add_to_selectfont:n { \ctex_update_kanjisize: }
%</luatex>
%    \end{macrocode}
% \end{macro}
%
% \begin{macro}{space}
% 在导言区或正文中设置忽略空格方式。
% \pdfTeX{} 和 \XeTeX{} 下初始设置为 \opt{auto},\LuaTeX{}、\upTeX{} 下是无效
% 选项。
%    \begin{macrocode}
\keys_define:nn { ctex }
  {
%<*pdftex|xetex>
    space .choice: ,
    space / true  .code:n =
%<pdftex>      { \ctex_ignorespaces_case:N \prg_do_nothing: } ,
%<xetex>      { \xeCJKsetup { CJKspace = true } } ,
    space / auto  .code:n =
%<pdftex>      { \ctex_ignorespaces_case:N \ctex_auto_ignorespaces: } ,
%<xetex>      { \xeCJKsetup { CJKspace = false } } ,
    space / false .code:n =
%<pdftex>      { \ctex_ignorespaces_case:N \tex_ignorespaces:D } ,
%<xetex>      { \xeCJKsetup { CJKspace = false } } ,
    space .default:n = { true } ,
    space .initial:n = { auto }
%</pdftex|xetex>
%<*luatex|uptex|aptex>
    space .code:n =
      { \msg_warning:nn { ctex } { invalid-option } }
%</luatex|uptex|aptex>
  }
%    \end{macrocode}
% \end{macro}
%
% \begin{macro}{punct}
% 在导言区或正文中设置标点符号输出格式。\pkg{LuaTeX-ja} 设置的是字体的默认 \texttt{JFM},
% 只会影响到之后设置的字体。\upTeX{} 暂时无效。
%    \begin{macrocode}
\keys_define:nn { ctex }
  {
    punct .code:n =
      {
        \tl_set:Nx \l_@@_punct_tl {#1}
%<pdftex>        \punctstyle { \l_@@_punct_tl }
%<xetex>        \xeCJKsetup { PunctStyle = \l_@@_punct_tl }
%<luatex>        \ctex_mono_jfm:o { \l_@@_punct_tl }
%<uptex|aptex>        \msg_warning:nn { ctex } { invalid-option }
      } ,
    punct .default:n = { quanjiao } ,
  }
%    \end{macrocode}
% \end{macro}
%
%    \begin{macrocode}
%</pdftex|xetex|luatex|uptex|aptex>
%    \end{macrocode}
%
% \subsection{载入引擎定义文件}
%
% 最后载入各个编译引擎的定义文件。
%    \begin{macrocode}
%<class|ctex>\ctex_file_input:n { \c_@@_engine_file_str }
%    \end{macrocode}
