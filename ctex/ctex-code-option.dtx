% \iffalse meta-comment
%
% Copyright (C) 2003--2020
% CTEX.ORG and any individual authors listed elsewhere in this file.
% --------------------------------------------------------------------------
%
% This work may be distributed and/or modified under the
% conditions of the LaTeX Project Public License, either
% version 1.3c of this license or (at your option) any later
% version. This version of this license is in
%    http://www.latex-project.org/lppl/lppl-1-3c.txt
% and the latest version of this license is in
%    http://www.latex-project.org/lppl.txt
% and version 1.3 or later is part of all distributions of
% LaTeX version 2005/12/01 or later.
%
% This work has the LPPL maintenance status `maintained'.
%
% The Current Maintainers of this work are Leo Liu, Qing Lee and Liam Huang.
%
% --------------------------------------------------------------------------
%
% \fi
%
% \section{宏包选项}
%
%   \begin{macrocode}
%<*class|style>
\keys_define:nn { ctex / option }
  {
%</class|style>
%    \end{macrocode}
%
% \begin{macro}{zihao}
%    \begin{macrocode}
%<*class|ctex|ctexsize>
    zihao .choice: ,
    zihao .value_required:n = true ,
    zihao /     5  .code:n = { \int_gset:Nn \g_@@_font_size_int { 0 } } ,
    zihao /    -4  .code:n = { \int_gset:Nn \g_@@_font_size_int { 1 } } ,
    zihao / false  .code:n = { \int_gset:Nn \g_@@_font_size_int { 2 } } ,
%<ctexsize>  }
%</class|ctex|ctexsize>
%<*class|ctex>
    c5size  .code:n =
      {
        \msg_warning:nnn { ctex } { deprecated-option }
          { Option~ `zihao=5'~ is~ set. }
        \keys_set:nn { ctex / option } { zihao = 5 }
      } ,
    cs4size  .code:n =
      {
        \msg_warning:nnn { ctex } { deprecated-option }
          { Option~ `zihao=-4'~ is~ set. }
        \keys_set:nn { ctex / option } { zihao = -4 }
      } ,
    c5size  .value_forbidden:n = true ,
    cs4size .value_forbidden:n = true ,
%    \end{macrocode}
% \end{macro}
%
% \begin{macro}{linespread}
% 行距初始值为标志 \texttt{nan},用于检查用户是否设置了 \opt{linespread} 选项。
%    \begin{macrocode}
    linespread  .fp_set:N = \l_@@_line_spread_fp ,
    linespread .initial:n = { \c_nan_fp } ,
    linespread .value_required:n = true ,
%    \end{macrocode}
% \end{macro}
%
% \begin{macro}{autoindent}
% 自动调整段落的首行缩进功能。
%    \begin{macrocode}
    autoindent .choice: ,
    autoindent .default:n = { true } ,
    autoindent / true    .code:n =
      {
        \tl_set:Nn \l_@@_autoindent_tl { 2 \ccwd }
        \cs_set_eq:NN \ctex_if_autoindent_touched:F \use_none:n
      } ,
    autoindent / false   .code:n =
      {
        \tl_clear:N \l_@@_autoindent_tl
        \cs_set_eq:NN \ctex_if_autoindent_touched:F \use_none:n
      } ,
    autoindent / unknown .code:n =
      {
        \ctex_set_default_ccwd:Nn \l_@@_autoindent_tl {#1}
        \cs_set_eq:NN \ctex_if_autoindent_touched:F \use_none:n
      } ,
%    \end{macrocode}
% \end{macro}
%
% \begin{macro}{indent}
% 仅为兼容性保留,已过时。
%    \begin{macrocode}
    indent .code:n =
      {
        \msg_warning:nnn { ctex } { deprecated-option }
          {
            The~ functionality~ has~ been~ removed.\\
            It's~ better~ to~ set~ the~ heading~ styles~ via~ `afterindent'~
            options.~
          }
      } ,
    indent .value_forbidden:n = true ,
    noindent .code:n =
      {
        \msg_warning:nnn { ctex } { deprecated-option }
          {
            The~ functionality~ has~ been~ removed.\\
            It's~ better~ to~ set~ the~ heading~ styles~ via~ `afterindent'~
            options.
          }
      } ,
    noindent .value_forbidden:n = true ,
%    \end{macrocode}
% \end{macro}
%
% \changes{v2.5}{2019/11/10}{所有引擎下默认编码均设为 UTF-8。}
%
% \begin{macro}{GBK,UTF8}
% 文档编码,默认为 UTF-8。
%   \begin{macrocode}
    GBK  .code:n =
      {
        \sys_if_engine_pdftex:TF
          { \tl_gset:Nn \g_@@_encoding_tl { GBK } }
          {
            \msg_warning:nn { ctex } { invalid-option }
            \tl_gset:Nn \g_@@_encoding_tl { UTF8 }
          }
      } ,
    UTF8 .code:n = { \tl_gset:Nn \g_@@_encoding_tl { UTF8 } } ,
    GBK  .value_forbidden:n = true ,
    UTF8 .value_forbidden:n = true ,
%    \end{macrocode}
% \end{macro}
%
% \begin{macro}{fontset}
% 初始值为空。若用户未指定,则根据操作系统载入对应字体配置,可以区分 Windows、
% Mac~OS~X 和其它。
%   \begin{macrocode}
    fontset    .tl_gset:N = \g_@@_fontset_tl ,
    nofonts    .code:n =
      {
        \msg_warning:nnn { ctex } { deprecated-option }
          {
            Option~ `fontset=none'~ is~ set.~ It~ is~ better~ to~ use~
            fontset~ option.
          }
        \keys_set:nn { ctex / option } { fontset = none }
      } ,
    adobefonts .code:n =
      {
        \msg_warning:nnn { ctex } { deprecated-option }
          {
            Option~ `fontset=adobe'~ is~ set.~ It~ is~ better~ to~ use~
            fontset~ option.
          }
        \keys_set:nn { ctex / option } { fontset = adobe }
      } ,
    winfonts   .code:n =
      {
        \msg_warning:nnn { ctex } { deprecated-option }
          {
            Option~ `fontset=windows'~ is~ set.~ It~ is~ better~ to~ use~
            fontset~ option.
          }
        \keys_set:nn { ctex / option } { fontset = windows }
      } ,
    nofonts    .value_forbidden:n = true ,
    winfonts   .value_forbidden:n = true ,
    adobefonts .value_forbidden:n = true ,
%    \end{macrocode}
% \end{macro}
%
% \begin{macro}{zhmap}
%   \begin{macrocode}
    zhmap .choice: ,
    zhmap .default:n = { true } ,
    zhmap / zhmCJK .code:n =
      {
        \bool_gset_true:N \g_@@_zhmCJK_bool
        \cs_gset_eq:NN \ctex_zhmap_case:nnn \use_i:nnn
      } ,
    zhmap / true   .code:n =
      {
        \bool_gset_false:N \g_@@_zhmCJK_bool
        \cs_gset_eq:NN \ctex_zhmap_case:nnn \use_ii:nnn
      } ,
    zhmap / false  .code:n =
      {
        \bool_gset_false:N \g_@@_zhmCJK_bool
        \cs_gset_eq:NN \ctex_zhmap_case:nnn \use_iii:nnn
      } ,
    nozhmap   .code:n =
      {
        \msg_warning:nnn { ctex } { deprecated-option }
          { Option~ `zhmap=false'~ is~ set. }
        \keys_set:nn { ctex / option } { zhmap = false }
      } ,
    nozhmap   .value_forbidden:n = true ,
%    \end{macrocode}
% \end{macro}
%
% \begin{macro}{punct}
% 设置标点符号输出格式。
%   \begin{macrocode}
    punct   .tl_set:N = \l_@@_punct_tl ,
    punct  .default:n = { quanjiao } ,
    punct  .initial:n = { quanjiao } ,
    nopunct   .code:n =
      {
        \msg_warning:nnn { ctex } { deprecated-option }
          { Option~ `punct=plain'~ is~ set.  }
        \keys_set:nn { ctex / option } { punct = plain }
      } ,
    nopunct   .value_forbidden:n = true ,
%    \end{macrocode}
% \end{macro}
%
% \begin{macro}{space}
%   \begin{macrocode}
    space .choices:nn =
      { true , auto , false }
      {
        \exp_args:Nx \ctex_at_end:n
          { \keys_set:nn { ctex } { space = \l_keys_choice_tl } }
      } ,
    space  .default:n = { true } ,
    nospace   .code:n =
      {
        \msg_warning:nnn { ctex } { deprecated-option }
          { Option~ `space=false'~ is~ set. }
        \keys_set:nn { ctex / option } { space = false }
      } ,
    nospace   .value_forbidden:n = true ,
%    \end{macrocode}
% \end{macro}
%
% \begin{macro}{heading}
%   \begin{macrocode}
    heading .bool_set:N = \l_@@_heading_bool ,
%    \end{macrocode}
% \end{macro}
%
%    \begin{macrocode}
%</class|ctex>
%<*class|ctex|ctexheading>
%    \end{macrocode}
%
% \begin{macro}{sub3section,sub4section}
% \begin{macrocode}
%<*!beamer>
    sub3section .code:n =
      { \int_gset:Nn \g_@@_section_depth_int { 3 } } ,
    sub4section .code:n =
      { \int_gset:Nn \g_@@_section_depth_int { 4 } } ,
    sub3section .value_forbidden:n = true ,
    sub4section .value_forbidden:n = true ,
%</!beamer>
%    \end{macrocode}
% \end{macro}
%
% \begin{macro}{scheme}
%    \begin{macrocode}
    scheme .tl_set:N  = \l_@@_scheme_tl ,
%<*ctexheading>
    scheme .default:n = { plain } ,
    scheme .initial:n = { plain }
  }
%</ctexheading>
%<*!ctexheading>
    scheme .default:n = { chinese } ,
    scheme .initial:n = { chinese } ,
%</!ctexheading>
%    \end{macrocode}
% \end{macro}
%
%    \begin{macrocode}
%</class|ctex|ctexheading>
%<*class|ctex>
%    \end{macrocode}
%
% \begin{macro}{cap,nocap}
% \opt{cap} 和 \opt{nocap} 是过时选项。
% \begin{macrocode}
    cap    .code:n    =
      {
        \msg_warning:nnn { ctex } { deprecated-option }
          { Option~ `scheme = chinese' ~ is~ set. }
        \keys_set:nn { ctex / option } { scheme = chinese }
      } ,
    nocap  .code:n    =
      {
        \msg_warning:nnn { ctex } { deprecated-option }
          { Option~ `scheme = plain' ~ is~ set. }
        \keys_set:nn { ctex / option } { scheme = plain }
      } ,
    cap    .value_forbidden:n = true ,
    nocap  .value_forbidden:n = true ,
%    \end{macrocode}
% \end{macro}
%
% \begin{macro}{fntef,fancyhdr,hyperref}
% 这些都是过时的宏包兼容选项,原选项功能总是打开的。
%   \begin{macrocode}
    fntef .code:n =
      {
        \sys_if_engine_xetex:TF
          {
            \msg_warning:nnn { ctex } { deprecated-option }
              { `xeCJKfntef'~package~is~loaded. }
            \RequirePackage { xeCJKfntef }
          }
          {
            \sys_if_engine_pdftex:TF
              {
                \msg_warning:nnn { ctex } { deprecated-option }
                  { `CJKfntef'~package~is~loaded. }
                \RequirePackage { CJKfntef }
              }
              {
                \msg_warning:nnn { ctex } { deprecated-option }
                  { Furthermore,~option~`fntef'~is~invalid~in~current~mode. }
              }
          }
      } ,
%    \end{macrocode}
% \end{macro}
%
% \begin{macro}{fancyhdr}
%   \begin{macrocode}
    fancyhdr .code:n =
      {
        \msg_warning:nnn { ctex } { deprecated-option }
          { `fancyhdr'~package~is~loaded. }
        \RequirePackage { fancyhdr }
      } ,
%    \end{macrocode}
% \end{macro}
%
% \begin{macro}{hyperref}
% \changes{v2.1}{2015/06/03}{补充定义 \tn{hypersetup}。}
%   \begin{macrocode}
    hyperref .code:n =
      {
        \msg_warning:nnn { ctex } { deprecated-option }
          { `hyperref'~package~will~be~loaded. }
        \ctex_at_end:n
          {
            \cs_if_exist:NF \hypersetup
              { \cs_new_eq:NN \hypersetup \ctex_hypersetup:n }
          }
        \ctex_at_end_preamble:n { \RequirePackage { hyperref } }
      } ,
  }
%    \end{macrocode}
% \end{macro}
%
%    \begin{macrocode}
%</class|ctex>
%<*class|ctex|ctexsize>
%    \end{macrocode}
%
% \begin{macro}{10pt,11pt,12pt}
% 使 \pkg{ctex} 和 \pkg{ctexsize} 可以接受文档类的全局选项,不修改默认字体大小。
% 在文档类下还将参数传给标准文档类。
%    \begin{macrocode}
\tl_clear_new:N \l_@@_tmp_tl
\clist_map_inline:nn
  {
    10pt , 11pt , 12pt ,
     8pt ,  9pt , 14pt , 17pt , 20pt , 25pt , 30pt , 36pt , 48pt , 60pt
  }
  {
    \tl_put_right:Nn \l_@@_tmp_tl
      {
        #1 .code:n =
%<*!class>
          { \int_gset:Nn \g_@@_font_size_int { 2 } } ,
%</!class>
%<*class>
          {
            \int_gset:Nn \g_@@_font_size_int { 2 }
            \clist_gput_right:Nn \g_@@_std_options_clist {#1}
          } ,
%</class>
        #1 .value_forbidden:n = true ,
      }
  }
\use:x { \keys_define:nn { ctex / option } { \exp_not:o { \l_@@_tmp_tl } } }
\tl_clear:N \l_@@_tmp_tl
%    \end{macrocode}
% \end{macro}
%
% 将未知选项传给标准文档类。
%    \begin{macrocode}
%<*class>
\keys_define:nn { ctex / option }
  {
    unknown .code:n =
      { \clist_gput_right:No \g_@@_std_options_clist { \CurrentOption } }
  }
%</class>
%    \end{macrocode}
%
% 载入选项配置文件。
%    \begin{macrocode}
%<!ctexsize>\ctex_file_input:n { ctexopts.cfg }
%</class|ctex|ctexsize>
%    \end{macrocode}
%
% 处理宏包选项。
%    \begin{macrocode}
%<*class|style>
\ProcessKeysOptions { ctex / option }
%</class|style>
%    \end{macrocode}
%
% \pdfLaTeX{} 下,如果没有显式指定编码为 |UTF8|,则给出警告信息。
%    \begin{macrocode}
%<*class|ctex>
\msg_new:nnn { ctex } { pdftex-utf8 }
  { UTF8~will~be~used~as~the~default~encoding. }
\sys_if_engine_pdftex:T
  {
    \clist_if_in:NnF \CurrentOption { UTF8 }
      {
        \clist_if_in:NnF \CurrentOption { GBK }
          { \msg_warning:nn { ctex } { pdftex-utf8 } }
      }
  }
%</class|ctex>
%    \end{macrocode}
%
%    \begin{macrocode}
%<*class>
%    \end{macrocode}
%
% 五号字使用标准文档类的 |10pt| 字体大小设置,小四号字则使用 |12pt|。
%    \begin{macrocode}
\if_case:w \g_@@_font_size_int
  \clist_gput_right:Nn \g_@@_std_options_clist { 10pt }
\or:
  \clist_gput_right:Nn \g_@@_std_options_clist { 12pt }
\fi:
%    \end{macrocode}
%
% 使用 \tn{PassOptionsToClass} 是为了预防可能存在的选项冲突。
%    \begin{macrocode}
%<*article>
\tl_const:Nn \c_@@_class_tl { article }
\PassOptionsToClass { \g_@@_std_options_clist } { article }
\LoadClass { article }
%</article>
%<*book>
\tl_const:Nn \c_@@_class_tl { book }
\PassOptionsToClass { \g_@@_std_options_clist } { book }
\LoadClass { book }
%</book>
%<*report>
\tl_const:Nn \c_@@_class_tl { report }
\PassOptionsToClass { \g_@@_std_options_clist } { report }
\LoadClass { report }
%</report>
%<*beamer>
\tl_const:Nn \c_@@_class_tl { beamer }
\PassOptionsToClass { \g_@@_std_options_clist } { beamer }
\LoadClass { beamer }
%</beamer>
%    \end{macrocode}
%
%    \begin{macrocode}
%</class>
%    \end{macrocode}
