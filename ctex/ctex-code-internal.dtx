% \iffalse meta-comment
%
% Copyright (C) 2003--2020
% CTEX.ORG and any individual authors listed elsewhere in this file.
% --------------------------------------------------------------------------
%
% This work may be distributed and/or modified under the
% conditions of the LaTeX Project Public License, either
% version 1.3c of this license or (at your option) any later
% version. This version of this license is in
%    http://www.latex-project.org/lppl/lppl-1-3c.txt
% and the latest version of this license is in
%    http://www.latex-project.org/lppl.txt
% and version 1.3 or later is part of all distributions of
% LaTeX version 2005/12/01 or later.
%
% This work has the LPPL maintenance status `maintained'.
%
% The Current Maintainers of this work are Leo Liu, Qing Lee and Liam Huang.
%
% --------------------------------------------------------------------------
%
% \fi
%
% \section{内部函数与变量}
%
% \begin{variable}{\l_@@_tmp_tl,\l_@@_tmp_int,\l_@@_tmp_box,\l_@@_tmp_dim}
% 临时变量。
%    \begin{macrocode}
%<*class|ctex|ctexheading>
\tl_clear_new:N \l_@@_tmp_tl
\int_new:N \l_@@_tmp_int
\box_new:N \l_@@_tmp_box
%<!ctexheading>\dim_new:N \l_@@_tmp_dim
%    \end{macrocode}
% \end{variable}
%
% \begin{macro}[int]{\ctex_file_wrapper:nnn}
% 设置文件操作的 \tn{catcode} 环境,参数 |#1| 是设置,|#2| 是文件操作,|#3| 是恢复。
% 默认关闭 \LaTeXiii{} 语法环境,并设置 |@| 的 \tn{catcode} 为 $11$。
%    \begin{macrocode}
\cs_new_protected:Npn \ctex_file_wrapper:nnn #1#2#3
  {
    \use:x
      {
        \ExplSyntaxOff
        \char_set_catcode_letter:n { 64 }
        #1
        \exp_not:n {#2}
        \bool_if:NTF \l__kernel_expl_bool
          { \ExplSyntaxOn }
          { \ExplSyntaxOff }
        \char_set_catcode:nn { 64 } { \char_value_catcode:n { 64 } }
        #3
      }
  }
%    \end{macrocode}
% \end{macro}
%
% \begin{macro}[int]{\ctex_file_input:n}
% 输入文件。
%    \begin{macrocode}
\cs_new_protected_nopar:Npn \ctex_file_input:n #1
  { \ctex_file_wrapper:nnn { } { \file_input:n {#1} } { } }
%    \end{macrocode}
% \end{macro}
%
% \begin{macro}[int]{\ctex_scheme_input:n}
% 输入 \opt{scheme} 文件。先查找当前文档类下的 \meta{scheme},找不到再查找一般的文件。
%    \begin{macrocode}
\cs_new_protected_nopar:Npn \ctex_scheme_input:n #1
  {
    \ctex_file_wrapper:nnn
      { }
      {
        \tl_if_exist:NTF \c_@@_class_tl
          {
            \file_if_exist_input:nF { ctex-scheme- #1 - \c_@@_class_tl .def }
              { \file_input:n  { ctex-scheme- #1 .def } }
          }
          { \file_input:n  { ctex-scheme- #1 .def } }
      }
      { }
  }
\cs_generate_variant:Nn \ctex_scheme_input:n { o }
%    \end{macrocode}
% \end{macro}
%
% \begin{variable}{\g_@@_section_depth_int}
% 若大于 |3|,则 \tn{paragraph} 和 \tn{subparagraph} 标题单独占一行;若为 |3|,则
% \tn{paragraph} 单独占一行。
%    \begin{macrocode}
%<*!beamer>
\int_new:N \g_@@_section_depth_int
\int_gset:Nn \g_@@_section_depth_int { 2 }
%</!beamer>
%    \end{macrocode}
% \end{variable}
%
%    \begin{macrocode}
%</class|ctex|ctexheading>
%<*class|ctex>
%    \end{macrocode}
%
% 对旧版本的宏包给出错误信息。
%    \begin{macrocode}
\msg_new:nnnn { ctex } { package-too-old }
  { Support~package~`#1'~too~old. }
  {
    Please~update~an~up~to~date~version~of~the~package~`#1'\\
    using~your~TeX~package~manager~or~from~CTAN.
  }
%    \end{macrocode}
%
% \begin{macro}[int]{\ifctexpdf}
% 在 \pkg{zhmetrics} 映射文件中使用。
%    \begin{macrocode}
\sys_if_output_pdf:TF
  { \cs_new_eq:NN \ifctexpdf \if_true: }
  { \cs_new_eq:NN \ifctexpdf \if_false: }
%    \end{macrocode}
% \end{macro}
%
% \begin{macro}[int]{\ctex_if_preamble:TF}
% 测试是否在 \LaTeXe{} 的导言区。在宏包内部初始为真,文档最开始位置再设置为假。
% 注意,钩子 \cs{ctex_after_end_preamble:n} 在 \tn{AtBeginDocument} 之后执行,
% 可以与 \tn{@onlypreamble} 的行为一致。
%    \begin{macrocode}
\cs_new_eq:NN \ctex_if_preamble:TF \use_i:nn
\ctex_after_end_preamble:n { \cs_set_eq:NN \ctex_if_preamble:TF \use_ii:nn }
%    \end{macrocode}
% \end{macro}
%
% \begin{macro}[int]{\ctex_set_default_ccwd:Nn}
% 若参数 |#2| 带长度单位,则设置它为 |tl| 变量 |#1| 的值,否则以 \tn{ccwd} 为单位。
%    \begin{macrocode}
\cs_new_protected:Npn \ctex_set_default_ccwd:Nn #1#2
  { \tl_set:Nx #1 { \@@_default_ccwd_aux:n {#2} } }
\cs_new:Npn \@@_default_ccwd_aux:n #1
  {
    \exp_not:n {#1}
    \exp_after:wN \@@_default_ccwd_aux:w
      \dim_use:N \tex_dimexpr:D #1 pt \scan_stop: \q_stop
  }
\exp_last_unbraced:NNNNo
  \cs_new:Npn \@@_default_ccwd_aux:w #1 { \tl_to_str:n { pt } } #2 \q_stop
    { \tl_if_empty:nT {#2} { \ccwd } }
%    \end{macrocode}
% \end{macro}
%
% \begin{variable}{\g_@@_encoding_tl}
% 所有引擎下默认编码均设为 UTF-8。
%    \begin{macrocode}
\tl_new:N \g_@@_encoding_tl
\tl_set:Nn \g_@@_encoding_tl { UTF8 }
%    \end{macrocode}
% \end{variable}
%
% \begin{variable}{\g_@@_zhmCJK_bool}
% 是否使用 \pkg{zhmCJK} 宏包。
%    \begin{macrocode}
\bool_new:N \g_@@_zhmCJK_bool
%    \end{macrocode}
% \end{variable}
%
% \begin{variable}{\l_@@_autoindent_tl}
% 保存 \opt{autoindent} 选项的值,空值表示不自动调整首行缩进。
%    \begin{macrocode}
\tl_new:N \l_@@_autoindent_tl
%    \end{macrocode}
% \end{variable}
%
% \begin{macro}[int]{\ctex_if_autoindent_touched:F}
% 检查 \opt{autoindent} 选项是否被用户设置。
%    \begin{macrocode}
\cs_new_eq:NN \ctex_if_autoindent_touched:F \use:n
%    \end{macrocode}
% \end{macro}
%
% \begin{macro}[int]{\ctex_zhmap_case:nnn}
% 参数 |#1| 是 \pkg{zhmCJK} 的内容,|#2| 是 \pkg{zhmetrics}。
%    \begin{macrocode}
\cs_new_eq:NN \ctex_zhmap_case:nnn \use_ii:nnn
%    \end{macrocode}
% \end{macro}
%
% \begin{macro}[int]{\ctex_at_end:n}
% 区分 \tn{AtEndOfClass} 和 \tn{AtEndOfPackage},虽然它们的意思都是一样的。
%    \begin{macrocode}
%<class>\cs_new_protected_nopar:Npn \ctex_at_end:n { \AtEndOfClass }
%<ctex>\cs_new_protected_nopar:Npn \ctex_at_end:n { \AtEndOfPackage }
%    \end{macrocode}
% \end{macro}
%
% \begin{variable}{\g_@@_std_options_clist}
% 保存传递给标准文档类的选项。
%    \begin{macrocode}
%<*class>
\clist_new:N \g_@@_std_options_clist
%</class>
%    \end{macrocode}
% \end{variable}
%
% 对无效选项给出警告。
%    \begin{macrocode}
\msg_new:nnn { ctex } { invalid-option }
  { Option~`\l_keys_key_tl'~is~invalid~in~current~mode. }
\msg_new:nnn { ctex } { invalid-value }
  { Value~`#1'~is~invalid~for~the~key~`\l_keys_key_tl'. }
%    \end{macrocode}
%
% 对过时选项或命令给出警告。
%    \begin{macrocode}
\msg_new:nnn { ctex } { deprecated-option }
  { Option~ `\l_keys_key_tl'~ is~ deprecated.\\ #1 }
\msg_new:nnn { ctex } { deprecated-command }
  { Command~ #1 is~ deprecated.\\ #2 }
\msg_new:nnn { ctex } { deprecated-environment }
  { Environment~ `#1'~ is~ deprecated.\\ #2 }
%    \end{macrocode}
%
%    \begin{macrocode}
%</class|ctex>
%    \end{macrocode}
%
% \begin{variable}{\g_@@_font_size_int}
% |0| 表示修改默认字体大小为五号,|1| 为小四号,大于 1 则不作修改。初始值 |-1|
% 表示 \opt{zihao} 选项未初始化,会在将来根据文档类决定初值。
%    \begin{macrocode}
%<*class|ctex|ctexsize>
\int_new:N \g_@@_font_size_int
\int_set:Nn \g_@@_font_size_int { -1 }
%</class|ctex|ctexsize>
%    \end{macrocode}
% \end{variable}
%
% \section{宏包选项}
