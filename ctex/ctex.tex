% !TeX program  = XeLaTeX
% !TeX encoding = UTF-8
%
% Copyright (C) 2003--2020
% CTEX.ORG and any individual authors listed elsewhere in this file.
% --------------------------------------------------------------------------
%
% This work may be distributed and/or modified under the
% conditions of the LaTeX Project Public License, either
% version 1.3c of this license or (at your option) any later
% version. This version of this license is in
%    http://www.latex-project.org/lppl/lppl-1-3c.txt
% and the latest version of this license is in
%    http://www.latex-project.org/lppl.txt
% and version 1.3 or later is part of all distributions of
% LaTeX version 2005/12/01 or later.
%
% This work has the LPPL maintenance status `maintained'.
%
% The Current Maintainers of this work are Leo Liu, Qing Lee and Liam Huang.
%
% --------------------------------------------------------------------------
%

\documentclass{ctxdoc}

\GetFileId{ctex.sty}

\title{\bfseries \CTeX{} 宏集手册}
\author{\href{http://www.ctex.org}{CTEX.ORG}}
\date{\filedate\qquad\fileversion}


\begin{document}

\maketitle

\begin{abstract}
\CTeX{} 宏集是面向中文排版的通用 \LaTeX{} 排版框架,为中文 \LaTeX{} 文档
提供了汉字输出支持、标点压缩、字体字号命令、标题文字汉化、中文版式调整、数字
日期转换等支持功能,可适应论文、报告、书籍、幻灯片等不同类型的中文文档。

\CTeX{} 宏集支持 \LaTeX、\pdfLaTeX、\XeLaTeX、\LuaLaTeX、\upLaTeX{} 等多种不同
的编译方式,并为它们提供了统一的界面。主要功能由宏包 \pkg{ctex} 和中文文档类
\cls{ctexart}、\cls{ctexrep}、\cls{ctexbook} 和 \cls{ctexbeamer} 实现。
\end{abstract}

\tableofcontents

\clearpage
\setlength{\parskip}{0.8ex}

\begin{documentation}

\section{介绍}

\subsection*{历史}

\CTeX{} 宏集的源头有两个:一是王磊编写的 \cls{cjkbook} 文档类,二是吴凌云编写的
\file{GB.cap}。
这些工作没有经过认真、系统的设计,也没有用户文档,不利于维护和改进。

2003 年,吴凌云使用 \pkg{doc} 和 \pkg{DocStrip} 重构了整个工程,并增加了许多新的功能,
称为 \pkg{ctex} 宏包。2007 年,oseen 和王越在 \pkg{ctex} 宏包的基础上,
增加了对 UTF-8 编码的支持,开发出了 \pkg{ctexutf8} 宏包。

2009 年 5 月,我们在 Google Code 建立了 ctex-kit 项目^^A
\footnote{\nolinkurl{http://code.google.com/p/ctex-kit/}},
对 \pkg{ctex} 宏包及相关脚本进行了整合,并加入了对 \XeTeX{} 引擎的支持。
在开发新版本时,考虑到合作开发和调试的方便,我们放弃了 \pkg{doc} 和 \pkg{DocStrip},
采取了直接编写宏包代码的方式。

2014 年 3 月,为了适应 \LaTeX{} 的最新发展,特别是 \LaTeXiii{} 的逐渐成熟,李清用
\LaTeXiii{} 重构了整个宏包的代码,并重新使用 \pkg{doc} 和 \pkg{DocStrip} 工具进行代码
的管理,升级版本号为 2.0,并改称 \CTeX{} 宏集。

2015 年 3 月,由于 Google Code 即将停止服务,ctex-kit 项目迁移至
\href{https://github.com/CTeX-org/ctex-kit}{GitHub}^^A
\footnote{\url{https://github.com/CTeX-org/ctex-kit}}。

最初,Knuth 在设计开发 \TeX{} 的时候没有考虑到多国语言支持,特别是对多字节的中日韩
语言的支持。这使得 \TeX{} 以至后来的 \LaTeX{} 对中文的支持一直不是很好。即使在
\pkg{CJK} 宏包解决了中文字符处理的问题以后,中文用户使用 \LaTeX{} 仍然要面对许
多困难。
这些困难里,以章节标题的中文化为最。由于中文和西文语言习惯的差异,用户很难使用标准
文档类中的代码结构来表达中文标题。于是,用户不得不对标准文档类做较大的修改。
除此之外,日期格式、首行缩进、中文字号和字距等细节问题,也需要精细的调校。
我们设计 \CTeX{} 宏集的目的之一就是解决这些 \LaTeX{} 文档的汉化难题。

另一方面,随着 \TeX{} 引擎和 \LaTeX{} 宏包的不断发展,\LaTeX{} 的中文支持方式从早期的
专用系统(如 \pkg{CCT})发展为适用于不同引擎的多种方式^^A
\footnote{比如:\pdfTeX{} 引擎下的 \pkg{CJK}、\pkg{zhmCJK}宏包,
\XeTeX{} 引擎下的 \pkg{xeCJK} 宏包和 \LuaTeX{} 引擎下的 \pkg{LuaTeX-ja} 宏
包}。这些方式的适用情况和使用方式有不少细节上的差异,同时操作系统的不同、语言环境的不同等
客观情况又进一步带来了更多的细节差异。我们设计 \CTeX{} 宏集的另一个主要目的就是
尽可能消除这些差异带来的影响,使用户能够以一个统一的接口来使用不同的中文支持方式,
使得同一份文档能够在不同环境下交换使用。

\CTeX{} 宏集的许多实现细节离不开热心朋友们在 \url{bbs.ctex.org} 论坛上的讨论,
在此对参与讨论的朋友们表示感谢。

\subsection*{关于宏集名字的说明}

\CTeX{} 之名是英文单词 China(中国)或 Chinese(中文)的首字母“C”与 “\TeX{}”
结合而成的。在纯文本环境下,该名字应写作“CTeX”。

\CTeX{} 宏集是由 \href{http://bbs.ctex.org}{\CTeX{} 社区} 发起并维护的
\LaTeX{} \emph{宏包和文档类的集合}。
社区另有发布名为 \href{http://www.ctex.org/CTeX}{\CTeX{} 套装}
的 \TeX{} 发行版,与本文档所述的 \CTeX{} 宏集并非是同一事物。

\pkg{ctex} 则是本宏集中的 \pkg{ctex.sty} 的名字。这一完全小写的名称,在过去
也被用来指代整个 \CTeX{} 宏集,不过现在则特指 \pkg{ctex.sty} 这一宏包。
在不引起歧义的情况下,它也可以沿用过去的习惯,代指整个宏集。

\section{简明教程}

\subsection{\CTeX{} 宏集的组成}

为了适应用户不同的需求,我们将 \CTeX{} 宏集的主要功能分散在四个中文文档类和
三个宏包当中,具体的组成见表~\ref{tab:ctex}。

\begin{table}[htbp]
\centering
\caption{\CTeX{} 宏集的组成}\label{tab:ctex}
\begin{tabularx}{\linewidth}{llX}
\toprule
  类别   & 文件 & 说明 \\
\midrule
  文档类 & \file{ctexart.cls}  & 标准文档类 \cls{article} 的汉化版本,一般适用于
                                 短篇幅的文章 \\
         & \file{ctexrep.cls}  & 标准文档类 \cls{report} 的汉化版本,一般适用于
                                 中篇幅的报告 \\
         & \file{ctexbook.cls} & 标准文档类 \cls{book} 的汉化版本,一般适用于
                                 长篇幅的书籍 \\
         & \file{ctexbeamer.cls} & 文档类 \cls{beamer} 的汉化版本,适用于
                                   幻灯片演示 \\
\midrule
  宏包   & \file{ctex.sty}     & 提供全部功能,但\emph{默认不开启章节标题设置功能},
                                 需要使用 \opt{heading} 选项来开启 \\
         & \file{ctexsize.sty} & 定义和调整中文字号,在 \pkg{ctex} 宏包
                                 或 \CTeX{} 中文文档类之外单独调用 \\
         & \file{ctexheading.sty} & 提供章节标题设置功能(见 \ref{sec:secstyle}
                                    节),在 \pkg{ctex} 宏包或 \CTeX{} 中文
                                    文档类之外单独调用 \\
\bottomrule
\end{tabularx}
\end{table}

\subsection{\CTeX{} 宏集的安装和更新}
\label{subsec:easy-ins}

\CTeX{} 宏集依赖的宏包和宏集已被最常见的 \TeX{} 发行版 \TeXLive{} 和 \MiKTeX{}
所收录。如果本地安装 \TeXLive{} 或 \MiKTeX{} 不是完全版本,就可能需要通过这
两个发行版提供的\emph{宏包管理器}来安装宏包。

\TeXLive{} 的宏包管理器是 tlmgr。用户可以在系统命令行中^^A
\footnote{Windows 系统的命令行是 CMD 命令提示符,你可以使用 Win + R 组合键^^A
打开“运行”对话框,然后输入 cmd 确认打开命令提示符窗口。}^^A
执行
\begin{frameverb}
  tlmgr gui
\end{frameverb}
启动管理器的图形界面(Windows 用户也可以通过开始菜单的
TeX Live 2015 \ding{212} TeX Live Manager 打开)。
连接上远程仓库之后,搜索 ctex 安装即可。
tlmgr 的图形界面使用 Perl 编写,容易造成系统假死。遇到这种问题的用户,
也可以直接在系统命令行执行
\begin{frameverb}
  tlmgr install ctex
\end{frameverb}
来安装 \CTeX{} 宏集\footnote{*nix 用户可能需要超级用户权限才能正确安装宏集。}。

\MiKTeX{} 的宏包管理器是 mpm (\MiKTeX{} Package Manager)。用户可以在开始菜单
找到 MiKTeX \ding{212} Maintenance (Admin) \ding{212} Package Manager (Admin),
打开管理器,连接上远程仓库之后,搜索 ctex 安装即可。

若希望了解 \CTeX{} 宏集具体的依赖情况和手工安装宏集的方法,
请参阅第 \ref{sec:dep-ins}~节。

当宏包发布新版本,并被发行版在远程仓库安装之后,在本地就可以通过宏包管理器来
取得新版本。

对于 \TeXLive{},可以在 tlmgr 的图形界面点击“更新全部已安装的”按钮或者在
命令行执行
\begin{frameverb}
  tlmgr update --all
\end{frameverb}
来完整更新已安装的宏包。

对于 \MiKTeX{},在开始菜单找到
MiKTeX \ding{212} Maintenance (Admin) \ding{212} Update (Admin),
按照界面说明更新即可。

\subsection{使用 \CTeX{} 文档类}

\emph{如果用户需要在标准文档类的基础上添加中文支持和中文版式支持,我们建议用户使用 \CTeX{}
宏集提供的四个中文文档类。}

\CTeX{} 宏集提供了四个中文文档类:\cls{ctexart}、\cls{ctexrep}、\cls{ctexbook} 和
\cls{ctexbeamer},分别对应 \LaTeX{} 的标准文档类 \cls{article}、\cls{report}、
\cls{book} 和 \cls{beamer}。使用它们的时候,需要将涉及到的所有源文件使用 UTF-8
编码保存\footnote{使用 (pdf)\LaTeX{} 时也能够使用 GBK 编码,但不推荐。(见
\ref{subs:encoding}~节)}。

\begin{ctexexam}
  \documentclass{ctexart}
  \begin{document}
  中文文档类测试。你需要将所有源文件保存为 UTF-8 编码。

  你可以使用 XeLaTeX、LuaLaTeX 或 upLaTeX 编译,也可以使用 (pdf)LaTeX 编译。
  推荐使用 XeLaTeX 或 LuaLaTeX 编译。
  \end{document}
\end{ctexexam}

以下是使用 \cls{ctexbeamer} 文档类编写中文演示文稿的一个示例。
\begin{ctexexam}
  \documentclass{ctexbeamer}
  \begin{document}
  \begin{frame}{中文演示文档}
  \begin{itemize}
    \item 你需要将所有源文件保存为 UTF-8 编码
    \item 你可以使用 XeLaTeX、LuaLaTeX 或 upLaTeX 编译
    \item 也可以使用 (pdf)LaTeX 编译
    \item 推荐使用 XeLaTeX 或 LuaLaTeX 编译
  \end{itemize}
  \end{frame}
  \end{document}
\end{ctexexam}

\subsection{使用 \pkg{ctex} 宏包}

\emph{用户在使用非标准文档类时,如果需要添加中文支持或中文版式支持,则可以使用 \pkg{ctex}
宏包。}

有些文档类是建立在 \LaTeX{} 标准文档类之上开发的。这时,给 \pkg{ctex} 宏包
加上 \opt{heading} 选项,可以将章节标题设置为中文风格。
\begin{ctexexam}
  \documentclass{ltxdoc}
  \usepackage[heading=true]{ctex}
  \begin{document}
  \section{简介}
  章节标题中文化的 \LaTeX{} 手册。
  \end{document}
\end{ctexexam}

\section{宏包选项与 \tn{ctexset} 命令}
\label{sec:options}

\CTeX{} 宏集已经尽可能就中文的行文和版式习惯做了调整和配置,通常而言,这些配置
已经够用。因此,除非必要,我们不建议普通用户修改这些默认配置。如果你认为 \CTeX{} 宏集
的默认配置还可以完善,可以在项目主页上%
\href{https://github.com/CTeX-org/ctex-kit/issues}{提交 issue},
向我们反映,我们会酌情在后续版本中予以改进。

不过,\CTeX{} 宏集也提供了一系列选项。用户可以使用这些选项来控制 \CTeX{} 宏集的行为。
具体来说,这些选项里,有的以传统的方式提供,
也有的以 \meta{key}|=|\meta{value} 的形式提供。对于以键值对形式提供的选项,
在下面的说明中使用\textbf{粗体}来表示 \CTeX{} 的默认设置。

另一方面,这些选项可以分为以下三类:
\begin{itemize}
\item 名字后带有 \rexptarget\rexpstar{} 号的选项,只能作为宏包/文档类选项,需要
  在引入宏包/文档类的时候指定;
\item 名字后带有 \exptarget\expstar{} 号的选项,只能通过 \CTeX{} 宏集提供的
  用户接口 \tn{ctexset} 来设定;
\item 名字后不带有特殊符号的选项,既可以作为宏包/文档类选项,也可以通过
  \tn{ctexset} 来设定。
\end{itemize}
后续文档将在使用说明中对某些特殊的选项加以说明。

\begin{function}[added=2014-03-18]{\ctexset}
  \begin{syntax}
    \tn{ctexset} \Arg{键值列表}
  \end{syntax}
是 \CTeX{} 宏集的通用控制命令,用来在宏包载入后控制宏包的各项功能。
\tn{ctexset} 的参数是一个键值列表,以通用的接口完成各项设置。
\end{function}

\tn{ctexset} 的参数是一组由逗号分隔的选项列表,列表中的选项通常是一个
\meta{key}|=|\meta{value} 格式的定义。例如设置摘要与参考文献标题名称
(\ref{subs:capname}~节)就可以使用:
\begin{ctexexam}[labelref=exam:capname]
  \ctexset{
    abstractname = {本文概要},
    bibname      = {文\quad 献}
  }
\end{ctexexam}

\tn{ctexset} 采用 \LaTeXiii{} 风格的键值设置,支持不同类型的选项与层次化的选
项设置,相关示例见 \ref{sec:secstyle}~节。

\section{编译方式、编码与中文字库}
\label{sec:chinese}

\subsection{编译方式}
\label{subs:compile}

\CTeX{} 宏集会根据用户使用的编译方式\footnote{\LaTeX、\pdfLaTeX、\XeLaTeX、
\LuaLaTeX{} 及 \upLaTeX。},在底层选择不同的中文支持方式(见
表~\ref{tab:chinese-support})。

\begin{table}[htbp]
\centering
\begin{threeparttable}
\caption{\CTeX{} 宏集的中文支持方式}
\label{tab:chinese-support}
\begin{tabular}{*6c}
  \toprule
  编译方式 & (pdf)\LaTeX & \XeLaTeX & \LuaLaTeX & \upLaTeX\tnote{*} \\
  \midrule
  支持宏包 & \pkg{CJK} & \pkg{xeCJK} & \pkg{LuaTeX-ja} & 原生 \\
  \bottomrule
\end{tabular}
\begin{tablenotes}
\item[*] p\LaTeX-ng(或称 \ApLaTeX)与 \upLaTeX{} 兼容。使用 p\LaTeX-ng 编译
时,\pkg{ctex} 采用与 \upLaTeX{} 相同的设置。
\end{tablenotes}
\end{threeparttable}
\end{table}

不同的编译方式和中文支持方式会在一定程度上影响 \CTeX{} 宏集的行为,比如宏包对
文档编码、字体选择、空格、标点等的处理。具体细节将在本文档后续内容中进行阐述。

\subsection{中文编码}
\label{subs:encoding}

\begin{function}[rEXP,updated=2019-11-10]{GBK, UTF8}
  指明编写文档时使用的编码。\CTeX{} 宏集无法检测文档实际的编码格式,因此需要
  用户通过选项声明。如果没有显式指定,则默认采用 UTF-8 编码。

  使用 \XeLaTeX{}、\LuaLaTeX{} 或 \upLaTeX{} 编译时,\CTeX{} 宏集强制使用
  UTF-8 编码,此时 \opt{GBK} 选项无效;使用 (pdf)\LaTeX{} 编译时,用户可以
  显式声明 \opt{GBK} 选项,使 \CTeX{} 宏集按 GBK 编码处理文档。用户需要^^A
  \emph{保证编译方式、源文件编码、宏包编码选项三者一致}。

  \emph{我们建议用户编写新文档时始终使用 UTF-8 编码,而仅把 GBK 编码留给
  历史遗留文档。}
\end{function}

\subsection{中文字库}
\label{subs:options-CJK-font}

以往,为 \LaTeX{} 文档配置中文支持是一件相当繁琐的事情。默认情况下,
\CTeX{} 宏集能自动检测用户使用的编译方式(参见 \ref{subs:compile}~节)和
操作系统\footnote{\CTeX{} 宏集现在能够识别 Mac~OS~X 系统以及 Windows 系统。},
选择合适的底层支持和字库,从而简化配置过程。自动配置的情况参见
表~\ref{tab:default-font-select}。

\begin{table}[htbp]
\centering
\begin{threeparttable}
\caption{\CTeX{} 宏集自动配置字体策略}
\label{tab:default-font-select}
\begin{tabular}{*{5}{c}}
  \toprule
             & macOS & Windows New\tnote{1} & Windows Old\tnote{2} &
              其他 \\
  \midrule
  \XeLaTeX{} & \makecell{\pkg{xeCJK}\\华文字库} &
    \makecell{\pkg{xeCJK}\\中易字库 + 微软雅黑} & \makecell{\pkg{xeCJK}\\中易字库} &
    \makecell{\pkg{xeCJK}\\Fandol 字库\tnote{3}} \\
  \cmidrule(lr){1-5}
  \LuaLaTeX{}\tnote{4} & \makecell{\pkg{LuaTeX-ja}\\华文字库} &
    \makecell{\pkg{LuaTeX-ja}\\中易字库 + 微软雅黑} &
    \makecell{\pkg{LuaTeX-ja}\\中易字库} &
    \makecell{\pkg{LuaTeX-ja}\\Fandol 字库} \\
  \cmidrule(lr){1-5}
  \pdfLaTeX{} & 不可用\tnote{5} &
    \makecell{\pkg{CJK} + \pkg{zhmetrics}\\中易字库 + 微软雅黑} &
    \makecell{\pkg{CJK} + \pkg{zhmetrics}\\中易字库} & 不可用\tnote{5} \\
  \cmidrule(lr){1-5}
  \makecell{\LaTeX{} + \\\dvipdfmx{}} & 不可用\tnote{6} &
    \makecell{\pkg{CJK} + \pkg{zhmetrics}\\中易字库 + 微软雅黑} &
    \makecell{\pkg{CJK} + \pkg{zhmetrics}\\中易字库} &
    \makecell{\pkg{CJK} + \pkg{zhmetrics}\\Fandol 字库} \\
  \cmidrule(lr){1-5}
  \makecell{\upLaTeX{} + \\\dvipdfmx{}} & 不可用\tnote{6} &
    \makecell{\pkg{zhmetrics-uptex}\\中易字库 + 微软雅黑} &
    \makecell{\pkg{zhmetrics-uptex}\\中易字库} &
    \makecell{\pkg{zhmetrics-uptex}\\Fandol 字库} \\
  \bottomrule
\end{tabular}
\begin{tablenotes}
  \item [1] Windows Vista 及以后的 Windows 操作系统。
  \item [2] Windows XP 及以前的 Windows 操作系统。
  \item [3] 由马起园、苏杰、黄晨成等人开发的开源中文字体,
    参见:\url{https://github.com/clerkma/fandol-fonts}。
  \item [4] \LuaLaTeX{} 编译时使用 \pkg{LuaTeX-ja} 宏包。对此,
    第 \ref{sec:lualatex-chinese}~节有特别说明。
  \item [5] 受 \pdfTeX{} 的限制,无法嵌入 OpenType 字体。
  \item [6] 目前受 \dvipdfmx{} 的限制,macOS 系统上的黑体和仿宋无法读取。
\end{tablenotes}
\end{threeparttable}
\end{table}

通常,由 \CTeX{} 宏集进行的自动配置已经足够使用,无需用户手工干预;但
是 \CTeX{} 仍然提供了一系列选项,供在 \CTeX{} 的自动选择机制因为
意外情况失效,或者在用户有特殊需求的情况下使用。\emph{除非必要,用户不
应使用这些选项。}

\begin{function}[rEXP,updated=2014-03-08]{zhmap}
  \begin{syntax}
    zhmap = <\TTF|zhmCJK>
  \end{syntax}
  指定字体映射机制。本选项只在使用 \pdfLaTeX/\LaTeX{} 编译时有意义。
\end{function}
\begin{optdesc}
  \item[true] 使用 \pkg{zhmetrics} 宏包,将 CJK 字库通过 \tn{special}
  命令映射到 \file{.ttf} 文件。

  \item[false] 使用传统的 CJK 字库(Type 1)^^A
  \footnote{如果需要使用自定义的字体映射文件,或者希望使用 Type1 字库,请禁用本选项。}。

  \item[zhmCJK] 载入 \pkg{zhmCJK} 宏包^^A
  \footnote{\pkg{zhmCJK} 宏包基于 \pkg{zhmetrics} 和 \pkg{CJK} 宏包,提供与
  \pkg{xeCJK} 宏包类似的用户接口。}^^A
  ,由 \pkg{zhmCJK} 宏包提供从 CJK 字库到 \file{.ttf} 的映射。
\end{optdesc}

\begin{function}[added=2014-03-08]{fontset}
  \begin{syntax}
    fontset = <adobe|fandol|founder|mac|macnew|macold|ubuntu|windows|none|...>
  \end{syntax}
  指定 \CTeX{} 宏集加载的字库。

  如果没有指定 \opt{fontset} 的值,\CTeX{} 宏集将自动检测用户使用的操作系统,配置
  相应的字体(参见表~\ref{tab:default-font-select})。
\end{function}

\CTeX{} 预定义了以下六种中文字库。

\begin{optdesc}
  \item[adobe] 使用 Adobe 公司的四款中文字体,\emph{不支持 \pdfLaTeX}。
  \item[fandol] 使用 Fandol 中文字体,\emph{不支持 \pdfLaTeX}。
  \item[founder] 使用方正公司的中文字体。
  \item[mac] 使用 macOS 系统下的字体,\emph{不支持 \pdfLaTeX},根据版本分为
    |macnew| 和 |macold| 两种。
  \item[macnew] 使用 El Capitan 或之后的多字重华文字体和苹方字体。
  \item[macold] 使用 Yosemite 或之前的华文字体。
  \item[ubuntu] 使用 Ubuntu 系统下的思源宋体、思源黑体和 \TeX{} 发行版自带的
    文鼎楷体,\emph{不支持 \pdfLaTeX}。
  \item[windows] 使用 Windows 系统下的中易字体和微软雅黑字体。
\end{optdesc}

注意:使用 (pdf)\LaTeX{} 编译的时候,若设置 \opt{zhmap = false}(比如需要
使用 \LaTeX{} + Dvips 编译),则需要按照传统方式^^A
\footnote{可以使用 \pkg{zhmetrics} 宏包提供的脚本
\href{https://github.com/CTeX-org/ctex-kit/blob/master/zhmetrics/CTeXFonts.lua}
{\file{CTeXFonts.lua}}。}^^A
在本地安装好 CJK 字体。

如果不想使用 \CTeX{} 预定义的中文字库,可以设置 \opt{fontset} 为下述值之一。

\begin{optdesc}
  \item[none] 不配置中文字体,需要用户自己配置。
  \item[\meta{name}] 这里 \meta{name} 为自定义的名字。
  \CTeX{} 宏集将载入名为 |ctex-fontset-|\meta{name}|.def| 的文件作为字体配置
  文件。因此,请先保证文件的存在。可以在当前工作目录或者本地 \texttt{TDS} 目录
  树下合适位置建立一个名为 |ctex-fontset-|\meta{name}|.def| 的文件,在这个文件
  里面自定义中文字体。然后通过使用 |fontset=|\meta{name} 选项来调用它。字体配置
  文件的具体写法可以参考 \CTeX{} 宏集 \texttt{fontset} 目录下的字体配置文件。
\end{optdesc}

注意:\emph{如果希望使用 \tn{ctexset} 在导言区指定字库,则需要先在宏包/文档类选项中指定
\opt{fontset = none}}。例如:
\begin{ctexexam}
  \documentclass[fontset = none]{ctexart}
  \ctexset{fontset = founder}
  \begin{document}
  在文档类选项中声明 \verb|fontset = none|,随后在导言区用 \verb|\ctexset|
  指定字体。
  \end{document}
\end{ctexexam}

\CTeX{} 宏集预定义的中文字库还定义了一些字体命令。除了在 \opt{ubuntu} 字库中没有
\tn{fangsong} 的定义外,所有字库都有以下四个字体命令:
\begin{optdesc}
  \item[\tn{songti}] 宋体,CJK 等价命令 |\CJKfamily{zhsong}|。
  \item[\tn{heiti}] 黑体,CJK 等价命令 |\CJKfamily{zhhei}|。
  \item[\tn{fangsong}] 仿宋,CJK 等价命令 |\CJKfamily{zhfs}|。
  \item[\tn{kaishu}] 楷书,CJK 等价命令 |\CJKfamily{zhkai}|。
\end{optdesc}
在 \opt{windows}、\opt{founder} 和 \opt{macnew} 字库中,额外定义了 \tn{lishu}
和 \tn{youyuan}:
\begin{optdesc}
  \item[\tn{lishu}] 隶书,CJK 等价命令 |\CJKfamily{zhli}|。
  \item[\tn{youyuan}] 圆体,CJK 等价命令 |\CJKfamily{zhyou}|。
\end{optdesc}
在 \opt{windows} 字库中还定义了 \tn{yahei}。出于兼容性的考虑,\tn{yahei} 命令
在 \opt{macnew} 字库中也有定义,但实际调用苹方黑体:
\begin{optdesc}
  \item[\tn{yahei}] 微软雅黑,CJK 等价命令 |\CJKfamily{zhyahei}|。
\end{optdesc}
在 \opt{macnew} 字库中,还定义了 \tn{pingfang}:
\begin{optdesc}
  \item[\tn{pingfang}] 苹方黑体,CJK 等价命令 |\CJKfamily{zhpf}|。
\end{optdesc}

\section{排版格式设定}
\label{sec:format-settings}
\subsection{文档默认字号}
\label{subs:options-class}

\begin{function}[rEXP,added=2015-05-06]{zihao}
  \begin{syntax}
    zihao = <-4|5|false>
  \end{syntax}
  将文章默认字号(\tn{normalsize})设置为小四号字或五号字,
  具体情况见表 \ref{tab:fontsize}。\opt{false} 禁用本功能。
  本选项可以用于四个 \CTeX{} 文档类和 \pkg{ctex} 宏包,
  也可以用于 \pkg{ctexsize} 宏包。

  \opt{scheme = chinese} 时,对标准文档类默认值为 |5|,即设置
  \tn{normalsize} 为五号字;对 \cls{beamer} 则为 |false|,使用文档类原有的设
  置。
\end{function}

\begin{table}[htbp]
\centering
\setlength\tabcolsep{1em}
\caption{标准字体命令与字号的对应}\label{tab:fontsize}
\begin{tabular}{l*2{c>{\ttfamily}r}*3{>{\ttfamily}c}}
\toprule
& \multicolumn2c{|zihao = 5|} & \multicolumn2c{|zihao = -4|} &
\multicolumn1c{|10pt|} & \multicolumn1c{|11pt|} & \multicolumn1c{|12pt|} \\
\cmidrule(lr){2-3} \cmidrule(lr){4-5}
\cmidrule(lr){6-6} \cmidrule(lr){7-7} \cmidrule(lr){8-8}
字体命令        & 字号 & \si{bp} & 字号 & \si{bp}
                       & \si{pt} & \si{pt} & \si{pt} \\
\midrule
|\tiny|         & 七号 & 5.5  & 小六 & 6.5  & ~5 & ~6 & ~6 \\
|\scriptsize|   & 小六 & 6.5  & 六号 & 7.5  & ~7 & ~8 & ~8 \\
|\footnotesize| & 六号 & 7.5  & 小五 & 9~~  & ~8 & ~9 & 10 \\
|\small|        & 小五 & 9~~  & 五号 & 10.5 & ~9 & 10 & 11 \\
|\normalsize|   & 五号 & 10.5 & 小四 & 12~~ & 10 & 11 & 12 \\
|\large|        & 小四 & 12~~ & 小三 & 15~~ & 12 & 12 & 14 \\
|\Large|        & 小三 & 15~~ & 小二 & 18~~ & 14 & 14 & 17 \\
|\LARGE|        & 小二 & 18~~ & 二号 & 22~~ & 17 & 17 & 20 \\
|\huge|         & 二号 & 22~~ & 小一 & 24~~ & 20 & 20 & 25 \\
|\Huge|         & 一号 & 26~~ & 一号 & 26~~ & 25 & 25 & 25 \\
\bottomrule
\end{tabular}
\end{table}

\begin{function}[rEXP]{10pt, 11pt, 12pt}
  \CTeX{} 文档类是在 \LaTeX{} 标准文档类之上开发的。因此,除了可以使用 \CTeX{}
  宏包定义的字号选项之外,还可以使用标准文档类的同类选项(\opt{10pt}、\opt{11pt}
  和 \opt{12pt})。在使用这些来自标准文档类的选项的时候,\CTeX{} 文档类的字号
  选项会被抑制。亦即,在 \opt{zihao} 选项之后设置 \opt{10pt} 选项,
  \opt{zihao} 选项将不再起作用。
\end{function}

标准文档类的其他选项在 \CTeX{} 文档类中依旧有效。例如,设置纸张大小和方向的
\opt{a4paper} 和 \opt{landscape},设置单双面的 \opt{oneside} 和
\opt{twoside} 等。\CTeX{} 会将这些选项传给标准文档类^^A
\footnote{事实上,\LaTeX{} 在文档类中的选项是全局设定的,除了对使用的文档类有
影响外,也可能会影响到随后使用的宏包。如果这些宏包中有某些选项出现在文档类的
选项列表中,那么该选项将会被自动激活。}。

\subsection{章节标题风格}
\label{subs:options-heading}

\begin{function}[rEXP,added=2014-03-08]{heading}
  \begin{syntax}
    heading = <\TFF>
  \end{syntax}
  本选项只能在调用 \pkg{ctex.sty} 时作为宏包选项使用。

  \CTeX{} 宏集提供了一套用于修改文档章节标题格式的接口。该选项用于选择是否
  启用该功能。详细的设置方法请参见
  \ref{sec:secstyle}~节和 \ref{subs:pagestyle}~节。
\end{function}

\CTeX{} 宏集提供的四个文档类总是启用该功能。如果在 \pkg{ctex.sty} 下启用该选项,
将会检查当前是否使用 \LaTeX{} 标准文档类。
若然,则该选项将会使得 \pkg{ctex.sty} 宏包的行为和 \CTeX{} 宏集提供的
四个中文文档类\emph{完全}一致;若不然,则会根据 \tn{chapter}
是否有定义来使用 \cls{ctexbook} 或者 \cls{ctexart} 的标题设置。

\begin{function}[rEXP]{sub3section, sub4section}
  修改 \tn{paragraph} 和 \tn{subparagraph} 的格式。

  默认情况下,\tn{paragraph} 和 \tn{subparagraph} 会将标题与随后的正文
  排版在同一个段落。启用 \opt{sub3section} 会将 \tn{paragraph} 的格式修改为
  类似 \tn{section} 的格式,并将 \tn{subparagraph} 的格式修改为原本 \tn{paragraph}
  的格式。启用 \opt{sub4section} 会将 \tn{paragraph} 和 \tn{subparagraph} 的格式
  都修改为类似 \tn{section} 的格式。

  启用该选项通常需要将计数器 |secnumdepth| 的值为设置为 4 或 5。

  具体格式可参考 \ref{sec:secstyle}~节中的 \opt{runin} 和 \opt{afterskip} 选项。

  注意,上述两个选项只有在非 \cls{beamer} 文档类下 \opt{heading} 选项启用的时候
  才有意义。亦即,只有在使用除了 \cls{ctexbeamer} 的三个 \CTeX{} 文档类或启用了
  \opt{heading} 的 \pkg{ctex.sty} 的时候才有意义。
\end{function}

\subsection{排版方案选项}
\label{subs:options-type-style}

\begin{function}[rEXP,added=2015-04-15]{scheme}
  \begin{syntax}
    scheme = <(chinese)|plain>
  \end{syntax}
  选择文章的排版方案,预设有 \opt{chinese} 和 \opt{plain} 两种方案。%
\end{function}
  \begin{optdesc}[itemsep=\medskipamount]
    \item[chinese] 对 \cls{beamer} 以外的文档类,调整默认字号为五号字,并调
      整行距为 |1.3|;汉化文档中的标题名字(如“图”、“表”、“目录”和“参
      考文献”等,见 \ref{subs:capname}~节);
      在 \opt{heading = true} 的情况下^^A
      \footnote{使用 \CTeX{} 文档类,或者使用 \pkg{ctex} 宏包且开启该选项时。}^^A
      (\ref{subs:options-heading}~节),还会将章节标题的风格修改为
      中文样式(见 \ref{sec:secstyle}~节)。

      当关闭 \opt{heading} 选项的 \pkg{ctex} 宏包与标准文档类或其衍生文档类
      联用时,会载入 \pkg{indentfirst} 宏包,以实现章节标题后的段首缩进。
    \item[plain] 不调整默认字号和行距,不会汉化文档中的标题名字,也不会将章节
      标题风格修改为中文样式,同时不会调整 \tn{pagestyle},并禁用 \opt{autoindent}
      选项。事实上,此时的 \CTeX{} 宏集只提供了中文支持功能,而不对文章版式进行任何修改。
  \end{optdesc}

\begin{function}[updated=2014-04-11]{punct}
  \begin{syntax}
    punct = <(quanjiao)|banjiao|kaiming|CCT|plain>
  \end{syntax}
  设置标点处理格式。预定义好的格式有:
\end{function}
  \begin{optdesc}
    \item[quanjiao] 全角式:所有标点占一个汉字宽度,相邻两个标点占 1.5 汉字宽度;
    \item[banjiao]  半角式:所有标点占半个汉字宽度;
    \item[kaiming]  开明式:句末点号^^A
      \footnote{标点符号分为标号与点号。点号分为两类,一共七种:句末点号有^^A
      句号、问号和叹号;句内点号有逗号、顿号、冒号和分号。}用占一个汉字宽度,标^^A
      号和句内点号占半个汉字宽度;
    \item[CCT]   CCT 式:所有标点符号的宽度略小于一个汉字宽度;
    \item[plain] 原样(不调整标点间距)。
  \end{optdesc}

\begin{function}[updated=2014-03-08]{space}
  \begin{syntax}
    space = <\TF|(auto)>
  \end{syntax}
  是否在生成的 PDF 中保留汉字后面的空格。
\end{function}

\begin{optdesc}
  \item[true] 总是保留汉字后的空格。此时,用户需要自行在行尾加上~|%|~处理换行产生
    的空格\footnote{\LaTeX{} 将单个换行视作一个空格。}。
  \item[false] 总是忽略掉汉字后面的空格,不论汉字后是什么(使用 (pdf)\LaTeX{}
    编译时);等同于 \opt{auto} 的效果(使用 \XeLaTeX{} 编译时)。不建议使用该选项。
  \item[auto] 根据空格后面的情况决定是否保留:如果空格后面是汉字,则忽略该
  空格,否则保留。
\end{optdesc}

  例如,使用
  \begin{ctexexam}
  \ctexset{space=true}
  汉字 分词
  技术 English
  \end{ctexexam}
  将得到“{\ctexset{space=true}汉字 分词 技术 English}”;使用
  \begin{ctexexam}
  \ctexset{space=auto}
  汉字 分词
  技术 English
  \end{ctexexam}
  则会得到“{\ctexset{space=auto}汉字 分词 技术 English}”。

\emph{使用 \textup{\LuaLaTeX{} 及 \upLaTeX} 编译的时候,该选项无效:汉字间的
空格以及汉字与西文字符之间的空格总是有效,不会被忽略,但可以自动忽略掉由换行
产生的空格。}

\begin{function}[rEXP,added=2014-04-23]{linespread}
  \begin{syntax}
    linespread = <数值>
  \end{syntax}
  接受一个浮点数值,设置行距倍数。本选项的初始值与 \opt{scheme} 有关。
\end{function}
\begin{optdesc}
  \item[scheme = chinese] 对标准文档类初始值为 $1.3$,即 $1.3$ 倍行距。此
  时,相邻两行的基线(\tn{baselineskip})距离为 $1.3\times 1.2=1.56$ 倍字体
  高度。对 \cls{beamer} 不改变行距,即使用默认的单倍行距。

  \item[scheme = plain] \CTeX{} 宏集默认不调整行距倍数,文档中的行距由所选文档类和
  其他宏包或用户设置决定。
\end{optdesc}

\begin{function}[added=2014-03-13]{autoindent}
  \begin{syntax}
    autoindent = <\TTF|数值|带单位的数值>
  \end{syntax}
  在字体大小发生变化时,是否自动调整段首缩进(\tn{parindent})的大小。
\end{function}
\begin{optdesc}
  \item[\meta{数值或带单位的数值}] 用于设置段首缩进的长度。如果不带单位,则默认
  单位是单个汉字字宽 \tn{ccwd};如果带单位,则使用该单位。

  \item[true] 等价于设置 \opt{autoindent = 2}。

  \item[false] 禁用自动调整功能,可以设置固定长度的段首缩进。如设置每段缩进 40 点:
  \begin{ctexexam}
  \ctexset{autoindent=false}
  \setlength\parindent{40pt}
  \end{ctexexam}
\end{optdesc}

\begin{function}[EXP,added=2014-03-26]{linestretch}
  \begin{syntax}
    linestretch = <数值或长度>
  \end{syntax}
  \opt{linestretch} 是一个比较特殊的选项,它用来设置汉字之间弹性间距的弹性程
  度。如果有单位,则可以在选项中直接写;如果是数字,单位则是汉字宽度
  \tn{ccwd} 的倍数。
\end{function}

  如果行宽不是汉字宽度的整数倍,为了让段落左右两端对齐,自然就要求伸展汉字之
  间的间距,而 \opt{linestretch} 选项就是设置每行总的允许伸行量。初始值是允
  许每行伸行一个汉字的宽度 \tn{ccwd},并且此宽度能根据字号变化动态调整。

  过小的 \opt{linestretch} 可能导致段落文字右侧可能参差不齐;较大的
  \opt{linestretch} 选项则可以帮助拥有较长不可断行内容的复杂段落方便地断行,
  而不会产生大量编译警告;但很大的 \opt{linestretch} 则会掩盖段落不良断行产
  生的坏盒子警告。

  如果将 \opt{linestretch} 选项的值设置为 \tn{maxdimen},则可以禁止按字号自
  动修改每行的允许伸长量。此时汉字间的弹性间距则固定为 \tn{baselineskip} 的
  $0.08$ 倍。

\section{文档汉化}
\label{sec:chinese-localization}
\subsection{日期汉化}

\CTeX 宏包对显示当前日期的 \tn{today} 命令进行了汉化,使之以中文的方式显示今
天的日期。如编译本文档的日期就是“\today”。

\begin{function}[EXP]{today}
  \begin{syntax}
    today = <(small)|big|old>
  \end{syntax}
该选项用来控制 \tn{today} 命令的输出格式:
\begin{optdesc}
  \item[small] \ctexset{today=small}
    效果为“\today”。使用阿拉伯数字和汉字的日期格式。
  \item[big] \ctexset{today=big}
    效果为“\today”。使用全汉字的日期格式。
  \item[old] \ctexset{today=old}
    效果为“\today”。使用文档原来的(英文)日期格式。
\end{optdesc}
\end{function}

设置日期格式使用 \tn{ctexset} 命令完成,例如设置全汉字的日期格式:
\begin{ctexexam}
  \ctexset{today=big}
\end{ctexexam}

\CTeX 宏包的中文日期功能实际上是调用 \pkg{zhnumber} 宏包完成的。如果需要更
多有关日期、时间的命令和更复杂的设置,可以查阅 \pkg{zhnumber} 宏包的文档。

\subsection{文档标题汉化}
\label{subs:capname}

这里主要介绍由宏包 \opt{scheme} 选项(\ref{subs:options-type-style}~节)控制
的文档标题汉化功能。

设置文档标题名的示例可见例~\ref{exam:capname}。下面的选项(如
\opt{contentsname})主要用来重新定义与选项同名的宏(如 \tn{contentsname})的
定义。

\begin{defaultcapconfig}

\begin{function}[EXP]{contentsname}
  \begin{syntax}
    contentsname = <名字>
  \end{syntax}
设置目录标题名 \tn{contentsname}。中文默认为“\contentsname”。
\end{function}

\begin{function}[EXP]{listfigurename}
  \begin{syntax}
    listfigurename = <名字>
  \end{syntax}
设置插图目录标题名 \tn{listfigurename}。中文默认为“\listfigurename”。
\end{function}

\begin{function}[EXP]{listtablename}
  \begin{syntax}
    listtablename = <名字>
  \end{syntax}
设置表格目录标题名 \tn{listtablename}。中文默认为“\listtablename”。
\end{function}

\begin{function}[EXP]{figurename}
  \begin{syntax}
    figurename = <名字>
  \end{syntax}
设置图片环境标题名 \tn{figurename}。中文默认为“\figurename”。
\end{function}

\begin{function}[EXP]{tablename}
  \begin{syntax}
    tablename = <名字>
  \end{syntax}
设置表格环境标题名 \tn{tablename}。中文默认为“\tablename”。
\end{function}

\begin{function}[EXP]{abstractname}
  \begin{syntax}
    abstractname = <名字>
  \end{syntax}
设置摘要 \env{abstract} 环境标题名 \tn{abstractname}。中文默认
为“\abstractname”。注意 \cls{book} 类没有摘要,该选项无效。
\end{function}

\begin{function}[EXP]{indexname}
  \begin{syntax}
    indexname = <名字>
  \end{syntax}
设置索引标题名 \tn{indexname}。中文默认为“\indexname”。
\end{function}

\begin{function}[EXP]{appendixname}
  \begin{syntax}
    appendixname = <名字>
  \end{syntax}
设置附录标题名 \tn{appendixname}。中文默认为“\appendixname”。
\end{function}

\begin{function}[EXP]{bibname}
  \begin{syntax}
    bibname = <名字>
  \end{syntax}
设置参考文献标题名 \tn{refname}(对 \cls{article})或 \tn{bibname}(对
\cls{report}、\cls{book} 和 \cls{beamer})。中文默认为“\refname”。
\end{function}

\begin{function}[EXP]{proofname}
  \begin{syntax}
    proofname = <名字>
  \end{syntax}
设置证明环境的名称 \tn{proofname}。中文默认为“\proofname”。
\end{function}

如果使用 \cls{ctexbeamer} 文档类或者在 \cls{beamer} 文档类下使用 \pkg{ctex} 包,
还会汉化常用定理类环境的诸如“定义”、“定理”和“引理”等名称。此时,还有下列
三个选项。

\begin{function}[EXP]{refname}
  \begin{syntax}
    refname = <名字>
  \end{syntax}
设置参考文献标题名 \tn{refname}。中文默认为“\refname”。
\end{function}

\begin{function}[EXP]{algorithmname}
  \begin{syntax}
    algorithmname = <名字>
  \end{syntax}
设置算法环境标题名 \tn{algorithmname}。中文默认为“算法”。
\end{function}

\begin{function}[EXP]{continuation}
  \begin{syntax}
    continuation = <名字>
  \end{syntax}
设置 \cls{beamer} 可断页的帧在续页标题中的延续标识 \tn{insertcontinuationtext}。
中文默认为“(续)”。
\end{function}

\end{defaultcapconfig}

\subsection{页面格式设置与汉化}
\label{subs:pagestyle}

当使用了 \CTeX 的文档类或是用 \pkg{ctex} 宏包加载了 \opt{heading} 选项时,会
设置整个文档的页面格式(page style)为 |headings|,即相当于设置了
\begin{frameverb}
  \pagestyle{headings}
\end{frameverb}
在页眉中显示当前章节的编号与标题。

同时,\CTeX 宏包也会对默认的 |headings| 页面格式进行修改,使之调用
\tn{CTEXthechapter}、\tn{CTEXthesection} 等宏来正确显示中文的章节编号。

\CTeX 宏包的默认页面格式设置是经过汉化的 |headings|,其基本效果如本文档所
示,只在页眉一侧显示章节编号和标题,另一侧显示页码。

更复杂的页面格式可以通过调用 \pkg{fancyhdr}、\pkg{titleps} 等宏包来设
置。\CTeX 宏包同时也为这些自定义页面格式
的包提供了以下宏供使用:
\begin{itemize}
\item \tn{CTEXthechapter}、\tn{CTEXthesection} 等章节编号(见
\ref{sec:secstyle} 节)。它们用来代替英文文档类中的
\tn{thechapter}、\tn{thesection} 等宏。

\item \tn{leftmark}、\tn{rightmark},它们是在使用章节标题命令后,自动设置的
宏。它们实际是在与章节标题命令对应的标记命令
\tn{chaptermark}、\tn{sectionmark} 中调用 \tn{markright} 或 \tn{markboth} 生
成的。
\end{itemize}
有关 \LaTeX 页面标记的涵义与使用细节,已经超出了本文档讨论的范围。可以参考
\cite[Chapter~23]{knuthtex1986}、\cite[\S4.3, \S4.4]{mittelbach2004} 等书籍。

这里举一个例子,说明通过重定义 \tn{sectionmark},在 \cls{ctexart} 文档类中的
标准 |headings| 页面格式下控制页眉的方式:
\begin{ctexexam}
  \documentclass{ctexart}
  \pagestyle{headings}
  \ctexset{section={
      name={第,节},
      number=\arabic{section},
    }
  }
  \renewcommand\sectionmark[1]{%
    \markright{\CTEXifname{\CTEXthesection——}{}#1}}

  \begin{document}

  \section{天地玄黄}
  \newpage

  \section{宇宙洪荒}

  \end{document}
\end{ctexexam}
在上例中,我们设置了页眉的形式是用破折号分开的节编号与节标题,即“第 1 节
——天地玄黄”、“第 2 节——宇宙洪荒”。

\CTeX 宏包已经对 \pkg{fancyhdr} 宏包进行了补丁,载入 \pkg{fancyhdr} 后,其
|fancy| 页面格式将使用 \tn{CTEXthechapter} 等宏显示中文章节编号。

关于 \pkg{fancyhdr} 的具体用法可以参见其宏包手册。通常也只要像在标准的英文文
档类中使用 \pkg{fancyhdr} 一样定义页眉页脚格式即可,并不需要额外的定义。

下面我则给出一个与前例类似而稍复杂的例子,展示如何在文档中设置页眉内容与页眉
的格式。
\begin{ctexexam}
  \documentclass{ctexart}
  \ctexset{section={
      name={第,节},
      number=\arabic{section},
    }
  }
  \usepackage{fancyhdr}
  \fancyhf{}
  \lhead{\textnormal{\kaishu\rightmark}}
  \rhead{--\ \thepage\ --}
  \pagestyle{fancy}
  % \sectionmark 的重定义需要在 \pagestyle 之后生效
  \renewcommand\sectionmark[1]{%
    \markright{\CTEXifname{\CTEXthesection——}{}#1}}

  \begin{document}

  \section{天地玄黄}
  \newpage

  \section{宇宙洪荒}

  \end{document}
\end{ctexexam}
本例的页眉效果大致如下(有页眉线):
\begin{trivlist}\item
\textnormal{\kaishu 第 1 节——天地玄黄}\hfill -- 1 --\par
\smallskip\hrule
\end{trivlist}

\section{章节标题格式设置}
\label{sec:secstyle}

\CTeX 宏包对 \LaTeX 的标准文档类(\cls{article}、\cls{report} 和
\cls{book})和 \cls{beamer} 进行了扩充。当以 \opt{heading} 选项调用 \CTeX
宏包时(\ref{subs:options-heading}~节),则会启用章
节标题的格式设置功能。本节就来介绍有关章节标题的格式选项,所有选项使用
\tn{ctexset} 命令设置。

第 \ref{subs:pagestyle} 节和本节介绍的功能已经被提取到 \pkg{ctexheading}
宏包之中,可以在 \pkg{ctex} 宏包和 \cls{ctexart} 等文档类之外独立使用。
各选项的默认值与 \opt{scheme = plain} 时的情形相同。

章节标题的格式选项是分层设置的。顶层的选项是章节标题名称,次一级的选项是章节
标题的格式。章节标题名包括 |part|, |chapter|, |section|, |subsection|,
|subsubsection|, |paragraph|, |subparagraph|;而可用的格式包括 \opt{numbering},
\opt{name}, \opt{number}, \opt{format}, \opt{nameformat}, \opt{numberformat},
\opt{aftername}, \opt{titleformat}, \opt{aftertitle}, \opt{runin},
\opt{afterindent}, \opt{beforeskip}, \opt{afterskip}, \opt{fixskip},
\opt{lotskip}, \opt{lofskip}, \opt{indent}, \opt{hang},
\opt{pagestyle}, \opt{break}, \opt{tocline} 等。

注意,对 \cls{article} 及其衍生的 \cls{ctexart} 等文档类,没有 |chapter|
级别的标题。而对于 \cls{beamer} 文档类,这些选项控制的是由 \tn{partpage},
\tn{sectionpage} 和 \tn{subsectionpage} 产生的标题格式,此时只有 |part|,
|section| 和 |subsection| 这三层级别,并且 \opt{runin}, \opt{afterindent},
\opt{fixskip}, \opt{hang}, \opt{break} 和 \opt{tocline} 这六个格式无效。

多级选项之间用斜线分开,例如,\opt{part/name} 选项设置 \tn{part} 标题的在数
字前后的名称,而 \opt{section/number} 选项设置 \tn{section} 标题的数字类型。
注意,斜线 |/| 前后不能有空格或者换行。

使用 \tn{ctexset} 设置多级选项时,可以在同一个上级选项下设置多个下级选项。

例如,同时设置 |part| 一级标题的 \opt{pagestyle} 选项,|chapter| 一级标题的
\opt{format} 与 \opt{pagestyle} 选项和 |section| 一级标题的 \opt{name} 与
\opt{number} 选项:
\begin{ctexexam}
  \ctexset {
    part/pagestyle = empty,
    chapter = {
      format    = \raggedright,
      pagestyle = empty,
    },
    section = {
      name   = {第,节},
      number = \chinese{section},
    }
  }
\end{ctexexam}

\begin{function}[EXP,added=2015-06-21]{part/numbering, chapter/numbering,
  section/numbering, subsection/numbering, subsubsection/numbering,
  paragraph/numbering, subparagraph/numbering}
  \begin{syntax}
    numbering = \TTF
  \end{syntax}
  控制是否对章节标题编号。对各级标题的默认值均为 \opt{true}。

  我们知道,\LaTeX{} 带星号的章节标题命令(如 \tn{section*})不会对标题编号,
  但也不会将该没编号的标题编入目录中。本选项控制的是不带星号的标题命令是否编号。
  设置本选项为 \opt{false},除了不对标题编号以外,功能与正常标题一致,
  比如可以编目录,正确的 \pkg{hyperref} 目录超链接位置和页眉标记。

  注意,章节标题的编号深度受 \LaTeX{} 计数器 |secnumdepth| 的控制。
  \opt{numbering} 选项在 |secnumdepth| 的控制下起作用。
\end{function}

如果没有特别说明,以下将用 “|...|” 代表各级章节标题名。

\begin{function}[EXP,updated=2014-03-08]{.../name}
  \begin{syntax}
    name = \{<前名字>,<后名字>\}
    name = \Arg{前名字}
  \end{syntax}
  设置章节的名字。名字可以分为前后两部分,即章节编号前后的词语,两个词之间用
  一个半角逗号分开;也可以只有一部分,表示只有章节编号之前的名字。例如:
  \begin{ctexexam}
  \ctexset{
    chapter/name = {第,章},
    section/name = {\S},
  }
  \end{ctexexam}
  会使得 \tn{chapter} 标题使用形如“第一章”的名字,而 \tn{section} 标题则使
  用形如“\S1”的名字。
\end{function}

\begin{table}[htbp]
\small\centering
\caption{\opt{name} 选项的默认设置}
\begin{tabular}{llll}
\toprule
标题名 & \opt{scheme = chinese} & \opt{scheme = plain} & 注 \\
\midrule
part & |{第,部分}| & |{\partname\space}| & 原 \tn{partname} 为 Part \\
chapter & |{第,章}| & |{\chaptername\space}|
  & 原 \tn{chaptername} 为 Chapter \\
section (beamer) & |{}| & |{\sectionname\space}|
  & \BSTACK 原 \tn{sectionname} 为\\ |\translate{Section}| \ESTACK \\
section & 同右 & |{}| & \\
subsection (beamer) & |{}| & |{\subsectionname\space}|
  & \BSTACK 原 \tn{subsectionname} 为\\ |\translate{Subsection}| \ESTACK \\
subsection & 同右 & |{}| & \\
subsubsection & 同右 & |{}| & \\
paragraph & 同右 & |{}| & \\
subparagraph & 同右 & |{}| & \\
\bottomrule
\end{tabular}
\end{table}

\begin{function}[EXP]{.../number}
  \begin{syntax}
    number = \Arg{数字输出命令}
  \end{syntax}
  设置章节编号的数字输出格式。\meta{数字输出命令} 通常是对应章节编号计数器的
  输出命令,如 \tn{thesection} 或 |\chinese{chapter}| 之类。
  \begin{ctexexam}
  \ctexset{
    section/number = \Roman{section}
  }
  \end{ctexexam}
\end{function}

  \opt{number} 选项的定义同时将控制对章节计数器的交叉引用。在引用计数器时,
  记录在 \LaTeX{} 辅助文件中的是 \opt{number} 选项的定义。

  但是,\opt{number} 选项不会影响计数器本身的输出。即设置 |section/number|
  不会影响 \tn{thesection} 的定义。(但该选项会影响 \tn{CTEXthesection} 的定
  义,见后。)

\begin{table}[htbp]
\small\centering
\caption{\opt{number} 选项的默认设置}
\setlength\leftskip{0pt plus 1 fil minus \marginparwidth}
\begin{tabular}{llll}
\toprule
标题名 & \opt{scheme = chinese} & \opt{scheme = plain}
  & 原 |\the|\meta{标题} 等价定义 \\
\midrule
part (beamer) & |\chinese{part}| & |\insertromanpartnumber| & 意义为 |\Roman{part}| \\
part & |\chinese{part}| & |\thepart| & |\Roman{part}| \\
chapter & |\chinese{chapter}| & |\thechapter| & |\arabic{chapter}| \\
section (beamer) & 同右 & |\insertsectionnumber| & 意义为 |\arabic{section}| \\
section & 同右 & |\thesection| & |\arabic{section}| \\
subsection (beamer)
  & \BSTACK |\arabic{section}.|\\ |\arabic{subsection}| \ESTACK
  & |\insertsubsectionnumber| & 意义为 |\arabic{subsection}| \\
subsection & 同右 & |\thesubsection| & |\thesection.\arabic{subsection}| \\
subsubsection & 同右 & |\thesubsubsection|
  & |\thesubsection.\arabic{subsubsection}| \\
paragraph & 同右 & |\theparagraph|
  & |\thesubsubsection.\arabic{paragraph}| \\
subparagraph & 同右 & |\thesubparagraph|
  & |\theparagraph.\arabic{subparagraph}| \\
\bottomrule
\end{tabular}
\end{table}

\begin{function}{\CTEXthepart, \CTEXthechapter, \CTEXthesection,
  \CTEXthesubsection, \CTEXthesubsubsection, \CTEXtheparagraph,
  \CTEXthesubparagraph}
  以 |\CTEXthe| 开头的这组宏给出结合了 \opt{name} 与 \opt{number} 选项的章节
  编号输出格式。例如在 \opt{scheme = chinese} 时,默认章编号输出格式就是
  \tn{CTEXthechapter},形如“第一章”。

  这组宏在 \CTeX 文档类中将代替 \tn{thechapter} 等宏的作用,在章节中引用本章
  节的完整编号。例如用于帮助定义自定义的目录格式、页眉格式等。
\end{function}

\begin{function}[added=2016-09-18]{\CTEXifname}
  \begin{syntax}
    \tn{CTEXifname} \Arg{有名字时的格式命令} \Arg{无名字时的格式命令}
  \end{syntax}
  \tn{CTEXifname} 用于根据当前章节的名字的有无设置不同的格式。

  它可用于 \opt{format}, \opt{titleformat}, \opt{aftertitle}, \opt{afterskip},
  \opt{indent} 这五个选项和 \tn{chapter} 标题 \opt{beforeskip} 选项的格式设置之中。
  也可用于帮助定义自定义的目录格式、页眉格式等。

  例如,设置章的标题有名字时左对齐,无名字时居中对齐,并且在标题后画一条横线。
  \begin{ctexexam}
  \ctexset{
    chapter/format = \CTEXifname{\raggedright}{\centering},
    chapter/aftertitle = \par\CTEXifname{}{\hrule},
  }
  \end{ctexexam}
\end{function}

\begin{function}[EXP,updated=2015-06-30]{.../format, .../format+}
  \begin{syntax}
    format = \Arg{格式命令}
    format+= \Arg{格式命令}
  \end{syntax}
  \opt{format} 选项用于控制章节标题的全局格式,作用域为章节名字和随后的标题
  内容。可以用于控制章节标题的对齐方式、整体字体字号等格式。

  带加号的 \opt{format+} 选项则用于在已有格式之后追加新的格式命令。

  例如,设置章格式为无衬线字体左对齐,为节格式增加无衬线字体设置:
  \begin{ctexexam}
  \ctexset{
    chapter/format  = \sffamily\raggedright,
    section/format += \sffamily
  }
  \end{ctexexam}
\end{function}

\begin{table}[htbp]
\small\centering
\caption{\opt{format} 选项的默认设置}
\begin{tabular}{lll}
\toprule
标题名 & \opt{scheme = chinese} & \opt{scheme = plain} \\
\midrule
part (article) & |\Large\bfseries\centering| & |\raggedright| \\
part (beamer) & 同右 & |\centering| \\
part & |\huge\bfseries\centering| & |\centering| \\
chapter & |\huge\bfseries\centering| & |\raggedright| \\
section (beamer) & 同右 & |\centering| \\
section & |\Large\bfseries\centering| & |\Large\bfseries| \\
subsection (beamer) & 同右 & |\centering| \\
subsection & 同右 & |\large\bfseries| \\
subsubsection & 同右 & |\normalsize\bfseries| \\
paragraph & 同右 & |\normalsize\bfseries| \\
subparagraph & 同右 & |\normalsize\bfseries| \\
\bottomrule
\end{tabular}
\end{table}

\begin{function}[EXP,updated=2015-06-30]{.../nameformat, .../nameformat+}
  \begin{syntax}
    nameformat = \Arg{格式命令}
    nameformat+= \Arg{格式命令}
  \end{syntax}
  \opt{nameformat} 用于控制章节名字的格式,作用域为章节名字,包括编号。它一
  般用于章节名(包括编号)与章节标题的字体、字号等设置不一致的情形。参见
  \opt{titleformat} 选项。

  \opt{nameformat+} 用于在已有的章节名字格式后附加内容。

  \opt{nameformat} 选项的最后一个格式命令可以是一个有一个参数的命令。
  这个命令的参数用于接受章节名字和编号,实现特殊效果(见例~\ref{exam:miscopt})。

  \opt{nameformat} 选项的默认值,在 \opt{scheme} 选项的不同取值下相同。
\end{function}

\begin{table}[htbp]
\small\centering
\caption{\opt{nameformat} 选项的默认设置}
\begin{tabular}{lll}
\toprule
标题名 & \opt{scheme = chinese} & \opt{scheme = plain} \\
\midrule
part (article) & |{}| & |\Large\bfseries| \\
part (beamer) & 同右
  & \BSTACK |\usebeamerfont{part name}| \\
            |\usebeamercolor[fg]{part name}| \ESTACK \\
part & |{}| & |\huge\bfseries| \\
chapter & |{}| & |\huge\bfseries| \\
section (beamer) & 同右
  & \BSTACK |\usebeamerfont{section name}| \\
            |\usebeamercolor[fg]{section name}| \ESTACK \\
section & 同右 & |{}| \\
subsection (beamer) & 同右
  & \BSTACK |\usebeamerfont{subsection name}| \\
            |\usebeamercolor[fg]{subsection name}| \ESTACK \\
subsection & 同右 & |{}| \\
subsubsection & 同右 & |{}| \\
paragraph & 同右 & |{}| \\
subparagraph & 同右 & |{}| \\
\bottomrule
\end{tabular}
\end{table}

\begin{function}[EXP,updated=2015-06-19]{.../numberformat, .../numberformat+}
  \begin{syntax}
    numberformat = \Arg{格式命令}
    numberformat+= \Arg{格式命令}
  \end{syntax}
  \opt{numberformat} 选项用于控制章节编号的格式,作用域仅为编号数字本身。对
  各级标题默认均为空,当你需要编号的格式和前后的章节名字不一样时可以使用。

  \opt{numberformat+} 选项用于在原有编号格式后面附加格式命令。

  \opt{numberformat} 选项的最后一个格式命令可以是一个有一个参数的命令。
  这个命令的参数用于接受编号数字。
\end{function}

  例如,我们可以使用 \opt{numberformat} 特别强调章标题中的数字:
  \begin{ctexexam}
  \ctexset{
    chapter/number = \arabic{chapter},
    chapter/numberformat = \color{blue}\zihao{0}\itshape,
  }
  \end{ctexexam}
  上面的代码在 \opt{scheme = chinese} 时可以做出类似这样的章标题效果:
  \begin{center}
  \huge\bfseries 第 \textit{\color{blue}\zihao{0}4} 章
  \end{center}

\begin{function}[EXP,updated=2014-03-08]{.../aftername, .../aftername+}
  \begin{syntax}
  aftername = \Arg{代码}
  aftername+= \Arg{代码}
  \end{syntax}
  \opt{aftername} 选项的参数 \meta{代码} 将被插入到章节编号与其后的标题内容之
  间,用于控制格式变换。常用于控制章节编号与标题内容之间的距离,或者控制标题
  是否另起一行。

  \opt{aftername+} 选项用于在原有代码后面附加代码。
\end{function}

\begin{table}[htbp]
\small\centering
\caption{\opt{aftername} 选项的默认设置}
\begin{tabular}{lll}
\toprule
标题名 & \opt{scheme = chinese} & \opt{scheme = plain} \\
\midrule
part (article) & |\quad| & |\par\nobreak| \\
part (beamer) & 同右 & |\vskip 1em \par| \\
part & 同右 & |\par\vskip 20pt| \\
chapter & |\quad| & |\par\nobreak\vskip 20pt| \\
section (beamer) & 同右 & |\vskip 1em \par| \\
section & 同右 & |\quad| \\
subsection (beamer) & 同右 & |\vskip 1em \par| \\
subsection & 同右 & |\quad| \\
subsubsection & 同右 & |\quad| \\
paragraph & 同右 & |\quad| \\
subparagraph & 同右 & |\quad| \\
\bottomrule
\end{tabular}
\end{table}

\begin{function}[EXP,updated=2015-06-30]{.../titleformat, .../titleformat+}
  \begin{syntax}
    titleformat = \Arg{格式命令}
    titleformat+= \Arg{格式命令}
  \end{syntax}
  \opt{titleformat} 选项用于控制标题内容的格式,作用域为章节标题内容。

  \opt{titleformat+} 选项用于在原有标题格式后面附加格式命令。

  需要注意的是,\opt{titleformat} 选项的最后一个格式命令可以是一个有一个
  参数的命令。这个命令的参数接受标题内容,用于实现特殊效果。
  例如,实现多行标题的居中悬挂对齐:
\end{function}
\begin{ctexexam}
  \usepackage{varwidth} %% 提供 varwidth 环境
  \ctexset{
    chapter/name = {第,回},
    chapter/titleformat = \chaptertitleformat
  }
  \newcommand\chaptertitleformat[1]{%% 以标题内容为参数
    \begin{varwidth}[t]{.7\linewidth}#1\end{varwidth}}
  ......
  \chapter{情中情因情感妹妹\\错里错以错劝哥哥}
\end{ctexexam}
上面的代码可以做出类似这样的章标题效果:
\begin{center}\Large\bfseries
第三十四回\quad
\begin{tabular}[t]{l}
  情中情因情感妹妹\\
  错里错以错劝哥哥
\end{tabular}
\end{center}

\begin{table}[htbp]
\small\centering
\caption{\opt{titleformat} 选项的默认设置}
\begin{tabular}{lll}
\toprule
标题名 & \opt{scheme = chinese} & \opt{scheme = plain} \\
\midrule
part (article) & |{}| & |\huge\bfseries| \\
part (beamer) & 同右 & |\usebeamerfont{part title}| \\
part & |{}| & |\Huge\bfseries| \\
chapter & |{}| & |\Huge\bfseries| \\
section (beamer) & 同右 & |\usebeamerfont{section title}| \\
section & 同右 & |{}| \\
subsection (beamer) & 同右 & |\usebeamerfont{subsection title}| \\
subsection & 同右 & |{}| \\
subsubsection & 同右 & |{}| \\
paragraph & 同右 & |{}| \\
subparagraph & 同右 & |{}| \\
\bottomrule
\end{tabular}
\end{table}

\begin{function}[EXP,added=2015-06-19]{.../aftertitle, .../aftertitle+}
  \begin{syntax}
  aftertitle = \Arg{代码}
  aftertitle+= \Arg{代码}
  \end{syntax}
  \opt{aftertitle} 选项的参数 \meta{代码} 将被插入到章节标题内容之后。

  \opt{aftertitle+} 选项用于在原有代码后面附加代码。

  \opt{aftertitle} 选项的默认值,在 \opt{scheme} 选项的不同取值下相同。

  \opt{sub3section} 或 \opt{sub4section} 宏包选项(见
  \ref{subs:options-heading}~节)会影响 \opt{aftertitle} 选项的默认值。
\end{function}

\begin{table}[htbp]
\begin{minipage}[t]{.5\linewidth}
\small\centering
\caption{\opt{aftertitle} 选项的默认设置}
\begin{tabular}{ll}
\toprule
标题名 & 默认值 \\
\midrule
part & |\par| \\
chapter & |\par| \\
section & |\@@par| \\
subsection & |\@@par| \\
subsubsection & |\@@par| \\
paragraph & |{}| \\
\qquad(sub3section) & |\@@par| \\
\qquad(sub4section) & 同上 \\
subparagraph & |{}| \\
\qquad(sub4section) & |\@@par| \\
\bottomrule
\end{tabular}
\end{minipage}%
\begin{minipage}[t]{.5\linewidth}
\small\centering
\caption{\opt{runin} 选项的默认设置}
\begin{tabular}{ll}
\toprule
标题名 & 默认值 \\
\midrule
part & 无效 \\
chapter & 无效 \\
section & |false| \\
subsection & |false| \\
subsubsection & |false| \\
paragraph & |true| \\
\qquad(sub3section) & |false| \\
\qquad(sub4section) & 同上 \\
subparagraph & |true| \\
\qquad(sub4section) & |false| \\
\bottomrule
\end{tabular}
\end{minipage}
\end{table}

\begin{function}[EXP,added=2015-06-27]{section/runin, subsection/runin,
  subsubsection/runin, paragraph/runin, subparagraph/runin}
  \begin{syntax}
  runin = \TF
  \end{syntax}
  \opt{runin} 选项只对 \tn{section} 级以下标题有意义。
  用于确定标题与随后的正文是否排在同一段之上。

  \opt{runin} 选项的默认值,在 \opt{scheme} 选项的不同取值下相同。

  默认情况下,\tn{paragraph}、\tn{subparagraph} 两级标题是与后面正文排在同一
  段的,\opt{runin} 选项为 \opt{true};但使用 \opt{sub3section} 或
  \opt{sub4section} 宏包选项(见 \ref{subs:options-heading}~节)后,
  将对这两级标题设 \opt{runin} 选项为 \opt{false},这两级标题会改为排在不同段。
\end{function}

\begin{function}[EXP,added=2015-06-27]{.../afterindent}
  \begin{syntax}
  afterindent = \TF
  \end{syntax}
  \opt{afterindent} 选项用于设置章节标题后首段的缩进。

  \cls{book} 和 \cls{report} 类的 \tn{part} 标题被单独排在一页之上,
  \opt{afterindent} 选项没有意义。

  对于 \tn{section} 级以下标题,若设置了 \opt{runin} 选项为 \opt{true},
  即标题与随后正文排在同一段,\opt{afterindent} 选项也就没有了意义。
\end{function}

\begin{table}[htbp]
\small\centering
\caption{\opt{afterindent} 选项的默认设置}
\begin{tabular}{lll}
\toprule
标题名 & \opt{scheme = chinese} & \opt{scheme = plain} \\
\midrule
part (article) & |true| & |false| \\
part & 无效 & 无效 \\
chapter & |true| & |false| \\
section & |true| & |false| \\
subsection & |true| & |false| \\
subsubsection & |true| & |false| \\
paragraph & |true| & |false| \\
subparagraph & |true| & |false| \\
\bottomrule
\end{tabular}
\end{table}

\begin{function}[EXP,updated=2016-05-10]{.../beforeskip}
  \begin{syntax}
  beforeskip = \Arg{弹性间距}
  \end{syntax}
  \opt{beforeskip} 选项用于设置章节标题前的垂直间距。

  \opt{beforeskip} 选项的默认值,在 \opt{scheme} 选项的不同取值下相同。
\end{function}

\begin{table}[htbp]
\setlength\leftskip{0pt plus 1 fil minus \marginparwidth}
\begin{minipage}[t]{.6\linewidth}
\small\centering
\caption{\opt{beforeskip} 选项的默认设置}
\begin{tabular}{ll}
\toprule
标题名 & 默认值 \\
\midrule
part (article) & |4ex| \\
part (beamer) & |0pt| \\
part & |0pt plus 1fil| \\
chapter & |50pt| \\
section (beamer) & |0pt| \\
section & |3.5ex plus 1ex minus .2ex| \\
subsection (beamer) & |0pt| \\
subsection & |3.25ex plus 1ex minus .2ex| \\
subsubsection & |3.25ex plus 1ex minus .2ex| \\
paragraph & |3.25ex plus 1ex minus .2ex| \\
subparagraph & |3.25ex plus 1ex minus .2ex| \\
\bottomrule
\end{tabular}
\end{minipage}%
\begin{minipage}[t]{.5\linewidth}
\small\centering
\caption{\opt{afterskip} 选项的默认设置}
\begin{tabular}{ll}
\toprule
标题名 & 默认值 \\
\midrule
part (article) & |3ex| \\
part (beamer) & |0pt| \\
part & |0pt plus 1fil| \\
chapter & |40pt| \\
section (beamer) & |0pt| \\
section & |2.3ex plus .2ex| \\
subsection (beamer) & |0pt| \\
subsection & |1.5ex plus .2ex| \\
subsubsection & |1.5ex plus .2ex| \\
paragraph & |1em| \\
\qquad(sub3section) & |1ex plus .2ex| \\
\qquad(sub4section) & 同上 \\
subparagraph & |1em| \\
\qquad(sub4section) & |1ex plus .2ex| \\
\bottomrule
\end{tabular}
\end{minipage}
\end{table}

\begin{function}[EXP,updated=2015-06-27]{.../afterskip}
  \begin{syntax}
  afterskip = \Arg{弹性间距}
  \end{syntax}
  \opt{afterskip} 选项控制章节标题与后面下方之间的距离。

  对于 \tn{section} 级以下标题,\opt{runin} 选项会影响 \opt{afterskip} 选项的意义:
  若 \opt{runin} 为 \opt{true},标题与随后正文排在同一段,\meta{弹性间距} 给出水平间距。
  否则,正文另起一段,\meta{弹性间距} 给出的是垂直间距。

  \opt{afterskip} 选项的默认值,在 \opt{scheme} 选项的不同取值下相同。

  \opt{sub3section} 或 \opt{sub4section} 宏包选项(见
  \ref{subs:options-heading}~节)会影响 \opt{aftertitle} 选项的默认值。
\end{function}

\begin{function}[EXP,added=2016-06-03]{.../fixskip}
  \begin{syntax}
  fixskip = \TFF
  \end{syntax}
  默认情况下,\cls{article}、\cls{book} 和 \cls{report} 类的标题与正文的距离除了由
  \opt{beforeskip} 和 \opt{afterskip} 选项设置的垂直间距外,还会有一些多余的间距。
  \opt{fixskip} 选项用于抑制这些多余间距。
\end{function}

\begin{function}[EXP,added=2016-10-01]{chapter/lofskip, chapter/lotskip}
  \begin{syntax}
  lofskip = \Arg{弹性间距}
  lotskip = \Arg{弹性间距}
  \end{syntax}
  \opt{lofskip} 选项控制插图目录(\file{.lof})中,章之间的插图标题的距离。

  同样,\opt{lotskip} 选项控制表格目录(\file{.lot})中,章之间的表格标题的距离。

  目前,这两个选项只在 \opt{chapter} 标题下有定义。
  他们的默认值,在 \opt{scheme} 选项的不同取值下都为 \SI{10}{pt}。
\end{function}

\begin{function}[EXP,updated=2015-06-27]{.../indent}
  \begin{syntax}
  indent = \Arg{缩进间距}
  \end{syntax}
  \opt{indent} 选项用于设置章节标题本身的首行缩进。

  \opt{indent} 选项的默认值,在 \opt{scheme} 选项的不同取值下相同。

  如果 \opt{indent} 的值是以 \texttt{em}、\texttt{ex} 或 \cs{ccwd} 为单位,
  那么缩进间距的大小是相对于 \opt{format} 中指定的字号大小。

  例如,设置 \tn{part} 标题缩进三个字、\tn{section} 标题缩进 \SI{20}{pt}:
\end{function}
\begin{ctexexam}
  \ctexset{
    part={
      format+=\raggedright,
      indent=3\ccwd,
    },
    section={
      format=\Large\bfseries,
      indent=20pt,
    }
  }
  \part{首行缩进的标题}
  \noindent 无缩进的正文。
  \section{首行缩进的标题}
  \noindent 无缩进的正文。
\end{ctexexam}

\begin{table}[htbp]
\begin{minipage}[t]{.5\linewidth}
\small\centering
\caption{\opt{indent} 选项的默认设置}
\begin{tabular}{ll}
\toprule
标题名 & 默认值 \\
\midrule
part & |0pt| \\
chapter & |0pt| \\
section & |0pt| \\
subsection & |0pt| \\
subsubsection & |0pt| \\
paragraph & |0pt| \\
subparagraph & |\parindent| \\
\qquad(sub3section) & |0pt| \\
\qquad(sub4section) & 同上 \\
\bottomrule
\end{tabular}
\end{minipage}%
\begin{minipage}[t]{.5\linewidth}
\small\centering
\caption{\opt{hang} 选项的默认设置}
\begin{tabular}{ll}
\toprule
标题名 & 默认值 \\
\midrule
part & |false| \\
chapter & |false| \\
section & |true| \\
subsection & |true| \\
subsubsection & |true| \\
paragraph & 无意义 \\
\qquad(sub3section) & |true| \\
\qquad(sub4section) & |true| \\
subparagraph & 无意义 \\
\qquad(sub4section) & |true| \\
\bottomrule
\end{tabular}
\end{minipage}%
\end{table}

\begin{function}[EXP,updated=2019-03-31]{part/hang, chapter/hang, section/hang,
  subsection/hang, subsubsection/hang, paragraph/hang, subparagraph/hang}
  \begin{syntax}
  hang = \TF
  \end{syntax}
  \opt{hang} 选项用于设置是否对章节标题实施悬挂缩进(缩进的宽度为名字宽度和 \opt{indent} 选项
  设置的宽度之和)。

  本选项对 \cls{beamer}/\cls{ctexbeamer} 文档类无效。
  对于 \tn{section} 级以下标题,若设置了 \opt{runin} 选项为 \opt{true},
  即标题与随后正文排在同一段,\opt{hang} 选项没有意义。
\end{function}

\begin{function}[EXP,added=2014-03-21]{part/pagestyle, chapter/pagestyle}
  \begin{syntax}
    pagestyle = \Arg{页面格式}
  \end{syntax}
  设置 \cls{book}/\cls{ctexbook} 或 \cls{report}/\cls{ctexrep} 文档类
  中,\tn{part} 与 \tn{chapter} 标题所在页的页面格式(page style)。
\end{function}

\begin{table}[htbp]
\small\centering
\caption{\opt{pagestyle} 选项的默认设置}
\begin{tabular}{ll}
\toprule
标题名 & 默认值 \\
\midrule
part (article) & 无效 \\
part & |plain| \\
chapter & |plain| \\
\bottomrule
\end{tabular}
\end{table}

\begin{function}[EXP,added=2016-09-19]{.../break, .../break+}
  \begin{syntax}
    break = \Arg{格式命令}
    break+= \Arg{格式命令}
  \end{syntax}
  \opt{break} 选项用于控制章节标题与之前正文的分隔关系。一般用于设置是否在标题之前分页或者设置行间罚点。

  带加号的 \opt{break+} 选项则用于在已有格式之后追加新的格式命令。

  \opt{break} 选项的默认值,在 \opt{scheme} 选项的不同取值下相同。

  例如,若当前页剩余高度小于正文高度的一半时,则另起一页输出 \tn{section} 标题:
  \begin{ctexexam}
  \usepackage{needspace}
  \ctexset{section/break = \Needspace{.5\textheight}}
  \end{ctexexam}
\end{function}

\begin{table}[htbp]
\small\centering
\caption{\opt{break} 选项的默认设置}
\begin{tabular}{ll}
\toprule
标题名 & 默认值 \\
\midrule
part (article) & |{}| \\
part & |\if@openright\cleardoublepage\else\clearpage\fi| \\
chapter & 同上 \\
section & |\addpenalty{\@secpenalty}| \\
subsection & 同上 \\
subsubsection & 同上 \\
paragraph & 同上 \\
subparagraph & 同上 \\
\bottomrule
\end{tabular}
\end{table}

\begin{function}[EXP,added=2016-10-25]{.../tocline}
  \begin{syntax}
    tocline = \Arg{格式定义}
  \end{syntax}
  \opt{tocline} 选项用于定义章节标题在目录文件(\file{.toc})中的格式。\meta{格式定义}有两个参数:
  参数 |#1| 是 |part|、|chapter| 等名字,参数 |#2| 是标题内容。
\end{function}

\begin{table}[htbp]
\small\centering
\caption{\opt{tocline} 选项的默认设置}
\begin{tabular}{ll}
\toprule
标题名 & 默认值 \\
\midrule
part & |\CTEXifname{\CTEXthepart\hspace{1em}}{}#2| \\
chapter (\opt{chinese})
  & |\CTEXifname{\protect\numberline{\CTEXthechapter\hspace{.3em}}}{}#2| \\
chapter (\opt{plain})
  & |\CTEXnumberline{#1}#2| \\
section & |\CTEXnumberline{#1}#2| \\
subsection & 同上 \\
subsubsection & 同上 \\
paragraph & 同上 \\
subparagraph & 同上 \\
\bottomrule
\end{tabular}

\medskip
其中 \tn{CTEXnumberline} 的意义是若标题 |#1| 没有名字,则不输出 |\numberline{\CTEXthe#1}|
等编号:
\begin{verbatim}
  \CTEXifname{\protect\numberline{\csname CTEXthe#1\endcsname}}{}
\end{verbatim}
\end{table}

\begin{function}[EXP,added=2015-06-21]{appendix/numbering}
  \begin{syntax}
    numbering = \TTF
  \end{syntax}
  控制是否对附录章(对 \cls{book} 与 \cls{report})或附录节(对 \cls{article})
  进行编号。

  用法与普通章节 \opt{numbering} 选项类似。
\end{function}

\begin{function}[EXP,updated=2014-03-08]{appendix/name}
  \begin{syntax}
    name = \{<前名字>,<后名字>\}
    name = \Arg{前名字}
  \end{syntax}
  设置附录章(对 \cls{book} 与 \cls{report})或附录节(对 \cls{article})的
  名字。

  用法与普通章节 \opt{name} 选项类似。

  注意该选项与 \opt{appendixname} 选项(\ref{subs:capname}~节)在意义上有
  些重叠,但意义不同。\opt{appendixname} 选项只用来重定义
  \tn{appendixname},而不管 \tn{appendixname} 如何使用;该选项则决定在章节标
  题中输出的名字,可以调用 \tn{appendixname} 设置。
\end{function}

\begin{table}[htbp]
\small\centering
\caption{\opt{appendix/name} 选项的默认设置}
\begin{tabular}{llllll}
\toprule
文档类 & 影响命令 & \opt{scheme = chinese} & 实际定义
  & \opt{scheme = plain} & 实际定义 \\
\midrule
article & \tn{section} & |{}| & & |{}| & \\
book, report & \tn{chapter} & |\appendixname\space| & \verb*|附录 |
                            & |\appendixname\space| & \verb*|Appendix | \\
\bottomrule
\end{tabular}
\end{table}

\begin{function}[EXP]{appendix/number}
  \begin{syntax}
    number = \Arg{数字输出命令}
  \end{syntax}
  设置附录章(对 \cls{book} 与 \cls{report})或附录节(对 \cls{article})编
  号的数字输出格式。

  用法与普通章节的 \opt{number} 选项类似。

  该选项也同时控制附录章节计数器的交叉引用。

  与普通章节的 \opt{number} 选项类似,同样需要注意,该选项不会影响计数器本身
  的输出,即不影响 \tn{thesection} 或 \tn{thechapter} 的定义。
\end{function}

\begin{table}[htbp]
\small\centering
\caption{\opt{appendix/number} 选项的默认设置}
\begin{tabular}{llllll}
\toprule
文档类 & 影响命令 & 默认值 \\
\midrule
article & \tn{section} & |\Alph{section}| \\
book, report & \tn{chapter} & |\Alph{chapter}| \\
\bottomrule
\end{tabular}
\end{table}

我们最后举一个稍微复杂的例子,来看看上述选项的综合应用。

\begin{ctexexam}[labelref=exam:miscopt]
  \ctexset {
    chapter = {
      beforeskip = 0pt,
      fixskip    = true,
      format     = \Huge\bfseries,
      nameformat = \rule{\linewidth}{1bp}\par\bigskip\hfill\chapternamebox,
      number     = \arabic{chapter},
      aftername  = \par\medskip,
      aftertitle = \par\bigskip\nointerlineskip\rule{\linewidth}{2bp}\par
    }
  }
  \newcommand\chapternamebox[1]{%
    \parbox{\ccwd}{\linespread{1}\selectfont\centering #1}}
  ......
  \chapter{熟悉 \LaTeX}
\end{ctexexam}
本例的设置效果大致如下:
\begin{center}
\begin{minipage}{.75\linewidth}\Large\bfseries
\hrule height .5bp \relax
\medskip
\hfill\parbox{\ccwd}{\linespread{1}\selectfont\centering 第 1 章}
\par\smallskip
\noindent 熟悉 \LaTeX
\medskip\hrule height 1bp \relax
\end{minipage}
\end{center}

\section{实用命令}
\label{sec:useful-commands}
\subsection{字号与间距}

\begin{function}[updated = 2014-03-08, label = ]{\zihao}
  \begin{syntax}
    \tn{zihao} \Arg{字号}
  \end{syntax}
  用于调整字号大小。其中 \meta{字号} 的有效值共有 16 个,如表 \ref{tab:zihao}
  所示。使用 \tn{zihao} 命令调整字体大小时,西文字号大小会始终和中文字号保持一致。
\end{function}

\begin{table}[!htbp]
\centering
\def\ZH#1{\zihaopt{#1} & \zihao{#1}}
\tabcolsep=1em
\caption{中文字号}\label{tab:zihao}
\begin{tabular}{>{\ttfamily}S[mode=text,detect-family,table-format=-1]
                            S[table-format=2.1]S[table-format=2.5]l}
\toprule
{\meta{字号}} & {大小(bp)} & {大小(pt)} & 意义 \\
\midrule
 0   & 42   & \ZH{0}    初号    \\
{−0} & 36   & \ZH{-0}   小初号  \\
 1   & 26   & \ZH{1}    一号    \\
-1   & 24   & \ZH{-1}   小一号  \\
 2   & 22   & \ZH{2}    二号    \\
-2   & 18   & \ZH{-2}   小二号  \\
 3   & 16   & \ZH{3}    三号    \\
-3   & 15   & \ZH{-3}   小三号  \\
 4   & 14   & \ZH{4}    四号    \\
-4   & 12   & \ZH{-4}   小四号  \\
 5   & 10.5 & \ZH{5}    五号    \\
-5   &  9   & \ZH{-5}   小五号  \\
 6   &  7.5 & \ZH{6}    六号    \\
-6   &  6.5 & \ZH{-6}   小六号  \\
 7   &  5.5 & \ZH{7}    七号    \\
 8   &  5   & \ZH{8}    八号    \\
\bottomrule
\end{tabular}
\end{table}

\begin{function}[updated=2014-03-28]{\ziju}
  \begin{syntax}
    \tn{ziju} \Arg{中文字符宽度的倍数}
  \end{syntax}
  用于调整相邻汉字之间的间距,即(在正常中文行文中)前一个汉字的右边缘与后一个汉字
  的左边缘之间的距离。其中参数可以是任意浮点数值;而中文字符宽度指的是实际汉字的
  宽度,不包含当前字距。

  这个命令会影响 \tn{ccwd} 的值,但不会影响英文字距。
\end{function}

\begin{function}[updated=2014-03-27]{\ccwd}
  当前汉字的字宽保存在长度寄存器 \tn{ccwd} 之中。汉字字宽是相邻两个汉字中心
  之间的距离,包含字距在内。因此修改字距会间接修改字宽。
\end{function}

\subsection{中文数字转换}

\CTeX{} 宏集的中文数字转换功能实际上是调用 \pkg{zhnumber} 宏包来完成。下面只
介绍一些基本的用法,更高级的用法可以查阅 \pkg{zhnumber} 宏包的文档。

\begin{function}[updated=2016-05-01]{\chinese}
  \begin{syntax}
    \tn{chinese} \Arg{counter}
    \tn{pagenumbering} \{chinese\}
  \end{syntax}
  \tn{chinese} 命令与 \tn{roman} 等命令的用法类似,作用在一个 \LaTeX{}
  计数器上,将计数器的值以中文数字的形式输出。
\end{function}

\begin{function}[added=2014-03-08]{\zhnumber}
  \begin{syntax}
    \tn{zhnumber} \Arg{number}
  \end{syntax}
  以中文格式输出数字。这里的数字可以是整数、小数和分数。
\end{function}

\begin{function}[added=2014-03-08]{\zhdigits}
  \begin{syntax}
    \tn{zhdigits} \Arg{number}
  \end{syntax}
  将阿拉伯数字转换为中文数字串。
\end{function}

\begin{function}{\CTEXnumber}
  \begin{syntax}
    \tn{CTEXnumber} "\"<macro> \Arg{number}
  \end{syntax}
  |\|<macro> 必须是一个 \TeX{} 宏,不需预先定义。\tn{CTEXnumber} 通过
  \tn{zhnumber} 将 \meta{number} 转为中文数字,最后将结果存储在 |\|<macro>
  里。对 |\|<macro> 的定义是局部的,将它展开一次就可以得到转换结果。
\end{function}

一般来说,并不需要使用 \tn{CTEXnumber},直接使用 \tn{zhnumber} 即可。但是,如果
在文档中需要多次使用同一个数字 \meta{number} 的中文形式,就可以先用
\tn{CTEXnumber} 将结果保存起来备用,而不是每次使用时都用 \tn{zhnumber} 现场
转换一次。

\begin{function}{\CTEXdigits}
  \begin{syntax}
    \tn{CTEXdigits} "\"<macro> \Arg{number}
  \end{syntax}
  \tn{CTEXdigits} 与 \tn{CTEXnumber} 类似,但其转换的结果是中文数字串,而不是
  中文数字。
\end{function}

\subsection{杂项}

\begin{function}{\CTeX}
用于显示 \CTeX 标志。
\end{function}

\section{\LuaLaTeX{} 下的中文支持方式}
\label{sec:lualatex-chinese}

在 \LuaLaTeX{} 下,\CTeX{} 宏集依赖 \pkg{LuaTeX-ja} 宏包来完成中文支持。
该宏包是日本 \TeX{} 社区的北川弘典、前田一贵、八登崇之等人开发的,设计目的主要
是在 \LuaTeX{} 引擎下实现日本 p\TeX{} 引擎的(大部分)功能。它为了兼容 p\LaTeX
的使用习惯,对 \LaTeXe 的 \pkg{NFSS} 作了不少修改和扩充。这对于简体中文用户来说
不是必要的,因而 \CTeX{} 禁用了它在 \LaTeX{} 格式下的大部分设置,只保留了必要的
部分。同时修改了它的字体设置方式,使得相关命令与 \pkg{xeCJK} 宏包大致相同。

20150420 版以后的 \pkg{LuaTeX-ja} 宏包开始支持竖排,但 \CTeX{} 暂不支持竖排。

\subsection{\LuaLaTeX{} 下替代字体的设置}

\begin{function}[added=2014-04-14]{AlternateFont}
  \begin{syntax}
    \tn{setCJKfamilyfont} \Arg{family}
    \  [
    \    AlternateFont =
    \      \{
    \        \Arg{character range_1} \oarg{alternate font features_1} \Arg{alternate font name_1} ||
    \        \Arg{character range_2} \oarg{alternate font features_2} \Arg{alternate font name_2} ||
    \        ......
    \      \} ,
    \    <base font features>
    \  ] \Arg{base font name}
  \end{syntax}
  在设置字体族 \meta{family} 的时候,同时设置该字体族在字符范围
  \meta{character range_n} 内,对应字形的替代字体。
\end{function}

\begin{function}[added=2014-04-14]{CharRange}
  \begin{syntax}
    \tn{setCJKfamilyfont} \Arg{family}
    \  [
    \    CharRange = \Arg{character range} ,
    \    <alternate font features>
    \  ] \Arg{alternate font name}
  \end{syntax}
  只设置字体族 \meta{family} 在字符范围 \meta{character range} 内,对应字形的
  替代字体。
\end{function}

一个 \tn{setCJKfamilyfont} 里只能使用一次 \opt{CharRange} 或者
\opt{AlternateFont},但可以将它们分开重叠使用。例如下面的方式是有效的。

\begin{ctexexam}
  \setCJKmainfont[AlternateFont={...}{...}, ...]{...}
  \setCJKmainfont[CharRange={"4E00->"67FF,-2}, ...]{...}
  \setCJKmainfont[CharRange={"6800->"9FFF}, ...]{...}
\end{ctexexam}

\begin{function}[EXP,added=2014-04-14]{declarecharrange}
  \begin{syntax}
    \tn{ctexset}
    \  \{
    \    declarecharrange =
    \      \{
    \        \Arg{name_1} \Arg{character range_1} ,
    \        \Arg{name_2} \Arg{character range_2} ,
    \        ...
    \      \}
    \  \}
  \end{syntax}
  预先声明字符范围。声明字符范围 \meta{name} 之后,它的名字 \meta{name} 可以
  用在 \opt{AlternateFont} 和 \opt{CharRange} 选项的 \meta{character range}
  之中,表示对应的字符范围。
\end{function}

在声明字符范围 \meta{name} 的同时,还为 \tn{setCJKmainfont} 等字体设置命令定义
了选项 \meta{name},用于设置对应字符的替代字体:
\begin{quote}\linespread{1}\small\ttfamily
  \meta{name} = \oarg{alternate font features} \Arg{alternate font name}
\end{quote}
\meta{name} 选项可以与 \opt{AlternateFont} 共同使用,但不能与 \opt{CharRange}
一起使用。如果没有给 \meta{name} 设置值,则等价于设置 \opt{CharRange=\meta{name}},
即只设置 \meta{name} 对应的字符范围的替代字体。

\begin{function}[EXP,added=2014-04-15]{clearalternatefont,resetalternatefont}
  \begin{syntax}
    \tn{ctexset}
    \  \{
    \    clearalternatefont = \Arg{family_1, family_2, ...} ,
    \    resetalternatefont = \Arg{family_1, family_2, ...} ,
    \    clearalternatefont ,
    \    resetalternatefont
    \  \}
  \end{syntax}
  清除与重置 CJK 字体族 \meta{family} 的替换字体设置。如果没有给定值,则作用于
  当前 CJK 字体族。清除与重置操作总是全局的。
\end{function}

\section{\CTeX{} 宏集的配置文件}

\CTeX{} 宏集提供了不同的配置文件,可以通过修改配置文件来改变 \CTeX{} 宏集的
默认行为。

在多数情况下,并不需要修改配置文件,\CTeX{} 宏集的默认设置已经能满足大多数用
户的需要。不恰当地修改 \CTeX{} 宏集的默认行为也可能导致同一文件在别处无法正
常编译或排版效果完全不同,因此修改应该慎重。

但在一些情况下,直接修改配置文件仍是必要的,例如:
\begin{itemize}
\item 系统没有安装默认设置的字体文件,无法编译。
\item 需要经常编译来自其他系统的中文 \TeX{} 文件,但对方的操作系统或默认设置
与本机不同。
\end{itemize}

与 \CTeX{} 宏集的源代码一样,配置文件采用 \LaTeXiii{} 的语法编写。

\CTeX{} 宏集的配置文件随宏包其他文件一起安装在 \TeX{} 系统 TDS 目录树中,文
件后缀是 \file{.cfg}。为了避免本地配置文件内容因 \CTeX{} 宏集的更新而丢失,
不要直接修改系统 TDS 目录树中的配置文件,而应该将系统自带的配置文件复制到本
地的或用户私有的 TDS 目录树中修改,并运行 \bashcmd{texhash} 命令刷新文件名数据库。

例如对于 \TeX{} Live,系统自带的配置文件就在 \TeX{} Live 安装目录下的
\path{texmf-dist/tex/latex/ctex/config/} 子目录下,可以修改它的副本,保存在
本地 TDS 树的 \path{texmf-local/tex/latex/ctex/} 目录下,或者用户 TDS 树的
\path{~/.texlive2015/texmf-var/tex/latex/ctex/} 目录下,作为本地/用户专有的
配置文件。复制配置文件后需要运行 |texhash| 命令使本地配置文件生效。

\MiKTeX{} 的配置文件也保存在类似的目录结构中,\MiKTeX{} 管理的
几个 TDS 根目录可以在 \MiKTeX{} Options 设置项中查看到,这里不再赘述。

除了修改本地 \TeX{} 系统中的配置文件,对于特定文档,也可以将修改过的配置文件
保存在文档的工作目录下。此时配置文件就只对工作目录下的所有文档生效。

\subsection{修改宏包默认选项}

配置文件 \file{ctexopts.cfg} 可以用来修改宏包的默认选项。随系统安装的配置文
件除了文件信息声明外没有实际的内容,但在注释中给出了一个简单的示例,只要取消
注释就可以生效。

\begin{ctexexam}
  % 系统自带 ctexopts.cfg 注释中的示例语句,固定默认字体集为 windows。
  % 该设置可以用在安装了 Windows 字体的非 Windows 系统中。
  \keys_set:nn { ctex / option } { fontset = windows }
\end{ctexexam}
如上例所示,宏包选项通常使用 \LaTeXiii{} 的 \cs{keys_set:nn} 命令完成键值设置,
第一个参数是固定的子模块 |ctex/option|,第二个参数中是用户定义的新的默认宏包
选项。

\file{ctexopts.cfg} 中的设置将在 \CTeX{} 宏集的开始处,定义过宏包选项之后,
\tn{ProcessKeysOptions} 命令之前生效。最好只使用此配置文件修改宏包默认选项。

\subsection{宏包载入后的配置}

配置文件 \file{ctex.cfg} 将在宏包的末尾被载入生效。可以用它完成任意的设置,
或是覆盖已有的定义。随系统安装的配置文件除版本信息外没有实际内容,注意配置文
件中也使用 \LaTeXiii{} 语法。

\begin{ctexexam}
  % 简单的 ctex.cfg 内容示例。
  % 修改默认的页面格式设置。
  \pagestyle{plain}
\end{ctexexam}

\begin{ctexexam}
  % 略复杂的 ctex.cfg 内容示例:禁止段末孤字成行。
  % 在使用 XeTeX 编译时,打开 xeCJK 的 CheckSingle 选项。
  \sys_if_engine_xetex:T
    {
      \xeCJKsetup { CheckSingle }
    }
  % 在使用 LuaTeX 编译时,设置 LuaTeX-ja 的 jcharwidowpenalty 参数。
  \sys_if_engine_luatex:T
    {
      \ltjsetparameter { jcharwidowpenalty = 10000 }
    }
\end{ctexexam}

\subsection{配置标题中文翻译}

由于 \CTeX{} 宏集需要同时支持 GBK 和 UTF-8 两种编码,因此对标题的中文翻译写
在两个配置文件当中:\file{ctex-name-gbk.cfg} 和 \file{ctex-name-utf8.cfg}。
两个文件的设置相同,只是编码不同。

为了同一文档在不同电脑上编译效果的一致性,通常不建议修改默认的中文翻译。

\subsection{自定义字体集}

\ref{subs:options-CJK-font}~节介绍的用于 |fontset| 选项的自定义字库文件,
类似于 \CTeX{} 宏集的配置文件,也应该与其他本地配置文件一起保存在本地
\texttt{TDS} 目录树下,并可以配合 \file{ctexopts.cfg} 等配置文件使用。

\section{对旧版本的兼容性}

\subsection{\CTeX\ 0.8a 及以前的版本}

在 ctex-kit 项目成立之前,\CTeX 宏包的最后一个版本是 \CTeX\
0.8a(2007/05/06)。

第 2 版未考虑对这些很早版本的兼容性。

\subsection{\CTeX\ 0.9--\CTeX\ 1.0d}

在 2009 年在 ctex-kit 项目成立后,新增了 \XeTeX{} 引擎的支持,并增加了不少控
制字体的命令和选项。

这里主要介绍新版本 \CTeX 宏包相对 1.02d 版本(2014/06/09)的兼容性。

第 2 版的 \CTeX 宏包已尽力保证对 1.0x 版本的兼容性,原有为 1.0x 编写的代码,
在第 2 版的 \CTeX 宏包下保证仍能编译,并且在大多数情况下保持编译效果不变。

\CTeX 宏包在 0.8a 以前的版本支持以 \pkg{CCT} 作为底层中文支持方式,从 0.9 版
之后即不再推荐使用,只保留向后兼容。在 \CTeX 宏包第 2 版中则完全不再支持
\pkg{CCT}。

下面这些是在旧版本 \CTeX 宏包中存在,而在新版本中已不建议使用的选项和命令,
在未来版本中可能会删去它们的支持。

在多数情况下它们的功能仍将保留,但也有部分选项命令功能已失效。

\begin{function}{cs4size, c5size}
  分别相当于 |zihao=-4| 和 |zihao=5|,过时选项。
\end{function}

\begin{function}{CCT, CCTfont}
相关选项已删除。
\end{function}

\begin{function}{indent, noindent}
\opt{indent} 和 \opt{noindent} 什么也不做,过时选项。

在中文版式下,\pkg{ctex} 宏包的相关功能在与标准文档类及其衍生文档类联用时
默认打开。\CTeX{} 文档类的相关功能由章节标题的 \opt{afterindent} 选项的值
来确定。
\end{function}

\begin{function}[label = ]{zhmap, nozhmap}
\opt{zhmap} 宏包选项增加了参数,扩充了功能,除了支持真假值参数外,还支持选择
\pkg{zhmCJK} 作为底层中文处理宏包。(\ref{subs:options-CJK-font}~节)

\opt{nozhmap} 选项相当于 |zhmap=false|。过时选项。
\end{function}

\begin{function}{winfonts, adobefonts, nofonts}
宏包选项 \opt{winfonts} 相当于 |fontset=windows|,\opt{adobefonts} 相当于
|fontset=adobe|,\opt{nofonts} 相当于 |fontset=none|。这几个选项是过时选项,
对于新文档,应使用 \opt{fontset} 选项设置不同字体集。

另外,第 2 版 \CTeX 宏包的默认字体不再是 Windows 系统字体,而是根据检测到的
操作系统选择使用 Windows、Mac 的系统字体还是 Fandol 字体
(\ref{subs:options-CJK-font}~节)。
\end{function}

\begin{function}[label = ]{punct, nopunct}
旧版本中宏包 \opt{punct} 选项没有参数,现在可以用参数设定标点风格
(\ref{subs:options-type-style}~节)。原有无参形式的 \opt{punct} 选项相当
于 |punct=quanjiao|。

旧版宏包中 \opt{nopunct} 选项的效果大致相当于 |punct=plain|。过时选项,不推荐使用。
\end{function}

\begin{function}{cap, nocap}
原有的 \opt{cap} 和 \opt{nocap} 选项由新的 \opt{scheme} 选项代替。
(\ref{subs:options-type-style}~节)

\opt{cap} 选项相当于 |scheme = chinese|,\opt{nocap} 选项相当
于 |scheme = plain|。它们均已过时,仅因兼容性而保留。
\end{function}

\begin{function}[label = ]{space, nospace}
新版本宏包 \opt{space} 选项增加真假值参数。
(\ref{subs:options-type-style}~节)

\opt{nospace} 选项相当于 |space=false|,成为过时选项。
\end{function}

\begin{function}{fancyhdr}
新版本宏包中总是自动处理对 \pkg{fancyhdr} 宏包的兼容性,而由用户自己使用
\tn{usepackage} 载入 \pkg{fancyhdr} 宏包。

\opt{fancyhdr} 选项过时,因兼容性保留,功能是载入 \pkg{fancyhdr} 宏包。
\end{function}

\begin{function}{hyperref}
新版本宏包中总是自动处理对 \pkg{hyperref} 宏包的兼容性,而由用户自己使用
\tn{usepackage} 载入 \pkg{hyperref} 宏包。

\opt{hyperref} 选项过时,因兼容性保留,功能是在导言区末尾载入 \pkg{hyperref}
宏包。
\end{function}

\begin{function}{fntef}
旧版本的 \opt{fntef} 选项用于统一 \pkg{CCTfntef} 与 \pkg{CJKfntef} 的界面,
新版本 \CTeX{} 宏集不再支持 \pkg{CCT},也不再自动载入 \pkg{CJKfntef} 或
\pkg{xeCJKfntef} 宏包,而仅在其末尾做适当格式调整。

\opt{fntef} 选项过时,因兼容性保留,功能是根据引擎载入 \pkg{CJKfntef}
(\pdfTeX{}) 或 \pkg{xeCJKfntef} (\XeTeX{}) 宏包。
\end{function}

\begin{function}{\CTEXunderdot, \CTEXunderline, \CTEXunderdblline,
  \CTEXunderwave, \CTEXsout, \CTEXxout, \CTEXfilltwosides}
在调用 \opt{fntef} 宏包选项的同时,旧版本 \CTeX{} 宏包由于需要支持 \pkg{CCT}
系统,会将以 |\CJK| 开头的 \tn{CJKunderline} 等宏换名为以 |\CTEX| 开头的
\tn{CTEXunderline} 等宏。此功能在新版本的 \CTeX{} 宏集中已失去意义。此外,
在 \pdfTeX{} 引擎下,用于设置格式的 \tn{CJKunderdotbasesep} 等宏也被更名为
\tn{CTEXunderdotbasesep} 等宏。

在新版本中,上述由 \opt{fntef} 衍生的相关命令和环境均被移除。
\end{function}

\begin{function}{\CTEXsetfont}
更新当前的中文字体信息,包括当前字距(\tn{ccwd})和段首缩进(\tn{parindent})。
一般来说,用户无需使用这个命令。
\end{function}

\begin{function}{\CTEXindent}
更新 \tn{ccwd} 宽度后设置 |\parindent=2\ccwd|。过时命令。
\end{function}

\begin{function}{\CTEXnoindent}
设置 |\parindent=0pt|。过时命令。
\end{function}

\begin{function}{\CTEXsetup}
\begin{syntax}
  \tn{CTEXsetup}\oarg{选项}\Arg{标题}
\end{syntax}
相当于设置了
\texttt{\tn{ctexset}\{ \meta{标题} = \Arg{选项} \}}。
过时命令。
\end{function}

\begin{function}{\CTEXoptions}
\begin{syntax}
  \tn{CTEXoptions}\oarg{选项}
\end{syntax}
相当于设置了
\texttt{\tn{ctexset}\Arg{选项}}。
过时命令。
\end{function}

\begin{function}{\Chinese}
\begin{syntax}
  \tn{Chinese}\Arg{counter}
\end{syntax}
新版宏集中 \tn{chinese} 统一了旧版本中 \tn{chinese} 和 \tn{Chinese} 的功能。因此,
该命令已过时。
\end{function}

\begin{function}{captiondelimiter}
原为 \tn{CTEXoptions} 命令的选项,用于控制 \tn{caption} 编号后面的标点。此选
项已过时,并在新版本的 \CTeX 宏包中失效。

可以使用 \pkg{caption} 宏包的 \opt{labelsep} 选项来完成同样的功能。
\begin{ctexexam}
  % 代替 \CTEXoptions[captiondelimiter={:}]
  \usepackage{caption}
  \captionsetup{labelsep=colon}
\end{ctexexam}
\end{function}

\subsection{\CTeX\ 1.02c 以后的 SVN 开发版}

\CTeX 宏包在 1.02c 版本(2011/03/11)之后在 Google code 上的 SVN 开发版本,
内部版本号一直升到 1.11 版,但从未正式发布。SVN 开发版在 1.02c 版本的基础上
新增的功能在第 2 版中大多继承了过来,但新增的命令与选项都不再保持兼容。

\CTeX 宏包第 2 版不保证对未发布的 SVN 开发版兼容。

\subsection{\CTeX\ 2.2 之前的版本}

\begin{function}{part/beforeskip, chapter/beforeskip, section/beforeskip,
  subsection/beforeskip, subsubsection/beforeskip, paragraph/beforeskip,
  subparagraph/beforeskip}
  在 \CTeX\ 2.2 之前的版本中,\opt{beforeskip} 选项的符号还用于确定章节标题后
  首段的缩进。当 \opt{beforeskip} 是负值时,章节标题后的第一段按英文文档的排版
  习惯,没有首行缩进,否则保留首行缩进。

  这一特性在 2.2 版和后续版本中不再保留,相应的功能通过新的 \opt{afterindent}
  选项来设置。如果原先设置 \opt{beforeskip} 为负值,在新版本中需要改为正值,
  并设置相应的 \opt{afterindent} 选项为 \opt{false}。
\end{function}

\begin{function}{section/afterskip, subsection/afterskip,
  subsubsection/afterskip, paragraph/afterskip, subparagraph/afterskip}
  在 \CTeX\ 2.2 之前的版本中,对于 \tn{section} 级以下标题,\opt{afterskip}
  选项的符号用于确定标题与随后正文是否排在同一段。
  如果是正值,则正文另起一段,否则标题与随后正文排在同一段,
  \opt{afterskip} 的绝对值给出水平间距。

  这一特性在 2.2 版和后续版本中不再保留,相应的功能通过新的 \opt{runin}
  选项来设置。如果原先设置 \opt{afterskip} 为负值,在新版本中需要改为正值,
  并设置相应的 \opt{runin} 选项为 \opt{true}。
\end{function}

\subsection{\CTeX\ 2.4.1 和 2.4.2}

\begin{function}{part/fixbeforeskip, chapter/fixbeforeskip}
  这两个选项已经被删除,相应功能由新的选项 \opt{fixskip} 提供。
\end{function}

\section{宏集依赖情况与手工安装方法}
\label{sec:dep-ins}

本节介绍 \CTeX{} 宏集的依赖情况,并介绍手工编译安装的具体方法。
通常用户只需参照第 \ref{subsec:easy-ins}~节介绍的方法,使用发行版自带的宏包管理器安装
本宏集。

\CTeX{} 宏集有两个源文件:\file{ctex.dtx}、\file{ctexpunct.spa}。
使用不同的编译方式时,\CTeX{} 依赖的宏包略有不同。在手工安装 \CTeX{} 宏集之前,请确保
你的 \TeX{} 发行版中已经正确安装了这些宏包。\CTeX{} 依赖宏包的详情叙述如下:

\begin{itemize}
  \item \pkg{expl3}、\pkg{xparse} 和 \pkg{l3keys2e} 宏包。它们属于 \pkg{l3kernel}
  和 \pkg{l3packages} 宏集。
  \item \pkg{indentfirst} 宏包,属于 \pkg{tools} 宏集。
  \item \pkg{everysel} 宏包,属于 \pkg{ms} 宏集。
  \item \pkg{ulem} 宏包。
  \item \pkg{zhnumber} 宏包。
  \item[\ding{229}] 以上是各种编译方式都必需的依赖项。
  \item \pkg{CJK} 宏集。
  \item \pkg{CJKpunct} 宏包。
  \item \pkg{xCJK2uni} 宏包。
  \item \pkg{zhmetrics} 宏包。
  \item \pkg{zhmCJK} 宏包,它还依赖 \pkg{oberdiek} 宏集。
  \item[\ding{229}] 以上是使用 \pdfLaTeX{} 或 \LaTeX{} + \dvipdfmx{} 的编译方式所需要
  的依赖项,其中 \pkg{zhmCJK} 是可选的。
  \item \pkg{xeCJK} 宏集,它还依赖
  \begin{itemize}
    \item \pkg{xtemplate} 宏包,它属于 \pkg{l3packages} 宏集。
    \item \pkg{fontspec} 宏包。
  \end{itemize}
  \item \pkg{environ} 宏包,它还依赖 \pkg{trimspaces} 宏包。
  \item[\ding{229}] 以上是使用 \XeLaTeX{} 编译时的依赖项。
  \item \pkg{luatexja} 宏包,它还依赖
  \begin{itemize}
    \item \pkg{adobemapping} 宏包。
    \item \pkg{lualibs} 宏包。
    \item \pkg{luaotfload} 宏包。
    \item \pkg{luatexbase} 宏包,它还依赖 \pkg{ctablestack} 宏包。
    \item \pkg{oberdiek} 宏集。
    \item \pkg{xkeyval} 宏包。
    \item \pkg{etoolbox} 宏包。
  \end{itemize}
  \item \pkg{fontspec} 宏包。
  \item \pkg{xunicode-addon} 宏包,属于 \pkg{xeCJK} 宏集,它还依赖
  \begin{itemize}
    \item \pkg{xunicode} 宏包,它还依赖
    \begin{itemize}
      \item \pkg{graphics} 宏集。
      \item \pkg{graphics-cfg} 宏包。
      \item \pkg{graphics-def} 宏包。
    \end{itemize}
  \end{itemize}
  \item[\ding{229}] 以上是使用 \LuaLaTeX{} 编译时的依赖项。
  \item \pkg{pxeverysel} 宏包,属于 \pkg{platex-tools} 宏集。
  \item \pkg{zhmetrics-uptex} 宏包。
  \item[\ding{229}] 以上是使用 \upLaTeX{} 编译时的依赖项。
\end{itemize}

出于一些原因,\pkg{zhmCJK} 尚未被收入 \TeXLive{} 和 \MiKTeX。因此,若
你希望使用 \pkg{zhmCJK} 作为 \CTeX{} 宏集的底层中文支持方式,那么你需要自行安装该宏包。
\pkg{zhmCJK} 的安装较为复杂。我们建议你
\begin{enumerate}
  \item 从 CTAN 下载 \pkg{zhmCJK} 宏包的
  \href{http://mirrors.ctan.org/install/language/chinese/zhmcjk.tds.zip}
  {TDS 安装包},
  \item 按目录结构将文件复制到 \TeX{} 发行版的本地 TDS 根目录,
  \item 最后执行 \bashcmd{texhash} 刷新 \TeX{} 发行版的 ls-R 数据库以完成安装。
\end{enumerate}
其他细节,可参照其
\href{http://mirrors.ctan.org/language/chinese/zhmcjk/zhmCJK.pdf}{宏包手册}
中第 3 节的指导。

\emph{\CTeX{} 宏集已被 \TeXLive{} 和 \MiKTeX{} 收录,若无特别理由,
我们强烈建议用户使用宏包管理器安装本宏集。}

若要手工安装,请遵循如下步骤:
\begin{enumerate}
  \item 从 CTAN 下载 \CTeX{} 宏集的
  \href{http://mirrors.ctan.org/install/language/chinese/ctex.tds.zip}
  {TDS 安装包},
  \item 按目录结构将文件复制到 \TeX{} 发行版的本地 TDS 根目录,
  \item 最后执行 \bashcmd{texhash} 刷新 \TeX{} 发行版的 ls-R 数据库以完成安装。
\end{enumerate}

\section{开发人员}

\begin{itemize}
\item 吴凌云 (\email{aloft@ctex.org})
\item 江疆 (\email{gzjjgod@gmail.com})
\item 王越 (\email{yuleopen@gmail.com})
\item 刘海洋 (\email{LeoLiu.PKU@gmail.com})
\item 李延瑞 (\email{LiYanrui.m2@gmail.com})
\item 陈之初 (\email{zhichu.chen@gmail.com})
\item 李清 (\email{sobenlee@gmail.com})
\item 黄晨成 (\email{liamhuang0205@gmail.com})
\end{itemize}

目前比较活跃的开发维护人员是刘海洋、李清和黄晨成。


\begin{thebibliography}{9}
\bibitem{knuthtex1986}
\textsc{Donald~Ervin Knuth}.
\newblock \textit{The {{\TeX{}book}}}, \textit{Computers \& Typesetting},
  volume~A.
\newblock Addison-Wesley, 1986

\bibitem{mittelbach2004}
\textsc{Frank Mittelbach} and \textsc{Michel Goossens}.
\newblock \textit{The {{\LaTeX}} Companion}.
\newblock Tools and Techniques for Computer Typesetting. Boston:
  Addison-Wesley, second edition, 2004

\end{thebibliography}

\end{documentation}


\IndexLayout
\IndexPrologue{%
  \section{索引}%
}
\PrintIndex

\end{document}
