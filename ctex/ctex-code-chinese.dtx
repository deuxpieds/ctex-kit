% \iffalse meta-comment
%
% Copyright (C) 2003--2020
% CTEX.ORG and any individual authors listed elsewhere in this file.
% --------------------------------------------------------------------------
%
% This work may be distributed and/or modified under the
% conditions of the LaTeX Project Public License, either
% version 1.3c of this license or (at your option) any later
% version. This version of this license is in
%    http://www.latex-project.org/lppl/lppl-1-3c.txt
% and the latest version of this license is in
%    http://www.latex-project.org/lppl.txt
% and version 1.3 or later is part of all distributions of
% LaTeX version 2005/12/01 or later.
%
% This work has the LPPL maintenance status `maintained'.
%
% The Current Maintainers of this work are Leo Liu, Qing Lee and Liam Huang.
%
% --------------------------------------------------------------------------
%
% \fi
%
% \section{\opt{chinese} 方案的其他设置}
%
%    \begin{macrocode}
%<*scheme&chinese>
%    \end{macrocode}
%
% \opt{chinese} 在标准文档类下的页面格式总采用 \texttt{headings}。
%    \begin{macrocode}
%<article|book|report>\pagestyle { headings }
%    \end{macrocode}
%
% 日期格式。
%    \begin{macrocode}
\keys_set:nn { ctex } { today = small }
%    \end{macrocode}
%
% 若用户未设置宏包选项 \opt{autoindent},则自动调整首行缩进。
%    \begin{macrocode}
\ctex_if_autoindent_touched:F
  { \keys_set:nn { ctex } { autoindent = true } }
%    \end{macrocode}
%
% 使用标题定义时的设置。首先是命题名字汉化。\cls{beamer} 需要汉化定理名称。
%    \begin{macrocode}
%<*!generic>
\str_if_eq:onTF { \g_@@_encoding_tl } { GBK }
%<*beamer>
  {
    \uselanguage { ChineseGBK }
    \languagealias { chinese } { ChineseGBK }
    \ctex_file_input:n { ctex-name-gbk.cfg }
  }
  {
    \uselanguage { ChineseUTF8 }
    \languagealias { chinese } { ChineseUTF8 }
    \ctex_file_input:n { ctex-name-utf8.cfg }
  }
%    \end{macrocode}
% 让 \pkg{translator} 包优先查找中文翻译。
%    \begin{macrocode}
\clist_put_left:Nn \trans@languagepath { chinese }
%</beamer>
%<*!beamer>
  { \ctex_file_input:n { ctex-name-gbk.cfg } }
  { \ctex_file_input:n { ctex-name-utf8.cfg } }
%    \end{macrocode}
%
% 对 \cls{beamer} 以外的文档类,若用户未设置宏包选项 \opt{zihao},则设置 \tn{normalsize}
% 为五号字。\cls{beamer} 不调整默认字体大小。
%    \begin{macrocode}
\int_compare:nNnF \g_@@_font_size_int > { -1 }
  { \int_gset:Nn \g_@@_font_size_int { 0 } }
%    \end{macrocode}
%
% 对 \cls{beamer} 以外的文档类,若用户未设置宏包选项 \opt{linespread},则设置行
% 距初始值为 $1.3\times 1.2=1.56$ 倍字体大小。\cls{beamer} 不调整行距。
%    \begin{macrocode}
\fp_compare:nNnT { \l_@@_line_spread_fp } ? { \c_zero_fp }
  { \fp_set:Nn \l_@@_line_spread_fp { 1.3 } }
%</!beamer>
%</!generic>
%    \end{macrocode}
%
% 不使用标题定义时的通用设置。
%    \begin{macrocode}
%<*generic>
\tl_set:Nn \l_@@_tmp_tl { beamer }
\tl_if_eq:NNTF \c_@@_std_class_tl \l_@@_tmp_tl
  {
    \str_if_eq:onTF { \g_@@_encoding_tl } { GBK }
      {
        \uselanguage { ChineseGBK }
        \languagealias { chinese } { ChineseGBK }
        \ctex_file_input:n { ctex-name-gbk.cfg }
      }
      {
        \uselanguage { ChineseUTF8 }
        \languagealias { chinese } { ChineseUTF8 }
        \ctex_file_input:n { ctex-name-utf8.cfg }
      }
    \clist_put_left:Nn \trans@languagepath { chinese }
  }
  {
    \str_if_eq:onTF { \g_@@_encoding_tl } { GBK }
      { \ctex_file_input:n { ctex-name-gbk.cfg } }
      { \ctex_file_input:n { ctex-name-utf8.cfg } }
    \int_compare:nNnF \g_@@_font_size_int > { -1 }
      { \int_gset:Nn \g_@@_font_size_int { 0 } }
    \fp_compare:nNnT { \l_@@_line_spread_fp } ? { \c_zero_fp }
      { \fp_set:Nn \l_@@_line_spread_fp { 1.3 } }
%    \end{macrocode}
% 若 \pkg{ctex} 宏包与标准文档类及其衍生文档类联用,则将载入 \pkg{indentfirst} 宏包,
% 实现章节标题后首个段落的段首缩进。
%    \begin{macrocode}
    \tl_if_exist:NT \c_@@_std_class_tl
      { \RequirePackage { indentfirst } }
  }
%</generic>
%    \end{macrocode}
%
%    \begin{macrocode}
%</scheme&chinese>
%    \end{macrocode}
